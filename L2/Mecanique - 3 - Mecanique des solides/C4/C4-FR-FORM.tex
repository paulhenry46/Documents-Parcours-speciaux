% !TeX spellcheck = en_US
\documentclass[french]{yLectureNote}

\title{FICHE DE RÉVISION - ESSENTIEL}
\subtitle{Dynamique des solides en contact : Lois de Coulombs}
\author{Synthèse Générée}
\date{\today}
\yLanguage{Français}

\usepackage[utf8]{inputenc}
\usepackage{geometry}
\usepackage{awesomebox}
\usepackage{amsmath} % Pour align* et la structure des formules
\usepackage{amssymb}
\usepackage{yFlatTable}

\renewcommand{\vec}{\overrightarrow}
\newcommand{\dd}{\mathrm{d}}
\newcommand{\norm}[1]{||\vec{#1}||}

\begin{document}
\setcounter{chapter}{3}
\chapter{Lois de Coulombs et TMC}

\section{Actions de Contact et Modélisation}

\begin{definition}[Action de contact]
Action mécanique (force ou moment de force) exercée par deux solides dont les surfaces sont en contact.
\end{definition}


\begin{tabular}{|l|c|l|}
\hline
\textbf{Composante} & \textbf{Vecteur} & \textbf{Rôle / Condition} \\
\hline
Tangentielle (Frottement) & \(\vec{T}\) & Dans le plan tangent (\(P\)). \\
\hline
Normale (Pression) & \(\vec{N}\) & Perpendiculaire au plan (\(P\)) : \(\vec{N}\cdot \vec{n_{2/1}}\geq 0\). \\
\hline
Moment Normal & \(\vec{M_{I,N}}\) & S'oppose à la rotation autour de la normale \(\vec{n}\). \\
\hline
Moment Tangentiel & \(\vec{M_{I,T}}\) & S'oppose à la torsion dans le plan de contact \(P\). \\
\hline
\end{tabular}

\criticalInfo{Définition du Contact Ponctuel}{
Le contact est ponctuel en \(I\) si le moment des actions mécaniques de contact est nul en \(I\) : \(\vec{M_I}(\text{actions de contact}) = \vec{0}\).
}

\section{Les Lois Empiriques de Coulombs}

\subsection{Cas sans Glissement (Adhérence)}
\begin{theorem}[Condition d'Adhérence]
La vitesse de glissement est nulle (\(\vec{v_g}=\vec{0}\)) tant que :
\[\norm{T}\leq \mu_S \norm{N}\]
\end{theorem}
\checkInfo{Notions Clés}{
\(\vec{T}\) est une force de réaction. \(\mu_S\) est le \textbf{coefficient de frottement statique}.
}

\subsection{Cas avec Glissement (Glissement Effectif)}
\begin{theorem}[Force de Frottement en Glissement]
La vitesse de glissement est non nulle (\(\vec{v_g} \neq \vec{0}\)), et la force de frottement \(\vec{T}\) s'oppose au mouvement :
\[\vec{T} = -\mu_d \norm{N} \frac{\vec{v_g}}{\norm{v_g}}\]
\end{theorem}
\criticalInfo{Utilisation pour résoudre un système}{Dans le cas sans glissement, on utilise \(\vec{v_g}=\vec{0}\) et dans l'autre \(\vec{T} = -\mu_d \norm{N} \frac{\vec{v_g}}{\norm{v_g}}\).}

\section{Théorèmes de la Mécanique et Roulement}

\subsection{Théorème du Moment Cinétique (TMC)}

\criticalInfo{TMC au Centre de Masse \(C\)}{
\[\frac{\dd \vec{L_C}}{\dd t} = \sum \vec{M_{C,ex}}\]
Pour un disque en rotation autour de  C : \[\frac{1}{2} M R^2 \dot{\omega} = R T \quad \Rightarrow \quad T = \frac{1}{2} M R \dot{\omega}\]
}

\begin{theorem}[TMC à un Point Mobile \(I\) (Formule Générale)]
\[\frac{\dd \vec{L_I}}{\dd t}+\vec{v_I}\wedge \vec{p_S} = \sum \vec{M_{I,ex}}\]
Où \(\vec{p_S} = M\vec{v_C}\) est la quantité de mouvement et \(\vec{v_I}\wedge \vec{p_S}\) est le moment de transport.
\end{theorem}

\end{document}
