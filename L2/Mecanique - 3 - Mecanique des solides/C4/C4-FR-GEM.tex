% !TeX spellcheck = en_US
\documentclass[french]{yLectureNote}

\title{Mécanique du solide : Lois de Coulomb}
\subtitle{Dynamique des solides en contact}
\author{Paulhenry Saux}
\date{\today}
\yLanguage{Français}

\professor{F.Pettinari}
\usepackage{graphicx}%----pour mettre des images
\usepackage[utf8]{inputenc}%---encodage
\usepackage{geometry}%---pour modifier les tailles et mettre a4paper
\usepackage{awesomebox}%---Pour les boites info, danger et autres
\usepackage{tikz}%---pour deiffner + d\'ependance de chemfig
% \usepackage{tabularx}%---pour dimensionner automatiquement les tableaux avec variable X
\usepackage{menukeys}%---Pour deiffner les touches de Calculatrice
\usepackage{fancyhdr}%---pour les en-t\^ete personnalis\'ees
\usepackage{blindtext}%---pour les liens
\usepackage{hyperref}%---pour les liens (\`a mettre en dernier)
\usepackage{caption}%---pour la francisation de la l\'egende table vers Tableau
\usepackage{pifont}
\usepackage{array}%---pour les tableaux
\usepackage{lipsum}
\usepackage{yFlatTable}

\usepackage{multicol}

\newcommand{\Lim}[1]{\lim\limits_{\substack{#1}}\:}
\renewcommand{\vec}{\overrightarrow}
\newcommand{\N}[0]{\mathbb{N}}
\newcommand{\dd}{\mathrm{d}}
\newcommand{\norm}[1]{||\vec{#1}||}
\newcommand{\fo}{\psi(\vec{r},t)}
\newcommand{\foe}{\psi(\vec{r},t)\*}
\newcommand{\HH}{\hat{H}}
\newcommand{\hb}{\hbar}
\newcommand{\lap}{\nabla^2}
\newcommand{\lapcc}{\frac{\partial^2 }{\partial x^2}+\frac{\partial^2 }{\partial y^2}+\frac{\partial^2 }{\partial z^2}}
\begin{document}
\setcounter{chapter}{3}
\chapter{Dynamique des solides en contact - Lois de Coulombs}
Ce chapitre étudie les lois de Coulomb sur le frottement solide qui décrivent de manière rigoureuse les données observées expérimentalement : Ce sont des lois empiriques.

\section{Actions de contact et Modélisation}
\begin{definition}[Action de contact]
Les actions mécaniques (forces ou moments de force) qu'exercent deux solides en contact l'un sur l'autre dont les surfaces sont en contact.
\end{definition}
\subsection{Composantes de l'action mécanique de contact}
L'action mécanique entre deux solides en contact dans le plan $P$ est modélisée par:
\begin{itemize}
 \item La résultante des forces de contact
 \item La somme des moments des actions de contact (normales et dans le plan)
\end{itemize}

\begin{tabular}{|l|c|l|}
\hline
\textbf{Composante} & \textbf{Vecteur} & \textbf{Description} \\
\hline
Tangentielle (Frottement) & $\vec{T}$ & Composante tangentielle des actions de contact \\
\hline
Normale (Pression) & $\vec{N}$ & Composante normale des actions de contact avec $\vec{N}\cdot \vec{n_{2/1}}\geq 0$. \\
\hline
Moment Normal & $\vec{M_{I,N}}$ & Moment qui s'oppose au mouvement de rotation autour de $\vec{n}$. \\
\hline
Moment Tangentiel & $\vec{M_{I,T}}$ & Moment qui s'oppose au mouvement de torsion dans le plan $P$. \\
\hline
\end{tabular}

\subsection{Approximation du contact rigoureusement ponctuel}
Cela correspond aux situations où le contact est considéré s'effectuer en un point $I$. Très souvent, ce point appartient au plan contenant le centre de masse.

\criticalInfo{Définition du Contact Ponctuel}{
Lorsque le moment en un point $I$ des actions mécaniques de contact est nul, tout se passe comme si ces actions de contact étaient une seule force appliquée en $I$ : on parle de contact ponctuel.
}

\warningInfo{Condition pour choisir I (Exemple du Cube sur pente)}{
Pour un cube sur une pente, $I$ vérifie la condition $\vec{M_I} = \vec{0}$.
À l'équilibre $\vec{M_I}(Poids)+\vec{M_i}(\text{Actions de Contact}) = \vec{0}$ et donc $\vec{M_i}(\text{Actions de Contact})=\vec{0}$.
}

\section{Lois de Coulombs}

\subsection{Cas sans glissement (Adhérence)}
On a alors une vitesse de glissement $\vec{v_g}=\vec{0}$.

\begin{theorem}[Lois de Coulombs sans glissement]
La vitesse de glissement reste nulle tant que la force de traction $\vec{F}$ situé dans le plan $P$ tangent en $I$ aux surfaces limitant les 2 solides n'atteint pas une valeur limite $\norm{T_{max}}$ définie par :
$$\norm{T_{max}} = \mu_S \norm{N}$$
Avec $\mu_S$ le coefficient de frottement statique indépendant de l'aire de contact.
\end{theorem}
\criticalInfo{Condition d'Adhérence}{
On a alors : $$\norm{T}\leq \mu_S \norm{N}$$ et $\vec{v_g}=\vec{0}$.
}
\subsubsection{Remarques / exemples}
On étudie un cube sur un chemin pentu d'angle $\alpha$ avec l'horizontale. Il ne faut pas confondre $\vec{T}$ le frottement et $\vec{v_g}$ le glissement. Dans ce cas, $\vec{v_g} = \vec{0}$ mais $\vec{T}\neq \vec{0}$.

\subsection{Cas avec glissement (Glissement effectif)}
On a une vitesse de glissement $\vec{v_g} \neq \vec{0}$.

\begin{theorem}[Force de Frottement en Glissement]
$$\vec{T} = -\mu_d \norm{N} \frac{\vec{v_g}}{\norm{v_g}}$$
Avec $\mu_d$ le coefficient de frottement dynamique.
\end{theorem}

\subsubsection{Remarque}
En pratique, on considère souvent que $\mu_d = \mu_s$ même si c'est en général différent.

\subsection{Synthèse des Lois de Coulombs}
\criticalInfo{À retenir}{
\begin{itemize}
 \item Pas de glissement (Adhérence) : $\vec{v_g}=\vec{0} \quad \Rightarrow \quad \norm{T}\leq \mu_S \norm{N}$
 \item Glissement effectif : $\vec{v_g}\neq\vec{0} \quad \Rightarrow \quad \vec{T} = -\mu_d\norm{N}\frac{\vec{v_g}}{\norm{v_g}}$
\end{itemize}
}

\section{Application : Déplacement d'un Solide à Vitesse Constante}
On va comparer le déplacement d'un solide cubique et d'une roue pour déterminer la force $\vec{F}$ à appliquer pour les déplacer à vitesse constante ($\vec{a}_C = \vec{0}$), les 2 solides ayant la même masse $M$.

\subsection{Cas du Cube (Glissement)}
Le contact est ponctuel en $I$. La vitesse du point $I$ est $\vec{v}_I = \dot{x}\vec{e_x}$ (glissement effectif).
\subsubsection{Bilan des Forces et PFD}
Le référentiel d'étude est Galiléen. Bilan des forces : $\sum \vec{F} = \vec{P_C}+\vec{T_I}+\vec{N_I}+\vec{F}$ (où $\vec{F}$ est la force appliquée).
Application du PFD ($M\vec{a}_C = \sum \vec{F}_{ext} = \vec{0}$ car vitesse constante) :
\begin{flalign*}
M\ddot{x}&= F+T = 0 \quad \Rightarrow \quad F = -T \\
M\ddot{y} &= -Mg + N = 0 \quad \Rightarrow \quad N = Mg \quad \text{(contact permanent)}
\end{flalign*}
\subsubsection{Application des Lois de Coulomb}
On a glissement $\vec{v_g} = \vec{v_{i1}}-\vec{v_{i2}} = \dot{x}\vec{e_x}$ ($\dot{x}>0$).
$$\vec{T} = -\mu_d \norm{N} \frac{\vec{v_g}}{\norm{v_g}} = -\mu_d Mg \vec{e_x}$$
\checkInfo{Force à Appliquer pour le Cube}{
$$F = -T = \mu_d Mg$$
}

\subsection{Cas de la Roue (CRSG)}
\subsubsection{Bilan des Forces et PFD}
Bilan des forces $\vec{P}, \vec{T}, \vec{N}, \vec{F}$.
Théorème du Centre de Masse ($M\vec{a}_C = \vec{0}$) :
$$F = -T \quad \text{et} \quad N = Mg \quad \text{(vitesse constante, contact permanent)}$$
\subsubsection{Théorème du Moment Cinétique (TMC) en C}
Le TMC en $C$ donne : $\frac{\dd \vec{L_C}}{\dd t} = \sum \vec{M_{C,ex}}$.
\begin{itemize}
    \item $\vec{L_C} = [I]_c \vec{\omega} = I_{cz}\omega \vec{e_z} = \frac{MR^2}{2}\omega \vec{e_z}$.
    \item $\sum \vec{M_{C,ex}} = \vec{M_C}(\vec{T}+\vec{N}) = \vec{CI} \wedge (\vec{T}+\vec{N}) = \vec{CI} \wedge \vec{T} = -R\vec{e_y}\wedge T\vec{e_x} = RT\vec{e_z}$.
\end{itemize}
Par identification :
$$\frac{MR^2}{2} \dot{\omega}\vec{e_z} = RT\vec{e_z} \quad \Rightarrow \quad T = \frac{MR}{2} \dot{\omega}$$
\subsubsection{Condition de Roulement Sans Glissement (CRSG)}
En CRSG, $\vec{v}_I = \vec{v}_C + \vec{\Omega} \wedge \vec{CI} = \vec{0}$.
La vitesse de glissement est nulle $\vec{v_g}=\vec{0}$.
$$\vec{v}_I = (\dot{x}_C + R\omega)\vec{e_x} = \vec{0} \quad \Rightarrow \quad \ddot{x_c} = -R\dot{\omega}$$
Puisque le mouvement est à vitesse constante, $\ddot{x}_C = 0$, donc $\dot{\omega} = 0$.
\checkInfo{Force à Appliquer pour la Roue en CRSG}{
Si $\dot{\omega} = 0$, alors $T = 0$, et donc $\boldsymbol{F = 0}$.
}
\infoInfo{Commentaire}{
Si la roue n'est pas en CRSG (elle glisse), alors on applique la loi de Coulomb avec glissement $\norm{T}=\mu\norm{N}$. On retrouve $\norm{T}=\mu Mg \Rightarrow F = \mu Mg$, soit le cas du cube.
}
%
% \subsection{Application du TMC à un point mobile I}
% On peut appliquer le TMC au point mobile $I$ (point d'application des actions de contact), pour lequel $\vec{M_I}(\vec{T}+\vec{N}=\vec{0})$.
% La formule est :
% $$\frac{\dd \vec{L_I}}{\dd t}+\vec{v_i}\wedge \vec{p_s} = \sum \vec{M_{I,ext}}$$
% \begin{itemize}
%     \item $\vec{L_I} = [I]_I \vec{\Omega} = I_{iz}\omega \vec{e_z}$ ($I_{iz}$ est obtenu avec Huygens).
%     \item $\vec{v_i}\wedge \vec{p_s} = \dot{x}\vec{e_x}\wedge M\dot{x}\vec{e_x}=\vec{0}$ (si $\vec{v}_I$ est colinéaire à $\vec{p}_S = M\vec{v}_C$).
%     \item $\sum \vec{M_{I,ext}} = \vec{M}(\vec{P})+\vec{M_i}(\vec{F}) = \vec{IC}\wedge \vec{P} + \vec{IC}\wedge \vec{F}$ (où $\vec{M_i}(\vec{T}+\vec{N}) = \vec{0}$).
% \end{itemize}

\subsection{Application du TMC à un point mobile I}
Le Théorème du Moment Cinétique (TMC) est généralement appliqué à un point fixe ou au centre de masse $C$. Cependant, il peut également être appliqué à un point mobile $I$, à condition d'ajouter un terme correctif.

\criticalInfo{Formule du TMC en un Point Mobile $I$}{
$$\frac{\dd \vec{L_I}}{\dd t}+\vec{v_I}\wedge \vec{p_S} = \sum \vec{M_{I,ext}}$$
}
Où :
\begin{itemize}
    \item $\frac{\dd \vec{L_I}}{\dd t}$ est la dérivée temporelle du moment cinétique du solide en $I$.
    \item $\vec{v_I}$ est la vitesse du point mobile $I$.
    \item $\vec{p_S} = M\vec{v_C}$ est la quantité de mouvement du solide $S$ de masse $M$ et de centre de masse $C$.
    \item $\vec{v_I}\wedge \vec{p_S}$ est le terme correctif, souvent appelé moment de transport.
    \item $\sum \vec{M_{I,ext}}$ est la somme des moments des forces extérieures appliquées au solide, calculés au point $I$.
\end{itemize}

\subsubsection{Application au cas de la Roue en Translation Pure}
Dans le cas de la roue étudiée :
\begin{itemize}
    \item Point $I$ : On choisit $I$ comme le point d'application des actions de contact $\vec{T}$ et $\vec{N}$.
    \item Moment des actions de contact en $I$ : Par définition du point d'application, le moment des actions de contact est nul en $I$ :
    $$\vec{M_I}(\vec{T}+\vec{N}) = \vec{0}$$
    \item Moment de transport ($\vec{v_I}\wedge \vec{p_S}$) :
    La vitesse du point $I$ est $\vec{v_i} = \dot{x}\vec{e_x}$ (vitesse de translation).
    La quantité de mouvement est $\vec{p_S} = M\vec{v_C} = M\dot{x}\vec{e_x}$.
    Les deux vecteurs sont colinéaires, donc leur produit vectoriel est nul :
    $$\vec{v_i}\wedge \vec{p_s} = \dot{x}\vec{e_x}\wedge M\dot{x}\vec{e_x}=\vec{0}$$
    \item Moment Cinétique en $I$ ($\vec{L_I}$) :
    $\vec{L_I}$ est donné par : $$\vec{L_I} = [I]_I \vec{\Omega} = I_{iz}\omega \vec{e_z}$$
    $I_{iz}$ est le moment d'inertie du solide par rapport à l'axe $(I, \vec{e_z})$. Il peut être obtenu grâce au théorème de Huygens : $I_{iz} = I_{Cz} + M R^2$. Pour un disque ($I_{Cz} = \frac{1}{2}MR^2$), on a $I_{iz} = \frac{1}{2}MR^2 + MR^2 = \frac{3}{2}MR^2$.
    \item Somme des Moments Extérieurs en $I$ :
    $$\sum \vec{M_{I,ext}} = \vec{M_I}(\vec{P})+\vec{M_i}(\vec{T}+\vec{N}) + \vec{M_i}(\vec{F})$$
    $$\sum \vec{M_{I,ext}} = \vec{IC}\wedge \vec{P} + \vec{0} + \vec{IC}\wedge \vec{F}$$
    Avec $\vec{IC} = R\vec{e_y}$.
    $$\vec{M_I}(\vec{P}) = R\vec{e_y}\wedge (-Mg\vec{e_y}) = \vec{0}$$
    $$\vec{M_i}(\vec{F}) = R\vec{e_y}\wedge F\vec{e_x} = -RF\vec{e_z}$$
\end{itemize}
\criticalInfo{Équation Finale du TMC en $I$}{
L'application du TMC se simplifie alors pour donner :
$$\frac{\dd \vec{L_I}}{\dd t} = \sum \vec{M_{I,ext}}$$
En utilisant $I_{iz} = \frac{3}{2}MR^2$, on obtient :
$$\frac{3}{2}MR^2\dot{\omega}\vec{e_z}= -RF\vec{e_z} \quad \Rightarrow \quad F = -\frac{3}{2}MR\dot{\omega}$$
Ce résultat est différent de celui obtenu en $C$ ($T = \frac{1}{2}MR\dot{\omega}$) mais permet, combiné au PFD, de résoudre le système.
}
\subsubsection{Cohérence des Résultats : Le Rôle du PFD}
Les résultats obtenus par l'application du TMC au centre de masse $C$ et au point de contact $I$ sont a priori différents, mais ils décrivent le même mouvement et sont parfaitement cohérents lorsqu'ils sont combinés avec le Principe Fondamental de la Dynamique (PFD) en translation.

\paragraph{Démonstration de la cohérence :}
En substituant l'expression de $T$ (issue du TMC en $C$) et de $\ddot{x}_C$ (issue de la condition CRSG) dans l'équation du PFD, on retrouve l'expression obtenue par le TMC en $I$.

\begin{align*}
F &= M \ddot{x}_C - T \\
F &= M (-R \dot{\omega}) - \left( \frac{1}{2} M R \dot{\omega} \right) \quad \text{(Substitution des expressions)} \\
F &= -M R \dot{\omega} - \frac{1}{2} M R \dot{\omega} \\
F &= \left( -1 - \frac{1}{2} \right) M R \dot{\omega} \\
F &= -\frac{3}{2} M R \dot{\omega}
\end{align*}

Ce résultat est exactement celui obtenu directement par l'application du TMC au point $I$.

\checkInfo{Conclusion}{
Les deux approches (TMC en $C$ combiné au PFD, ou TMC en $I$) sont équivalentes et mènent aux mêmes équations du mouvement. L'utilisation du TMC en $I$ est souvent privilégiée car elle annule le moment des forces de contact ($\vec{T}$ et $\vec{N}$), simplifiant le calcul de la somme des moments.
}
\end{document}
