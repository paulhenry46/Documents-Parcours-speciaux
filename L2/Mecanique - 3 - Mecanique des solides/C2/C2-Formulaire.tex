% !TeX spellcheck = en_US
\documentclass[french]{yLectureNote}

\title{Mécanique du solide}
\subtitle{Physique}
\author{Paulhenry Saux}
\date{\today}
\yLanguage{Français}

\professor{F.Pettinari}
\usepackage{graphicx}%----pour mettre des images
\usepackage[utf8]{inputenc}%---encodage
\usepackage{geometry}%---pour modifier les tailles et mettre a4paper
%\usepackage{awesomebox}%---pour les boites d'exercices, de pbq et de croquis ---d\'esactiv\'e pour les TP de PC
\usepackage{tikz}%---pour deiffner + d\'ependance de chemfig
% \usepackage{tabularx}%---pour dimensionner automatiquement les tableaux avec variable X
\usepackage{awesomebox}%---Pour les boites info, danger et autres
\usepackage{menukeys}%---Pour deiffner les touches de Calculatrice
\usepackage{fancyhdr}%---pour les en-t\^ete personnalis\'ees
\usepackage{blindtext}%---pour les liens
\usepackage{hyperref}%---pour les liens (\`a mettre en dernier)
\usepackage{caption}%---pour la francisation de la l\'egende table vers Tableau
\usepackage{pifont}
\usepackage{array}%---pour les tableaux
\usepackage{lipsum}
\usepackage{yFlatTable}

\usepackage{multicol}

\newcommand{\Lim}[1]{\lim\limits_{\substack{#1}}\:}
\renewcommand{\vec}{\overrightarrow}
\newcommand{\N}[0]{\mathbb{N}}
\newcommand{\dd}{\mathrm{d}}
\newcommand{\norm}[1]{||\vec{#1}||}
\newcommand{\fo}{\psi(\vec{r},t)}
\newcommand{\foe}{\psi(\vec{r},t)\*}
\newcommand{\HH}{\hat{H}}
\newcommand{\hb}{\hbar}
\newcommand{\lap}{\nabla^2}
\newcommand{\lapcc}{\frac{\partial^2 }{\partial x^2}+\frac{\partial^2 }{\partial y^2}+\frac{\partial^2 }{\partial z^2}}
\begin{document}
\setcounter{chapter}{1}
\chapter{Éléments cinétiques et cinématiques des solides}
\section{Cinématique des solides}
\begin{theorem}[Formule du champ des vitesses]
\[\vec{v_B} = \vec{v_A}+\vec{BA}\wedge \vec{\omega}\]
 avec B et A 2 points quelconques du solide
\end{theorem}
\begin{theorem}[Vitesse de glissement des solides]
\[\vec{g} = \vec{v_{i\in S_1}}-\vec{v_{i\in S_2}}\]
\end{theorem}
\begin{theorem}[Condition de roullement sans glissement]
\[\vec{g} =\vec{0}\]
\end{theorem}
\section{Cinétique des solides}
\begin{definition}[Moment d'inertie]
\[I_{\Delta} = \int_{(Solide)} \vec{HA}^2 \dd m\] avec \(\dd m\) l'élément élémentaire de matière.
\end{definition}
\begin{theorem}[Théorème de Huygens]
 \[I_{o\Delta}=I_{c\Delta}+Md^2\] avec O un point quelconque et C le centre de masse
\end{theorem}
\subsection{Ëéterminer le moment cinétique}
\begin{theorem}[Moment cinétique d'un point fixe lié au solide]
 \[\vec{L_{O}} = [I(S)]_O \vec{\Omega}\]
\end{theorem}
\begin{theorem}[Moment cinétique d'un point pas fixe ou pas lié au solide avec Koening]
 \[\vec{L_{O}} =\vec{L^*}+\vec{OC}\wedge M\vec{v_c}\] avec \(\vec{L^*} = \vec{L_{c}} = [I(S)]_C\vec{\omega}\).
\end{theorem}
%TODO demander différence et coexistence des 2 formules
\subsection{Déterminer l'énergie cinétique}
\begin{theorem}[Énergie cinétique avec un point fixe et lié]
 \[E_k(S) = \frac12 \vec{L_O}\cdot \vec{\Omega} = \frac12([I]_O \vec{\Omega})\cdot \vec{\Omega}\]
\end{theorem}
\begin{theorem}[Énergie cinétique avec un point non fixe ou non lié]
 \[E_k(S) = E_k^*+\frac12 M\vec{v_c}^2\] avec \(E_k^* = \frac12 \vec{L_C}\cdot \vec{\Omega} = \frac12([I]_C\vec{\omega})\cdot\vec{\Omega}\)
\end{theorem}
 \end{document}
