% !TeX spellcheck = en_US
\documentclass[french]{yLectureNote}

\title{Mécanique du solide}
\subtitle{Physique}
\author{Paulhenry Saux}
\date{\today}
\yLanguage{Français}

\professor{F.Pettinari}
\usepackage{graphicx}%----pour mettre des images
\usepackage[utf8]{inputenc}%---encodage
\usepackage{geometry}%---pour modifier les tailles et mettre a4paper
%\usepackage{awesomebox}%---pour les boites d'exercices, de pbq et de croquis ---d\'esactiv\'e pour les TP de PC
\usepackage{tikz}%---pour deiffner + d\'ependance de chemfig
% \usepackage{tabularx}%---pour dimensionner automatiquement les tableaux avec variable X
\usepackage{awesomebox}%---Pour les boites info, danger et autres
\usepackage{menukeys}%---Pour deiffner les touches de Calculatrice
\usepackage{fancyhdr}%---pour les en-t\^ete personnalis\'ees
\usepackage{blindtext}%---pour les liens
\usepackage{hyperref}%---pour les liens (\`a mettre en dernier)
\usepackage{caption}%---pour la francisation de la l\'egende table vers Tableau
\usepackage{pifont}
\usepackage{array}%---pour les tableaux
\usepackage{lipsum}
\usepackage{yFlatTable}

\usepackage{multicol}

\newcommand{\Lim}[1]{\lim\limits_{\substack{#1}}\:}
\renewcommand{\vec}{\overrightarrow}
\newcommand{\N}[0]{\mathbb{N}}
\newcommand{\dd}{\mathrm{d}}
\newcommand{\norm}[1]{||\vec{#1}||}
\newcommand{\fo}{\psi(\vec{r},t)}
\newcommand{\foe}{\psi(\vec{r},t)\*}
\newcommand{\HH}{\hat{H}}
\newcommand{\hb}{\hbar}
\newcommand{\lap}{\nabla^2}
\newcommand{\lapcc}{\frac{\partial^2 }{\partial x^2}+\frac{\partial^2 }{\partial y^2}+\frac{\partial^2 }{\partial z^2}}
\newcommand{\ex}{\vec{e_x}}
\newcommand{\ey}{\vec{e_y}}
\newcommand{\ez}{\vec{e_z}}
\begin{document}
\setcounter{chapter}{4}
\chapter{Aspect énergétiques des solides}
\section{Puissance et travail d'une action mécanique}
Le système étudié peut ne pas \^etre indéformable. On distingue également les forces extérieures du système des forces intérieures.
\section{Puissnce et travail des actions extérieures}
\subsection{Puissance et travail des actions extérieures}
\[P = \sum_i \vec{F_i}\cdot \vec{v_{Ai}}\] et \[\delta W_{ex}=\sum_i \vec{f_{ex}}\cdot \dd \vec{OA_i}\]
\subsection{Forces intérieures}
\[P_{in} = \sum_{ij}\vec{F_{i,j}}\cdot \vec{OA_i}\]
\warningInfo{Propriété}{\[P_{in/R} = P_{in/R'}\] avec R' en translation /R. Ainsi, la puissance des forces intérieures ne dépend pas du référenrentiel d'étude : on peut donc l'appliquer au référentiel du solide.}
\section{Puissance et travail des actions mécaniques appliquées à un solide}
\subsection{Formule générale}
On considère 2 points A et P appartenants au solide S. A est un point particulier et P quelconque de S.

Le solide S est soumis à l'ensemble de forces volumiques décrit par une densité volumique de force \(\vec{f_v}\). On cherche la puissance associée à cet ensemble de forces dans le référentiel d'étude R : \(P = \int_{(S)}\vec{f_v}\cdot \vec{v_{P\in S}}\dd V\). On applique la formule de Varignon :
\[\vec{V_{P}} = \vec{v_A} + \vec{PA}\wedge \vec{\Omega}\]
Donc \[P = \int (\vec{f_v}\dd v)\cdot \vec{V_A}+\vec{\Omega}\cdot (\int(\vec{AP}\wedge \vec{f_v})\dd v)\]
On en déduit que \(P = \sum\vec{F}\cdot \vec{v_A}+\vec{\Omega}\cdot \vec{M_A}(\sum \vec{F})\)

\warningInfo{Remarque}{
\begin{itemize}
%  \item La puissance de l'ensmble des forces s'exerçants sur le solide est le produit du torseur force par le torseur cinématique
 \item Lorsque l'action mécanique est appliquée en un point A : \[\vec{M_A(\vec{F})} = \vec{0} \Rightarrow \vec{F}\cdot \vec{V_{A/R}}\]
 \item Si l'action mécanique est un couple \(\gamma\), la puissance associée est \(P = \vec{\Omega}\cdot \vec{\gamma}\)
\end{itemize}
}
\subsection{Puissance et travail des actions intérieures}
On choisit R' le référentiel lié au solide et on utilise la propriété \(P_{in/R} = P_{in/R'}\)
\criticalInfo{Puissance de force intérieure}{
Dans un solide, la puissance des forces intérieures est nulle.}
\subsection{Puissance et travail des forces de pesnateur terrestre}
Le poids est une force extérieure de densité volumique \(\vec{f_v} = \rho \vec{g}\)

Donc \[P_{poids} = \int_v \vec{v_a}\cdot \rho \vec{g}\dd v = \int_v \vec{v_a}\cdot \rho \dd v\cdot \vec{g} = \int \vec{v_a}\cdot \dd m \cdot \vec{g} = M\vec{v_c}\cdot \vec{g} = M\vec{g}\cdot \vec{v_c}\]

Tout se passe comme s'il n'y avait qu'une seule force appliquée au centre de masse.

Donc \(\delta w(Poids) = \int_v \rho \vec{g}\cdot \dd \vec{OA} = \int \dd m \dd \vec{AO}\cdot \vec{g} = M\dd \vec{OC}\cdot \vec{g} = M\vec{g}\cdot \dd \vec{OC} \Rightarrow w(Poids) = M\vec{g}\cdot{OC}+Cst\)

On utilisera en pratique ces 2 formules :
\[P_{poids} = M\vec{g}\cdot \vec{v_c}, W(Poids) = M\vec{g}\cdot \vec{OC}+Cst\]
\subsection{Puissance et travail des actions de contact}
On considère 2 solides en contact pontuel en I.
\subsubsection{Puissance de l'action de S2 sur S1 /R}
On applique la formule : \(P(S2\to S1) = \vec{R}\cdot \vec{V_{I_1}}+ M_i\cdot \vec{\Omega}\) avec \(\vec{R}\) la résultante des actions de contact (\(T+N\)) et \(\vec{M_I}\) le moment des actions de contact (AC) qui vaut 0.
\subsubsection{Puissance de l'action mécanique de S1 sur S2/R}
De la m\^eme façon, on a \(\vec{R_{S1\to S2}}\cdot \vec{V_{i2}}+(M_{i}\cdot \omega = \vec{0})\)

\subsubsection{puissance totale}
\[P_t^{AC} = \vec{R_{S2\to S_1}}\cdot \vec{V_{I_1}}+\vec{R_{S1\to S_2}} = \vec{R_{S2\to S1}}\cdot \vec{v_{g1/2}}\]

La puissance totale des actions de contact est nulle lorsque :
\begin{itemize}
 \item Il n'y a pas de glissement
 \item Il n'y a pas de frottement : \(\mu = 0\) ou \(T = 0\) o u \(\vec{R} = \vec{T}+\vec{N}\)
\end{itemize}
Lorsqu'il y a du glissement et T non nul : \(P_t^{AC}<0\).

Lorsque le support est fixe\marginCritical{i.e. 1 des 2 solides}, \(P_{support\to solide} = P^{AC} = \vec{R_0}\cdot \vec{v_g}=\vec{T}\cdot \vec{V}\)
\subsection{Notion de liaison pivot parfaite}
\subsubsection{Définition}
\begin{definition}[Liaison]
C'est une action de contact entre 2 solides. Elle est parfaite quand les actions mécaniques ont une puissance nulle
\end{definition}
\subsubsection{Liaison double parfaite}
Une liaison pivot autorise uniquement un mouvement de rotation autour d'un axe fixe \(\Delta\) passant par O. Si cette liaison est parfaite, \(P_{liaison} = 0 = \vec{R}\cdot \vec{V_O} + \vec{M_O^{AC}}\cdot \vec{\Omega}\). Si O est fixe car la liaison s'effectue en O : \(\vec{V_{O/R}}\) et on peut écrire \(P_{liaison} = 0 = \vec{M_O^{AC}}\cdot \vec{\Omega} \Rightarrow \vec{M_O}\perp \vec{\Omega}\).

Si \(\vec{\Omega}\) est porté par un axe \((Oz)\) : \(M_{oz} = 0\)

Conséquence : Avec \(M_{oz} = 0\), on applique le TMC en O que l'on projette :

\(\vec{M_{o,z}^{AC}} = 0, (\frac{\dd \vec{L_o}}{\dd t})\cdot \vec{e_z} = M_{o,z}(autres forces)\).

\section{Théorèmes énergétiques}
\subsection{Théorème de la puissance cinétique}
On note \(E_k\) l'énergie cinétique du solide par rapport à R.

\begin{theorem}[Théorème de la puissance cinétique]
\[\frac{\dd E_k}{\dd t} = P_{ex}+P_{in} = P_{ex}\] pour un solide indéformable
\end{theorem}
\subsection{Théorème de l'énergie cinétique}
\begin{theorem}[Théorème de l'énergie cinétique]
 \[\dd E_k = \delta W^{in}+\delta \omega^{ex}\] pour un solide indéformable
\end{theorem}
On peut bien sur intégrer pour trouver la variation totale :
\[\Delta E_k = W_{in} + W_{ex} = W_{ex}\]
Cest 3 formules sont appelées théorème de l'énergie cinétique.

\section{Théorème de la puissance mécanique et de l'énergie cinétique}

\subsection{Énergie potentielle}
Lorsque le travail élémentaire des forces peut se mettre sous la forme d'une différentielle d'une fonction, i.e. \(\delta W(\vec{F})= -\dd E_p\), alors \(E_p\) est l'énergie potentielle associée à cette force. Ces forces sont donc conservatives
\subsection{Exemples de forces}
\subsubsection{Énergie potentielle de pesanteur}
\(\dd E_{p,p}=-\delta W = -\int_{solide}\rho \vec{g}\dd v \cdot \dd \vec{OP} = -\vec{g}\int_{solide}\rho \dd v \dd \vec{OP} = -\vec{g}\int \dd m \dd \vec{OP}\). P est un point du solide de masse \(\dd m = \rho \dd v \Rightarrow \dd E_{pp} = -\vec{g}M\dd \vec{OC}\)

Donc \[E_{pp} = -M\vec{g}\cdot \vec{OC} + Cst\]
\subsubsection{Énergie potentielle aux forces d'inertie}
On rencontre des forces d'inertie lorsqu'on travaille dans un référentiel en mouvement par rapport à un référentiel galiléen.

\begin{enumerate}
 \item Rotation uniforme de R : \[\vec{a_e} = \vec{a_{e, R'/R}}= \vec{\Omega}\wedge \vec{\Omega}\wedge \vec{OA_i} = -\omega^2 \vec{H_iA_i}\]avec \(H_i\) est le projeté orthogonal de $A_i$ suivant l'axe de rotation.
\end{enumerate}
---------------Fin Eval Jeudi-------------
Pour le point matériel \(A_i\) : \(\dd E_{pie}(A_i) = -F_{ie}\cdot \dd \vec{OA_i}\) et \(\dd E_{pie} = -\vec{F_{ie}}\cdot-\dd \vec{OH_i}+\dd \vec{H_iA_i} = -m_i\omega^2 \vec{H_iA_i}\cdot \vec{\dd H_iAi} = -m_i\omega^2 \dd(\frac12 \vec{H_iA_i}^2)\)

Si on considère un solide S, alors \[\dd E_{pie} = \sum_i \dd E_{pie}(A_i) = \sum_i(-m_i\omega^2 \dd (\frac12 \vec{H_iA_i}^2))= -\frac12 \omega^2 \sum_i m_i \vec{H_iA_i}^2 + Cst = -\frac12 \omega^2 I_{oz}+Cst\]

On a donc trouvé l'énergue potentielle d'entrainement du solide

\begin{theorem}[Énergie potentielle d'entrainement pour un solide]
 \[E_{pie} = -\frac12 I_{\delta}\omega^2 + Cst\] avec \(\Delta\) l'axe de rotation du ref du solide par rapport à R
\end{theorem}

Si R' est en translation par rapport à R avec O' l'origine de R'.

Dans ce cas, l'accélération d'entrainement vaut \(\vec{ac'} = \vec{a_0}\) et donc \[\vec{Fie}(A_i) = -m_i \vec{a_e} = -m_i \vec{a_0}\] et on écrit
\[\dd E_{pie}(Solide) = \vec{a_0}\cdot M\dd \vec{OC} = M\vec{a_o}\cdot \vec{OC}+Cst\]
Donc \begin{theorem}[Énergie potentielle d'entrainemnt]
      \[\dd E_{pie}(Solide) = M\vec{a_o}\cdot \vec{OC}+Cst\]
     \end{theorem}
Remarque : Quand on étudie le mouvement dans R' on doit prendre en compte la force de coriolis. Mais d'un point de vue énergétique, cette force ne travaille pas.
\subsection{Énergie mécanique}
\subsubsection{Définition}
\begin{definition}[Énergie mécanique]
\(E_m = E_k+E_p\)
\end{definition}
\subsubsection{Théorème de la puissance mécanique et de l'énergie mécanique}
\begin{theorem}[Théorème de la puissance mécanique et de l'énergie mécanique]
\[\frac{\dd E_m}{\dd t} = M^{nc} \iff \dd E_m = \delta W^{nc}\]
\end{theorem}
\subsection{Application}
%TODO La puissance des forces non conservatives est nulle donc l'énrgie mécanique se conserve.
\subsubsection{Cylindre de centre sur un tapis roulant}
Soit S un cilindre plain, de centre C de masse M, de rayon a, posé sans vitesse initiale à t=0 sur un tapis roulant animé d'une vitesse \(\vec{v_0}=v_0\vec{e_x}\) par rapport au ref terrestre R. On considère que le contact du cylindre sur le tapis est caractérisé par le coefficient de frottement \(\mu\) et qu'il est ponctuel.

On fait le bilan des forces :

\(\vec{P} = M\vec{g}\) appliqué en C et \(\vec{R}=\vec{T}+\vec{N}\) en I.

Les forces et vitesses initiales sont dans le plan \((Cxy)\), on peut donc étudier la coupe du système.

Objectif : Déterminer la puissance des actions de contact

Dans ce problème, on a 3 inconnues : x, la position du centre de masse, \(\omega\) la rotation, T et N. On utlise le PFD et les lois de Coulomb

On utilise d'abord the TH de la quantité de mouvement : \(M\vec{a_c} = \vec{P}+\vec{T}+\vec{N}\)

Donc \(M\ddot{x} = T, M\ddot{y} = -Mg+N\)

On utilise maintenant le TMC appliqué en C : \(\frac{\dd \vec{L_c}}{\dd t} = \sum\vec{M_c,ex} = \vec{M_c(P)+\vec{M_c}(T+N)}\).

On a \(\vec{L_c} = [I]_C \vec{\omega} = I_{cz}\dot{\omega}\vec{e_z}\)

On a aussi \(\vec{M_C}(P) = 0, \vec{M_c}(T+N) = aT \vec{e_z}\)

On en déduit que \(T = \frac{I_{cz}}{a}\dot{\omega}\)

On utilise la loi de Coulomb pour la dernière relation :

On détermine d'abord \(\vec{v_g}\) au cours du temps. \(\vec{v_g} = \vec{v_{i1}}-\vec{v_{i2}}\).

On utilise Varignon pour déterminer \(\vec{v_1}\) et \(\vec{v_{i2}}=v_0\vec{e_x}\)

Donc \(\vec{v_g} = ((\dot{x}+a\omega)-v_0)\vec{e_x}\).

A t=0, \(\vec{v_g} = -v_0\vec{e_x}< 0\)

Donc T>0 et \(\vec{T} = \mu Mg \vec{e_x}\)

On déduit des relations 1 et 4 que  \(M\ddot{x} = T = \mu Mg \Rightarrow x(t) = \frac12 \mu gt^2\)

Avec les relations 3 et 4 : \(T = \frac{Ma\dot{\omega}}{2} = \mu Mg \Rightarrow w(t) = \frac{2}{a} \mu gt\).

On en déduit la vitesse de glissement : \((3\mu gt-v_0)\vec{e_x}\)

Partie énergétique


On cherche la puissance du tapis roulant sur le solide.

Donc \((\vec{T}+\vec{N})\cdot \vec{v_i1}  =\vec{T}\cdot \vec{v_i1} = 3\mu^2Mg^2t\)

On cherche maintenant la puissance du solide sur le tapis roulant. On a donc \(-\vec{T}\cdot v_0\vec{e_x} = -\mu Mg \vec{e_x} = T(\dot{x}+a\omega -v_0)<0\)


Application du TH de l'énergie cinétique

On travaille avec \(t_1\) l'instant pour lequel \(\vec{v_g}\) s'annule.

On cherche la variation d'énergie cinétique entre t=0 et t=t1.

On a \(v_g(t=0) = -v_0, V_g(t) = 0 \Rightarrow t_1 = \frac{v_0}{3\mu g}\)

Avec \(E_k = E_k*+\frac12MV_c^2 = \frac12 I_{cz}\omega^2+\frac12 M\dot{x}\). On a \(\omega(t1) = \frac{2}{a} \mu gt_1 = \frac{2}{3a}v_0\)

Et \(E_k(t_1) = \frac16MV_0^2\) et donc \(Delta E_k = \frac16 MV_0^2\)

On cherche \(W_{0\to t_1}(P+T+N)=\int^{t}P\dd t = 3m\mu^2Mg \frac12 (\frac{V_0}{3\mu g})^2\)

On trouve que \(W_{0\to t_1} = \Delta e_k\)

\subsubsection{Différents énergie potneitllees}
\begin{itemize}
 \item \(E_pp = -M\vec{g}\cdot \vec{OC}+Cst\)
 \item \(E_{pie} = -\frac12 I_{\Delta}\vec{\omega}^2 + Cst\) % Savoir démontrer
 \item \(E_{pe} = \frac12 k\Delta l^2 \) avec k la raudeu et \(\Delta l\) l'énergie du ressort.
\end{itemize}


\end{document}
