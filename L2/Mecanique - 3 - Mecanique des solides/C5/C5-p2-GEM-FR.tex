% !TeX spellcheck = en_US
\documentclass[french]{yLectureNote}

\title{Mécanique du solide}
\subtitle{Physique}
\author{Paulhenry Saux}
\date{\today}
\yLanguage{Français}

\professor{F.Pettinari}
\usepackage{graphicx}%----pour mettre des images
\usepackage[utf8]{inputenc}%---encodage
\usepackage{geometry}%---pour modifier les tailles et mettre a4paper
%\usepackage{awesomebox}%---pour les boites d'exercices, de pbq et de croquis ---d\'esactiv\'e pour les TP de PC
\usepackage{tikz}%---pour deiffner + d\'ependance de chemfig
% \usepackage{tabularx}%---pour dimensionner automatiquement les tableaux avec variable X
\usepackage{awesomebox}%---Pour les boites info, danger et autres
\usepackage{menukeys}%---Pour deiffner les touches de Calculatrice
\usepackage{fancyhdr}%---pour les en-t\^ete personnalis\'ees
\usepackage{blindtext}%---pour les liens
\usepackage{hyperref}%---pour les liens (\`a mettre en dernier)
\usepackage{caption}%---pour la francisation de la l\'egende table vers Tableau
\usepackage{pifont}
\usepackage{array}%---pour les tableaux
\usepackage{lipsum}
\usepackage{yFlatTable}

\usepackage{multicol}

\newcommand{\Lim}[1]{\lim\limits_{\substack{#1}}\:}
\renewcommand{\vec}{\overrightarrow}
\newcommand{\N}[0]{\mathbb{N}}
\newcommand{\dd}{\mathrm{d}}
\newcommand{\norm}[1]{||\vec{#1}||}
\newcommand{\fo}{\psi(\vec{r},t)}
\newcommand{\foe}{\psi(\vec{r},t)\*}
\newcommand{\HH}{\hat{H}}
\newcommand{\hb}{\hbar}
\newcommand{\lap}{\nabla^2}
\newcommand{\lapcc}{\frac{\partial^2 }{\partial x^2}+\frac{\partial^2 }{\partial y^2}+\frac{\partial^2 }{\partial z^2}}
\newcommand{\ex}{\vec{e_x}}
\newcommand{\ey}{\vec{e_y}}
\newcommand{\ez}{\vec{e_z}}
\begin{document}
\setcounter{chapter}{4}
\chapter{Aspect énergétique des solides}
\section{Puissance et Travail des Actions Mécaniques}

\subsection{Puissance et Travail des Actions Extérieures}
\begin{theorem}[Puissance totale]
$$P = \sum_i \vec{F_i}\cdot \vec{v_{A_i}}$$
\end{theorem}
\begin{theorem}[Travail élémentaire]
$$\delta W_{ex}=\sum_i \vec{f_{ex}}\cdot \dd \vec{OA_i}$$
\end{theorem}

\subsection{Formule Générale}
La puissance $P$ des actions mécaniques agissant sur le solide $S$ est donnée par :
\begin{theorem}[Puissance d'une action mécanique]
$$P = \vec{R} \cdot \vec{v_A} + \vec{M_A} \cdot \vec{\Omega}$$
\end{theorem}
\begin{itemize}
    \item $\vec{R} = \sum \vec{F}$ (Résultante), $\vec{M_A}$ (Moment en A), $\vec{v_A}$ (Vitesse en A), $\vec{\Omega}$ (Vecteur Rotation).
\end{itemize}
\warningInfo{Simplifications}{\begin{itemize}
    \item \textbf{Force Ponctuelle} : Si l'action mécanique est appliquée en $A$ ($\vec{M_A}(\vec{F}) = \vec{0}$), alors $P = \vec{F}\cdot \vec{V_{A/R}}$.
    \item \textbf{Couple pur ($\vec{\gamma}$)} : La puissance associée est $P = \vec{\Omega}\cdot \vec{\gamma}$.
\end{itemize}}

\section{Cas Spécifiques de Puissance et Travail}

\subsection{Puissance des Actions Intérieures}
\begin{theorem}[Puissance des forces intérieures dans un solide]
Dans un solide, la puissance des forces intérieures est nulle.
$$P_{int} = 0$$
\end{theorem}
(Rappel: La puissance des forces intérieures ne dépend pas du référentiel d'étude.)

\subsection{Puissance et Travail du Poids ($M\vec{g}$)}
Le poids agit comme une force unique $M\vec{g}$ appliquée au Centre de Masse ($C$).
\begin{theorem}[Puissance du Poids]
$$P_{poids} = M\vec{g}\cdot \vec{v_c}$$
\end{theorem}
\begin{theorem}[Travail du Poids]
$$W(Poids) = M\vec{g}\cdot \vec{OC} + Cst$$
\end{theorem}

\subsection{Puissance des Actions de Contact (AC)}
Soient $S_1$ et $S_2$ en contact ponctuel en $I$.
\begin{theorem}[Puissance totale des Actions de Contact]
$$P_t^{AC} = \vec{R_{S_2 \to S_1}}\cdot \vec{v_{g_{1/2}}}$$
\end{theorem}
$P_t^{AC}$ est nulle si :
\begin{itemize}
 \item Pas de glissement ($\vec{v_{I_{1/2}}} = \vec{0}$) OU
 \item Pas de frottement ($T = 0$).
\end{itemize}
\section{Liaisons Parfaites}

\subsection{Définition}
\begin{definition}[Liaison Parfaite]
C'est une action de contact entre 2 solides. Elle est parfaite quand les actions mécaniques qu'elle exerce ont une puissance nulle.
\end{definition}

\subsection{Liaison Pivot Parfaite}
Une liaison pivot parfaite s'effectuant en un point fixe $O$ implique :
\begin{theorem}[Propriété d'une liaison pivot]
$$\vec{M_O^{AC}}\cdot \vec{\Omega} = 0$$
\end{theorem}
\warningInfo{Conséquences}{
\begin{itemize}
 \item Le moment $\vec{M_O^{AC}}$ est \textbf{perpendiculaire} au vecteur rotation $\vec{\Omega}$.
 \item Si $\vec{\Omega}$ est porté par $(Oz)$, alors $M_{O,z}^{AC} = 0$.
\end{itemize}
}
\section{Théorèmes Énergétiques}

\subsection{Théorème de la Puissance Cinétique}
On note $E_k$ l'énergie cinétique du solide par rapport à $R$.
\begin{theorem}[Théorème de la puissance cinétique]
$$\frac{\dd E_k}{\dd t} = P_{ex}+P_{in} = P_{ex} \quad (\text{pour un solide indéformable})$$
\end{theorem}

\subsection{Théorème de l'Énergie Cinétique}
\begin{theorem}[Théorème de l'énergie cinétique]
$$\Delta E_k = W_{in} + W_{ex} = W_{ex} \quad (\text{pour un solide indéformable})$$
\end{theorem}
L'expression élémentaire est $\dd E_k = \delta W^{in}+\delta W^{ex}$.

\section{Énergie Potentielle ($E_p$)}

\subsection{Forces Conservatives}
\begin{definition}[Force Conservative]
Une force est conservative si le travail élémentaire qu'elle produit se met sous la forme d'une différentielle d'une fonction, i.e., $\delta W(\vec{F})= -\dd E_p$. $E_p$ est l'énergie potentielle associée à cette force.
\end{definition}

\subsection{Énergie Potentielle de Pesanteur ($E_{pp}$)}
\begin{theorem}[Énergie Potentielle de Pesanteur]
$$E_{pp} = -M\vec{g}\cdot \vec{OC} + Cst$$
\end{theorem}
\subsection{Énergie potentielle liée aux forces d'inertie}
\begin{theorem}[Énergie potentielle d'entrainement]
Pour une transalatiobn :
      \[E_{pie} = M\vec{a_o}\cdot \vec{OC}+Cst\]
Pour une rotation autour d'un axe (Oz, ici) :
\[E_{pie} = -\frac12 \omega^2 I_{oz}\]
\end{theorem}

\end{document}
