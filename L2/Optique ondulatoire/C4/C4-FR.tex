% !TeX spellcheck = fr_FR
\documentclass[french]{yLectureNote}

\title{Optique ondulatoire}
\subtitle{Réseaux de Diffraction}
\author{Paulhenry Saux}
\date{\today}
\yLanguage{Français}

\professor{F.Pettinari}
\usepackage{graphicx}%----pour mettre des images
\usepackage[utf8]{inputenc}%---encodage
\usepackage{geometry}%---pour modifier les tailles et mettre a4paper
\usepackage{tikz}%---pour deiffner + dépendance de chemfig
\usepackage{awesomebox}%---Pour les boites info, danger et autres
\usepackage{menukeys}%---Pour deiffner les touches de Calculatrice
\usepackage{fancyhdr}%---pour les en-tête personnalisées
\usepackage{blindtext}%---pour les liens
\usepackage{hyperref}%---pour les liens (à mettre en dernier)
\usepackage{caption}%---pour la francisation de la légende table vers Tableau
\usepackage{pifont}
\usepackage{array}%---pour les tableaux
\usepackage{lipsum}
\usepackage{yFlatTable}
\usepackage{multicol}
\newcommand{\Lim}[1]{\lim\limits_{\substack{#1}}\:}
\renewcommand{\vec}{\overrightarrow}
\newcommand{\N}[0]{\mathbb{N}}
\newcommand{\dd}{\mathrm{d}}
\newcommand{\norm}[1]{||\vec{#1}||}
\newcommand{\fo}{\psi(\vec{r},t)}
\newcommand{\foe}{\psi(\vec{r},t)\*}
\newcommand{\HH}{\hat{H}}
\newcommand{\hb}{\hbar}
\newcommand{\lap}{\nabla^2}
\newcommand{\lapcc}{\frac{\partial^2 }{\partial x^2}+\frac{\partial^2 }{\partial y^2}+\frac{\partial^2 }{\partial z^2}}
\newcommand{\mpsi}{\(\psi\)}
\newcommand{\und}{\underline}
\DeclareMathOperator{\sinc}{sinc}

% Définition de la fonction de la fente unique pour plus de clarté
\newcommand{\psifente}{\und{\psi}_{\mathrm{fente}}}

\begin{document}

\setcounter{chapter}{3}
\chapter{Réseaux de Diffraction}
\begin{definition}[Réseau optique]
Un réseau optique est constitué de la répétation périodique d'un motif diffractant
\end{definition}
\begin{theorem}[Intensité à l'infini d'un réseau illuminé par une onde plane]
\[I(u) = I_0 \cdot \underbrace{\sinc^2(ua)}_{\text{Diffraction}} \cdot \underbrace{\frac{\sin^2(\pi N up)}{\sin^2(\pi up)}}_{\text{Interférence}}\] ou
\[I(u) = N^2 I_0\sinc^2(ua)R(2\pi u p)\]
\end{theorem}
\begin{definition}[Fonction Réseau]
La Fonction Réseau $R(\varphi)$ normalisée est :
$$R(\varphi) = \frac{\sin^2(N\varphi/2)}{N^2\sin^2(\varphi/2)}$$
Elle décrit l'intensité de l'interférence de $N$ ondes avec un déphasage successif $\varphi = 2\pi u' p$.
\end{definition}
\begin{proposition}[Représentation]
Une succession de pics centrés en \(2\pi m, m\in \mathbb{Z}\) et de largeur \(\frac{4\pi}{N}\)
\end{proposition}
\begin{theorem}[Formule Fondamentale des Réseaux]
Les angles de diffraction vérifient la relation :
$$\mathbf{p}(\sin(\theta_d)-\sin(\theta_i))=\mathbf{m}\lambda$$
\end{theorem}
\begin{definition}[Pouvoir de réolsution d'un réseau]
\[PR = \frac{\lambda}{\Delta \lambda} = mN\] avec m l'ordre de diffraction et N le nombre de fentes
\end{definition}
\end{document}
