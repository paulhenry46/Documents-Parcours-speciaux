% !TeX spellcheck = fr_FR
\documentclass[french]{yLectureNote}

\title{Optique ondulatoire}
\subtitle{Réseaux de Diffraction}
\author{Paulhenry Saux}
\date{\today}
\yLanguage{Français}

\professor{F.Pettinari}
\usepackage{graphicx}%----pour mettre des images
\usepackage[utf8]{inputenc}%---encodage
\usepackage{geometry}%---pour modifier les tailles et mettre a4paper
\usepackage{tikz}%---pour deiffner + dépendance de chemfig
\usepackage{awesomebox}%---Pour les boites info, danger et autres
\usepackage{menukeys}%---Pour deiffner les touches de Calculatrice
\usepackage{fancyhdr}%---pour les en-tête personnalisées
\usepackage{blindtext}%---pour les liens
\usepackage{hyperref}%---pour les liens (à mettre en dernier)
\usepackage{caption}%---pour la francisation de la légende table vers Tableau
\usepackage{pifont}
\usepackage{array}%---pour les tableaux
\usepackage{lipsum}
\usepackage{yFlatTable}
\usepackage{multicol}
\newcommand{\Lim}[1]{\lim\limits_{\substack{#1}}\:}
\renewcommand{\vec}{\overrightarrow}
\newcommand{\N}[0]{\mathbb{N}}
\newcommand{\dd}{\mathrm{d}}
\newcommand{\norm}[1]{||\vec{#1}||}
\newcommand{\fo}{\psi(\vec{r},t)}
\newcommand{\foe}{\psi(\vec{r},t)\*}
\newcommand{\HH}{\hat{H}}
\newcommand{\hb}{\hbar}
\newcommand{\lap}{\nabla^2}
\newcommand{\lapcc}{\frac{\partial^2 }{\partial x^2}+\frac{\partial^2 }{\partial y^2}+\frac{\partial^2 }{\partial z^2}}
\newcommand{\mpsi}{\(\psi\)}
\newcommand{\und}{\underline}
\DeclareMathOperator{\sinc}{sinc}

% Définition de la fonction de la fente unique pour plus de clarté
\newcommand{\psifente}{\und{\psi}_{\mathrm{fente}}}

\begin{document}

\setcounter{chapter}{3}
\chapter{Méthode - Étude d'un réseau}
\subsection{ Calcul de la Différence de Marche Optique ($\delta$)}

On considère deux fentes successives, $F_n$ et $F_{n+1}$, séparées par une distance $p$ (le pas du réseau). La différence de marche totale ($\delta$) entre les ondes issues de ces deux fentes est la somme de deux contributions :
\begin{itemize}
 \item La différence de marche due à l'**incidence** ($\delta_{inc}$) de l'onde plane sur le réseau.
 \item La différence de marche due à la **diffraction** ($\delta_{d}$) vers l'observateur.
\end{itemize}
\subsubsection{Contribution due à l'incidence ($\delta_{inc}$)}

L'onde incidente est plane et arrive avec un angle $\theta_i$ par rapport à la normale au réseau.

Considérons le front d'onde incident qui arrive au point $A$. Le rayon qui arrive au point $B$ (correspondant à la fente suivante) doit parcourir une distance supplémentaire $AC$ pour atteindre le même front d'onde.

$$AC = p \cdot \sin(\theta_i)$$

$$\mathbf{\delta_{inc} = p \sin(\theta_i)}$$

\subsubsection{Contribution due à la diffraction ($\delta_{d}$)}

Les ondes diffractées se propagent dans la direction $\theta_d$ (angle de diffraction).

Considérons un front d'onde diffracté. Le rayon issu de $A$ parcourt une distance supplémentaire pour rejoindre le front d'onde de référence par rapport au rayon issu de $A$ (fente $F_n$).

$$AD = p \cdot \sin(\theta_d)$$

$$\mathbf{\delta_{d} = p \sin(\theta_d)}$$

\subsubsection{Différence de Marche Totale ($\delta$)}

La différence de marche totale est la somme algébrique des deux contributions :

$$\delta = \delta_{inc} - \delta_{d}$$

$$\mathbf{\delta = p (\sin \theta_i - \sin \theta_d)}$$

\subsection{Calcul du Déphasage ($\phi$)}

Le déphasage $\phi$ est directement lié à la différence de marche $\delta$ par la relation fondamentale :

$$\phi = \frac{2\pi}{\lambda} \delta$$

En substituant l'expression de $\delta$ trouvée précédemment :

$$\mathbf{\phi = \frac{2\pi p}{\lambda} (\sin \theta_i - \sin \theta_d)}$$

Ce déphasage $\phi$ est l'argument utilisé dans la fonction réseau :

$$I(\theta) \propto \left(\frac{\sin(N\phi/2)}{N\sin(\phi/2)}\right)^2$$

On rappelle que les maxima principaux (raies brillantes) se produisent lorsque le déphasage $\phi$ est un multiple entier de $2\pi$, ce qui ramène à la formule du réseau :

$$\phi = 2m\pi \implies \frac{2\pi p}{\lambda} (\sin \theta_i + \sin \theta_d) = 2m\pi$$
$$m\lambda = p (\sin \theta_i + \sin \theta_d)$$
\subsection{Annulation de la fonction réseau}
On souhaite exprimer la valeur de \(u = \frac{\sin\theta_d-\sin\theta_i}{\lambda}\) correspondant à la première annulation de la fonction réseau. En déduire
la demi-largeur angulaire \(\Delta \theta_a\) de ce pic quand on l’exprime en fonction de \(\theta_d\) .

La fonction réseau est nulle quand l'argument du sinus du numérateur est un multiple de \(\pi\). Donc \(N\frac{\phi}{2} =\pi \Rightarrow
\phi = \frac{2\pi}{N}\).

Or, \(\phi = 2\pi up = \frac{2\pi}{N} \Rightarrow u = \frac{1}{NP} \). On le note \(\Delta u\) car c'est la varitaion depuis le centre du pic : \[\Delta u = \frac{1}{NP}\]

Mais on sait aussi que \(u = \frac{\sin\theta_d-\sin\theta_i}{\lambda}\). Pour faire apparaitre \(\Delta \theta_a\) qui est la variation de \(\theta\) depuis le centre du pic, on va différention la relation.

On sait que\marginCheck{cf Outils Mathématiques 1} \(\dd f(x) = f'(x)\dd x\). C'est cette relation que l'on va utiliser, avec comme variable \(\theta_d\) pour exprimer \(f(\theta_d) = u(\theta_d)\).

On écrit donc, en traitant \(\sin(\theta_i)\) comme uns constante :
\begin{flalign*}
u(\theta_x) = \frac{\sin(\theta_d) - \sin(\theta_i)}{\lambda}\\
\dd u(\theta_x) = \frac{\cos(\theta_d)}{\lambda} \dd \theta_d
\end{flalign*}
Comme la variation de u et \(\theta_d\) sont petites, on peut raisonnablement écrire, avec \(\Delta \theta_d = \Delta \theta_a\) \[\Delta u = \frac{\cos(\theta_d)}{\lambda}\Delta \theta_a\]

En combinant les 2 expressions de \(\Delta u\), on obtient \(\Delta \theta_a = \frac{\lambda}{pN\cos(\theta_d)}\)
\subsection{Écart entre les pics de diffraction}
La relation fondamentale des réseaux permet de voir comment, pour un ordre de diffraction m
donné, l’angle \(\theta_d\) donnant un pic de diffraction est fonction de la longueur d’onde \(\lambda\). En prenant
la différentielle de cette relation, exprimer l’écart \(\Delta \theta\) entre les pics de diffraction à l’ordre m pour
deux longueurs d’ondes \(\lambda\) et \(\lambda+\Delta \lambda\) proches.

On va différencier de la m\^eme façon pour obtenir la relation :
\begin{flalign*}
p(\sin\theta_d-\sin\theta_i) &= m\lambda\\
\dd (p(\sin\theta_d-\sin\theta_i)) &= \dd (m\lambda)\\
p \cos(\theta_d)\dd \theta_d &= m\dd \lambda\\
p \cos(\theta_d)\Delta \theta &= m\Delta \lambda\\
\Delta \theta &= \frac{m\Delta \lambda}{p\cos(\theta_d)}
\end{flalign*}
\subsection{Séparation de longueurs d'onde par le réseau}
 Deux longueurs d’ondes sont séparables avec le réseau si les pics de diffraction correspondant sont séparés par un écart angulaire supérieur à la demi largeur de chaque pic. Exprimer
cette condition en fonction de $\Delta \theta_a$ et $\Delta \theta$ définis précédemment.

L'écart minimal est défini par \(\Delta \theta_a = \Delta \theta\), soit en remplaçant par les expressions : \[\Delta \lambda = \frac{\lambda}{mN}\]

\criticalInfo{Commentaires sur l'expression trouvée}{
\begin{itemize}
 \item Plus le réseau a de fentes illuminées (N grand), plus les pics sont fins, et donc plus le réseau est capable de séparer des longueurs d'onde proches.
 \item Plus l'ordre de diffraction (m) est élevé, plus les pics sont angulairement espacés, et plus le pouvoir de résolution est élevé. C'est pourquoi les réseaux sont souvent utilisés à des ordres m élevés pour la spectroscopie de haute précision.
  \item Remarquablement, le pouvoir de résolution du réseau ne dépend pas directement de la longueur d'onde λ, mais uniquement des caractéristiques du dispositif (N) et de l'ordre utilisé (m).
\end{itemize}}
\end{document}
