% !TeX spellcheck = en_US
\documentclass[french]{yLectureNote}

\title{Méthode}
\subtitle{Méthode}
\author{Paulhenry Saux}
\date{\today}
\yLanguage{Français}
%Au programme de l'examen
\professor{J.Daudé}%Jérémi Daudé

\usepackage{graphicx}%----pour mettre des images
\usepackage[utf8]{inputenc}%---encodage
\usepackage{geometry}%---pour modifier les tailles et mettre a4paper
%\usepackage{awesomebox}%---pour les boites d'exercices, de pbq et de croquis ---d\'esactiv\'e pour les TP de PC
\usepackage{tikz}%---pour deiffner + d\'ependance de chemfig
\usepackage{tkz-tab}
\usepackage{awesomebox}%---Pour les boites info, danger et autres
\usepackage{menukeys}%---Pour deiffner les touches de Calculatrice
\usepackage{fancyhdr}%---pour les en-t\^ete personnalis\'ees
\usepackage{blindtext}%---pour les liens
\usepackage{hyperref}%---pour les liens (\`a mettre en dernier)
\usepackage{caption}%---pour la francisation de la l\'egende table vers Tableau
\usepackage{pifont}
\usepackage{array}%---pour les tableaux
\usepackage{yFlatTable}
\usepackage{multicol}

\newcommand{\Lim}[1]{\lim\limits_{\substack{#1}}\:}
\renewcommand{\vec}{\overrightarrow}
\newcommand{\N}[0]{\mathbb{N}}
\newcommand{\R}[0]{\mathbb{R}}
\newcommand{\C}[0]{\mathbb{C}}
\newcommand{\dd}[0]{\mathrm{d}}
\begin{document}

%\titleOne
\setcounter{chapter}{3}
	\chapter{Espaces vectoriels}
\section{Espace vectoriel}
\checkInfo{Montrer qu'un ensemble est un EV}{
\begin{itemize}
 \item On revient à la définition avec les 8 axiomes (non)
 \item On démontre que F est un SEV d'un EV déjà connu (oui)
\end{itemize}
Pour ce faire, on détermine une famille génératrice des F, ou on montre que que l'ensemble vérifie les propriétés de linéarités (\(\lambda u_1 + u_2 \in F\)).
}
\checkInfo{Montrer qu'une famille est libre}{
On revient à la définition. On doit donc montrer que

\(\forall (\lambda_1,\dots,\lambda_N),  \lambda_1v_1+\dots+\lambda_Nv_N = 0_E \Rightarrow \lambda_1=\lambda_N = 0\)
}
\checkInfo{Trouver la dimension d'un EV}{Il faut trouver une base E. Le cardinal de cette famille est la dimension de E.}
\checkInfo{Montrer que 2 SEV sont égaux}{
\begin{itemize}
 \item On montre que \(F\subset G \wedge G\subset F\)
 \item On montre que \(\dim(F) = \dim(G)\) et que \(F\subset G\) ou \(G\subset F\)
\end{itemize}
}
\checkInfo{Montrer que 2 SEV sont supplémentaires}{On peut
\begin{itemize}
 \item Revenir à la définition, i.e montrer que \(E=F+G\) et que \(F\cap G = \{0_E\}\)
 \item Montrer que tout élément de E se décompose de manière unique comme une somme d'un vecteur de F et d'un vecteur G
 \item Si E est de dimension finie, montrer que (\(F\cap G = \{0_E\}\) ou \(E=F+G\)) et \(\dim(E) = \dim(F)+\dim(G)\)
\end{itemize}
}
\section{Famille}
\checkInfo{Montrer qu'une famille est une base}{
En dimension quelconque, on montre que la famille est libre et génératrice dans E.

En dimension finie, on montre que \(Card(F)=\dim(E)\), puis que F est libre}
\end{document}
