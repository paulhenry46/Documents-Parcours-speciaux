% !TeX spellcheck = en_US
\documentclass[french]{yLectureNote}

\title{Mathématiques}
\subtitle{MPS2}
\author{Paulhenry Saux}
\date{\today}
\yLanguage{Français}
%Au programme de l'examen
\professor{J.Daudé}%Jérémi Daudé

\usepackage{graphicx}%----pour mettre des images
\usepackage[utf8]{inputenc}%---encodage
\usepackage{geometry}%---pour modifier les tailles et mettre a4paper
%\usepackage{awesomebox}%---pour les boites d'exercices, de pbq et de croquis ---d\'esactiv\'e pour les TP de PC
\usepackage{tikz}%---pour deiffner + d\'ependance de chemfig
\usepackage{tkz-tab}
\usepackage{awesomebox}%---Pour les boites info, danger et autres
\usepackage{menukeys}%---Pour deiffner les touches de Calculatrice
\usepackage{fancyhdr}%---pour les en-t\^ete personnalis\'ees
\usepackage{blindtext}%---pour les liens
\usepackage{hyperref}%---pour les liens (\`a mettre en dernier)
\usepackage{caption}%---pour la francisation de la l\'egende table vers Tableau
\usepackage{pifont}
\usepackage{array}%---pour les tableaux
\usepackage{yFlatTable}
\usepackage{multicol}

\newcommand{\Lim}[1]{\lim\limits_{\substack{#1}}\:}
\renewcommand{\vec}{\overrightarrow}
\newcommand{\N}[0]{\mathbb{N}}
\newcommand{\R}[0]{\mathbb{R}}
\newcommand{\C}[0]{\mathbb{C}}
\newcommand{\dd}[0]{\mathrm{d}}
\begin{document}

%\titleOne
\setcounter{chapter}{3}
	\chapter{EV de dimension finie}
\section{Familles de vecteurs}
\begin{definition}[Familles de vecteurs]
Soit I un ensemble. On appelle famille de vecteurs de E, une collection de vecteurs de E indexée sur I.
\[F = \{v_i,i\in I\}\] avec \(\forall i\in I, v_i \in E\). C'est comme un array.
\end{definition}
% Exemple : \(\{v_1=(1.0.1), v_2=(2.1.0),v_3=(1.01)\}\) est une famille finie de vecteurs de \(\R^3\) avec \(I=\{1,2,3,4,5\}\)
%
% Exemple \(\{X^k, k\in \N\}\) est une famille infinie dénombrable de vecteurs de \(\R[X]\).
%
% Exemple : \(\{\alpha,\alpha^3\}\) est infinie et indénombrable de vecteur \(\R^2\).

% \begin{definition}
% On note Card(F) = Card(I) le cardinal de F.
% \end{definition}
\begin{definition}[Combinaison linéaire de vecteur d'une famille]
Soit F une famille de vecteurs de E, indexée sur I. On appelle combinaison linéaire de vecteurs de F tout vecteur de E s'écrivant \(\lambda_1\cdot v_{i1}+\lambda_2\cdot v_{i2}+\dots+\lambda_n\cdot v_{in}\) avec \(N\leq Card(I)\in \N, i_k\in I\).

Si \(N=0\) , la combinaison linéaire vaut le vecteur nul.
\end{definition}
On note \(Vect(F)\) l'ensemble du résultat des combinaisons linéaires de F (qui vérifient \(Vect(F)\subset E\)).
\begin{proposition}
Soit F une famille de vecteurs de E. Alors un Vect(F) est le plus petit sev de E contenant tous les vecteurs de F
\end{proposition}
% \begin{myproof}
% Le vecteur nul est dedans.
%
% \end{myproof}
% \begin{proposition}
% Vect(F) est le plus petit sev de E contenant tous les vecteurs de F
% \end{proposition}

\begin{definition}[Famille génératrice]
Soit F une famille de vecteurs de E et G un sev de E.

On dit que F est une famille génétratrice de G \(\iff\) Vect(F) = G.
\end{definition}
% Exemple : \(F=\{(1.0.0),(0.1.0),(0.1.1)\}\) est une famille génératrice de R3. Tous les vecteurs peuvent sé'écrire comme une combinaison linéaire des 3 vecteurs.

\criticalInfo{Enlever des vecteurs dans une famille génératrice}{Si on en enlève, il est possible de casser le caractère générateur de la famille}

\begin{definition}[Famille libre/liée]
Soit F une famille de vecteurs de E. On dit que la famille F est libere \(\iff \forall N\in\N, \forall(i_1\dotsi_N)\in I^{\N}, i_p\neq i_q \Leftarrow p\neq q, \lambda_1v_{i1}+\dots + \lambda_N v_{iN} =0_E \iff \lambda_N = 0\). On dit alors que les vecteurs de F sont linéairement indépendants.

On dit que F est liée si elle n'est pas libre. On dit que les vecteurs de F sont linéairement dépendants.

Par convention, la famille vide est libre.
\end{definition}
\begin{proposition}
Soit F \(\{v_1, i\in I\}\) une famille finie. F est libre \(\iff \sum_{i\in I}\lambda_i v_i = 0_E \iff \lambda_i = 0 \forall i\in I\)
\end{proposition}

\begin{proposition}
Soit F une famille de vecteur de E.

Si \(\exists i\in I, tq v_i = 0_E\), F est liée

Si \(\exists i,j\in I, i\neq j, v_i=v_j\) F est liée
\end{proposition}
\begin{proposition}
Soit F une famille.

F est liée \(\iff \exists k\) tq \(v_k \in Vect(F \setminus v_k)\).
\end{proposition}
\begin{proposition}
Si F est liée, \(\exists (i_1,\dots, i_N)\in I^N\) et \(\lambda_1,\dots,\lambda_N\), \(\lambda_i\) non tous nul

\end{proposition}
\begin{proposition}
Soit 2 familles de vecteurs de E

\begin{itemize}
 \item F est génératrice de Vect(F)
 \item Si F est génératrice, et \(F\subset G\), alors G est aussi génératrice.
 \item Si G est libre, et \(G\subset F\), alors F est libre.
 \item On suppose F génératrice. Pour \(k\in I\), on note \(H =\{v_i, i\in I \setminus k\}\),  H est génératrice  \(\iff v_k\in\) Vect(H).
 \item On suppose F libre. Soit \(v \in E\) et \(s\notin I\), alors \(v_s=v\) et \(H = F\cup \{s\}\). I est libre \(\iff\) \(v_s \notin Vect(F)\). (preuve en exo)
 \item Soit \(p\in \N\). et F une famille de vecteurs à p éléments. Si G est une famille d'au moins p+1 éléments appartenant à Vect(F), alors G est liée
\end{itemize}
\end{proposition}
\begin{definition}[Base]
Soit F une famille de vecteurs de et G un sev de E.

F est une base de \(G \iff Vect(F) = G\) et F est libre.
\end{definition}
\section{Espace vectoriel de dimension finie}
\begin{definition}[Espace vectoriel de dimension finie]
Soit E un R-ev. E est de dimension finie si il admet une famille génératrice finie. (Il existe un nombre fini de vecteur tel que je peux créer tous les autres à partir de ceux-ci). Dans le cas contraire, E est de dimension infinie.
\end{definition}
\begin{proposition}
Soit F une famille finie de vecteurs. Il existe une famille B finie de vecteurs de F telle que B est une base de Vect(F). (Si une famille engendre qqc, on peut enlever des vecteurs de cette famille pour obtenir une famille libre génératrice, une base)
\end{proposition}
\begin{proposition}
Soit E un ev de dim finie. Il admet une base finie
\end{proposition}
\begin{theorem}[Coordonnées]
 Soit E un espace vectoriel de dim finie et F une famille finie. F est une base de E \(\iff \forall u\in E, \exists! (\lambda_1,\dots,\lambda_d)\in \R^d tq u = \lambda_1e_1+\dots+\lambda_de_d)\). Les réels \(\lambda\) sont appelés coordonnées du vecteur u dans la base F.
\end{theorem}
\begin{theorem}[Dimension]
 Soit E un espace vectoriel. Si E est de dimension finie, alors toutes les bases de E ont le m\^eme nombre d'éléments. Ce nombre est un entier appelé dimension de l'espace noté $\dim(E)$.
\end{theorem}
\begin{proposition}
Soit E un ev de dimension finie. Soit F une famille de vecteurs de E.

Si F est libre\( \Rightarrow Card(F)\leq \dim(E)\)

\(Card(F)>\dim(E) \Rightarrow F\) est liée

\(F\) est générateur \(\Rightarrow Card(F) \geq \dim(E)\).
\end{proposition}
\begin{theorem}[]
 Soit E un ev de dim finie et B une famille de vecteurs de E. Alors
 \begin{itemize}
  \item \(B\) est une base
  \item \(\iff B\) est une famille libre avec \(Card(B) = \dim(E)\)
  \item \(\iff B\) est une famille génératrice avec \(Card(B) = \dim(E)\)
 \end{itemize}
\end{theorem}
\warningInfo{En pratique}{On n'utilise que la deuxième équivalence}

\warningInfo{Dimension}{On peut utiliser l'argument de la dimension si uniquement on l'a démontré par ailleur.}

\criticalInfo{Coordonnées}{Il ne faut pas confondre le vecteur et ses coordonnées}

\begin{proposition}
Soit E un ev et F un sev de E. Alors \(\dim(F)\leq \dim(E)\). En particulier, si E est de dimension finie, F est de dimension finie.
\end{proposition}
\begin{proposition}
E est un ev de dimension finie, F sev de E. Alors F=E \(\iff \dim(F) = \dim(E)\).
\end{proposition}
Exercice : Soit E ev, \(\dim(E)<+\infty\). Mq $\forall d\in \{0,\dots, \dim(E)\},\exists F$, sev de E tq $\dim(F)=d$. (par l'absurde).

\begin{proposition}
E ev de dimension finie. F et G sev de E en somme directe. Alors \(\dim(F+G) = \dim(F)+\dim(G)\).
\end{proposition}
\begin{proposition}[Existence du supplémentaire]
E ev de dimension finie, et F sev de E. Alors \(\exists G\) sev de E tq \(G\) est un supplémentaire de F dans E.
\end{proposition}
\begin{proposition}[Formule de Grassmann]
E est un ev de dimension finie. Fet G 2 sev de E.

\(\dim(F)+\dim(G) = \dim(F+G) + \dim(F\cap G)\)
\end{proposition}
\begin{proposition}
Soit E un \(\R\) ev de dimension finie et F et G 2 sev de E. Alors les propriétés suivantes sont équivalentes :
\begin{itemize}
 \item F et G sont supplémentaires dans E
 \item \(F\cap G = \{0_E\}\) et \(\dim(F)+\dim(G) = \dim(E)\)
 \item \(G+F = E\) et \(\dim(F)+\dim(G) = \dim(E)\).
\end{itemize}
\end{proposition}
\begin{proposition}
Soit E un ev de dimension finie. Soit L une famille libre de vecteurs de E et G une famille génératrice de vecteurs de E

\(\exists B\) base de E tq \(L\subset B\subset L\cup G\)
\end{proposition}
\begin{theorem}[Théorème de la base incomplète]
 E ev de dimension finie
 \begin{itemize}
  \item Soit L famille libre de vecteurs de E. Alors \(\exists B\) base de \(E\) tq \(L\subset B\).
  \item Soit G une famille génératrice de E. Alors \(\exists B\) une base tell que \(B\subset G\).
 \end{itemize}
\end{theorem}
\end{document}
