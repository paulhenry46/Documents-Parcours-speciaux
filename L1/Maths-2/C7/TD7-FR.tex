% !TeX spellcheck = en_US
\documentclass[french]{yLectureNote}

\title{Mathématiques}
\subtitle{MPS2}
\author{Paulhenry Saux}
\date{\today}
\yLanguage{Français}

\professor{J.Dardé}%Jérémi Daudé

\usepackage{graphicx}%----pour mettre des images
\usepackage[utf8]{inputenc}%---encodage
\usepackage{geometry}%---pour modifier les tailles et mettre a4paper
%\usepackage{awesomebox}%---pour les boites d'exercices, de pbq et de croquis ---d\'esactiv\'e pour les TP de PC
\usepackage{tikz}%---pour deiffner + d\'ependance de chemfig
\usepackage{tkz-tab}
\usepackage{awesomebox}%---Pour les boites info, danger et autres
\usepackage{menukeys}%---Pour deiffner les touches de Calculatrice
\usepackage{fancyhdr}%---pour les en-t\^ete personnalis\'ees
\usepackage{blindtext}%---pour les liens
\usepackage{hyperref}%---pour les liens (\`a mettre en dernier)
\usepackage{caption}%---pour la francisation de la l\'egende table vers Tableau
\usepackage{pifont}
\usepackage{array}%---pour les tableaux
\usepackage{yFlatTable}
\usepackage{multicol}
\usetikzlibrary{matrix,arrows,decorations.pathmorphing}
\usepackage{verbatim}

\newcommand{\Lim}[1]{\lim\limits_{\substack{#1}}\:}
\renewcommand{\vec}{\overrightarrow}
\newcommand{\N}[0]{\mathbb{N}}
\newcommand{\R}[0]{\mathbb{R}}
\newcommand{\C}[0]{\mathbb{C}}
\newcommand{\dd}[0]{\mathrm{d}}
\newcommand{\tq}[0]{\text{ tel que }}
\newcommand{\mc}{\mathcal}
\begin{document}

%\titleOne
\setcounter{chapter}{7}
\chapter{Déterminant}
\section{Application p-linéaires}
\begin{definition}[Application p-linéaire]
Soient E et F deux R-EV et \(L:E^p\to F\). On dit que L est p-liénaire si elle est linéaire en chacune de ses variable, i.e. \(\forall (u_1,\dots, u_{p-1}\in E^{p-1}), \forall I\in \{1,\dots, p\}, L_I : v\in E \to L(u_1,\dots, u_{I-1},v,\dots u_I,\dots, u_{p-1})\).
\end{definition}
Exemple : Toute application linéaire de E dans F est 1-linéaire

Dans \(\R^3\), el produit scalaire : \(L : (x_1,x_2,x_3),(y_1,y_2,y_3)\in \R^3\times R^3 \to x_1y_1+x_2y_2+x_3y_3\) est 2 linéaire.

En effet, fixons \((y_1,y_2,y_3)\in \R^3\). Soit \((x_1,x_2,x_3)\in \R^3\), \((x_4,x_5,x_6)\in \R^3\), \(\lambda\in \R\)

\(L(\lambda(x_1,x_2,x_3) + (x_4,x_5,x_6), (y_1,y_2,y_3)) = L((\lambda x_1+x_4, \lambda x_2+x_5, \lambda x_3+x_6), (y_1,y_2,y_3)) = \lambda (x_1y_1+x_2y_2+x_3y_3)+(x_4y_1+x_5y_2+x_6y_3) = \lambda L() + L()\)

Il faut montrer la linéairité avec l'autre variable pour que la preuve soit complète.
\begin{proposition}[Application p-linéaire d'un vecteur nul]
Soit une application p-linéaire

si on met en position i le vecteur nul, le résultat est le vecteur nul de l'espace d'arrivé.
\end{proposition}
\begin{definition}[Forme p-linéaire]
L est une forme p-linéaire si l'espace d'arrivé est l'espace des réels.
\end{definition}
\begin{definition}[Forme p-linéaire alternée (FPA)]
Soit L une forme p-linéaire. L est alternée si \(\forall (u_1,\dots, u_p)\in E^p\), si \(\exists (i\neq j)\tq u_i = u_j\), alors \(L(u_1,\dots, u_p) = 0\)
\end{definition}
\begin{proposition}[Caractérisation d'une forme p-linéaire alternée]
Soit \(L : E^p \to \R\) une forme p-linéaire alternée. Soit \(\forall (u_1,\dots, u_p)\in E^p\, \forall (i,j)\in \{1,\dots, p\}\). Alors \(L(u_1,\dots, u_i,\dots, u_j,\dots, u_p) = - L(u_1,\dots, u_j,\dots, u_i,\dots, u_p)\)
\end{proposition}
\begin{proposition}[Famille liée dans une FPA]
Soit E un ev et L une forme p-linéaire alternée sur E. \(\forall (u_1,\dots, u_p)\) famille liée de E, alors \(L(u_1,\dots, u_p) = 0\)
\end{proposition}
\section{Déterminant d'une famille de vecteurs}
\subsection{En dimension 2}
Soit E un REV de dimension 2, \(B = \{e_1,e_2\}\).
\begin{proposition}[Forme linéaire en dimension 2]
Soit L une forme 2-linéaire alternée. Alors \(\forall u = a_1+be_2,\forall v = \alpha e_1+\beta e_2, L(u,v) = (a\beta-\alpha b)L(e_1,e_2)\). Ainsi, la connaissance de \(L(e_1,e_2)\) équivaut à connaitre L
\end{proposition}

\begin{definition}[Déterminant en dimension 2]

Posons \(L_B : (u,v) \to a\beta -\alpha b\). C'est l'unique forme 2-linéaire alternée vériffiant \(L(e_1,e_2) = 1\). On l'appelle le Déterminant en base B, noté \(\det_B\).
\end{definition}
\begin{proposition}[Forme linéaire (d2) et déterminant]
Soit L une forme 2-linéaire alternée, alors \(L(u,v) = \det_b(u,v)L(e_1,e_2)\)
\end{proposition}
\begin{proposition}[Déterminant et base]
Soit (u,v) une famille de E. (u,v) est une base de E \(\iff\) son Déterminant dans la base B est \(\neq 0\)
\end{proposition}
\begin{proposition}[Multiplication de det de 2 bases]
Soit B et C 2 bases de E. \(det_B(b_1,b_2)det_C(e_1,e_2) = 1\)
\end{proposition}
\subsection{En dimension 3}
E un ev de dimension 3, B = \(b_1,b_2,b_3\) une base de E.

\begin{proposition}
Soit L une forme 3-linéaire alternée et \(u_2,u_2,u_3\in E \tq \forall i \in \{1.2.3\}, u_i = \sum^3_{k = 1}\alpha_{ki}b_k\). On a

\(L(u_1,u_2,u_3) = (\alpha_{11}\alpha_{22}\alpha_{33}+\alpha_{21}\alpha_{32}\alpha_{13}+ \alpha_{23}\alpha_{31}\alpha_{12}-\alpha_{12}\alpha_{21}\alpha_{33}-\alpha_{13}\alpha_{22}\alpha_{31}-\alpha_{11}\alpha_{23}\alpha_{32})\)
\end{proposition}
\begin{definition}
On note \(L(u_1,u_2,u_3) = (\alpha_{11}\alpha_{22}\alpha_{33}+\alpha_{21}\alpha_{32}\alpha_{13}+ \alpha_{23}\alpha_{31}\alpha_{12}-\alpha_{12}\alpha_{21}\alpha_{33}-\alpha_{13}\alpha_{22}\alpha_{31}-\alpha_{11}\alpha_{23}\alpha_{32})\) le déterminant est base B.
\end{definition}
\begin{proposition}
\(\forall L\) formes 3-linéaires alternées, \(L(u,v,w) = L(b_1,b_2,b_3)det_B(u,v,w)\) et \(det_B\) est l'unique fome 3-linéaire alternée vérifiant \(det_B(b_1,b_2,b_3) = 1\)
\end{proposition}
\begin{proposition}
Soit \((u,v,w)\) une famille de E. C'est une base de E \(\iff det_B(u,v,w) \neq 0\).

Soit \(C = (c_1,c_2,c_3)\) une base de E. \(1 = det_B(c_1,c_2,c_3)\det_C(b_1,b_2,b_3)\)
\end{proposition}
\subsection{Dimension quelconque}
E est un ev de dimension p, B une base de E
\begin{theorem}[admis]
 Il existe une unique forme p-linéaire alternée, notée \(det_B\) est appelée déterminant en Base B, vérifiant :

 \(det_B(b_1,\dots, b_p) = 1\)

 \(\forall L\) forme p-linéaire alternée, \(\forall\) famille de vecteurs de E, \(L(u_1,\dots, u_p) = det_B(u_1,\dots, u_p)L(b_1,\dots, b_p)\).
\end{theorem}
\begin{proposition}
Soit une famille \((u_1,u_2,\dots,u_p)\) de vecteurs de E. C'est une base \(\iff det_B(u_1,u_2,\dots,u_p)\neq 0\)
\end{proposition}
\begin{proposition}
Soit C une base de E. \(det_B(c_1,\dots,c_p)\det_C(b_1,\dots, b_p) = 1\).
\end{proposition}
\begin{proposition}
Soit une famille de p-1 vecteurs de E.

\(u_k = \sum_{m=1}^p \alpha_{mk}b_m\). Posons \(v_k = u_k - \alpha_{ik}b_i\). Si on note \(C_I = (b_1,\dots,\b{i-1},b_{i+1},b_p)\). On a \(v_k \in Vect(C_i\).

Alors
\(det_B(u_1,\dots, u_{j-1}, b_i,u_j,\dots, u_{p-1}) = (-1)^{i+j}det_C (v_1,\dots, v_{p-1})\)
\end{proposition}
\begin{proposition}
Soit \((u_1,\dots, u_p)\in E^p, u_k = \sum^p_{l=1}\alpha_{lk}b_l\). Soient \((I,J)\in \{1,\dots, p\}\). On pose \(B_i = (b_1,\dots,b_{I-1},b_{I+1}n\dots, b_l)\) et \(u_k^i = u_k-\alpha_{ik}b_I\in vect(B_i)\). Alors \(det_B(u_1,\dots, u_p) = \sum_{i=1}^p (-1)^{i+j}\alpha_{ij}det_{B_i}(u_1^I,\dots, u_{J-1}^I, u_{J+1}^I,\dots, u_p^I)\)
\end{proposition}
\begin{proposition}
Soit \((u_1,\dots, u_p)\in E^p, u_k = \sum^p_{l=1}\alpha_{lk}b_l\). Soient \((I,J)\in \{1,\dots, p\}\). On pose \(B_i = (b_1,\dots,b_{I-1},b_{I+1}n\dots, b_l)\) et \(u_k^i = u_k-\alpha_{ik}b_I\in vect(B_i)\). Alors \(det_B(u_1,\dots, u_p) = \sum_{j=1}^p (-1)^{i+j}\alpha_{ij}det_{B_i}(u_1^I,\dots, u_{J-1}^I, u_{J+1}^I,\dots, u_p^I)\)
\end{proposition}
\begin{proposition}
Soit E et F 2 EV de m\^eme dimension finie = p.

Soit B une base de E et C une base de F. On pose \(\phi\in L(E,F)\) par \(\phi(b_i) = \mu_i\). Alors \(\forall (u_1,\dots, u_p)\in E^p, det_B(u_1,\dots, u_p) = det_C(\phi(u_1),\dots, \phi(u_p))\)
\end{proposition}
\section{Déterminant d'un endomorphisme}
\begin{theorem}[Déterminant d'endomorphisme dans plusieurs bases]
 Soit E un EV de dimension p, B et C 2 bases de E. Soit \(f\in l(E)\). Alors \(det_B(f(b_1),\dots, f(b_p)) = det_C(f(e_1)\dots, f(e_p))\)
\end{theorem}

\begin{definition}[Déterminant de l'endomorphisme]
Noté \(det(f)\), c'est le réel \(det_B(f(b_1),\dots, f(b_p))\) avec \(B = (b_1,\dots, b_p)\) une base quelconque
\end{definition}
\begin{proposition}[Déterminant de vecteurs par endomorphisme et déterminant d'endomorphisme]
\(\forall (u_1,\dots, u_p), det_B(f(u_1), \dots, f(u_p)) = det(f)det_b(u_1,\dots, u_p)\)
\end{proposition}
\begin{proposition}[Propriétés des det d'endomorphisme]
\begin{enumerate}
 \item Soient 2 endomorphismes. \(det(f\circ g) = det(f)\times det(g)\)
 \item \(det(I_d) = 1\)
 \item \(f\in L(E).\) f est bijectif \(\iff det(f)\neq 0\). Dans ce cas, \(det(f^{-1})=det(f)^{-1}\)
\end{enumerate}
\end{proposition}
\warningInfo{Rappel}{Soit \(A\in M_p(\R)\). On note \(f_A : X\in M_p(\R)\to AX\). \(f_A\in L(M_p(\R))\)}
\begin{definition}[déterminant d'une matrice d'application linéaire]
On appelle dtéerminant de A, noté \(det(A)\) le réel \(det(f_A)\)
\end{definition}
\begin{proposition}[Propriétés de déterminants d'applications linéaires]
\begin{enumerate}
 \item \(det(AB) = det(A)det(B)\)
 \item \(det(Id) = 1\)
 \item A est inversible \(det(A)\neq 0\). Alors, \(det(A^{-1}) = det(A)^{-1}\)
\end{enumerate}
\end{proposition}
Notons \(e_k\) la matrice colonne remplie de 0 sauf en k où il y a un 1 et \(\mc(E) = (E_1,\dots, E_p)\) la base canonique de \(M_{p1}(\R)\).
\begin{proposition}
Soit \(A\in M_p(\R), A=[C1,C_2,\dots, C_p]\) où \(C_k\) est la k-eme colonne de A. Alors \(det(A) = det_{\mc{E}}(C_1,\dots, C_p)\).
\end{proposition}
Exemple : On prend p réels \(\lambda_i\) et \(D\) la matrice avec les p réels sur la diognale et 0 sur les autres endroits. On a \(det(D) = \Pi_{i=1}^p \lambda_i\).

\begin{theorem}[Lien entre déterminant de fonction et de matrice de fonction]
 Soit E un ev de dimension p, B une base de E, \(f\in L(E)\) et \(A = Mat_B(f)\). Alors \(det(f) = det(Mat_B(f)) = det(A)\)
\end{theorem}
Pour IJ fixé, on note \(R_{ij} : A \in M_p \to M_{p-1}\). (On enlève la ligne I et la colonne J)
\begin{proposition}[Développement du déterminant de matrice par rapport à une ligne]
Soit une matrice carrée et on fixe I une ligne. Alors \(det(A) = \sum_{j=1}^p (-1)^{I+j}A_{ij}det(R_{ij}A)\)
\end{proposition}

\begin{proposition}[Développement du déterminant de matrice par rapport à une colonne]
Soit une matrice carrée et on fixe J une colonne. Alors \(det(A) = \sum_{i=1}^p (-1)^{I+j}A_{ij}det(R_{ij}A)\)
\end{proposition}
\begin{proposition}[Déterminant d'une matrice et de sa transposée]
\(det(A) = det(A^T)\).
\end{proposition}
\begin{proposition}[Sert à rien sauf pour la prochaine propriété]
Soit B une base de E et une famille de vecteurs de E. On pose f définie par \(f(b_i) = u_i,\). Alors \(det(f) = det(u_1,\dots, u_p)\)
\end{proposition}
\begin{proposition}[Corollaire ]
Soit B une base de E et une famille de vecteurs u de E. On pose \(u_j = \sum_{k=1}^p a_{kj}b_k\) et on définit \(A\in M_p(\R) : A_{ij} = a_{ij}\). On remarque que chaque colonne comporte les coordonnées de chaque vecteurs de la famille dans la base. Alors \(det(B(u_1,\dots, u_p)=det(A))\). En particulier, \((u_1,\dots, u_p)\) est une base ssi A est inversible.
\end{proposition}
On a ainsi une nouvelle manière de dire si la famille est une base.
\begin{definition}[Comatrice]
On appelle comatrice de A, matrice carré la matrice définie par \(a_{ij} = (-1)^{i+j}\cdot det(R_{ij(A)})\)
\end{definition}
\begin{proposition}[Formule théorique]
\(A\cdot Com(A)^T = det(A)I_p\), donc \(A^{-1} = \frac{Com(A)^T}{det(A)}\)
\end{proposition}
À ne pas utiliser en pratique

\section{Calcul pratique}
\begin{proposition}[Déterminant d'une matrice 22]
Le déterminant d'une matrice 22  est \(\begin{vmatrix}
a&b\\
c&d
                \end{vmatrix} = ad-bc\)
\end{proposition}
\begin{proposition}
\(\begin{vmatrix}
\lambda_1&0&0\\
0&\lambda_2&0\\
0&0&\lambda_3
                \end{vmatrix} = \lambda_1\times\dots\times \lambda_p\)
\end{proposition}
\begin{proposition}[Déterminant de matrices triangulaires]
\(\begin{vmatrix}
\lambda_1&*&*\\
0&\lambda_2&*\\
0&0&\lambda_3
                \end{vmatrix} =\begin{vmatrix}
\lambda_1&0&0\\
.&\lambda_2&0\\
.&.&\lambda_3
                \end{vmatrix} = \lambda_1\times\dots\times \lambda_p\)
\end{proposition}
\begin{proposition}[Propriétés calculatoires du détermiant de matrice]
On prend une matrice carrée A que l'on écrit sous la forme  de matrice de matrice colonne ou ligne.

\begin{itemize}
 \item \(\begin{vmatrix}
        C_1&\dots&\lambda C_i&C_p
       \end{vmatrix} = \lambda det(A)\)
\item \(\begin{vmatrix}
        C_1&\dots& C_j&\dots C_i &\dots &C_p
       \end{vmatrix} = -1 det(A)\)
\item \(\begin{vmatrix}
        C_1&\dots&\lambda C_i+\lambda C_j&C_p
       \end{vmatrix} = det(A)\) : pratique car permet de faire apparaitre des 0.
\item Cela vaut pour les lignes également
\end{itemize}

\end{proposition}

\end{document}
