% !TeX spellcheck = en_US
\documentclass[french]{yLectureNote}

\title{Méthode}
\subtitle{Méthode}
\author{Paulhenry Saux}
\date{\today}
\yLanguage{Français}
%Au programme de l'examen
\professor{J.Daudé}%Jérémi Daudé

\usepackage{graphicx}%----pour mettre des images
\usepackage[utf8]{inputenc}%---encodage
\usepackage{geometry}%---pour modifier les tailles et mettre a4paper
%\usepackage{awesomebox}%---pour les boites d'exercices, de pbq et de croquis ---d\'esactiv\'e pour les TP de PC
\usepackage{tikz}%---pour deiffner + d\'ependance de chemfig
\usepackage{tkz-tab}
\usepackage{awesomebox}%---Pour les boites info, danger et autres
\usepackage{menukeys}%---Pour deiffner les touches de Calculatrice
\usepackage{fancyhdr}%---pour les en-t\^ete personnalis\'ees
\usepackage{blindtext}%---pour les liens
\usepackage{hyperref}%---pour les liens (\`a mettre en dernier)
\usepackage{caption}%---pour la francisation de la l\'egende table vers Tableau
\usepackage{pifont}
\usepackage{array}%---pour les tableaux
\usepackage{yFlatTable}
\usepackage{multicol}

\newcommand{\Lim}[1]{\lim\limits_{\substack{#1}}\:}
\renewcommand{\vec}{\overrightarrow}
\newcommand{\N}[0]{\mathbb{N}}
\newcommand{\R}[0]{\mathbb{R}}
\newcommand{\C}[0]{\mathbb{C}}
\newcommand{\dd}[0]{\mathrm{d}}
\begin{document}

%\titleOne
\setcounter{chapter}{4}
	\chapter{Espaces vectoriels}
\checkInfo{Calculer ker et Im}{
Soient \(f\in L(E,F),B=(e_1,\dots, e_n)\) une base de E.
\begin{itemize}
 \item Pour déterminer le noyau, on peut :
 \begin{itemize}
  \item revenir à la définition, i.e. chercher le sev \(\{x\in E, f(x)=0_E\}\)
  \item En connaissant \(dim(Ker(f)) = p \neq 0\), on cherche une famille libre de p vecteurs tels que \(f(x_1)=\dots=f(x_p) = 0_F\) et conclure avec \(Ker(f) = Vect(x_1,\dots, x_p)\)
 \end{itemize}
\item Pour déterminer Im(f):
\begin{itemize}
 \item revenir à la définition et chercher le sev \(Im(f) = vect(f(e_1),\dots, f(e_n))\)
 \item Avec \(rg(f) = p\neq 0,\) chercher une famille libre de p vecteurs \(\{f(e'_1), \dots, f(e'_p)\}\) avec \(\{e'_1,\dots, e'_p\}\subset B\) puis conclure avec \(Im(f) = Vect(f(e'_1), \dots, f(e'_p))\)
\end{itemize}

\end{itemize}}
% Exemple : \(f(x_1,x_2,x_3,x_4) = (x_1+2x_2+3x_3-2x_4,x_2+x_3-x_4,x_1-x_2+x_4)\)
%
% On calcule le noyau : \((x_1,x_2,x_3,x_4)\in Ker(f)\iff f((x_1,x_2,x_3,x_4)) = (0.0.0)\)

\checkInfo{Montrer l'injectivité ou la surjectivité}{
Pour l'injectivité, on montre au choix que :
\begin{itemize}
 \item \(\forall x\in E, f(x) = 0_F \Rightarrow x =0_E\)
 \item \(dim(Ker(f)) = 0\)
 \item Si S est libre, alors \(f(S)\) est libre
\end{itemize}
Pour la surjectivité :
\begin{itemize}
 \item Montrer que \(vect(f(e_1,\dots,e_n)) = F\)
 \item Montrer que \(rg(f) = dim(F)\)
 \item Si S est génératrice de E, \(f(S)\) est génératrice de F
\end{itemize}}

\checkInfo{Montrer que f est un isomorphisme}{Les espaces de départ et d'arrivée sont de m\^eme dimension n.

On vérifie que f est une application linéaire, puis on montre qu'elle est bijective en prouvant que
\begin{itemize}
 \item f est injective
 \item \(dim(Ker(f)) = 0\)
 \item f est surjective
 \item \(rg(f) = n\)
\end{itemize}}

\checkInfo{Caractériser un projecteur}{Pour montrer que f est un projecteur, on peut
\begin{enumerate}
 \item montrer sa linéarité
 \item déterminer \(f^2\) et conclure que f est un projecteur si \(f^2=f\)
 \item Si f est un projecteur, déterminer \(Ker(f), Im(f)\) et conclure que \(f\) est la projection sur \(Im(f)\) parrallèlement à \(Ker(f)\)
\end{enumerate}
}
\end{document}
