% !TeX spellcheck = en_US
\documentclass[french]{yLectureNote}

\title{Mathématiques}
\subtitle{MPS2}
\author{Paulhenry Saux}
\date{\today}
\yLanguage{Français}

\professor{J.Daudé}%Jérémi Daudé

\usepackage{graphicx}%----pour mettre des images
\usepackage[utf8]{inputenc}%---encodage
\usepackage{geometry}%---pour modifier les tailles et mettre a4paper
%\usepackage{awesomebox}%---pour les boites d'exercices, de pbq et de croquis ---d\'esactiv\'e pour les TP de PC
\usepackage{tikz}%---pour deiffner + d\'ependance de chemfig
\usepackage{tkz-tab}
\usepackage{awesomebox}%---Pour les boites info, danger et autres
\usepackage{menukeys}%---Pour deiffner les touches de Calculatrice
\usepackage{fancyhdr}%---pour les en-t\^ete personnalis\'ees
\usepackage{blindtext}%---pour les liens
\usepackage{hyperref}%---pour les liens (\`a mettre en dernier)
\usepackage{caption}%---pour la francisation de la l\'egende table vers Tableau
\usepackage{pifont}
\usepackage{array}%---pour les tableaux
\usepackage{yFlatTable}
\usepackage{multicol}

\newcommand{\Lim}[1]{\lim\limits_{\substack{#1}}\:}
\renewcommand{\vec}{\overrightarrow}
\newcommand{\N}[0]{\mathbb{N}}
\newcommand{\R}[0]{\mathbb{R}}
\newcommand{\C}[0]{\mathbb{C}}
\newcommand{\dd}[0]{\mathrm{d}}
\newcommand{\tq}[0]{\text{ tel que }}
\begin{document}

%\titleOne
\setcounter{chapter}{4}
	\chapter{Applications linéaires}
\section{Dimension quelconque}
\subsection{Définition}
On prend $E,F$ 2 R-ev
\begin{definition}
\(f:E\to F\) est une application linéaire si
\begin{itemize}
 \item \(f(u+_Ev) = f(u)+_Ff(v)\)
 \item \(f\lambda \cdot_E u = \lambda \cdot_F f(u)\)
\end{itemize}
\end{definition}
Exemple : \(f:p\in R[X]\to (P(0),P(1),P(2))\in \R^3\) est une application linéaire. En effet, \(f(p+q) = (p+q)(0)+(p+q)(1)+(p+q)(2)= (p(0)+p(1)+p(2))+(q(0)+q(1)+q(2)) = f(p)+f(q)\)

Soit $P \in \R[X]$ et $\lambda \in \R$ : $f(\lambda p) = (\lambda (P(0),P(1),P(2)))$

Exemple : $g : p\in\R[X]\to (p(0),p(1),p(0)p(1))$

Prenons $P(X)=1,\lambda=2 : \lambda g(P) = (2.2.2), g(\lambda P) = (2.2.4)$.
\warningInfo{Remarque}{On peut au\(\iff\) vérifier les 2 propriétés en m\^eme temps. On montre alors que \(f(\lambda u + v)= \lambda f(u)+f(v)\)}
\begin{proposition}
Soit \(f:E\to F\) linéaire. Alors \(f(0_E) = 0_F\)
\end{proposition}
\checkInfo{Notation}{On note \(L(E,F)\) l'ensemble des applications linéaires de E dans F et \(L(E) = L(E,E)\).}
\begin{definition}[Formes linéaires]
On dit que f est une forme linéaire si \(f\in L(E,\R)\).

Si une application linéaire est bijective, on dit que \(f\) est isomorphisme d'espace vectoriel.

Si \(f\in L(E)\), c'est un endomorphisme.

Si la fonction vérifie les 2, c'est un automorphisme.
\end{definition}
\begin{definition}
Soit \(f\in L(E,F)\).

On appelle image de \(f\), noté \(Im(f)\) ou \(R(f)\) l'ensemble \(\{f(x),x\in E \}\subset F\).

On appelle noyau \(f\), noté \(Ker(f)\) l'ensemble des antécédants du vecteur nul \(\{x\in E, f(x)=0_F\}\subset E\).
\end{definition}
\begin{proposition}
\(f\in L(E,F)\). Alors \(Ker(f)\) est sev de E et \(Im(f)\) est un sev de F.
\end{proposition}
\begin{proposition}
Soit \(f\in L(E,F)\). Alors \(f\) est surjective \(\iff\) \(Im(f)=F\). Elle est injective \(\iff\) \(Ker(f) = \{0_E\}\).
\end{proposition}
\begin{proposition}
la composée de 2 applications linéaires est une application linéaire
\end{proposition}
\begin{proposition}
Soit f un isomorphisme de E dans F. Alors \(f^{-1} \in L(F,E)\)
\end{proposition}
\begin{proposition}
Soit E et F des ev. \(L(E,F)\) est un ev dont le vecteur nul est l'application dont l'image est le vecteur nul de F.
\end{proposition}
\subsection{Exemples}
\begin{itemize}
 \item \(Id_E : x\in E \to x\in E\)
 \item les homotéthies : \(f_{\mu} : x\in E \to \mu X\in E\) (automorphisme si \(\mu\neq 0\))
 \item Les projecteur / projections
\end{itemize}
\subsubsection{Projecteurs}
\begin{definition}[Projecteurs]
On dit que \(p\in L(E)\) est un projecteur \(\iff p^2 (= p\circ p) = p\)
\end{definition}
Exemple : $p(a,b,c)\in \R^3 \to (a,b,0)$

Exemple : $Q = \sum_{m=0}^N a_mX^m \to \sum_{m=0}^{\min(K,N)} a_mX^m$

\begin{proposition}
Soit p un projecteur de E. Alors
\begin{enumerate}
 \item \(\forall v \in Im(p), p(v) = v\)
 \item \(E = Im(p)\oplus Ker(p)\)
\end{enumerate}
\end{proposition}
\begin{definition}[Projection]
E un ev, F et G supplémentaires dans E : \(\forall u \in E, \exists! u_F,u_G tq u = u_F+u_G\). On appelle projection sur F parallèlement à G l'application \(p: u_F+u_G \to u_F\). Alors
\begin{enumerate}
 \item \(p\in L(E)\)
 \item \(p^2=p\)
 \item Im(p) = F, Ker(p) = G
\end{enumerate}
\end{definition}
\begin{proposition}[Corollaire]
Soit \(p\in L(E)\)

\(p\) est un projecteur \(\iff\) p est une projecion sur Im(p) parallèlement à Ker(p).
\end{proposition}
\begin{proposition}
\(p\in L(E)\) est un projecteur \(\iff Id_E-p\) est un projecteur.
\end{proposition}
\begin{proposition}[Corollaire]
Si p est une projection, \(Im(p) = Ker(Id_E-p), Ker(p) = Im(Id_E-p)\)
\end{proposition}
\subsubsection{Symétrie}
\begin{definition}
Soit \(s\in L(E)\). C'est une symétrie \(s^2 = Id_E\).
\end{definition}
\begin{proposition}[Lien entre symétrie et projecteur]
\begin{enumerate}
 \item si s est une symétrie, \(p = 0.5(s+Id_E)\) est un projecteur
 \item si p est un projecteur, alors \(s = 2p-Id_E\) est une symétrie
\end{enumerate}
\end{proposition}
\begin{proposition}[Corollaire]
Soit s une symétrie. ALors \(E = Ker(s+Id_E)\oplus Ker(s-Id_E)\).
\end{proposition}
\begin{proposition}
Soit E un ev, F et G 2 sev supplémentaires de E. On définit

\(s : u = u_F+u_G\in E \to u_F-u_G\in E\).

On appelle s symétrie par rapport à F parallèlement à G.

Alors \begin{enumerate}
       \item \(s\in L(E)\)
       \item \(s^2=Id_E \Leftarrow s est une symétrie\)
       \item \(Ker(s-Id_E) = F, Ker(s+Id_E) =G\)
      \end{enumerate}
\end{proposition}
\begin{proposition}[Corollaire]
Soit \(s\in L(E) tq s^2 = Id_E\). Alors s est une symétrie par rapport à \(Ker(s-Id_E)\) parallèlement à \(Ker(s+Id_E)\)
\end{proposition}
Exemple : $(a,b,c)\to (a,b,-c)$

Exemple : $\sum a_k X^k \to \sum (-1)^k a_k X_k$
\section{Dimension finie}
E est un R-EV de dimension finie et F de dimension quelconque

On note N la dimension de E

\begin{proposition}
Soit \(F\in L(E,F)\). Alors \(Im(f)\) est de dimension finie \(\leq N\). On appelle cette dimension rang de \(f\), noté \(rg(f)\).
\end{proposition}
\begin{definition}
Soit B une base de E. Soit \(p\in \{1,\dots,\N\}\). On note \(C_{pi}u = \lambda_1e_1+\dots+\lambda_Ne_N \in E \to \lambda_p\in \R\). Alors \(C_p\) est une forme liénaire, et pour tout \(u\in E, u = c_1(u)e_1+\dots+c_N(u)e_N\). \(C_p\) est appellée application p-ème coordonnée dans la base B.
\end{definition}
\begin{theorem}[]
 Soient B une base de dimension N quelconque de E et F une famille de N vecteur de G.

 Alors Il existe une unique application linéaire telle que \(\forall i \in \{1,\dots,\N\}, f(e_i) = v_i\)
\end{theorem}
\begin{proposition}[Corollaire]
Soit B une base de E et f une application linéaire. Alors f est complètement déterminée par la connaissance de \((f(e_1),\dots,f(e_N))\). En effet, \(f : u\in E \to \sum_{p=1}^N c_p(u)f(e_p)\).
\end{proposition}
\begin{proposition}
Soit f une application linéaire. Si f est un isomorphisme (bijective), alors F a la dimension de E.
\end{proposition}
\begin{theorem}[Théorème du rang]
 Soit \(f\in L(E,F)\). Alors \(\dim(Ker(f))+rg(f) = \dim(E)\)
\end{theorem}
\begin{proposition}
Si f est injective, alors \(\dim(E)\leq \dim(F)\).

Si f est surjective, alors \(\dim(F)\geq \dim(F)\)
\end{proposition}

\begin{proposition}[Corollaire]
E et F de dimension finie et égale. Soit \(f\in L(E,F)\). Alors f est bijective \(\iff\) f surjective, \(\iff\) f injective.
\end{proposition}
\begin{proposition}[Cororllaire]
Soit \(f\in L(E)\), alors \(f\) est injective \(\iff\) f est surjective ou surjective
\end{proposition}
Exercice : Soit E un EV de dimension finie. Mq toute symétrie de E est bijective et mq le seul projecteur bijectif de E est \(p=Id_E\)
\begin{proposition}
Soit f une fonction linéaire de E dans F
f est bijective \(\iff \) L'image d'une base de E par f est une base de F
\end{proposition}
\begin{proposition}
Supposons \(\dim(F)< +\infty\). Alors \(L(E,F)\) est de dimension finie et \(\dim(L(E,F)) = \dim(E)\times \dim(F)\)
\end{proposition}
\end{document}
