% !TeX spellcheck = en_US
\documentclass[french]{yLectureNote}

\title{Mathématiques}
\subtitle{MPS2}
\author{Paulhenry Saux}
\date{\today}
\yLanguage{Français}

\professor{J.Daudé}%Jérémi Daudé

\usepackage{graphicx}%----pour mettre des images
\usepackage[utf8]{inputenc}%---encodage
\usepackage{geometry}%---pour modifier les tailles et mettre a4paper
%\usepackage{awesomebox}%---pour les boites d'exercices, de pbq et de croquis ---d\'esactiv\'e pour les TP de PC
\usepackage{tikz}%---pour deiffner + d\'ependance de chemfig
\usepackage{tkz-tab}
\usepackage{chemfig}%---pour deiffner formules chimiques
\usepackage{chemformula}%---pour les formules chimiques en \'equation : \ch{...}
\usepackage{tabularx}%---pour dimensionner automatiquement les tableaux avec variable X
\usepackage{awesomebox}%---Pour les boites info, danger et autres
\usepackage{menukeys}%---Pour deiffner les touches de Calculatrice
\usepackage{fancyhdr}%---pour les en-t\^ete personnalis\'ees
\usepackage{blindtext}%---pour les liens
\usepackage{hyperref}%---pour les liens (\`a mettre en dernier)
\usepackage{caption}%---pour la francisation de la l\'egende table vers Tableau
\usepackage{pifont}
\usepackage{array}%---pour les tableaux
\usepackage{lipsum}
\usepackage{yFlatTable}
\usepackage{multicol}
\newcommand{\Lim}[1]{\lim\limits_{\substack{#1}}\:}
\renewcommand{\vec}{\overrightarrow}
\newcommand{\N}[0]{\mathbb{N}}
\newcommand{\R}[0]{\mathbb{R}}
\newcommand{\C}[0]{\mathbb{C}}
\newcommand{\dd}[0]{\mathrm{d}}
\begin{document}

%\titleOne

	\chapter{Développements limités}
\section{Défintion et premières propriétés}
\begin{definition}[Développement limité en 0]
Soit $I$ un intervalle ouvert de \(\R\) tel que \(0\in I\). Soit \(f:I\to \R\). On dit que f ademet un développement limité en 0 à l'ordre m si il existe un p\^olynome \(P\in\R_m[X]\) et une fonction \(\varepsilon : I\to \R\) telle que \(\varepsilon \to 0\) tel que \(\forall x\in I, f(x) = P(x) + x^m\varepsilon(x)\).
\end{definition}
On appelle le p\^olynome la partie régulière du développement et le \(o(\dots)\) est le reste. Les coefficients du polynome sont appelés les coefficient du développement limité.
\begin{definition}[En $x\neq 0$]
Soit \(f : I-\{x_0\} \to \R, t\to g(t+x_0)\) Alors \(g\) admet un DL à l'ordre $m$ en $x_0 \iff$ f admet un DL à l'ordfe m en 0.
\end{definition}
On peut donc faire un changement de variable $t+x_0 = x$.
\begin{lemma}
Si f admet un DL à l'ordre m en 0, alors il admet un DL à l'odre k en 0 \(\forall k \in \{0,\dots,m\}\) qui s'obtient en troquant le DL initial à partir du coeffcient k
\end{lemma}

\begin{theorem}[Unicité du DL]
 Les coeffcicients du DL d'une fonction sont unique.
\end{theorem}

\begin{lemma}
\(f\) admet un DL limité en 0 à l'ordre 0 \(\iff f\) est continue en 0. Dans ce cas, \(a_0 = f(0)\)
\end{lemma}
\begin{lemma}
\(f\) admet un DL à l'ordre 1 en 0 \(\iff f\) est dérivable en 0. Dans ce cas, \(a_1 = f'(0)\).
\end{lemma}
Exemples :\begin{itemize}
           \item  \(\sin(x) = 0 + 1x + o(x) = x+o(x)\)
           \item \(\cos(x) = 1+o(x)\)
           \item \(e^x = 1+x+o(x)\)
          \end{itemize}
          %\TODO
% Exercice : Soit $f :I\to \R$ admettant un DL à l'ordre $p$ en 0. Alors $f$ est paire $\Rightarrow a_{2k+1} =0$. et Alors $f$ est impaire $\Rightarrow a_{2k} =0$. Montrer ce résultat.
\begin{proposition}[DL et ``primitivation'']
Soit \(f : I\to \R\). On suppose qu'elle admet une primitive \(F\). Si \(f\) admet un DL en 0 à l'ordre \(p\) : \(f(x) = a_0+a_1x+\dots+a_px^p+o(x^p)\), alors \(F\) admet un DL à l'ordre \(p+1\) : \(F(x) = F(0) + a_0x+a_1\frac{x^2}{2}+a_2\frac{x^3}{3}+\dots+a_p\frac{x^{p+1}}{p+1}+o(x^{p+1})\)
\end{proposition}
\begin{proposition}
Soit \(f : I\to \R\), dérivable sur I, tel que \(\exists p\in\N, f'(x) = o(x^p)\). Alors \(f(x) = f(0) + o(x^{p+1}) = f(0) + (x^{p+1})\varepsilon_2(x)\)
\end{proposition}
\section{Notation de Landeau}
\begin{definition}
Soit \(I\) intervalle de \(\mathbb{R}\).

On dit que f est négligeable devant g en \(x_0\), noté \(f= o_{x\to x_0}(g)\) si il existe \(\eta >0\) et \(\varepsilon : ]x_0-\eta,x_0+\eta[\to \R\) telle que \(\varepsilon \to x\) et \(\forall x\in]x_0-\eta,x_0+\eta[, f(x) = g(x)\varepsilon\)

On dit que f est dominée par g en \(x_0\), noté \(f = O(g)_{x\to x_0}\)  si il existe \(\eta >0\) et \(\varepsilon : ]x_0-\eta,x_0+\eta[\to \R\) telle que \(|\varepsilon| \leq C\) et \(\forall x\in]x_0-\eta,x_0+\eta[, f(x) = g(x)\varepsilon\)

On dit que f est dominée par g en \(x_0\), noté \(f \sim_{x\to x_0} g\)  si il existe \(\eta >0\) et \(\varepsilon : ]x_0-\eta,x_0+\eta[\to \R\) telle que \(\varepsilon \to 0\) et \(\forall x\in]x_0-\eta,x_0+\eta[, f(x) = g(x)(1+\varepsilon)\)
\end{definition}

\criticalInfo{Sommes}{
$o(x^m) + o(x^m) = o(x^m)$ mais on a aussi $o(x^m)-o(x^m) = o(x^m) \neq 0$.}
\section{Fonctions p fois dérivables}
Dans cette partie, on note I un intervalle ouvert non vide de $\R$.
\begin{definition}
pour $p\in \N,$ la dérivée $p^{ene}$ de $f:I\to \R$, notée $f^{(p)}$ est définie récursivement : $f^{(0)} = f$ et si $f^{(p-1)}$ est dérivable sur I, alors $f^{(p)}  = (f^{(p-1)})'$.
\end{definition}
On dit que $f$ est p fois dérivable si $f^{(p)}$ est définie. Comme la dérivabilité implique la continuité, et $f^{(p)}$ déifinie, les dérivées précédentes sont continues sur tout l'intervalle.
\begin{definition}[Fonction p fois dérivables en un point]
On dit que f est p fois dérivabilité en \(x_0 \in I\) si \(\exists I_1\) intervalle ouvert tel que \(x_0\in\i_1\in I\) tel que f est p-1 fois dérivable sur \(I_1\) et \(f^{(p-1)}\) est dérivable en \(x_0\).
\end{definition}
\begin{definition}[Classe ]
On dit que f est de classe \(C^p\) sur I si f est p fois dérivable sur I et la dérivée pème est continue.

On dit que f est de classe \(C^{\infty}\) si \(f\in C^p \forall p\in\N\).
\end{definition}
\begin{proposition}
Soit \(f\) et \(g\) p fois dérivables sur I. Alors $f+g$ p fois dérivables sur $I$ et $(f+g)^{(p)} = f^{(p)} + g^{(p)}$.
\end{proposition}
Cela fonctionne aussi pour $f$ et $g$ p fois dérivables en un point et de classe $C^p$.

\begin{proposition}
\(\lambda f\) est p fois dérivables et \(\lambda f^{(p)} = (\lambda f)^{(p)}\).
\end{proposition}
\begin{proposition}
\(fg\) est p fois dérivable sur I et \((fg)^{(p)} = \sum^p_{m=0} C^m_p f^{(m)}g^{(p-m)}\)
\end{proposition}
\begin{proposition}
Soient \(I,J\) 2 intervalles ouverts non vides de \(\mathbb{R}\). \(f:I\to J\subset \mathbb{R}\) et \(g : J\to \mathbb{R}\). Si f est n fois dérivable sur I et g est n fois déruvable sur J, alors \(g\circ f : I\to \mathbb{R}\) est n fois dérivable sur I.
\end{proposition}

\begin{proposition}
Si \(g : I\to \mathbb{R}\) est n fois dérivable sur I et ne s'annule pas sur I, alors la fonction \(f: x\in I\to \frac{1}{g(x)}\) est n fois dérivable aussi.
\end{proposition}
% \begin{myproof}[En exercice]
% Par récurrence : Idée : $f' = \frac{-g'}{g^2}$.
% \end{myproof}
\begin{proposition}[]
Soit \(f:I\to\mathbb{R}\) $n\neq 0$ fois dérivable telle que f' ne s'annule pas sur I. Alors f est une bijection de I dans \(f(I) = J\in\mathbb{R}\). On a alors l'application réciproque \(f^{-1}\) est n fois dérivable. On a aussi \(f^{-1} : J\to I\)
\end{proposition}

\section{Formules de Taylor}
\begin{theorem}[Théorème de Taylor-Young]
 Si f est $(n\geq 1)$ fois dérivable en 0, alors \(f(x) = f(0)+xf'(0)+\frac{x^2}{2} f''(0)+\dots + \frac{x^n}{n!}f^{(n)}(0)+o(x^n)\)
\end{theorem}

\begin{theorem}[Théorème de taylor-Lagranges]
 Si f est \((n\geq 1)\) fois dérivable sur I, alors

 \(\forall x\in I, \exists c_x\)\( \in[\min(0,x), \max(0,x)]\) tel que \(f(x) = f(0) + xf'(0) + \frac{x^2}{2}f''(0)+\dots+\frac{x^{n-1}}{(n-1)!}f^{(n-1)}(0)\frac{x^n}{n!}f^{(n)}(c_x)\)
\end{theorem}
\warningInfo{Remarques}{Quand n=1, on obtient le TAF

Si $x\neq0,c_x\in$ l'intervalle ouvert.}
\begin{theorem}[Formule de Taylor-Laplace (avec reste intégrale)]
 Si f \(I\to \R\) tel que f tes de classe \(C^p(I)\), alors \(\forall x\in I\), \(f(x) = f(0)+xf'(0)+\dots+\frac{x^{p-1}}{(p-1)!}f^{(p-1)}(0) + \)
 \[\int^x_0f^{(p)}(s) \frac{(x-s)^{p-1}}{(p-1)!}\dd s\]
\end{theorem}
\subsection{DL de fonctions usuelles}
\begin{eqnarray*}
e^x&=&1+x+\frac{x^2}2+\dots+\frac{x^n}{n!}+o(x^n)\\
\cos x&=&1-\frac{x^2}{2!}+\dots+\frac{(-1)^n x^{2n}}{(2n)!}+o(x^{2n+1})\\
\sin x&=&x-\frac{x^3}{3!}+\dots+\frac{(-1)^n x^{2n+1}}{(2n+1)!}+o(x^{2n+2})\\
\cosh x&=&1+\frac{x^2}{2!}+\dots+\frac{x^{2n}}{(2n)!}+o(x^{2n+1})\\
\sinh x&=&x+\frac{x^3}{3!}+\dots+\frac{ x^{2n+1}}{(2n+1)!}+o(x^{2n+2})\\
\frac{1}{1-x}&=&1+x+x^2+\dots+x^n+o(x^n)\\
\frac{1}{1+x}&=&1-x+x^2+\dots+(-1)^nx^n+o(x^n)\\
\ln(1+x)&=&x-\frac{x^2}2+\dots+\frac{(-1)^{n+1}}{n}x^n+o(x^n)\\
(1+x)^\alpha&=&1+\alpha x+\frac{\alpha(\alpha-1)}2x^2+\dots+\frac{\alpha(\alpha-1)\cdots (\alpha-n+1)}{n!}x^n+o(x^n)\\
\tan(x) &=& x+\frac{x^3}{3}+o(x^4)\\
\end{eqnarray*}
\end{document}

