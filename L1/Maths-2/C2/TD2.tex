% !TeX spellcheck = en_US
\documentclass[french]{yLectureNote}

\title{Mathématiques}
\subtitle{MPS2}
\author{Paulhenry Saux}
\date{\today}
\yLanguage{Français}

\professor{J.Daudé}%Jérémi Daudé

\usepackage{graphicx}%----pour mettre des images
\usepackage[utf8]{inputenc}%---encodage
\usepackage{geometry}%---pour modifier les tailles et mettre a4paper
%\usepackage{awesomebox}%---pour les boites d'exercices, de pbq et de croquis ---d\'esactiv\'e pour les TP de PC
\usepackage{tikz}%---pour deiffner + d\'ependance de chemfig
\usepackage{tkz-tab}
\usepackage{chemfig}%---pour deiffner formules chimiques
\usepackage{chemformula}%---pour les formules chimiques en \'equation : \ch{...}
\usepackage{tabularx}%---pour dimensionner automatiquement les tableaux avec variable X
\usepackage{awesomebox}%---Pour les boites info, danger et autres
\usepackage{menukeys}%---Pour deiffner les touches de Calculatrice
\usepackage{fancyhdr}%---pour les en-t\^ete personnalis\'ees
\usepackage{blindtext}%---pour les liens
\usepackage{hyperref}%---pour les liens (\`a mettre en dernier)
\usepackage{caption}%---pour la francisation de la l\'egende table vers Tableau
\usepackage{pifont}
\usepackage{array}%---pour les tableaux
\usepackage{lipsum}
\usepackage{yFlatTable}
\usepackage{multicol}
\newcommand{\Lim}[1]{\lim\limits_{\substack{#1}}\:}
\renewcommand{\vec}{\overrightarrow}
\newcommand{\N}[0]{\mathbb{N}}
\newcommand{\R}[0]{\mathbb{R}}
\newcommand{\C}[0]{\mathbb{C}}
\newcommand{\dd}[0]{\mathrm{d}}
\begin{document}

%\titleOne

	\chapter{Développements limités}
\section{Défintion et premières propriétés}
\begin{definition}[Développement limité en 0]
Soit $I$ un intervalle ouvert de \(\R\) tel que \(0\in I\). Soit \(f:I\to \R\). On dit que f ademet un développement limité en 0 à l'ordre m si il existe un p\^olynome \(P\in\R_m[X]\) et une fonction \(\varepsilon : I\to \R\) telle que \(\varepsilon \to 0\) tel que \(\forall x\in I, f(x) = P(x) + x^m\varepsilon(x)\).
\end{definition}
On appelle le p\^olynome la partie régulière du développement et le \(o(\dots)\) est le reste. Les coefficients du polynome sont appelés les coefficient du développement limité.
\begin{definition}[En $x\neq 0$]
Soit \(f : I-\{x_0\} \to \R, t\to g(t+x_0)\) Alors \(g\) admet un DL à l'ordre $m$ en $x_0 \iff$ f admet un DL à l'ordfe m en 0.
\end{definition}
On peut donc faire un changement de variable $t+x_0 = x$.
\begin{lemma}
Si f admet un DL à l'ordre m en 0, alors il admet un DL à l'odre k en 0 \(\forall k \in \{0,\dots,m\}\) qui s'obtient en troquant le DL initial à partir du coeffcient k
\end{lemma}
\begin{myproof}
Si $k=m$, c'est bon. Si $k<m$. $f(x) = a_0+a_1x+\dots + a_kx^k + a_{k+1}x^{k+1} + \dots  + a_mx^m +o(x^m) = a_0+a_1x+\dots+a_kx^k + x^k(a_{k+1}x+\dots+a_mx^{m-k} + o(x^{m-k}))$
\end{myproof}
\begin{theorem}[Unicité du DL]
 Les coeffcicients du DL d'une fonction sont unique.
\end{theorem}
\begin{myproof}[Par récurrence sur $m$]
Initialisation : Soit $f$ telle que $f(x) = a_0+\varepsilon(x) = b_0+\eta(x)$ En $x=0$, on obtient $a_0=b_0$.

Hérédité : Supposons qu'il existe $m$ tel que $H_n$ vraie.

Soit $f(x) = a_0+a_1x+\dots + a_mx^m + o(x^m) = b_0+b_1x+\dots + b_mx^m + o(x^m)$. Par le m\^eme précédent $f(x) = a_0+a_1x_1+\dots +a_mx^m = b_0+b_1x_1+\dots b_mx^m o(x)$

Donc par $H_n$ c'est égal. Donc en particulier $a_{m+1}x^{m+1} = b_{m+1}x^{m+1}$.

Pour $x\neq 0$, on a $a_{m+1}+o(x) =b_{m+1}+o(x)$. En prenant la limite quand $x\to$ mais $x\neq 0$

Donc $H_{n+1}$.
\end{myproof}
\begin{lemma}
\(f\) admet un DL limité en 0 à l'ordre 0 \(\iff f\) est continue en 0. Dans ce cas, \(a_0 = f(0)\)
\end{lemma}
\begin{myproof}
On a $f(x) = a_0 + \varepsilon(x)$ avec $\varepsilon \to 0$ Donc en 0, $a_0 = a_0+\varepsilon = f(0)$ et $\forall x\in I, |f-x)-f(0)| = |f(x)-a_0| = |\varepsilon|\to 0$. Donc $f$ est continue en 0.

Soit $x\in I, f(x) = f(0) + f(x) - f(0) = f(0) + \varepsilon(x)$. Comme $f$ est continue en 0, $\varepsilon(x) = + f(x) - f(0) \to 0$. Par unicité du DL, $a_0 = f(0)$.
\end{myproof}
\begin{lemma}
\(f\) admet un DL à l'ordre 1 en 0 \(\iff f\) est dérivable en 0. Dans ce cas, \(a_1 = f'(0)\).
\end{lemma}
\begin{myproof}
On a $\forall x\in I,f(x) = a_0+a_1x+o(x) = f(0)+a_1x+o(x)$. Donc $\forall x\neq 0, \frac{f(x)-f(0)}{x-0} =  \frac{f(x)-f(0)}{x} =  \frac{a_1x+o(x)}{x} = a_1+o(1) \to a_1$. Donc $f$ est dérivable en 0 et $f'(0) = \Lim{x\to0} = a_1$.

$f(x) = f(0) + xf'(0) + (f(x)-f(0)-xf'(0))$. Posons $\varepsilon = \frac{f(x)-f(0)-xf'(0)'}{x}\forall x\neq 0$ et $0$ en $x=0$. On a bien $f(x) = f(0)+xf'(0)+o(x)$. Il faut montrer que $\varepsilon \to 0$. On sait par hypothèse que $f$ est dérivable en 0. On a donc $|\varepsilon| = |\frac{f(x)-f(0)}{x-0} - f'(0)| \to 0$
\end{myproof}
Exemples :\begin{itemize}
           \item  \(\sin(x) = 0 + 1x + o(x) = x+o(x)\)
           \item \(\cos(x) = 1+o(x)\)
           \item \(e^x = 1+x+o(x)\)
          \end{itemize}
          %\TODO
% Exercice : Soit $f :I\to \R$ admettant un DL à l'ordre $p$ en 0. Alors $f$ est paire $\Rightarrow a_{2k+1} =0$. et Alors $f$ est impaire $\Rightarrow a_{2k} =0$. Montrer ce résultat.
\begin{proposition}[DL et ``primitivation'']
Soit \(f : I\to \R\). On suppose qu'elle admet une primitive \(F\). Si \(f\) admet un DL en 0 à l'ordre \(p\) : \(f(x) = a_0+a_1x+\dots+a_px^p+o(x^p)\), alors \(F\) admet un DL à l'ordre \(p+1\) : \(F(x) = F(0) + a_0x+a_1\frac{x^2}{2}+a_2\frac{x^3}{3}+\dots+a_p\frac{x^{p+1}}{p+1}+o(x^{p+1})\)
\end{proposition}
\begin{proposition}
Soit \(f : I\to \R\), dérivable sur I, tel que \(\exists p\in\N, f'(x) = o(x^p)\). Alors \(f(x) = f(0) + o(x^{p+1}) = f(0) + (x^{p+1})\varepsilon_2(x)\)
\end{proposition}
\begin{myproof}
Par le TAF, on a $\forall x\neq 0,\exists c_x\in ]-|x|,|x|[$ tel que $f(x)-f(0) = f'(c_x)(x-0) = f'(c_x)$. Donc $f(x) = f(0)+xf'(x) = f(0)+xc_x^p\varepsilon(c_x)$. Posons $\varepsilon_2(x) = \frac{xc_c^p\varepsilon(c_x)}{x^{p+1}}$ si $x\neq 0$ ou 0 si $x=0$. On a par construction, $f(x) = f(0) + x^p\varepsilon_2(x)$.

Montrons maintenant que $\varepsilon_2(x) \to 0$ : on a $|\varepsilon_2(x)-0| =  \frac{|c_x|^p|\varepsilon(x)}{|x|^p} \leq \frac{|x|^p|\varepsilon(x)}{|x|^p} = |\varepsilon(c_x)| \to 0$ car $c_x\to 0$ quand $x\to 0$ (car $c_x\in ]-|x|,|x|[$.).
\end{myproof}
\begin{myproof}[De la proposition 1]
Posons $G: x\in I \to F(x) - a_0-a_1x^2/2-\dots-a_p\frac{x^{p+1}}{p+1}$. On a $G$ dérivable sur $I$ et $\forall x\in I,, G'(x) = f(x) - a_0-a_1x-a_px^p = x^p\varepsilon(x)$ par hypothèse. Par la proposition précédente, $G(x) = G(0) + x^{p+1}\varepsilon_2(x)$. Fini par la défintion de $G$.
\end{myproof}
\section{Notation de Landeau}
\begin{definition}
Soit \(I\) intervalle de \(\mathbb{R}\).

On dit que f est négligeable devant g en \(x_0\), noté \(f= o_{x\to x_0}(g)\) si il existe \(\eta >0\) et \(\varepsilon : ]x_0-\eta,x_0+\eta[\to \R\) telle que \(\varepsilon \to x\) et \(\forall x\in]x_0-\eta,x_0+\eta[, f(x) = g(x)\varepsilon\)

On dit que f est dominée par g en \(x_0\), noté \(f = O(g)_{x\to x_0}\)  si il existe \(\eta >0\) et \(\varepsilon : ]x_0-\eta,x_0+\eta[\to \R\) telle que \(|\varepsilon| \leq C\) et \(\forall x\in]x_0-\eta,x_0+\eta[, f(x) = g(x)\varepsilon\)

On dit que f est dominée par g en \(x_0\), noté \(f \sim_{x\to x_0} g\)  si il existe \(\eta >0\) et \(\varepsilon : ]x_0-\eta,x_0+\eta[\to \R\) telle que \(\varepsilon \to 0\) et \(\forall x\in]x_0-\eta,x_0+\eta[, f(x) = g(x)(1+\varepsilon)\)
\end{definition}
\tipsInfo{Remarque}{
Si $\varepsilon \to 0$, $x^p\varepsilon(x) = o(x^p)$.

De plus, f admet un DL à l'ordre p en 0 $\iff f(x)  = a_0+a_1x+\dots+a_px^p+o_{x\to 0}(x^p)$.

Enfin, $f(x) = o(1) \iff f\to 0$.

$f = x+o(x) \iff f \sim x$.

$\forall m,p \in \N, m\leq p, o(x^m) \Rightarrow O(x^m), o(x^p) \Rightarrow o(x^m)$.

$o(x^p)o(x^m) = o(x^{m+p})$.}
\criticalInfo{Sommes}{
$o(x^m) + o(x^m) = o(x^m)$ mais on a aussi $o(x^m)-o(x^m) = o(x^m) \neq 0$.}
\section{Fonctions p fois dérivables}
Dans cette partie, on note I un intervalle ouvert non vide de $\R$.
\begin{definition}
pour $p\in \N,$ la dérivée $p^{ene}$ de $f:I\to \R$, notée $f^{(p)}$ est définie récursivement : $f^{(0)} = f$ et si $f^{(p-1)}$ est dérivable sur I, alors $f^{(p)}  = (f^{(p-1)})'$.
\end{definition}
On dit que $f$ est p fois dérivable si $f^{(p)}$ est définie. Comme la dérivabilité implique la continuité, et $f^{(p)}$ déifinie, les dérivées précédentes sont continues sur tout l'intervalle.
\begin{definition}[Fonction p fois dérivables en un point]
On dit que f est p fois dérivabilité en \(x_0 \in I\) si \(\exists I_1\) intervalle ouvert tel que \(x_0\in\i_1\in I\) tel que f est p-1 fois dérivable sur \(I_1\) et \(f^{(p-1)}\) est dérivable en \(x_0\).
\end{definition}
\begin{definition}[Classe ]
On dit que f est de classe \(C^p\) sur I si f est p fois dérivable sur I et la dérivée pème est continue.

On dit que f est de classe \(C^{\infty}\) si \(f\in C^p \forall p\in\N\).
\end{definition}
\begin{proposition}
Soit \(f\) et \(g\) p fois dérivables sur I. Alors $f+g$ p fois dérivables sur $I$ et $(f+g)^{(p)} = f^{(p)} + g^{(p)}$.
\end{proposition}
\begin{myproof}[Par récurrence sur p]
Initialisation : pour $p=0$, on a $(f+g)^{(0)} = (f+g) = f+g$

Hérédité : Supposons qu'elles sont $p$ fois dérivables. On a $(f+g)^{(p)}$ et dérivons les. On a $f^{(p)'} + g^{(p)'} = f^{(p+1)} + g^{(p+1)}$.
\end{myproof}
Cela fonctionne aussi pour $f$ et $g$ p fois dérivables en un point et de classe $C^p$.

\begin{proposition}
\(\lambda f\) est p fois dérivables et \(\lambda f^{(p)} = (\lambda f)^{(p)}\).
\end{proposition}
\begin{proposition}
\(fg\) est p fois dérivable sur I et \((fg)^{(p)} = \sum^p_{m=0} C^m_p f^{(m)}g^{(p-m)}\)
\end{proposition}
\newpage
\explanation{step_1}{On applique l'hypothèse aux deux termes de la somme}
\explanation{step_2}{On fait passer f' en f en rajoutant unn degré de dévrivation. On fait la m\^eme chose pour g}
\explanation{step_3}{On change l'indice supérieur de la première somme. Elle ne va maintenant plus que jusqu'à p-1. Pour compenser on ajoute le terme en p pour maintenir l'égalité. De la m\^eme façon, on ajoute 1 à l'indice de départ de la deuxième somme. Pour maintenir l'égalité, on ajoute le premier terme de la somme}
\explanation{step_4}{On effectue un changement d'indice pour transformer la somme.}

\explanation{step_5}{On peut maintenant fusionner les 2 sommes car elles s'expriment avec les m\^mes indices.}

\explanation{step_6}{On remarque qu'en additionnant les termes qui restent, on obtient une somme allant de 0 à p+1, ce qui est recherché dans notre hérédité}
\begin{myproof}[Par récurrence sur p]
Ini : OK pour p = 0

Hérédité : Supposons $H_n$ vraie pour un entier $p$. On prend $f$ et $g$ p+1 fois dérivables sur I. Alors f et g sont une fois dérivable. Donc $fg$ est une fois dérivable et $(fg)' = fg'+f'g$. Donc par hypothèse $(fg)$ est p fois dérivable $\Rightarrow$ $fg$ est p+1 fois dérivable.

De plus, $fg^{(p+1)} = ((gf)')^{(p)} = (f'g+fg')^{(p)}  = (f'g)^{(p)} + (fg')^{(p)}$, soit par hypothèse
\begin{flalign*}
&= \sum_{m=0}^{p} C_p^m (f')^{(m)} g^{(p-m)} + \sum_{m=0}^p C_p^m f^{(m)} (g')^{(p-m)}\explain{step_1}{right}{0}{0.5}{} \\
&=\sum_{m=0}^{p} C_p^m (f)^{(m+1)} g^{(p-m)} + \sum_{m=0}^p C_p^m f^{(m)} (g)^{(p-m+1)}\explain{step_2}{right}{0}{0.5}{} \\
&= C^p_p f^{(p+1)} g^{(0)} + \sum^{p-1}_{m=0} C_p^m f^{(m+1)} g^{(p-m)}\\
&+ C_p^0 f^{(0)} g^{(p+1)} + \sum^p_{m=1} C_p^m f^{(m)} g^{(p-m+1)}\explain{step_3}{right}{0}{0.5}{} \\
&= C^p_p f^{(p+1)} g^{(0)} + \sum_{m=1}^{p} C_p^{m-1} f^{(m)} g^{(p-(m-1))}\\
&+ C_p^0 f^{(0)}g^{(p+1)} + \sum^p_{m=1}C_p^mf^{(m)}g^{(p-m+1)}\explain{step_4}{right}{0}{0.5}{} \\
&= (C^p_p= 1) f^{(p+1)}g^{(0)} + \sum^p_{m=1} (C^{m-1}_{p} + (C_{p}^{m}) = C^{m}_{p+1})f^{(m)} g^{(p-m+1)}\\
&+ (C^0_p = 1) f^{(0)}g^{(p+1)}\explain{step_5}{right}{0}{0.5}{}\\
&= \sum^{p+1}_{m=0} (C^{m}_{p+1})f^{(m)} g^{((p+1)-m)}\explain{step_6}{right}{0}{0.5}{}
\end{flalign*}
\end{myproof}
\begin{proposition}
Soient \(I,J\) 2 intervalles ouverts non vides de \(\mathbb{R}\). \(f:I\to J\subset \mathbb{R}\) et \(g : J\to \mathbb{R}\). Si f est n fois dérivable sur I et g est n fois déruvable sur J, alors \(g\circ f : I\to \mathbb{R}\) est n fois dérivable sur I.
\end{proposition}
\begin{myproof}[Par récurrence]
Ini : pour n=0, OK

Soient f n+1 fois dérivable sur I, et g n+1 fois dérivable sur J. On a f et g dérivables sur I et J, donc \(g\circ f\) est dérivable sur I et \((g\circ f)' = (g'\circ f)f'\).

f est n+1 fois dérivable, donc f' est n fois dérivable.

g est n+1 fois dérivbale donc g' est n fois dérivable.

Par l'hypothèse de récurrence \(g\circ f\) est n fois dérivable, donc le produit \((g'\circ f)f'\) est n fois dérivable, donc \(g\circ f\) est n+1 fois dérivable.

Pour la formule, on utilise la formule de Fadi Bruno.
\end{myproof}
\begin{proposition}
Si \(g : I\to \mathbb{R}\) est n fois dérivable sur I et ne s'annule pas sur I, alors la fonction \(f: x\in I\to \frac{1}{g(x)}\) est n fois dérivable aussi.
\end{proposition}
% \begin{myproof}[En exercice]
% Par récurrence : Idée : $f' = \frac{-g'}{g^2}$.
% \end{myproof}
\begin{proposition}[]
Soit \(f:I\to\mathbb{R}\) $n\neq 0$ fois dérivable telle que f' ne s'annule pas sur I. Alors f est une bijection de I dans \(f(I) = J\in\mathbb{R}\). On a alors l'application réciproque \(f^{-1}\) est n fois dérivable. On a aussi \(f^{-1} : J\to I\)
\end{proposition}
\begin{myproof}
Par récurrence ou direct : \(f^{-1'} = \frac{1}{f'(f^{-1})}\) On a \(f^{-1'} = \frac{1}{f'\circ f^{-1}}\). En effet, \(f'(f^{-1}(x))(f^{-1})(x) = 1 \iff f^{-1'} = \frac{1}{f'\circ f^{-1}}\) On peut aussi s'en servir pour retrouver la dérivée de arcsin
\end{myproof}
\section{Formules de Taylor}
\begin{theorem}[Théorème de Taylor-Young]
 Si f est $(n\geq 1)$ fois dérivable en 0, alors \(f(x) = f(0)+xf'(0)+\frac{x^2}{2} f''(0)+\dots + \frac{x^n}{n!}f^{(n)}(0)+o(x^n)\)
\end{theorem}
\begin{myproof}
Par récurrence :

Ini : Pour n=0,1 : OK.

Soit f n+1 fois dérivable en 0.

Alors f' est n fois dérivable en 0, donc par hypothèse $f'(x) = f'(0) + xf''(0)+\dots+x^n\frac{f^{(n+1)(0)}}{n!} + o(x^n)$. Par la propriété de 'primitivation' des DL, on obtient \(f(x) = f(0) + xf'(0) + \frac{x^2}{2}f''(0)+\dots+\frac{x^{n+1}}{(n+1)!}f^{(n+1)}(0)+o(x^{n+1})\)
\end{myproof}
\begin{theorem}[Théorème de taylor-Lagranges]
 Si f est \((n\geq 1)\) fois dérivable sur I, alors

 \(\forall x\in I, \exists c_x\)\( \in[\min(0,x), \max(0,x)]\) tel que \(f(x) = f(0) + xf'(0) + \frac{x^2}{2}f''(0)+\dots+\frac{x^{n-1}}{(n-1)!}f^{(n-1)}(0)\frac{x^n}{n!}f^{(n)}(c_x)\)
\end{theorem}
\warningInfo{Remarques}{Quand n=1, on obtient le TAF

Si $x\neq0,c_x\in$ l'intervalle ouvert.}
\begin{proposition}
Soit I un intervalle ouvert non vode de R, \(f:I\to \mathbb{R}\), p+1 fois dérivable. On suppose qu'il existe \((a,b)\in I\times I\) tels que \(f(a) = f(b)\) et \(0 = f'(a) = f''(a) = f^{(p)}(a)\). Alors \(\exists c \in]\min(a,b),\max(a,b)[\) tel que \(f^{(p+1)}(c) = 0\).
\end{proposition}
\begin{myproof}
Preuve dans les notes de cours
\end{myproof}
\begin{myproof}[de Taylor-Lagranges]
Si \(x=0\), on prend \(c_x =0\).

À partir de maintenant, on fixe \(x\in I, x\neq 0\) Pour \(k\in\mathbb{R}\), on pose \(G_k : t\in I\to f(t)-f(0)-tf'(0)-\dots-\frac{t^{n-1}}{(n-1)!}f^{(n-1)}(0) - \frac{t^n}{n!}K\). On choisi K tel que \(G(x) = 0\).

On a \(G_x\) est p fois dérivable sur I. On a aussi \(G_k(0) = g_k(x) = 0\). De plus, par constcrution \(G_k'(0) = G_k''(0) = G^{(n-1)}(0) = 0\)

Par Rolle généralisé, \(\exists c_x \in ]\min(0,x),\max(0,c)[\) tel que \(G_k^{(n)} = 0 = f^{(n)} - K \Rightarrow K = f^{(n)}(c_x)\).
\end{myproof}
\begin{theorem}[Formule de Taylor-Laplace (avec reste intégrale)]
 Si f \(I\to \R\) tel que f tes de classe \(C^p(I)\), alors \(\forall x\in I\), \(f(x) = f(0)+xf'(0)+\dots+\frac{x^{p-1}}{(p-1)!}f^{(p-1)}(0) + \)
 \[\int^x_0f^{(p)}(s) \frac{(x-s)^{p-1}}{(p-1)!}\dd s\]
\end{theorem}
\begin{myproof}[Par récurrence]
Ini pour p = 1.
\(\int^x_0f^{(p)}(s) \frac{(x-s)^{0}}{(0)!}\dd s = f(x)-f(0)\)

Hérédite : On suppose qu'il existe un p tel que l'hypothèse est vraie. L'application \(s\in I : \to f^{(p+1)}(s) \frac{(x-s)^{p}}{(p)!}\) est continue donc intégrable sur [0,x]. On peut donc écrire \[\int^x_0f^{(p+1)}(s) \frac{(x-s)^{p-1}}{(p-1)!}\dd s\]

On fait une intégration par parties : On pose \(u'(s) = f^{(p+1)}(s) \to u = f^{(p)}(s)\) et \(v = \frac{(x-s)^p}{p!} \to v'(s) = -\frac{1}{(p-1)!}(x-s)^{p-1}\). En intégrant, on obtient

\(u(x)v(x)-u(0)v(0) - \int^x_0\u(s)v'(s)\dd s = -f^{(p)(0)\frac{x^p}{p!}-\int^x_0}f^{(p)}(s)\frac{-1}{(p-1)!}(x-s)^{p+1}\dd s\). Par l'hypothèse de récurrence \( = -f^{(p)}(0)\frac{x^p}{p!}+f(x)-f(0)+xf'(0)-\dots-\frac{x^{p-1}}{(p-1)!}f^{(p-1)}(0)\). Donc $H_p$.

\end{myproof}
\subsection{DL de fonctions usuelles}
\begin{eqnarray*}
e^x&=&1+x+\frac{x^2}2+\dots+\frac{x^n}{n!}+o(x^n)\\
\cos x&=&1-\frac{x^2}{2!}+\dots+\frac{(-1)^n x^{2n}}{(2n)!}+o(x^{2n+1})\\
\sin x&=&x-\frac{x^3}{3!}+\dots+\frac{(-1)^n x^{2n+1}}{(2n+1)!}+o(x^{2n+2})\\
\cosh x&=&1+\frac{x^2}{2!}+\dots+\frac{x^{2n}}{(2n)!}+o(x^{2n+1})\\
\sinh x&=&x+\frac{x^3}{3!}+\dots+\frac{ x^{2n+1}}{(2n+1)!}+o(x^{2n+2})\\
\frac{1}{1-x}&=&1+x+x^2+\dots+x^n+o(x^n)\\
\frac{1}{1+x}&=&1-x+x^2+\dots+(-1)^nx^n+o(x^n)\\
\ln(1+x)&=&x-\frac{x^2}2+\dots+\frac{(-1)^{n+1}}{n}x^n+o(x^n)\\
(1+x)^\alpha&=&1+\alpha x+\frac{\alpha(\alpha-1)}2x^2+\dots+\frac{\alpha(\alpha-1)\cdots (\alpha-n+1)}{n!}x^n+o(x^n)\\
\tan(x) &=& x+\frac{x^3}{3}+o(x^4)\\
\end{eqnarray*}
\section{Application à l'étude des extremums locaux de fonction}
Dans toute cette partie, $f : ]a,b[ \to \R$ avec $a<b$.

\begin{definition}[Minimum/maximum/extremum local]
Soit \(x_0\in]a,b[\). C'est un mimum local de f si \(\exists \varepsilon>0,\forall x\in ]x_0-\varepsilon,x_0+\varepsilon[\cap ]a,b[, f(x)\geq f(x_0)\)

C'est un mimum local de f si \(\exists \varepsilon>0,\forall x\in ]x_0-\varepsilon,x_0+\varepsilon[\cap ]a,b[, f(x)\leq f(x_0)\).

C'est un extremum local de f si c'est l'un des deux.
\end{definition}
\begin{definition}[Point critique]
Soit f dérivable. On dit que \(x_0\) est un point critique de f si \(f'(x_0)=0\).
\end{definition}
\begin{proposition}
Si \(x_0\) est un extremum local de f, alors c'est un point critique de f
\end{proposition}
\begin{myproof}
Supposons que $x_0$ est un minimum local. Il existe $\varepsilon>0, \forall x\in  ]x_0-\varepsilon,x_0+\varepsilon[\cap ]a,b[, f(x)\geq f(x_0)$.

On sait que dans un cas \(f(x)\geq f(x_0) \Rightarrow f(x)-f(x_0)\geq 0\) et \(x-x_0>0\). Donc le quotient est supérieur ou égal à 0 et la dérivée (taux d'accroissement) est supérieur ou égale à 0.

On sait que dans un cas \(f(x)\geq f(x_0) \Rightarrow f(x)-f(x_0)\geq 0\) et \(x-x_0<0\). Donc le quotient est inférieure ou égal à 0 et la dérivée (taux d'accroissement) est inférieure ou égale à 0.

Donc la dérivée est nulle
\end{myproof}
La récirproque est fausse (voir point d'inflexion).

Si l'intervalle ab est fermé, alors f admet un maximum/minium local qui n'est pas un point critique.
\begin{proposition}
Soit f dérivable et \(x_0\) un point critique de f. On suppose qu'il existe \(p\in \N, p\geq 2, a_p\in \R,a_p\neq 0\) tel que \(f(x) = f(x_0)+a_p(x-x_0)^p+o((x-x_0)^p)\). Alors :
\begin{itemize}
 \item p est pair et \(a_p > 0\), \(x_0\) est un minium local
  \item p est pair et \(a_p > 0\), \(x_0\) est un maxium local
    \item p est impair  \(x_0\) est un point d'inflexion.
\end{itemize}
\end{proposition}
\begin{myproof}
Pour le cas 1 : \(f(x)-f(x_0) = a_p(x-x_0)^p+o((x-x_0)^p) = (x-x_0)^p (a_p+\varepsilon(x))\). On a \(\exists \varepsilon >0, \forall x\in ]x_0-\varepsilon,x_0+\varepsilon[, \frac{-a_p}{2} \leq \varepsilon(x)\leq \frac{a_p}{2} \) donc \(f(x)-f(x_0) \geq (x-x_0)^p \frac{a_p}{2} > 0\).
\end{myproof}
\begin{corollary}
Soit \(f : ]a,b[\to \R\) dérivable et on suppose qu'il existe \(p\in\N, p\geq 2, a_0\in\R,a_1\in \R,a_p\in\R,a_p\neq 0\) tel que \(f(x) = a_0+a_1(x-x_0)+a_p(x-x_0)^p+o((x-x_0)^p)\) alors la droite d'équation \(a_0+a_1(x-x_0)=y\) est la tangente à la courbe de f en \(x_0\).

Si p est pair, et \(a_p>0\) alors la tangente est localement au dessous de la courbe

Si p est pair, et \(a_p<0\) alors la tangente est localement au dessus de la courbe

Si p est impair, et \(a_p>0\) alors la tangente est localement au dessus de la courbe pour les x>x_0 et en dessous pour les x<x_0

Si p est impair, et \(a_p<0\) alors la tangente est localement au dessus de la courbe pour les x<x_0 et en dessous pour les x>x_0
\end{corollart}
\end{document}

