% !TeX spellcheck = en_US
\documentclass[french]{yLectureNote}

\title{Mécanique 2}
\subtitle{Chimie}
\author{Paulhenry Saux}
\date{\today}
\yLanguage{Français}

\professor{IHallery}%isabelle.hallery@univ-tlse3.fr
\usepackage{graphicx}%----pour mettre des images
\usepackage[utf8]{inputenc}%---encodage
\usepackage{geometry}%---pour modifier les tailles et mettre a4paper
%\usepackage{awesomebox}%---pour les boites d'exercices, de pbq et de croquis ---d\'esactiv\'e pour les TP de PC
\usepackage{tikz}%---pour deiffner + d\'ependance de chemfig
% \usepackage{tabularx}%---pour dimensionner automatiquement les tableaux avec variable X
\usepackage{awesomebox}%---Pour les boites info, danger et autres
\usepackage{menukeys}%---Pour deiffner les touches de Calculatrice
\usepackage{fancyhdr}%---pour les en-t\^ete personnalis\'ees
\usepackage{blindtext}%---pour les liens
\usepackage{hyperref}%---pour les liens (\`a mettre en dernier)
\usepackage{caption}%---pour la francisation de la l\'egende table vers Tableau
\usepackage{pifont}
\usepackage{array}%---pour les tableaux
\usepackage{lipsum}
\usepackage{yFlatTable}
\usepackage{multicol}
\newcommand{\Lim}[1]{\lim\limits_{\substack{#1}}\:}
\renewcommand{\vec}{\overrightarrow}
\newcommand{\N}[0]{\mathbb{N}}
\newcommand{\dd}{\mathrm{d}}
\newcommand{\norm}[1]{||\vec{#1}||}
\begin{document}
\setcounter{chapter}{3}
%Controle après les vacances (c1,c2,(c3 mais basique))
\chapter{Systèmes à plusieurs corps}
\section{Éléments cinétiques du système}
\subsection{Centre de masse}
On appelle centre de masse C, ou centre d'inertie, d'un système S de plusieurs points matériels le barycentre des points A, qui le constituent, affectés de leurs masses respectives \(m_i\).\marginCheck{Le centre de masse se rapproche du point le plus lourd}
\begin{proposition}[Caractérisation du centre de masse]
\[\sum_i m_iCA_i = 0\]
\end{proposition}
\begin{theorem}[Centre de masse]
 \[\vec{OC} = \frac{1}{M_{total}}\sum_i m_i OA_i\] avec \(M = \sum_i m_i\)
\end{theorem}
\subsection{Quantité de mouvement}
Par rapport à un référentiel R, la quantité de mouvement totale du système S est la somme vectorielle des quantités de mouvement de chacun de ses points A :
\begin{theorem}[Quantité de mouvement]
 \[\vec{P} = \sum\vec{p}_i = \sum m_i \vec{v}_i\]
\end{theorem}
\begin{proposition}[Quantité de mouvement du centre de masse]
\[\vec{P} = M_{total}\vec{V}_C\]
\end{proposition}
\subsubsection{Forces internes}
% \begin{flalign*}
% \frac{\dd \vec{P}}{\dd t} &= \sum^n_{i=1} \frac{\dd \vec{p_i}}{\dd t}\\
% &= \sum \vec{F_i} = 0
% \end{flalign*}
\begin{proposition}[Qt de mouvement dans un système isolé]
Les forces internes se compensent et il n'y a pas de forces externes, donc la quatité de mouvement reste constante
\end{proposition}
\subsection{Référentiel du centre de masse}
\warningInfo{Nouveau référentiel}{On intoduit le nouveau référentiel R* dont C, le centre de masse est l'origine}
Pour un système matériel, en mouvement par rapport au référentiel d'analyse R, le référentiel du centre de masse R* est le référentiel en translation par rapport à R tel que, dans R*, la quantité de mouvement totale du système S soit nulle. Ainsi, \[\vec{P*} = M\vec{v*_c} = \vec{0}\]
\subsection{Moment cinétique}
\subsubsection{Référentiel normal}
Par rapport au référentiel R, le moment cinétique total de S, au point O, est la somme vectorielle des moments cinétiques des points A qui le constituent :
\begin{theorem}[Moiment cinétique total]
 \[\vec{L_O} = \sum\vec{L_{O,i}} = \sum \vec{OA_i}\wedge \vec{p_i}\]
\end{theorem}
\begin{proposition}[Transport du mouvement]
\[\vec{L_O} = L_{O'} + \vec{OO'}\wedge \vec{P}\] d'où l'on déduit la relation entre le moment cinétique au point O et au centre de masse : \[\vec{L_O} = \vec{L_C} + \vec{OC}\wedge \vec{P}\]
\end{proposition}
\subsubsection{Référentiel R*}
Écrivons, par rapport à R*, la relation entre le moment cinétique en un point quelconque Q et le moment cinétique au centre de masse. Il vient, d'après ce qui précède : \(\vec{L*_Q} = \vec{L*C} + \vec{QC}\wedge \vec{P*}\). Or, comme dans ce référentiel, \(\vec{P} = \vec{0}\), on obtient :
\[\vec{L*_Q} = \vec{L*_C} = \vec{L*}\]

\section{Théorèmes de Koening}
\begin{theorem}[Théorème de Koeing du moment cinétique]
\[\vec{L_O} = \vec{L*}+\vec{OC}\wedge \vec{P}\]
\end{theorem}
On appelle \(\vec{OC}\wedge \vec{P}\) le moment cinétique orbital et \(\vec{L*}\) le moment cinétique intrinsèque.
\begin{theorem}[Théorème de Koeing de l'énergie cinétique]
L'énergie cinétique dans R est l'énergie cinétique dans R* augmentée de l'énergie cinétique du centre de masse dotée de la masse totale :
 \[E_k = E_k*+0.5MV^2_C\] avec \(E_k*\) l'énergie cinétique interne et \(0.5MV^2_C\) l'énergie cinétique externe.
\end{theorem}
\section{Théorème du moment cinétique}
\begin{theorem}[Théorème du moment cinétique]
Par rapport à un référentiel galiléen, la dérivée par rapport au temps du moment cinétique est égale à la somme des seuls moments des forces extérieures qui s'exercent sur le système. Les forces internes se compensent.
 \[\frac{\dd \vec{L_O}}{\dd t} = \vec{M_{O,ex}}\]
\end{theorem}
%TODO : Faire schema pour relation de chales pour v_i et v_c
\end{document}
