% !TeX spellcheck = en_US
\documentclass[french]{yLectureNote}

\title{Mécanique 2}
\subtitle{Chimie}
\author{Paulhenry Saux}
\date{\today}
\yLanguage{Français}

\professor{IHallery}%isabelle.hallery@univ-tlse3.fr
\usepackage{graphicx}%----pour mettre des images
\usepackage[utf8]{inputenc}%---encodage
\usepackage{geometry}%---pour modifier les tailles et mettre a4paper
%\usepackage{awesomebox}%---pour les boites d'exercices, de pbq et de croquis ---d\'esactiv\'e pour les TP de PC
\usepackage{tikz}%---pour deiffner + d\'ependance de chemfig
% \usepackage{tabularx}%---pour dimensionner automatiquement les tableaux avec variable X
\usepackage{awesomebox}%---Pour les boites info, danger et autres
\usepackage{menukeys}%---Pour deiffner les touches de Calculatrice
\usepackage{fancyhdr}%---pour les en-t\^ete personnalis\'ees
\usepackage{blindtext}%---pour les liens
\usepackage{hyperref}%---pour les liens (\`a mettre en dernier)
\usepackage{caption}%---pour la francisation de la l\'egende table vers Tableau
\usepackage{pifont}
\usepackage{array}%---pour les tableaux
\usepackage{lipsum}
\usepackage{yFlatTable}
\usepackage{multicol}
\newcommand{\Lim}[1]{\lim\limits_{\substack{#1}}\:}
\renewcommand{\vec}{\overrightarrow}
\newcommand{\N}[0]{\mathbb{N}}
\newcommand{\dd}{\mathrm{d}}
\newcommand{\norm}[1]{||\vec{#1}||}
\begin{document}
\setcounter{chapter}{1}
\chapter{Dynamique}
\section{Lois de Newton}
\subsection{Première loi}
\explanation{1}{Un objet est libre si aucune force n'y est appliqué et pseudo-isolé si la somme des forces appliquées est nulle}
\explanation{2}{Tout référentiel lié à un référentiel aboslu par une transformation (translation à vitesse constnate et rectiligne, avec le m\^eme temps) est aussi galiléen }

 Dans un référentiel galiléen\explain{2}{right}{0}{0.25}{}, tout objet libre\explain{1}{left}{0}{0.25}{} conserve son état de repos soit reste en mouvement uniforme rectiligne
\subsection{PDF}
Dans un référentiel R, si un objet est soumis à une force \(\vec{F}\), on a \[\sum \vec{F} = m \vec{a} = m \frac{\dd \vec{v}}{\dd t} = \frac{\dd \vec{P}}{\dd t}\] avec \(\vec{P} = m \vec{V}\) le moment cinétique.
\subsection{Force récirproque}
\[\vec{F_{A/B}} = -\vec{F_{B/A}}\]
\warningInfo{Loi de conservation du moment cinétique}{Dans un système isolé, \(\frac{\dd \vec{P}}{\dd t} = 0\)}

\section{Travail}
Voir fiche de meca 1

\subsection{Forces non conservatives}
Exemples : Frottement fluides : \(\vec{f} = -\alpha \vec{v}\), frottements solides, \(\vec{f} = -F_0 \frac{\vec{V}}{\norm{V}}\)
\subsection{Point d'équilibre}
Autour du point d'équilibre, on a un système oscillatoire
\end{document}
