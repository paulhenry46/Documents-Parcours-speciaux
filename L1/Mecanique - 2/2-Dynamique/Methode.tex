% !TeX spellcheck = en_US
\documentclass[french]{yLectureNote}

\title{Mécanique}
\subtitle{Mécanique du point}
\author{Paulhenry Saux}
\date{\today}
\yLanguage{Français}

\professor{S.Deheuvels}%sebastien.deveuhels.irap.omp.eu

\usepackage{graphicx}%----pour mettre des images
\usepackage[utf8]{inputenc}%---encodage
\usepackage{geometry}%---pour modifier les tailles et mettre a4paper
%\usepackage{awesomebox}%---pour les boites d'exercices, de pbq et de croquis ---d\'esactiv\'e pour les TP de PC
\usepackage{tikz}%---pour deiffner + d\'ependance de chemfig
\usepackage{tkz-tab}
\usepackage{chemfig}%---pour deiffner formules chimiques
\usepackage{chemformula}%---pour les formules chimiques en \'equation : \ch{...}
\usepackage{tabularx}%---pour dimensionner automatiquement les tableaux avec variable X
\usepackage{awesomebox}%---Pour les boites info, danger et autres
\usepackage{menukeys}%---Pour deiffner les touches de Calculatrice
\usepackage{fancyhdr}%---pour les en-t\^ete personnalis\'ees
\usepackage{blindtext}%---pour les liens
\usepackage{hyperref}%---pour les liens (\`a mettre en dernier)
\usepackage{caption}%---pour la francisation de la l\'egende table vers Tableau
\usepackage{pifont}
\usepackage{array}%---pour les tableaux
\usepackage{lipsum}
\usepackage{yFlatTable}
\usepackage{multicol}
\usepackage{cancel}
\usepackage{xcolor}
\newcommand\Ccancel[2][black]{\renewcommand\CancelColor{\color{#1}}\cancel{#2}}
\newcommand{\Lim}[1]{\lim\limits_{\substack{#1}}\:}
\renewcommand{\vec}{\overrightarrow}
\newcommand{\norm}[1]{||\vec{#1}||}
\newcommand{\dd}[0]{\mathrm{d}}
\newcommand{\ddp}[0]{\partial}
%\DeclareMathOperator\arctanh{arctanh}
\DeclareMathOperator\grad{grad}
\begin{document}

%\titleOne
\setcounter{chapter}{1}
	\chapter{Travail - Énergie}
\section{Exprimer le travail d'une force}
\subsection{Déplacement élémentaire}
On réécrit le déplacement élémentaire dans la base considérée (souvent la cartésienne) : $W_{AB}(\vec{F_i}) = \int_{C,AB}(\vec{F_i})\cdot (\dd x \vec{e_x} + \dd y \vec{e_y} + \dd z \vec{e_z})$

Le but est ensuite d'écrire le déplacement élémentaire en fonction d'une seule composante $\dd x, \dd y$ ou $\dd z$. Pour cela,
\begin{itemize}
 \item soit on remarque le déplacement s'effectue selon un axe seulement (cas d'un déplacement horizontal ou vertical): on peut alors supprimer les $\dd y, \dd x, \dd z$ où il n'y a pas de déplacement
 \item soit on utilise l'équation du mouvement (équation de droite, de parabole)
\end{itemize}
Dans le deuxième cas, on va se servir de l'équation fournie ou à créer.
\subsubsection{Droite}
Dans le cas d'une droite (on suppose dans notre exemple qu'elle est contenue dans le plan (oxy)), on peut écrire une relation entre les 2 axes de la forme \(y = ax+b\).

Pour déterminer $b$, qui est l'ordonnée à l'origine, on utilise la valeur de y quand $x=0$.

Pour déterminer $a$, il faut déterminer la pente de la droite. Pour cela, on prend 2 points dont on connait la valeur en y et on applique la formule : $a = \frac{y_2-y_1}{x_2-x_1}$

Une fois l'équation obtenue, on va exprimer le déplacement élémentaire en fonction de $\dd y$ puis en fonction de $\dd x$ :
\begin{flalign*}
&y = ax+b\\
&\Rightarrow \dd y = a \dd x\\
&\Rightarrow \dd r = \dd x(\vec{e_x} + a\vec{e_y})\\
&x = \frac{y-b }{a}\\
&\Rightarrow \dd x = \frac{1}{a}\dd y\\
&\Rightarrow \dd r = \dd y(\vec{e_y} + \frac{1}{a}\vec{e_x})
\end{flalign*}
\subsubsection{Autre fonction donnée}
On a une trajectoire dans le plan dont l'équation est définie par une expression quelconque. Pour notre exemple, nous prendrons $y = ax^2$

Ici, aussi on va exprimer le déplacement élémentaire en fonction de $\dd y$ puis en fonction de $\dd x$ :
\begin{flalign*}
&y = ax^2\\
&\Rightarrow \dd y = 2ax \dd x\\
&\Rightarrow \dd r = \dd x(\vec{e_x} + 2ax\vec{e_y})\\
&x = \sqrt{\frac{y}{a}} = \frac{1}{\sqrt{a}} \sqrt{y}\\
&\Rightarrow \dd x =\frac{1}{\sqrt{a}} \frac{1}{2\sqrt{y}}\dd y\\
&\Rightarrow \dd r = \dd y(\vec{e_y} + \frac{1}{\sqrt{a}} \frac{1}{2\sqrt{y}}\vec{e_x})
\end{flalign*}
\subsection{Calcul du travail}
On applique la formule $W_{AB}(\vec{F_i}) = \int_{C,AB}\vec{F}\cdot \dd \vec{r}$ en remplaçant le déplacement élémentaire par celui obtenu juste avant et la force par sa décomposition en fonction des vecteurs de la base utilisée.\marginTips{On préférera utiliser la décomposition du déplaceent élémentaire dont la variable $\dd$ apparait dans l'expression de la force. Par exemple, si la force dépend des coordonnées x, on utilise plutot l'expression avec $\dd x$}

\warningInfo{Remplacement du déplacement élémentaire}{Lorsque l'on remplace le déplacement élémentaire par son expression en fonction de  $\dd y, \dd x, \dd z$, il faut changer les bornes de l'intégrale par les coordonnées selon $y,x $ ou $z$ des point de départ et d'arrivée. En effet, nous n'intégrons plus sur le chemin mais selon une seule composante !  }
Exemple avec $\vec{F} = (ay^2+b)\vec{e_x}+cy\vec{e_y}$ et des $\dd$ calculés avec la méthode précédente de la forme $\dd r = \dd x(\vec{e_x}+\alpha \vec{e_y}) = \dd y(\frac{1}{\alpha}\vec{e_x} + \vec{e_y})$
\newpage
\explanation{1}{On remarque ici que la force s'exprime en fonction de y. Nous allons donc utiliser le déplacement élémentaire en fonction de y dans la prochaine étape }
\explanation{2}{On n'oublie le changement des bornes. Dans cet exemples, on suppose que les cordonnées en y du point de départ sont h et 0}
\explanation{3}{On fait le produit scalaire en se rappelant que dans la base cartésienne $\vec{e_x}\cdot \vec{e_y} = 0$ et que $\vec{e_y}\cdot \vec{e_y} =1$ }
\explanation{4}{Il ne nous reste plus qu'à intégrer}
On a :
\begin{flalign*}
W &= \int_C \vec{F} \cdot \dd \vec{r}\\
&= \int_C ((ay^2+b)\vec{e_x}+cy\vec{e_y}) \cdot \dd \vec{r}\explain{1}{right}{0}{0.5}{×}\\
&= \int_h^0 ((ay^2+b)\vec{e_x}+cy\vec{e_y}) \cdot (\dd y(\frac{1}{\alpha}\vec{e_x} + \vec{e_y}))\explain{2}{right}{0}{0.5}{×}\\
&= \int_h^0 \frac{1}{\alpha}(ay^2+b)+cy (\dd y)\explain{3}{right}{0}{0.5}{×}\\
&= \int_h^0 \frac{1}{\alpha}(ay^2) + \frac{1}{\alpha}(b)+cy (\dd y)\explain{4}{right}{0}{0.5}{×}\\
&= [ \frac{1}{3\alpha}(ay^3) + \frac{1}{\alpha}(by)+c\frac{y^2}{2} ]^0_h\\
&= \frac{1}{3\alpha}(ah^3) + \frac{1}{\alpha}(bh)+c\frac{h^2}{2}\\
\end{flalign*}
\section{Énergie potentielle}
\subsection{Déterminer l'énergie potentielle à partir d'une force}
\checkInfo{Méthode}{On sait que \(\vec{F} = -\nabla E_p\)

Donc, dans un repère cartésien,  \(E_p = -\int \vec{F}\cdot \vec{e_x}\dd x -\int \vec{F}\cdot \vec{e_y}\dd y -\int \vec{F}\cdot \vec{e_z}\dd z +Cst\)

Pour déterminer la constante, on peut se servir de la valeur de l'énergie potentielle en un point ou à l'infini.}
Exemple avec $\vec{F} = -a\cos(\alpha)\vec{e_x} - a\sin(\theta)\vec{e_y} - b\cos(\beta)\vec{e_z}$. On sait que l'objet reste sur l'axe z=0.

En intégrant, on obtient :
\[E_p = a \cos(\alpha)x + a\sin(\alpha)y + b\cos(\beta)z + K\]

Si l'on suppose que l'énergie potentielle est nulle à l'origine du repère, K = 0.

Comme on sait que z=0, on peut simplifier :

\[E_p = a \cos(\alpha)x + a\sin(\alpha)y\]
\subsection{Étudier des positions d'équilibre}
On a une force dont on connait l'énergie potentielle. On aimerait connaitre, en supposons que l'objet est uniquement soumis à cette force, ses positions d'équilibres.

Pour cela, il faut résoudre \(\vec{F} = \vec{0}\)

Pour connaitre la nature des positions d'équilibre trouvées, il faut calculer la dérivée seconde de l'énergie potentielle trouvée.

Si au point étudiée, la dérivée est positive, l'équilibre est stable. Dans le cas contraire, il est instable.

Autrement dit, si \(\frac{\dd E_p(x_0)}{\dd x} >0\), l'équilibre est stable, et si \(\frac{\dd E_p(x_0)}{\dd x} <0\) l'équilibre est instable.
\end{document}

