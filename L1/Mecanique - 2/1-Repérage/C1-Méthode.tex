% !TeX spellcheck = en_US
\documentclass[french]{yLectureNote}

\title{Mécanique 2}
\subtitle{Méthode}
\author{Paulhenry Saux}
\date{\today}
\yLanguage{Français}

\professor{IHallery}%isabelle.hallery@univ-tlse3.fr
\usepackage{graphicx}%----pour mettre des images
\usepackage[utf8]{inputenc}%---encodage
\usepackage{geometry}%---pour modifier les tailles et mettre a4paper
%\usepackage{awesomebox}%---pour les boites d'exercices, de pbq et de croquis ---d\'esactiv\'e pour les TP de PC
\usepackage{tikz}%---pour deiffner + d\'ependance de chemfig
% \usepackage{tabularx}%---pour dimensionner automatiquement les tableaux avec variable X
\usepackage{awesomebox}%---Pour les boites info, danger et autres
\usepackage{menukeys}%---Pour deiffner les touches de Calculatrice
\usepackage{fancyhdr}%---pour les en-t\^ete personnalis\'ees
\usepackage{blindtext}%---pour les liens
\usepackage{hyperref}%---pour les liens (\`a mettre en dernier)
\usepackage{caption}%---pour la francisation de la l\'egende table vers Tableau
\usepackage{pifont}
\usepackage{array}%---pour les tableaux
\usepackage{lipsum}
\usepackage{yFlatTable}
\usepackage{multicol}
\newcommand{\Lim}[1]{\lim\limits_{\substack{#1}}\:}
\renewcommand{\vec}{\overrightarrow}
\newcommand{\N}[0]{\mathbb{N}}
\newcommand{\dd}{\mathrm{d}}
\newcommand{\norm}[1]{||\vec{#1}||}
\newcommand{\ex}{\vec{e_x}}
\newcommand{\ey}{\vec{e_y}}
\newcommand{\ez}{\vec{e_z}}
\begin{document}
\chapter{Systèmes de coordonnées}
\section{Vecteurs}
\subsection{Construire le vecteur unitaire tangent à la trajectoire}
\begin{theorem}[Vecteur unitaire tangent à la trajectoire]
 \[e_t = \frac{\vec{v}}{\norm{v}}\]
\end{theorem}
Exemple pour un vecteur vitesse \(\vec{v} = at\vec{e_x}+b\vec{e_y}\):

On a :
\begin{flalign*}
\vec{e_t} &= \frac{\vec{v}}{\norm{v}}\\
&= \frac{at\vec{e_x}+b\vec{e_y}}{\sqrt{a^2t^2+b^2}}\\
&= \frac{at}{\sqrt{a^2t^2+b^2}}\ex + \frac{b}{\sqrt{a^2t^2+b^2}}\ey
\end{flalign*}
\subsection{Donner les composantes tangentes et normales d'un vecteur}
On cherche comment écrire le vecteur sous la forme \[\vec{v} = v_t\vec{e_t} + v_n\vec{e_n}\] en connaissant déjà son expression à partir d'autres vecteurs et celle du vecteur normal ou du vecteur tangent.
\newpage
\explanation{1}{Afin de trouver la composante d'un vecteur selon un autre vecteur, on réalise le produit scalaire entre les 2}
\explanation{2}{On écrit les 2 vecteurs en fonction des vecteurs de la base choisie}
\explanation{3}{Le fait d'avoir choisi des vecteurs de la m\^eme base pour les vecteurs simplifie les calculs et permet de ne pas avoir à chercher l'angle du produit scalaire.}
\warningInfo{Base utilisée}{La base dans laquelle les vecteurs normal et tangent doit \^etre la m\^eme que la base dans laquelle on connait une expression du vecteur}
On écrit :
\begin{flalign*}
\vec{v} &= v_t\vec{e_t} + v_n\vec{e_n} = v_x\ex + v_y\ey\\
v_t &= \vec{v}\cdot \vec{e_t}\explain{1}{right}{-0.5}{0.5}{}\\
&=( v_x\ex + v_y\ey) \cdot (e_{tx}\ex + e_{ty}\ey)\explain{2}{right}{0}{0.5}{}\\
&= v_x \times e_{tx} + v_y \times  e_{ty}\explain{3}{right}{0}{0.5}{}
\end{flalign*}
\explanation{4}{Avec la ligne précédente, on trouve une expression en fonction de plusieurs vecteurs. Pour trouver la valeur de $a_n$, il faut calculer la norme du vecteur en résultant. Si les vecteurs en fonction desquels l'expression obtenue est exprimée sont perpendiculaires, on peut simplement appliquer le théorème de Pythagore !}
Une fois que l'on connait la composante selon un des vecteurs (normal ou tangent), on peut déterminer l'autre composante sans m\^eme avoir à connaitre l'expression de l'autre vecteur !
En effet,
\begin{flalign*}
\vec{v} &= v_t\vec{e_t} + v_n\vec{e_n}\\
\vec{v} -v_t\vec{e_t} &=  v_n\vec{e_n}\\
v_n &= \sqrt{A^2+B^2}\explain{4}{right}{-0.5}{0.5}{}
\end{flalign*}

\subsection{Donner le rayon de courbure d'une trajectoire}

On applique la formule
\begin{theorem}[Rayon de courbure]
 \[R_c = \frac{v^2}{a_n}\]
\end{theorem}

\end{document}
