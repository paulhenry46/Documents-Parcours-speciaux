 % !TeX spellcheck = en_US
\documentclass[french]{yLectureNote}

\title{Optique Géométrique}
\subtitle{Niveau 1}
\author{Paulhenry Saux}
\date{\today}
\yLanguage{Français}

\professor{C.Gatel}%christophe.gatel@cemes.fr : section B sillon 7

\usepackage{graphicx}%----pour mettre des images
\usepackage[utf8]{inputenc}%---encodage
\usepackage{geometry}%---pour modifier les tailles et mettre a4paper
%\usepackage{awesomebox}%---pour les boites d'exercices, de pbq et de croquis ---d\'esactiv\'e pour les TP de PC
\usepackage{tikz}%---pour deiffner + d\'ependance de chemfig
\usepackage{tkz-tab}
\usepackage{chemfig}%---pour deiffner formules chimiques
\usepackage{chemformula}%---pour les formules chimiques en \'equation : \ch{...}
\usepackage{tabularx}%---pour dimensionner automatiquement les tableaux avec variable X
\usepackage{awesomebox}%---Pour les boites info, danger et autres
\usepackage{menukeys}%---Pour deiffner les touches de Calculatrice
\usepackage{fancyhdr}%---pour les en-t\^ete personnalis\'ees
\usepackage{blindtext}%---pour les liens
\usepackage{hyperref}%---pour les liens (\`a mettre en dernier)
\usepackage{caption}%---pour la francisation de la l\'egende table vers Tableau
\usepackage{pifont}
\usepackage{array}%---pour les tableaux
\usepackage{lipsum}
\usepackage{yFlatTable}
\usepackage{multicol}
\newcommand{\Lim}[1]{\lim\limits_{\substack{#1}}\:}
\renewcommand{\vec}{\overrightarrow}
\begin{document}
%\titleOne

\setcounter{chapter}{6}
\chapter{Miroirs}
\subsection{Définitions}
Comme les dioptres, il y a des miroirs convexes et concaves dépendant de \(\bar{SC}\).

Comme les lentilles, les indices objet et images sont les m\^emes.

On étudie les miroirs dans les conditions de Gauss.

On hachure le coté où la lumière ne passe pas.
\subsection{Réflexion sur un miroir sphérique}
Changer le sens de parcours de la lumière est équivalent à prendre l'opposé de l'indice.

On change le sens positif de parcours selon si le rayon appartient à l'objet ou à l'image.

On considère \(\bar{SC}\) dans le sens positif du milieu objet.

\begin{theorem}[Relation de conjugaison des miroirs]
 \(\frac{1}{\bar{Sa_i}} - \frac{1}{\bar{SA_o}} = \frac{-2}{\bar{SC}} = V\) avec le sens positif de Sai dans le sens opposé de SAo et de SC
\end{theorem}
Le grandissement vaut \(\frac{\bar{Sa_i}}{\bar{SA_o}}\) mais avec des sens différents.
\begin{theorem}[Foyers objet/image]
 \(\bar{SF_o} = \frac{\bar{SC}}{2}\). Il se trouve au milieu de SC. Les foyers objets/images sont au m\^eme endroit mais non confondus car ils gardent leur r\^ole respectif. La distance focale vaut \(f_i  = \bar{SF_i} = 1/V\) et la distance focale objet  vaut \(f_o  = \bar{SF_i} = -1/V\)
\end{theorem}
\warningInfo{Convergence et convexité}{Un miroir convexe est divergent et un miroir concave est convergent.}
\subsection{Construction}
On se sert des m\^emes règles de constrcution que pour les dioptres
\end{document}

