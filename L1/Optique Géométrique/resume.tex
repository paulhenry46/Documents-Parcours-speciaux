 % !TeX spellcheck = en_US
\documentclass[french]{yLectureNote}

\title{Optique Géométrique}
\subtitle{Niveau 1}
\author{Paulhenry Saux}
\date{\today}
\yLanguage{Français}

\professor{C.Gatel}%christophe.gatel@cemes.fr : section B sillon 7

\usepackage{graphicx}%----pour mettre des images
\usepackage[utf8]{inputenc}%---encodage
\usepackage{geometry}%---pour modifier les tailles et mettre a4paper
%\usepackage{awesomebox}%---pour les boites d'exercices, de pbq et de croquis ---d\'esactiv\'e pour les TP de PC
\usepackage{tikz}%---pour deiffner + d\'ependance de chemfig
\usepackage{tkz-tab}
\usepackage{chemfig}%---pour deiffner formules chimiques
\usepackage{chemformula}%---pour les formules chimiques en \'equation : \ch{...}
\usepackage{tabularx}%---pour dimensionner automatiquement les tableaux avec variable X
\usepackage{awesomebox}%---Pour les boites info, danger et autres
\usepackage{menukeys}%---Pour deiffner les touches de Calculatrice
\usepackage{fancyhdr}%---pour les en-t\^ete personnalis\'ees
\usepackage{blindtext}%---pour les liens
\usepackage{hyperref}%---pour les liens (\`a mettre en dernier)
\usepackage{caption}%---pour la francisation de la l\'egende table vers Tableau
\usepackage{pifont}
\usepackage{array}%---pour les tableaux
\usepackage{lipsum}
\usepackage{yFlatTable}
\usepackage{multicol}
\newcommand{\Lim}[1]{\lim\limits_{\substack{#1}}\:}
\renewcommand{\vec}{\overrightarrow}
\begin{document}
%\titleOne

\setcounter{chapter}{7}
\chapter{Résumé - Formulaire}
\section{Relations de base}
\begin{theorem}[Relations de conjugaison]
 \begin{itemize}
  \item Dioptre : \(\displaystyle \frac{n_i}{\bar{SA_i}}-\frac{n_0}{\bar{SA_o}} = \frac{n_i-n_o}{\bar{SC}} = V\)
  \item Lentille : \(\displaystyle \frac{1}{\bar{OA_i}}-\frac{1}{\bar{OA_o}}  = V\)
  \item Miroir sphérique : \(\displaystyle \frac{1}{\bar{SA_i}}-\frac{1}{\bar{SA_o}}  = \frac{-2}{\bar{SC}} = V\)
 \end{itemize}
\end{theorem}
\begin{definition}[Convexité/concavité]
Un dioptre est concave si $\bar{SC}<0$ et convexe $\bar{SC}>0$
\end{definition}
\begin{definition}[Convergence/divergence]
Un dioptre est convergent si $V>0$ et divergent $V<0$
\end{definition}
\begin{theorem}[Distances focales]
 \begin{itemize}
  \item Dioptre sphérique : \(\displaystyle\bar{SF_i} = \frac{n_i}{V}\) et \(\displaystyle\bar{SF_o} = -\frac{n_o}{V}\)
  \item Conséquence : Lentille et Miroir : \(\displaystyle\bar{SF_i} = \frac{1}{V}\) et \(\displaystyle\bar{SF_o} = -\frac{1}{V}\)
 \end{itemize}
\end{theorem}
\begin{theorem}[Grandissement transversal]
 \begin{itemize}
  \item Dioptre sphérique : \(\displaystyle G_t = \frac{n_0 \bar{SA_i}}{n_i \bar{SA_o}}\)
  \item Conséquence : Lentille  et miroir : \(\displaystyle G_t = \frac{ \bar{SA_i}}{\bar{SA_o}}\)
 \end{itemize}

\end{theorem}
\criticalInfo{Grandissement angulaire Ga / Grossissement Gr}{
\begin{itemize}
 \item \(\displaystyle G_a = \frac{\alpha_i}{\alpha_o}\) avec $\alpha_i$ et $\alpha_o$ les angles de sortie et d'entré des rayons dans le système optique.
 \item \(\displaystyle G_r = \frac{\alpha_i}{\alpha}\) avec $\alpha_i$ l' angle de sortie du système optique des rayons et $\alpha$ l'angle avec lequel on voit l'objet lorsque celui-ci est placé au PP, soit 25 cm.
\end{itemize}
}
\begin{definition}[Puissance]
C'est l'angle de sortie du système optique $\alpha_i$ sur la taille de l'objet : \(\displaystyle P = \frac{\alpha_i}{\bar{A_oB_o}}\)
\end{definition}
\section{Lentilles}
\criticalInfo{Combinaison impossibles}{
\begin{itemize}
 \item Lentille Convergente : Objet virtuel / Image virtuelle
 \item Lentille Divergente : Objet réel / Image réelle
\end{itemize}
}
\section{Oeil}
\begin{definition}[Punctun proximum (PP) / Punctun remotum (PR)]
Le PP est la distance minimale de vision nette lorsque l'oeil accomode au maximum et le PR est la distance maximale de vision sans accomodation
\end{definition}
\begin{definition}[Oeil emmétrope]
Un oeil emmétrope est un oeil qui respecte les conditions de PP  de 25 cm et de PR à l'infini.
\end{definition}
\begin{definition}[Myopie]
Le PR n'est plus l'infini. L'oeil est déjà trop convergent mais le PP se rapproche.
\end{definition}
\begin{definition}[Hypermetropie]
L'oeil n'est pas assez convergent. Au bout d'un moment, l'oeil n'arrive plus à accomoder : Le PP s'éloigne.
\end{definition}
\begin{definition}[Presbycie]
C'est la diminution de l'amplitude d'accomodation.
\end{definition}
\section{Miroir}
\criticalInfo{Sens algébrique}{
\begin{itemize}
                                \item On change le sens positif de parcours selon si le rayon appartient à
l'objet ou à l' image.
\item le sens positif de $\bar{SC}$ est celui du milieu objet
                               \end{itemize}
}
\warningInfo{Convergence et convexité}{Contrairement aux lentilles, un miroir convexe est divergent et un miroir concave est convergent.}
\end{document}

