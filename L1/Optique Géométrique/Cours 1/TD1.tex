 % !TeX spellcheck = en_US
\documentclass[french]{yLectureNote}

\title{Optique Géométrique}
\subtitle{Niveau 1}
\author{Paulhenry Saux}
\date{\today}
\yLanguage{Français}

\professor{C.Gatel}%christophe.gatel@cemes.fr : section B sillon 7

\usepackage{graphicx}%----pour mettre des images
\usepackage[utf8]{inputenc}%---encodage
\usepackage{geometry}%---pour modifier les tailles et mettre a4paper
%\usepackage{awesomebox}%---pour les boites d'exercices, de pbq et de croquis ---d\'esactiv\'e pour les TP de PC
\usepackage{tikz}%---pour deiffner + d\'ependance de chemfig
\usepackage{tkz-tab}
\usepackage{chemfig}%---pour deiffner formules chimiques
\usepackage{chemformula}%---pour les formules chimiques en \'equation : \ch{...}
\usepackage{tabularx}%---pour dimensionner automatiquement les tableaux avec variable X
\usepackage{awesomebox}%---Pour les boites info, danger et autres
\usepackage{menukeys}%---Pour deiffner les touches de Calculatrice
\usepackage{fancyhdr}%---pour les en-t\^ete personnalis\'ees
\usepackage{blindtext}%---pour les liens
\usepackage{hyperref}%---pour les liens (\`a mettre en dernier)
\usepackage{caption}%---pour la francisation de la l\'egende table vers Tableau
\usepackage{pifont}
\usepackage{array}%---pour les tableaux
\usepackage{lipsum}
\usepackage{yFlatTable}
\usepackage{multicol}
\newcommand{\Lim}[1]{\lim\limits_{\substack{#1}}\:}
\renewcommand{\vec}{\overrightarrow}
\begin{document}
%\titleOne

	\chapter{Introduction et notion de rayons lumineux}
	\section{Introduction}
	Compas à apporter !
	19/02 éval 1
	7/03 à 15h45 eval 2
	28/03 à 15h45 eval 3
	\subsection{Théories}
	\begin{itemize}
	 \item Physique quantique <- électromagnétisuqme <- optique ondulatoire <- optique (= science de la lumière) géométrique.

	\item Optique géométrique : nait naturellement, issue des observations : existe seule jusqu'au 17e

	\item Optique ondulatoire : phénomènes de diffraction, interférences, importante au 19e

	\item Électromégnétisme : On a mesuré la vitesse de la lumière, production de champ magnétique

	\item Physique quantique : Spectre de la lumière, répartission des ondes en fonction de leur énergie, explication de l'effet photoélectrique.

	\end{itemize}
\subsection{Nature de la lumière}
Grande évolution : jet de Dieu -> grain de couleur -> particule

Def électromagnétique : partie du spectre électromagnétique visible par l'oeil. Caractérisé par 2 champ vectoriels

\subsubsection{Grandeurs}
Amplitude A, longueur d'onde $\lambda$, période T, vitesse $v = c$ pour la lumière : $3\cdot 10^{8}$ m.s

Dans le vide : $c = \lambda f$, quelque soit les conditions.

Spectre électromagnétique visible : 750 nm (rouge) à 400 nm (violet)

Matérieux transparent à certains longueurs d'onde.

Première limitation de l'optique géométrique : spectre visible + travail avec des objets et images très supérieures à la longueur d'onde (pas de virus, bactéries, vers quelques centaines de micrométres)
	\section{Approximations}
	\subsection{1er principe : Ligne droite}
	La lumière se propage en ligne droite
	\subsection{Faisceaux}
	Elle se présente sous la forme d'un faisecau (= un ensemble de rayon). Un point de lumière envoie des rayons dans toutes les directions.
	\subsection{Théorème de Malus}
	Faisceaux convergents, divergents, parallèles (source à l'infini)

	La surface d'onde est une surface qui en tout poitn est perpendiuclaire au rayon.
	\subsection{2e principe}
	Les rayons lumineux sont rectilignes et indépendants les uns des autres.
	(faux car rayon rouge + bleu = violet)
	\subsection{3e principe}
	Principe du retour de inversse de la lumière : Le trajet suivi est indépendant du sens de propagation.
	\subsection{Chemin optique et principe de fermat}
	Indice de réfraction (sans dimension, sans unité) : \[n=\frac{c}{c_{milieu}}= \frac{c}[v]\] Toujours >1 car on ne peut pas aller plus vite que la vitesse de lumière. Plus l'indice optique est important, plus la vitesse dans le mileu est faible.
	Milieu ou l'indice varie beaucyoup = dispersif.
	Exemple : Dans l'eau : $n=1.33$, donc la lumière va trois fois moins vite.

	Dioptre : Surface de spération entre 2 milieux d'indice optique différents. 2 milieux différents mais avec le meme indice n'a pas de dioptre.

	Loi de Cauchy : \[n(\lambda) = A + \frac{B}{\lambda^2}\]

	\subsection{Chemin optique}
	Lie la distance parocurie dans le milieu et la vitesse à laquelle je la parcoure. Entre 2 points A et B, c'est entre crochet : $[AB] = \frac{c}{v} \times AB = c\times \Delta t = n \times AB$. Le chemin optique total est la somme de tous les chemins optiques. Il représente la distance qu'aurai parcourie le rayon pendant $\Delta t$ à la vitesse maximale, dans le vide.
	\subsection{Chemin optique}
	Principe de moindre action : Les processus de phénomènes se fait en minimant l'énergie pour aller vers un état de plus faible énergie. On cherche a minimiser le temps (et non la distance) durant lequel le rayon se ballade.

	Principe de Fermat : La lumière se ballade d etelle manière qu'elle va minimiser le chemin optique. Donne lieu à la réfraction et la réflexion
	\chapter{Rélfexion et réfraction}
	\subsection{réfexion}
	Premier milieu = milieu incident, deuxième milieu de l'autre coté, si on ne peut pas y rentrer, on le hachure

	Direction de référence : normale au miroir au point d'arrivée du rayon incident sur la surface. La lumière arrive en faisant un ange défini par rapport à la normale. ($0<\alpha<\frac{\pi}{2}$)

	Plan d'incidence : contient la normale et le rayon incident, perpendiculaire à la surface.

	Loi de la réflexion : le rayon est réflchi dans le m\^eme pan d'incidence et repart avec le m\^eme angle par rapport à la normale à l'angle d'incidence.

	Démonstration

	\subsection{Réfraction}
	On considère un dioptre séparant 2 milieux transparents (incidnets et réfractant). On a $n_1\sin(i_1) = n_2 \sin(i_2)$.

	Le rayon réfracté reste dans le plan d'incidence.

	Démonstration

	Une loi d'équilibre : un sinus entre 0 et pi/2 est croissant et positif. Si n auglente, l'angle diminue. Vient de la minimisation du trajet de la lumière.

	\subsubsection{Analyse}
	Si $n_2 > n_1$ Le milieu 2 est plus réfreingent et le rayon réfracté se rapproche de la normale.

	Cas extrèmes : $i_1 = 0 \to i_2 = 0$ et $i_1 = pi/2 \to i_2 = \arcsin(\frac{n_1}{n_2})$.

	Dans le cas contraire : Le rayon réfracté s'éloigne de la normale.

	Si $i_2 = \pi/2$, $i_{lim} = \arcsin(\frac{n_2}{n_1})$ Si $i_1 > i_{1,lim}$ : réflexion totale.

	\subsection{Construction de rayon réfracté par les surfaces d'indice}
	\subsubsection{Si $n_1 > n_2$}
	le rayon vaut $k\times n_1$ et $k\times n_2$ \[r_2 = \frac{n_2}{n_1} r_1\]

	Réfraction limite et réflexion totale.

	Variation linéaire (grandient) = trajectoire parabolique de la lumière.

	Mirage froid sur les objets avec capacité calorifique.
	\subsection{Le prisme}


\end{document}

