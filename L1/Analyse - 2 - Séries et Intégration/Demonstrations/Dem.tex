% !TeX spellcheck = en_US
\documentclass[french]{yLectureNote}

\title{Séries - Analyse 2}
\subtitle{Chimie}
\author{Paulhenry Saux}
\date{\today}
\yLanguage{Français}

\professor{IHallery}%isabelle.hallery@univ-tlse3.fr
\usepackage{graphicx}%----pour mettre des images
\usepackage[utf8]{inputenc}%---encodage
\usepackage{geometry}%---pour modifier les tailles et mettre a4paper
%\usepackage{awesomebox}%---pour les boites d'exercices, de pbq et de croquis ---d\'esactiv\'e pour les TP de PC
\usepackage{tikz}%---pour deiffner + d\'ependance de chemfig
\usepackage{tkz-tab}
\usepackage{chemfig}%---pour deiffner formules chimiques
\usepackage{chemformula}%---pour les formules chimiques en \'equation : \ch{...}
\usepackage{tabularx}%---pour dimensionner automatiquement les tableaux avec variable X
\usepackage{awesomebox}%---Pour les boites info, danger et autres
\usepackage{menukeys}%---Pour deiffner les touches de Calculatrice
\usepackage{fancyhdr}%---pour les en-t\^ete personnalis\'ees
\usepackage{blindtext}%---pour les liens
\usepackage{hyperref}%---pour les liens (\`a mettre en dernier)
\usepackage{caption}%---pour la francisation de la l\'egende table vers Tableau
\usepackage{pifont}
\usepackage{array}%---pour les tableaux
\usepackage{lipsum}
\usepackage{yFlatTable}
\usepackage{multicol}
\newcommand{\Lim}[1]{\lim\limits_{\substack{#1}}\:}
\renewcommand{\vec}{\overrightarrow}
\newcommand{\N}[0]{\mathbb{N}}
\begin{document}
\setcounter{chapter}{1}
\chapter{Démonstrations }
\begin{theorem}[Théorème de comparaison]
Soient $\sum u_k$ et $\sum v_k$ deux séries à termes positifs ou nuls. On
suppose qu'il existe $k_0\ge 0$ tel que, pour tout $k\ge k_0$,
$u_k \le v_k$.
\begin{itemize}
\item Si $\sum v_k$ converge alors $\sum u_k$ converge.
\item Si $\sum u_k$ diverge alors $\sum v_k$ diverge.
\end{itemize}
\end{theorem}
\begin{myproof}
La convergence ne dépendant pas des
premiers termes, on peut donc supposer $k_0=0$.


\begin{itemize}
\item Notons $S_n=u_0+\dots+u_n$ et $S'_n = v_0+\dots+v_n$.
Les suites $(S_n)$ et $(S'_n)$ sont croissantes, et de plus, pour tout $n \ge 0$,
$S_n\le S'_n$.
\item Si la série $\sum v_k$ converge, alors la suite
$(S'_n)$ converge. Soit $S'$ sa limite.
 \item La suite $(S_n)$ est croissante
et majorée par $S'$, donc elle converge, et ainsi la série $\sum u_k$
converge aussi.
 \item Inversement, si la série $\sum u_k$ diverge, alors
la suite $(S_n)$ tend vers $+\infty$, et il en est de même pour la
suite $(S'_n)$ et ainsi la série $\sum v_k$ diverge.
\end{itemize}
\end{myproof}
\begin{theorem}[Théorème des équivalents]
Soient $(u_k)$ et $(v_k)$ deux suites à termes strictement positifs.
Si $u_k \sim v_k$ alors les séries $\sum u_k$ et $\sum v_k$ sont de même nature.
\end{theorem}

\begin{myproof}
Comme les 2 suites sont équivalents, la limite de leur quotient vaut 1. Donc pour tout $\epsilon>0$, il existe $k_0$ tel que, pour
tout $k \ge k_0$,
$$\left|\frac{u_k}{v_k} -1\right| < \epsilon,$$
ou autrement dit
$$(1-\epsilon)v_k < u_k <(1+\epsilon) v_k.$$

Fixons un $\epsilon <1$.

Si $\sum u_k$ converge, alors par le théorème de comparaison,
$\sum(1-\epsilon) v_k$ converge, donc $\sum v_k$ également.

Réciproquement, si $\sum u_k$ diverge, alors
$\sum (1+\epsilon)v_k$ diverge, et $\sum v_k$ aussi.
\end{myproof}
\begin{theorem}
Toute série absolument convergente est convergente.
\end{theorem}

\begin{myproof}
Utilisons le critère de Cauchy.  Soit $\sum u_k$ une série absolument convergente.

La série $\sum |u_k|$ est convergente, donc la suite des sommes partielles $(S'_n)$ avec
$S'_n = \sum_{k=0}^{n} |u_k|$ est une suite
convergente, donc de Cauchy.

Soit $\epsilon>0$ fixé. Il existe donc $n_0 \in \N$ tel que
pour tout $n \geq n_0$ et pour tout $p \geq 0$ :
$$S'_{n+p}-S'_n=|u_n|+|u_{n+1}|+\dots+|u_{n+p}| < \epsilon.$$

Par suite, pour $n \geq n_0$ et  $p \geq 0$ on a par l'inégalité triangulaire :
$$\big|u_n+u_{n+1}+\dots+u_{n+p}\big| \leq |u_n|+|u_{n+1}|+\cdots+|u_{n+p}| < \epsilon.$$
Donc, d'après le critère de Cauchy,  $\sum u_k$ est convergente.
\end{myproof}

\begin{theorem}[Règle du quotient de d'Alembert]
Soit $\sum u_k$ une série à termes strictement positifs telle que \(\big(\frac{u_{k+1}}{u_{k}}\big)\) converge vers \(l\)
\begin{enumerate}
\item Si \(l < 1\), la série converge.
\item Si \(l > 1\), la série diverge.
\item Si \(l = 1\), on ne peut conclure.
\end{enumerate}
\end{theorem}
\begin{myproof}
Rappelons tout d'abord que la série géométrique $\sum q^k$
converge si $|q|<1$, diverge sinon.
\begin{itemize}
 \item Dans le premier cas du théorème, soit un réel q tel que \(l<q<1\). On a $\left|\frac{u_{k+1}}{u_k}\right| \leq q$ à partir d'un certain rang N, et donc \(u_{k+1}\leq u_kq\). Par récurrence, on obtient que \[u_k \leq u_{k-(k-N)}q^{k-N}=u_Nq^{-N}q^k=cq^k\], avec c constant.

Comme $0 < q < 1$, alors la série $\sum q^k$ converge, et,
par le théorème de comparaison :
la série $\sum u_k$ converge.
 \item Si $\left|\frac{u_{k+1}}{u_k}\right| > 1$, la suite $(|u_k|)$ est croissante : elle ne
peut donc pas tendre vers $0$ et la série diverge.
\end{itemize}
\end{myproof}
\begin{theorem}[Règle des racines de Cauchy]
 Soit une série à termes strictement positifs. Si il existe, on note \(l = \lim \sqrt[n]{u_n}\).
 \begin{itemize}
  \item  Si \(l<1\), la série converge
  \item Si \(l>1\), la série diverge.
 \end{itemize}
\end{theorem}
\begin{myproof}
Rappelons que la nature de la série ne dépend pas de ses premiers
termes. On peut ainsi trouver trouver un certaing rang $N$ à partir duquel les assertions suivantes sont vérifiées.

Dans le premier cas, soit un réel q tel que \(l<q<1\). On a $\sqrt[k]{|u_k|} \le q$
implique  $|u_k| \le q^k$. Comme $0<q<1$, alors $\sum q^k$ converge, donc la série aussi
par le théorème  de comparaison.

Dans le second cas, $\sqrt[k]{|u_k|} > 1$, donc $|u_k| > 1$.
Le terme général ne tend pas vers $0$, donc la série diverge.

% Enfin pour le dernier point du corollaire, on pose
% $u_k=\frac{1}{k}$, $v_k=\frac{1}{k^2}$.
% On a $\sqrt[k]{u_k}\to 1$ de même que $\sqrt[k]{v_k}\to 1$.
% Mais $\sum u_k$ diverge alors que $\sum v_k$ converge.
\end{myproof}
\begin{theorem}[Critère de Leibniz]
Supposons que $(u_k)_{k\ge0}$ soit une suite qui vérifie :
\begin{enumerate}
  \item $u_k  \ge 0$ pour tout $k \ge 0$,
  \item la suite $(u_k)$ est une suite décroissante,
  \item et $\Lim{k\to+\infty} u_k=0$.
\end{enumerate}
Alors la série alternée $\displaystyle \sum_{k=0}^{+\infty} (-1)^k u_k$ converge.

De plus, Soit $S$ la somme de cette série et soit $(S_n)$ la suite des sommes partielles.
\begin{enumerate}
  \item La somme $S$ vérifie les encadrements :
  $$ S_1\le S_5\le \cdots \le S_{2n+1} \le \cdots \le S
  \le  \cdots\le S_{2n} \le \cdots\le S_4\le S_0.$$
  \item En plus, si $\displaystyle R_n=S-S_n =\sum_{k=n+1}^{+\infty} (-1)^k u_k$ est le reste d'ordre $n$, alors on  a
$$\big|R_n\big|\le u_{n+1}.$$
\end{enumerate}
\end{theorem}

\begin{myproof}
Nous allons nous ramener à deux suites adjacentes.
\begin{itemize}
  \item La suite $(S_{2n+1})$ est croissante car
  $S_{2n+1}-S_{2n-1}=u_{2n}-u_{2n+1}\ge 0$.

  \item La suite $(S_{2n})$ est décroissante car
  $S_{2n}-S_{2n-2}= u_{2n}-u_{2n-1}\le 0$.

%   \item $S_{2n} \ge S_{2n+1}$ car
%   $S_{2n+1} - S_{2n} = -u_{2n+1} \le 0$.

  \item Enfin $S_{2n+1} - S_{2n}$ tend vers $0$
  car $S_{2n+1} - S_{2n} = -u_{2n+1} \to 0$
  (lorsque $n\to+\infty$).
\end{itemize}
En conséquence $(S_{2n+1})$ et $(S_{2n})$ convergent vers la même limite $S$.
Donc $(S_n)$ converge vers $S$.
\bigskip

De plus, comme les suites \(S_{2n+1}\) et \(S_{2n}\) sont adjacentes,  $S_{2n+1} \le S \le S_{2n}$ pour tout $n$.

Enfin on a aussi
 $$0\ge R_{2n}= S-S_{2n} \ge S_{2n+1}-S_{2n}=-u_{2n+1}$$ pour n pair et, pour n impair
 $$0\le R_{2n+1}= S-S_{2n+1} \le  S_{2n+2} -S_{2n+1}= u_{2n+2}.$$

 Ainsi, quelle que soit la parité de $n$, on a
 $|R_n|=|S-S_n|\le  u_{n+1}$.
 \end{myproof}
\end{document}

