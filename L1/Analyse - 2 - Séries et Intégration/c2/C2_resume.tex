% !TeX spellcheck = en_US
\documentclass[french]{yLectureNote}

\title{Mathématiques - Analyse 2}
\subtitle{Analyse}
\author{Paulhenry Saux}
\date{\today}
\yLanguage{Français}

\professor{C.Dartyge}

\usepackage{graphicx}%----pour mettre des images
\usepackage[utf8]{inputenc}%---encodage
\usepackage{geometry}%---pour modifier les tailles et mettre a4paper
%\usepackage{awesomebox}%---pour les boites d'exercices, de pbq et de croquis ---d\'esactiv\'e pour les TP de PC
\usepackage{tikz}%---pour deiffner + d\'ependance de chemfig
\usepackage{tkz-tab}
\usepackage{chemfig}%---pour deiffner formules chimiques
\usepackage{chemformula}%---pour les formules chimiques en \'equation : \ch{...}
\usepackage{tabularx}%---pour dimensionner automatiquement les tableaux avec variable X
\usepackage{awesomebox}%---Pour les boites info, danger et autres
\usepackage{menukeys}%---Pour deiffner les touches de Calculatrice
\usepackage{fancyhdr}%---pour les en-t\^ete personnalis\'ees
\usepackage{blindtext}%---pour les liens
\usepackage{hyperref}%---pour les liens (\`a mettre en dernier)
\usepackage{caption}%---pour la francisation de la l\'egende table vers Tableau
\usepackage{pifont}
\usepackage{array}%---pour les tableaux
\usepackage{lipsum}
\usepackage{yFlatTable}
\usepackage{multicol}
\newcommand{\Lim}[1]{\lim\limits_{\substack{#1}}\:}
\renewcommand{\vec}{\overrightarrow}
\begin{document}

\setcounter{chapter}{2}
	\chapter{Compléments sur les suites et fonctions }
\section{Théorème de Bolzano-Weierstrass}
\subsection{Suites}
\begin{proposition}[Caractérisation de la valeur d'adhérence]
l est une valeur d'adhérence de la suite \(\iff \forall \varepsilon >0,\forall m\in\mathbb{N},\exists n\geq m, |u_n-l|\leq \varepsilon\).
\end{proposition}
\begin{proposition}
Soit \(u_n\) une suite bornée avec une unique valeur d'adhérence. Alors cette suite converge vers cette valeur d'adhérence
\end{proposition}
\section{Fonctions continues sur un segment}
\subsection{Continuité}
\begin{definition}[Continuité]
une fonction est continue si \(\forall x\in I,\forall \varepsilon>0,\exists \delta>0,\forall y\in I, (|x-y|)<\delta \Rightarrow (|f(x)-f(y)|)<\varepsilon\)
\end{definition}
\begin{proposition}[Critère séquentiel de la continuité]
f est continue en un point \(x\in I \iff \forall (x_n)_n \subset I\) qui converge vers \(x\), \((f(x_n))_n\) converge vers \(f(x)\).
\end{proposition}
\begin{proposition}
Toute fonction continue sur un segment \(I=[a,b]\) avec \(a,b\in\mathbb{R}, a<b\) est bornée et atteint ses bornes sur I.
\end{proposition}
\begin{theorem}
 L'image d'un segment  (intervalle borné et fermé) par une fonction continue est un segment.
\end{theorem}
\subsection{Continuité uniforme}
\begin{definition}
Une fonction est uniformément continue sur I \(\iff \forall \varepsilon>0,\exists \delta >0, \forall x\in I, \forall y\in I, (|x-y|)<\delta \Rightarrow (|f(x)-f(y)|)<\varepsilon \)
\end{definition}
\begin{definition}
Une fonction  est lipschitzienne \(\iff \exists k\in\mathbb{R}⁺,\forall x,y\in I |f(x)-f(y)| \leq k|x-y|\)
\end{definition}
\begin{theorem}
 Toute fonction lipschitzienne sur I est uniformément continue sur I.
\end{theorem}
\begin{theorem}[Théorème de Heine]
 Toute fonction continue sur un segment I est uniformément continue sur I.
\end{theorem}
\section{Suites de Cauchy}
\begin{definition}
Une suite est de cachy si \(\forall \varepsilon>0,\exists N\in\mathbb{N},\forall p,q \in\mathbb{N}, (p\geq N,q\geq N \Rightarrow |u_p-u_q|\leq \varepsilon)\)
\end{definition}
\begin{proposition}
\begin{itemize}
 \item Toute suite convergente est de Cauchy
 \item Toute suite de Cauchy est bornée
 \item Une suite de Cauchy ademttant une valeur d'adhérence est convergente.
\end{itemize}
\begin{theorem}
 Toute suite de Cauchy converge
\end{theorem}
\end{proposition}
\end{document}

