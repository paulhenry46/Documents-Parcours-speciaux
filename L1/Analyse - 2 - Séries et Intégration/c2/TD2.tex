% !TeX spellcheck = en_US
\documentclass[french]{yLectureNote}

\title{Mathématiques - Analyse 2}
\subtitle{Analyse}
\author{Paulhenry Saux}
\date{\today}
\yLanguage{Français}

\professor{C.Dartyge}

\usepackage{graphicx}%----pour mettre des images
\usepackage[utf8]{inputenc}%---encodage
\usepackage{geometry}%---pour modifier les tailles et mettre a4paper
%\usepackage{awesomebox}%---pour les boites d'exercices, de pbq et de croquis ---d\'esactiv\'e pour les TP de PC
\usepackage{tikz}%---pour deiffner + d\'ependance de chemfig
\usepackage{tkz-tab}
\usepackage{chemfig}%---pour deiffner formules chimiques
\usepackage{chemformula}%---pour les formules chimiques en \'equation : \ch{...}
\usepackage{tabularx}%---pour dimensionner automatiquement les tableaux avec variable X
\usepackage{awesomebox}%---Pour les boites info, danger et autres
\usepackage{menukeys}%---Pour deiffner les touches de Calculatrice
\usepackage{fancyhdr}%---pour les en-t\^ete personnalis\'ees
\usepackage{blindtext}%---pour les liens
\usepackage{hyperref}%---pour les liens (\`a mettre en dernier)
\usepackage{caption}%---pour la francisation de la l\'egende table vers Tableau
\usepackage{pifont}
\usepackage{array}%---pour les tableaux
\usepackage{lipsum}
\usepackage{yFlatTable}
\usepackage{multicol}
\newcommand{\Lim}[1]{\lim\limits_{\substack{#1}}\:}
\renewcommand{\vec}{\overrightarrow}
\newcommand{\R}[0]{\mathbb{R}}
\begin{document}


	\chapter{Compléments sur les fonctions }
\section{Théorème de Bolzano-Weiestrass}
\begin{theorem}[Théorème de BW]
 Toute suite bornée admet une sous-suit convergente
\end{theorem}
\begin{myproof}
$U_n$ est une suite réelle bornée. Comme elle est bornée, il existe deux réels $a$ et $b$ avec $a<b$ tel que $\forall n\in\mathbb{N}, u_n \in[a,b]$. On pose $a_0 = a, b_0=b, \varphi(0) = 0, 8_0 = [a,b]$.

L'un, au moins, des deux intervalles $[a,\frac{a+b}{2}]$ et $[\frac{a+b}{2},b]$ contient une infinité de $n$ pour lesquels $u_n$ est dans cet intervalle. On note $I_1$ cet intervalle et $I_1 = [a_1,b_1]$. On a donc $a_1 = a_0, b_1 = \frac{a_0+b_0}{2}$ ou $a_1 = \frac{a_0+b_0}{2}, b_1=b_0$.

On prend pour $\varphi(1)$ un indice dans $I_1$ et $\varphi(1)$ plus grand que $\varphi(0)+1$.

On suppose alors construit $a_0 \leq a_1\leq a_n$. On a $b_n-a_n = \frac{a_0-b_0}{2^n}$

puis $\varphi(0)<\varphi(1)<\varphi(\dots)<\varphi(n)$.

L'intervalle $I_n = [a_n,b_n]$ contient une infinité de termes dans l'une des 2 moitiées. On répète le processus.

On construit ainsi une suite qui respecte les conditions, $u_{\varphi_n} \in I_n$. Les suites $a_n$ et $b_n$ sont adjacentes, donc convergent vers la m\^eme limite. Par l'encadrement, $u_{\varphi(n)}$ converge vers cette limite.
\end{myproof}
\section{Continuité Uniforme}
\begin{definition}
f est continue en $x$ si $\forall \varepsilon,\exists \delta >0, \forall y\in I (|y-x|)< \delta \Rightarrow |f(x)-f(y)| <\varepsilon$
\end{definition}
Exemple : $f(x) = 2x+3$ est continue sur $\R$. Soit $x\in \R, \varepsilon>0$. Posons $\delta = \frac{\varepsilon}{2}$

Soit $y\in\R$,

Si $|y-x| <\delta, |y-x| < \varepsilon/2$ donc $|2y-2x| <\varepsilon$ et $|2y+3-(2x+3)| <\varepsilon$.

On peut choisit $\delta$ en fonction de $\varepsilon$ uniquement et non $x$.

\begin{lemma}
Si il existe deux suites  $(x_n), (y_n)$ à valeurs dans $I$ telle que $x_n-y_n \to 0 \wedge f(x_n)-f(y_n) not \to 0$
\end{lemma}

Exemple pour $x^2$ : Posons $x_n = n$ et $y_n = n+\frac{1}{n}$. On a $x_n-y_n \to$ et $x_n^2-y_n^2 = -2-\frac{1}{n^2} \to -2$.

Donc $f$ n'est pas uniformément continue.
\begin{theorem}[Théorème de Heine]
 Toute fonction continue sur un segment $I$ est uniformément continue sur $I$.
\end{theorem}
\begin{definition}
Une fonction définie sur un intervalle I est lipshitienne si et seulement si \(\exists k\in\R,\forall x,y\in I, |f(x)-f(y)|\leq k|x-y|\)
\end{definition}
\begin{theorem}[]
 Une fonction lipschienne sur $I$ est uniformément continue sur $\R$
\end{theorem}
\begin{myproof}
On prend $\delta = \varepsilon/k$.
\end{myproof}
Remarque : Si $f$ est $C^1$ sur $I$ avec $f'$ bornée alors f est lipscheinne
\begin{myproof}
D'après le TAF, on a $|f(y)-f(x)| \leq \sup(f'(c)) |y-x|$.
\end{myproof}
Remarque : Soit $f C^1$ sur un segment. Alors elle est lipscheinne car $f'$ est continue sur $[a,b]$ donc bornée, puis par la première remarque.

$C^1$ dérivée bornée $\Rightarrow$ Lipshei $\Rightarrow$ uniformément continue $\Rightarrow$ continue

Mais continue n'implique pas uniformément continue ($x^2$), mais uniformément continue n'implique lipschiz (racine de x)
\section{Suites de Cauchy}
\begin{definition}
Une suite est de cachy si \(\forall \varepsilon>0,\exists N\in\mathbb{N},\forall p,q \in\mathbb{N}, (p\geq N,q\geq N \Rightarrow |u_p-u_q|\leq \varepsilon)\)
\end{definition}
\begin{lemma}
Une suite de Cauchy est bornée
\end{lemma}
On prend $\varepsilon = 1$ et on a $|u_p|=|U_p-U_n+U_n|\leq |U_p-U_n|+|U_n|\leq 1 + |U_n|$. On pose $M = \max(|U_0|\dots |U_n|,|U_n|+1)$. On a $|U_n|\leq M$
\begin{lemma}
Une suite de Cauchy avec une valeur d'adhérence converge.
\end{lemma}
\begin{theorem}
 Toute suite de Cauchy rélle ou complexe est convergente.
\end{theorem}
Cauchy ->lemme 1 -> bornée ->BW -> possède une VA -> lemme 2 ->VC.

Exemple : m'algorithme de Héron est une suite de rationnels qui converge vers $\sqrt{2}$ mais ne converge pas dans $Q$.
\warningInfo{Limite de 0}{$U_p-u_n \to 0$ n'implique pas que la suite est de cauchy. la suite ln diverge vers l'infini mais (donc pas de chauchy) mais respecte la condition précédente.}
\end{document}

