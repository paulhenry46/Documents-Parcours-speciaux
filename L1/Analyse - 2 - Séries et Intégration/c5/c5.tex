% !TeX spellcheck = en_US
\documentclass[french]{yLectureNote}

\title{Mathématiques - Analyse 2}
\subtitle{Analyse}
\author{Paulhenry Saux}
\date{\today}
\yLanguage{Français}

\professor{C.Dartyge}

\usepackage{graphicx}%----pour mettre des images
\usepackage[utf8]{inputenc}%---encodage
\usepackage{geometry}%---pour modifier les tailles et mettre a4paper
%\usepackage{awesomebox}%---pour les boites d'exercices, de pbq et de croquis ---d\'esactiv\'e pour les TP de PC
\usepackage{tikz}%---pour deiffner + d\'ependance de chemfig
\usepackage{tkz-tab}
\usepackage{chemfig}%---pour deiffner formules chimiques
\usepackage{chemformula}%---pour les formules chimiques en \'equation : \ch{...}
\usepackage{tabularx}%---pour dimensionner automatiquement les tableaux avec variable X
\usepackage{awesomebox}%---Pour les boites info, danger et autres
\usepackage{menukeys}%---Pour deiffner les touches de Calculatrice
\usepackage{fancyhdr}%---pour les en-t\^ete personnalis\'ees
\usepackage{blindtext}%---pour les liens
\usepackage{hyperref}%---pour les liens (\`a mettre en dernier)
\usepackage{caption}%---pour la francisation de la l\'egende table vers Tableau
\usepackage{pifont}
\usepackage{array}%---pour les tableaux
\usepackage{lipsum}
\usepackage{yFlatTable}
\usepackage{multicol}
\newcommand{\Lim}[1]{\lim\limits_{\substack{#1}}\:}
\renewcommand{\vec}{\overrightarrow}
\newcommand{\R}[0]{\mathbb{R}}
\newcommand{\N}[0]{\mathbb{N}}
\newcommand{\dd}[0]{\mathrm{d}}
\begin{document}
\setcounter{chapter}{4}

	\chapter{Intégrales impropres}

Exemples de fonctions qui ne sont pas cpm :
\begin{itemize}
 \item \(x\to \frac{1}{x} \text{ si } x\neq 0, 0\) car pas de limite finie en 0
\end{itemize}
\section{Introduction}
\subsection{CPM sur un intervalle}
\begin{definition}
Une fonction est cpm sur I si pour tout segement appartenant à I, la restrcition de cette fonction à ce segment est cpm.
\end{definition}
\begin{proposition}
Si f est cpm, alors \(|f|\) l'est aussi.
\end{proposition}
\subsection{Convergence}
\begin{definition}
Si \(I = [a,+\infty]\), l'intégrale converge si la limite quand \(x\to \infty\). On a alors \[\int^{+\infty}_{a} f(t)\dd t = \Lim{x\to +\infty} \int^x_a f(t)\dd t\]

Si \(I = ]a,b]\), l'intégrale converge si la limite quand \(x\to a\). On a alors \[\int^{b}_{a} f(t)\dd t = \Lim{x\to +\infty} \int^x_a f(t)\dd t\]
\end{definition}
\subsection{Propriétés}
\begin{proposition}
\begin{itemize}
 \item Chasles : \(\int^{+\infty}_a = \in^{a'}_a + \int^{+\infty}_{a'}\)
 \item Linéarité : À ne pas utiliser si l'une des deux intégrales et divergente
 \item Positivité et relation d'ordre
 \item Intégration par parties : Seules les bornes changent
\end{itemize}
\end{proposition}
\begin{proposition}
On suppose que f admet une limite. Si sa limite n'est pas nulle, l'intégrale vers \(\infty\) diverge. La contraposée est aussi valable.
\end{proposition}
\section{Fonctions de signe constants}
\begin{proposition}
Si \(\int^{+\infty}_af\) est convergente \(\iff x\to \int^x_a\) est majorée
\end{proposition}
\begin{proposition}[Intégrale de Riemann en \(+\infty\)]
\(\int^{+\infty}_1\frac{1}{t^{\alpha}}\) converge si \(\alpha > 1\). Dans le cas contraire, elle diverge.
\end{proposition}
\begin{proposition}[Intégrale de Riemann en \(0\)]
\(\int^{1}_0\frac{1}{t^{\alpha}}\) converge si \(\alpha < 1\). Dans le cas contraire, elle diverge.
\end{proposition}
\begin{proposition}[Inétgrale de bertrand en \(+\infty\)]
\(\int^{+\infty}_2\frac{1}{t(\ln(t))^{\beta}}\) converge si \(\beta > 1\) ou . Dans le cas contraire, elle diverge.
\end{proposition}
\begin{proposition}[Inétgrale de bertrand en \(0\)]
\(\int^{1/2}_2\frac{1}{t|(\ln(t))|^{\beta}}\) converge si \(\beta > 1\) ou . Dans le cas contraire, elle diverge.
\end{proposition}
\warningInfo{Théorème de comparaison}{Le théorème de comparaison s'applique aussi pour les intégrales généralisées dont les fonctions sont positives.}
\begin{proposition}[Théorème des équivalents]
Si 2 fonctions positives sont équivalents en un point incertain, leur intégrale sont de m\^eme nature près de ce point.
\end{proposition}
\section{Fonctions oscillantes}
\begin{definition}[Convergence absolue]
On dit que \(\int f\) est absolument convergente si \(\int |f|\) converge.
\end{definition}
\begin{theorem}
 Si l'intégrale est absolument convergente, elle est convergente. On a alors \(|\int f|\leq \int f\)
\end{theorem}
\begin{definition}[Semi-convergence]
Une intégrale est semi-convergente si elle est convergente mais pas absolument convergente.
\end{definition}
\begin{theorem}[Critère de cauchy]
 L'intégrale impropre converge si et seulement si si pour tout \(\varepsilon >0\), il existe \(x_e\) tel que \(|\int_u^vf(t)\dd t|\leq \varepsilon\)
\end{theorem}

\end{document}
