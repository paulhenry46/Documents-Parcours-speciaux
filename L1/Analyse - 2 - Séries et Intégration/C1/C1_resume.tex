% !TeX spellcheck = en_US
\documentclass[french]{yLectureNote}

\title{Mathématiques - Analyse 2}
\subtitle{Analyse}
\author{Paulhenry Saux}
\date{\today}
\yLanguage{Français}

\professor{C.Dartyge}

\usepackage{graphicx}%----pour mettre des images
\usepackage[utf8]{inputenc}%---encodage
\usepackage{geometry}%---pour modifier les tailles et mettre a4paper
%\usepackage{awesomebox}%---pour les boites d'exercices, de pbq et de croquis ---d\'esactiv\'e pour les TP de PC
\usepackage{tikz}%---pour deiffner + d\'ependance de chemfig
\usepackage{tkz-tab}
\usepackage{chemfig}%---pour deiffner formules chimiques
\usepackage{chemformula}%---pour les formules chimiques en \'equation : \ch{...}
\usepackage{tabularx}%---pour dimensionner automatiquement les tableaux avec variable X
\usepackage{awesomebox}%---Pour les boites info, danger et autres
\usepackage{menukeys}%---Pour deiffner les touches de Calculatrice
\usepackage{fancyhdr}%---pour les en-t\^ete personnalis\'ees
\usepackage{blindtext}%---pour les liens
\usepackage{hyperref}%---pour les liens (\`a mettre en dernier)
\usepackage{caption}%---pour la francisation de la l\'egende table vers Tableau
\usepackage{pifont}
\usepackage{array}%---pour les tableaux
\usepackage{lipsum}
\usepackage{yFlatTable}
\usepackage{multicol}
\newcommand{\Lim}[1]{\lim\limits_{\substack{#1}}\:}
\renewcommand{\vec}{\overrightarrow}
\begin{document}


	\chapter{Rappels d'analyse 1 }
\section{Bornes}
\subsection{Maximum, majorant et borne supérieure}
Les définitions et théorèmes s'appliquent aussi aux minorants, bornes inférieures, etc.
\begin{definition}[Majorant]
$M$ est un majorant de $A$ si $\forall x\in A, x\leq M$
\end{definition}
\begin{definition}[Maximum]
$m$ est un maxium de $A$ si $\forall x\in A, x\leq m$ et $m\in A$. C'est un majorant appartenant à $A$. On le note $\max(A)$.
\end{definition}
Pour montrer que $m$ est le maximum d'un ensemble, il faut montrer qu'il appartient à cet ensemble et que \(\forall x\in A, x\leq m\)
\begin{definition}[Borne supérieure]
$M$ est la borne supérieure de $A$ si $M$ est un majorant de $A$ et s'il en existe pas de plus petits. On le note $\sup(A)$. C'est le plus petit des majorants
\end{definition}
\begin{theorem}[Théorème fondamental de l'analyse]
 Toute partie de $\mathbb{R}$ non vide et majorée admet une borne supérieure.
\end{theorem}
\begin{proposition}[Caractérisation de la borne supérieure]
\[M=\sup(A) \iff M\text{ majore A } \wedge \forall \varepsilon > 0, \exists x \in A, M-\varepsilon < x\]
\end{proposition}
\section{Suites}
\subsection{Limites}
\begin{definition}
La suite a pour limite $l$ si \(\forall \varepsilon > 0, \exists N \in \mathbb{N}, n\geq N \Rightarrow |u_n-l|\leq \varepsilon\)
\end{definition}
\begin{definition}[Série géométrique]
\(1+a+a^2+\dots+a^n = \frac{1-a^{n+1}}{1-a}\)
\end{definition}
\begin{proposition}[Limite de la série géométrique]
$+\infty$ si $a>1$ et $\frac{1}{1-a}$ si $|a|<1$. Dans les autres cas, elle est divergente.
\end{proposition}
\subsection{Suites extraites}
\begin{definition}
C'est une suite de la forme $(u_{\varphi(n)})$ avec $\varphi : \mathbb{N} \to \mathbb{N}$ est strictement croissante.
\end{definition}
\begin{theorem}[Suite convergente et suite extraite]
 Toute suite extraite d'une suite convergente converge vers la m\^eme limite.
\end{theorem}
\begin{myproof}
Avec $\varphi \in \mathbb{R}$ Soit $\varphi : \mathbb{N} \to \mathbb{N}$ strictement croissante.

Definition de la limite : $\forall \varepsilon >0,\exists n \in \mathbb{N}, \forall n\in\mathbb{N}, x\geq N \Rightarrow |u_n-l| \leq \varepsilon$.

Soit $\varepsilon >0$. On sait qu'il exisye $N$ tel que pour tout $n\in\mathbb{N}, n\geq N \Rightarrow |U_n -l|\leq \varepsilon$.

Or, $\varphi$ est strciteent croissante donc $\varphi(n) \geq n$.

Donc $n\geq N \Rightarrow \varphi(n) \geq N$ et $|U_{\varphi}| - l| \leq \varepsilon$.
\end{myproof}
\begin{proposition}
La suite $(U_n)$ converge vers $l$ si $(U_{2n})$ et $(U_{2n+1})$ convergent vers $l$.
\end{proposition}
\begin{myproof}
On traduit les hypothèses : $\exists N_1\in\mathbb{N}, \forall n\geq N_1, |U_{2n} - l|\leq \varepsilon$ et  $\exists N_2\in\mathbb{N}, \forall n\geq N_2, |U_{2n+1} - l|\leq \varepsilon$

Posons $N = \max(2N_1,2N_2+1)$.
Si $n\geq N$, alors
\begin{itemize}
 \item si $n$ est pair, $n=2k$ et $n\geq 2N_1$ d'où $k\geq N_1$ donc $|U_n-l| = |U_{2k} - l| \leq \varepsilon$
 \item si $n$ est impair, $n=2k'+1$ et $n\geq 2N_2+1$ d'où $k'\geq N_2+1$ donc $|U_n-l| = |U_{2k'+1} - l| \leq \varepsilon$
\end{itemize}
\end{myproof}
\begin{definition}
$l$ est une valeur d'adhérence si une suite extraite de $u_n$ prend $l$ comme limite.
\end{definition}
\begin{theorem}[Théorème de Bolzano-Weierstrass]
 Toute suite bornée admet une sous-suite convergente.
\end{theorem}
\section{Comparaison}
\subsection{Négligeabilité}
\begin{definition}
On note $o(u_n)$ une suite pouvant s'écrire de la forme $(\varepsilon_nu_n)$ avec $(\varepsilon_n \to 0) = (v_n)$. On dit que $(v_n)$ est négligeable devant $u_n$.
\end{definition}
\begin{proposition}
\[v_n = o(u_n) \iff \lim \frac{v_n}{u_n} = 0\]
\end{proposition}
\subsection{Équivalence}
\begin{definition}
$u_n\sim v_n \iff u_n-v_n = o(v_n)$
\end{definition}
\begin{proposition}
$u_n\sim v_n \iff \lim \frac{u_n}{v_n} = 1$
\end{proposition}[Lien avec les limites]
\begin{proposition}[lien avec négligeance]
$u_n-v_n = o(v_n) \iff u_n\sim v_n$
\end{proposition}
\begin{proposition}[Équivalent de puissance]
$u_n\sim v_n \Rightarrow u_n^{\alpha} \sim v_n^{\alpha}$
\end{proposition}
\section{Fonctions}
\begin{theorem}[Théorème des accroissements finis]
Soit $f$ fonction continue sur $[a,b]$ et dérivable sur $]a,b[$. Il existe $c\in]a,b[$ tel que \(f(b)-f(a) = f'(c)(b-a)\)
\end{theorem}
\begin{theorem}[Théorème des bornes de Weierstrass]
Soit $f : [a,b] \to \mathbb{R}$ une fonction continue sur un segment. Alors il existe deux réels $m$ et $M$ tels que $f([a,b]) = [m,M]$. L'image d'un segment est un segment.
\end{theorem}
\begin{theorem}[Formule de Taylor-Young]
 \[f(x) = f(a) + f'(a)(x-a) + \frac{f''(a)}{2!}(x-a)^2+\dots+o((x-a)^n)\]
\end{theorem}
\subsection{Développements limités usuels}
\criticalInfo{
Ce sont les DL des fonctions au voisinage de 0 !
}

\warningInfo{Développements limités à conna\^itre}{
\begin{eqnarray*}
e^x&=&1+x+\frac{x^2}2+\dots+\frac{x^n}{n!}+o(x^n)\\
\cos x&=&1-\frac{x^2}{2!}+\dots+\frac{(-1)^n x^{2n}}{(2n)!}+o(x^{2n+1})\\
\sin x&=&x-\frac{x^3}{3!}+\dots+\frac{(-1)^n x^{2n+1}}{(2n+1)!}+o(x^{2n+2})\\
\cosh x&=&1+\frac{x^2}{2!}+\dots+\frac{x^{2n}}{(2n)!}+o(x^{2n+1})\\
\sinh x&=&x+\frac{x^3}{3!}+\dots+\frac{ x^{2n+1}}{(2n+1)!}+o(x^{2n+2})\\
\frac{1}{1-x}&=&1+x+x^2+\dots+x^n+o(x^n)\\
\frac{1}{1+x}&=&1-x+x^2+\dots+(-1)^nx^n+o(x^n)\\
\ln(1+x)&=&x-\frac{x^2}2+\dots+\frac{(-1)^{n+1}}{n}x^n+o(x^n)\\
(1+x)^\alpha&=&1+\alpha x+\frac{\alpha(\alpha-1)}2x^2+\dots+\frac{\alpha(\alpha-1)\cdots (\alpha-n+1)}{n!}x^n+o(x^n)\\
\end{eqnarray*}}
\subsection{Remarques}
\begin{itemize}
 \item Quand une fonction est paire, son DL ne comporte que des puissances de x paires.
\item Le DL d’une fonction impaire ne comporte que des puissances de x impaires.
\item Quand elle n’est ni paire ni impaire, elle comporte à priori toutes les puissances de x (sauf exception).
\item Le DL de ch(x) est la partie paire de $e^x$.
\item Le DL de sh(x) est la partie impaire de $e^x$.
\end{itemize}

\end{document}

