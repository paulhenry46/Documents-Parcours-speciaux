% !TeX spellcheck = en_US
\documentclass[french]{yLectureNote}

\title{Mathématiques}
\subtitle{Langage mathématique}
\author{Paulhenry Saux}
\date{\today}
\yLanguage{Français}

\professor{C.Dartyge}

\usepackage{graphicx}%----pour mettre des images
\usepackage[utf8]{inputenc}%---encodage
\usepackage{geometry}%---pour modifier les tailles et mettre a4paper
%\usepackage{awesomebox}%---pour les boites d'exercices, de pbq et de croquis ---d\'esactiv\'e pour les TP de PC
\usepackage{tikz}%---pour deiffner + d\'ependance de chemfig
\usepackage{tkz-tab}
\usepackage{chemfig}%---pour deiffner formules chimiques
\usepackage{chemformula}%---pour les formules chimiques en \'equation : \ch{...}
\usepackage{tabularx}%---pour dimensionner automatiquement les tableaux avec variable X
\usepackage{awesomebox}%---Pour les boites info, danger et autres
\usepackage{menukeys}%---Pour deiffner les touches de Calculatrice
\usepackage{fancyhdr}%---pour les en-t\^ete personnalis\'ees
\usepackage{blindtext}%---pour les liens
\usepackage{hyperref}%---pour les liens (\`a mettre en dernier)
\usepackage{caption}%---pour la francisation de la l\'egende table vers Tableau
\usepackage{pifont}
\usepackage{array}%---pour les tableaux
\usepackage{lipsum}
\usepackage{yFlatTable}
\usepackage{multicol}
\newcommand{\Lim}[1]{\lim\limits_{\substack{#1}}\:}
\renewcommand{\vec}{\overrightarrow}
\newcommand{\bz}{\overline{z}}
\DeclareMathOperator{\card}{card}
\begin{document}

\setcounter{chapter}{5}

	\chapter{Nombres complexes}
\section{Méthode}
\subsection{Solutions de $z^n=1$}
\begin{enumerate}
 \item On pose $z=\rho e^{i\theta}$
 \item On écrit sous forme polaire : $\rho^ne^{in\theta} = 1e^{i0}$
 \item L'égalité est vérifiée si le module des 2 nombres sont égaux, i.e. $\rho = 1$
 \item Il faut aussi que $n\theta = 0+2k\pi$, avec $0<k\leq n-1$
 \item Les solutions sont donc $\{1, e^{i\frac{2\pi}{n}}, e^{i\frac{4\pi}{n}}\dots e^{i\frac{2(n-1)\pi}{n}}\}$
\end{enumerate}
\subsection{Résoudre $z^n = w$}
\begin{enumerate}
 \item On pose $z=\rho e^{i\theta}$
 \item On écrit sous forme polaire : $\rho^ne^{in\theta} = |w|e^{i\varphi}$
 \item L'égalité est vérifiée si le module des 2 nombres sont égaux, i.e. $\rho = \sqrt[n]{|w|}$
 \item Il faut aussi que $\theta = \frac{\varphi}{n}$.
 \item On multiplie le résultat par les racines $n$-eme de l'unité associées.
\end{enumerate}
\subsection{Résoudre $z^n = w^n$}
On cherche $z$ pour $w$ donné. $z$ a $n$ solutions, qui sont $z$ mutipliées par chacunes des racines $n$ de l'unité associées.
% \subsection{Résoudre une équation de la forme $e^z = a+ib$}
\subsection{Calculer les racines d'un nombre complexe}
On cherche les racines $z_1,z_2$ d'un nombre complexe, noté $w$. Si $w=0, z=0$
\begin{enumerate}
 \item On calcule le module de $w$
 \item Comme $z^2=w \Rightarrow |z|^2 = |w| \Rightarrow \sqrt{a^2+b^2}^2 = |w|$, on en déduit finalement que $a^2+b^2 = |w|$
 \item De plus, $z^2 = (a+ib)^2 = (a^2-b^2)+2iab$. On en déduit que la partie réelle de $w$ est $a^2-b^2$ et que la partie imaginaire est $2ab$.
 \item On obtient un système à 3 équations. les 2 premières nous permettent de déterminer $\pm a$ et $\pm b$. La dernière nous donne le signe : si $2ab$ est positif, $a$ et $b$ sont de m\^eme signe, dans le cas contraire, ils sont de signe contraire.
\end{enumerate}
En résumé, on doit résoudre ce système pour trouver les solutions :\[ \left\{\begin{matrix}
a^2+b^2 &= |w|\\
a^2-b^2 &= Re(w)\\
2ab &= Im(w)
\end{matrix}\right.\]
% \subsection{Calcul des racines nieme de l'unité}
% $z^2 = 1 \iff z=1,-1$
%
% $z^3 \iff z=1, etc$. $z$ est un complexe de module $1$, i.e. $z= e^{i\theta}$, alors $z^3 = e^{3i\theta} = e^{i\times 0} \iff 3\theta = 0 + 2k\pi \iff \theta = \frac{2k\pi}{3}$ On prend $0\leq k\leq 2$. Les solutions sont donc $1,e^{2i\pi/3}, e^{4i\pi/3}$.
%
% La multiplication est stable dans le groupe des racines. On obtient une racine en multipliant/divisant 2 racines. Il y a $n$ racines nieme de l'unité
% \subsection{Utiliser les formules d'Euler (6.11)}
\subsection{Utiliser la formule de moivre pour exprimer des cosinus et sinus}
On sait que $(e^{i\theta})^n = e^{in\theta}$. On souhaite exprimer $\cos(4x)$ en fonction de sinus et cosinus. On introduit pour cela le nombre complexe $e^{i4x} = (e^{ix})^4 = (\cos(x)+i\sin(x))^4$. Par identification de la partie réelle pour le $\cos$ et imaginaire pour le $\sin$, on peut trouver le résultat demandé.

On se sert du binome de Newton pour retrouver les coefficients de notre développement si il le faut
\begin{flalign*}
(\cos(x)+i\sin(x))^4 &= \cos^4(x) + 4\cos^3(x)(i\sin(x))\\
&\:\:\:\:+6\cos^2(x)(i\sin(x))^2+4\cos(x)(i\sin(x))^3+(i\sin(x))^4\\
&= \cos^4(x) + 4\cos^3(x)(i\sin(x))-6\cos^2(x)\sin^2(x)\\
&\:\:\:\:-4\cos(x)(i\sin^3(x)+\sin^4(x)\\
&= \cos^4(x)+\sin^4(x) -6\cos^2(x)\sin^2(x)\\
&\:\:\:\:+ 4\cos^3(x)(i\sin(x))-4\cos(x)(i\sin^3(x)\\
\cos(4x) &= \cos^4(x)+\sin^4(x) -6\cos^2(x)\sin^2(x)\\
\end{flalign*}
\subsection{Linéariser une expression}
On utilise les formules d'Euler pour exprimer les $\cos$ et $\sin$. Ainsi, pour linéariser $\cos^n(x)\sin^m(x)$, on écrit
\begin{flalign*}
\cos^n(x)\sin^m(x) &= (\frac{e^{ix}+e^{-ix}}{2})^n(\frac{e^{ix}-e^{-ix}}{2i})^m
\end{flalign*}
On met en facteur tout ce qui se trouve au dénominateur pour l'enlever et on calcule pour retrouver d'autres expressions que l'on pourra retransformer avec les formules d'Euler.
\end{document}


