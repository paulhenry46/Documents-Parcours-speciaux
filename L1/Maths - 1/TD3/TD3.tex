% !TeX spellcheck = en_US
\documentclass[french]{yLectureNote}

\title{Mathématiques}
\subtitle{Langage mathématique}
\author{Paulhenry Saux}
\date{\today}
\yLanguage{Français}

\professor{C.Dartyge}

\usepackage{graphicx}%----pour mettre des images
\usepackage[utf8]{inputenc}%---encodage
\usepackage{geometry}%---pour modifier les tailles et mettre a4paper
%\usepackage{awesomebox}%---pour les boites d'exercices, de pbq et de croquis ---d\'esactiv\'e pour les TP de PC
\usepackage{tikz}%---pour deiffner + d\'ependance de chemfig
\usepackage{tkz-tab}
\usepackage{chemfig}%---pour deiffner formules chimiques
\usepackage{chemformula}%---pour les formules chimiques en \'equation : \ch{...}
\usepackage{tabularx}%---pour dimensionner automatiquement les tableaux avec variable X
\usepackage{awesomebox}%---Pour les boites info, danger et autres
\usepackage{menukeys}%---Pour deiffner les touches de Calculatrice
\usepackage{fancyhdr}%---pour les en-t\^ete personnalis\'ees
\usepackage{blindtext}%---pour les liens
\usepackage{hyperref}%---pour les liens (\`a mettre en dernier)
\usepackage{caption}%---pour la francisation de la l\'egende table vers Tableau
\usepackage{pifont}
\usepackage{array}%---pour les tableaux
\usepackage{lipsum}
\usepackage{yFlatTable}
\usepackage{multicol}
\newcommand{\Lim}[1]{\lim\limits_{\substack{#1}}\:}
\renewcommand{\vec}{\overrightarrow}
\begin{document}

\setcounter{chapter}{2}

	\chapter{Fonctions}
\section{Précisions sur les applications réciproques}
Une application est bijective de $E\rightarrow F$ si $\forall y\in F, \exists!x\in E,y=f(x)$.

On définit une application, appelée réciproque notée $f^{-1} : F\rightarrow F$ et $y\longmapsto x=f^{-1}(y)$, avec $\forall y\in F, x=f^{-1}(y) \iff x\in E$ et $y = f(x)$. f est bijective donc $f^{-1}$ est une application.

Propriétés :
\begin{itemize}
 \item $f\circ f^{-1}(y) = y = Id_F$
  \item $f^{-1}\circ f(x) = x = Id_E$
  \item $f\circ Id_E = f$ : $Id_E$ est le neutre à droite pour 0
  \item $Id_E \circ f = f$ : $Id_E$ est le neutre à gauche pour 0
\end{itemize}
En effet, $f\circ f^{-1}(y) = f(f^{-1}(y)) = f(x) = y$.

\begin{theorem}[Proposition]
Soit $f : E\rightarrow F$ une application. S'il existe $g : F\rightarrow E$ une application telle que : $g\circ f = Id_E$ et $ f\circ g = Id_F$. Alors $f$ est bijective et $g$ est la réciproque de $f$.
\end{theorem}
\begin{theorem}[Corrolaire]
Si l'application réciproque exsiste, elle est unique
\end{theorem}
Exemple : $Id : R\rightarrow R$. $Id_R (\sqrt{2}) = \sqrt{2}$.
\warningInfo{Conséquence}{On peut donc montrer qu'une application est bijective en exhibant sa réciproque}

\begin{myproof}[]
Je suppose :$f:E\rightarrow F$. Soit $g : F\rightarrow E$ avec $g\circ f = Id_E$ et $ f\circ g = Id_F$
\begin{itemize}
 \item Montrons que $f$ est bijective.

Soir $y\in F, E? x\in E,$ tel que $y=f(x)$, est unique ?

$y = f(x) \Rightarrow g(y) = g(f(x)) = x$, car $g\circ f = Id_E$.

Si $g=f(x)$, alors  $x=g(y)$. $y$ a au plus un antécédant et $f$ est injective.

De plus, $f(x) = f(g(y)) = y$ car $f\circ g = Id_F$. Donc $x = g(y)$ est bien un antécédant de $y$ par $f$ et c'est le seul.

Conslusion : $f$ est bijective de $E \rightarrow F$

\item Montrons $g=f^{-1}$.

$f$ bijective et $F \rightarrow E$ et $y \longmapsto x$.

$f^{-1}(x) = x \iff y = f(x)$ et $x\in E$.

Vérifions que $\forall y \in F, f^{-1}(y) = g(y)$.

$f^{-1}(y)=x \iff f(x) = y \iff g\circ f(x) = g(y) \Rightarrow x = g(y)$. Donc $g(y) = f^{-1}(y) \forall y\in F$.
\end{itemize}
Donc l'implication est démontrée.

\end{myproof}
\begin{theorem}[Proposition]
Soit $f : E\rightarrow F$ et $g : F\Rightarrow G$ deux applications bijectives. $g\circ f : E\rightarrow F \rightarrow G$ et $x \longmapsto f(x) \longmapsto g(f(x))$ Alors $g\circ f$ est bijective.
\end{theorem}

\begin{theorem}[Proposition]
Soit $f : E\rightarrow F$ bijective et notons $f^{-1} : F\Rightarrow E$ sa réciproque. Alors $f^{-1}$ est bijective de réciproque $f$.
\end{theorem}
\begin{myproof}[]
Si $f$ est bijective, alors $f\circ f^{-1} = Id_F$ et $f^{-1}\circ f = Id_E$. D'après la proposition précédante, $f^{-1}$ est bijective

\end{myproof}
\section{Généralités sur les fonctions de R dans R}
\subsection{}
Une fonction $f$ de R dans R est une correspondance de R dans R telle que tout élément de l'ensemble de départ a au plus une image dans l'ensemble d'arrivée. On ne change pas le domaine d'arrivé mais on s'autorise à changer le domaine de départ.

Exemples :
\begin{itemize}
 \item $R\rightarrow R$ et $x\longmapsto \sqrt{x-1}$ f(0) n'est pas définie
 \item $R\rightarrow R$ et $x\longmapsto \log(x)$. Log est une application de $R_+^*$ dans $R$.
\end{itemize}

On note $F(R,R)$ les fonctions de $R$ dans $R$

\begin{theorem}[Ensemble de définition]
Soit $f$ de R dans R une fonction. Le domaine de définition de $f$ est l'ensemble, noté $Df = \{x\in R, f(x) \text{existe}\}$ Alors, $Df\rightarrow R$ est une application.
\end{theorem}

Exemple : $f : R \rightarrow R$, $x\longmapsto \sqrt{x^2-x-1}$ $f$ est définie si le polynome est positif, i.e $x\leq\alpha$ et $\geq\tau$

Exemple : $f : R \rightarrow R$, $x\longmapsto \log(\frac{1-x}{1+x})$. $f$ est définie si la fraction est supérieure à 0 et $1+x\neq 0$ Donc le domaine de définition est $]-1,1[$
\checkInfo{Remarque}{Montrer que 2 fonctions sont égales : $\forall x\in D_f = D_g, f(x) = g(x)$}
\subsection{Monotonie}
Soit f de R dans R une fonction. Elle est croissante sur R, respectivement strictement croissante si $\forall x,x'\in D_f, x\leq x' \Rightarrow f(x)\leq g(x)$ respectivement $\forall x,x'\in D_f, x<x' \Rightarrow f(x)< g(x)$.

Exemple : $f(x) = 2x$ $\forall x,x' = 2x'-2x = 2(x'-x) >0$, donc elle est croissante.

Exemple : $g(x) = \frac{1}{x}$ est strictement décroissante : $\frac{1}{x'} - \frac{1}{x} = \frac{x-x'}{x'x} <0$. Donc $g$ est bien décroissante sur $R_+^*$.

Si $f$ est (strictement) croissante ou décroissante sur $D_f$, on dit qu'elle est (strictement) monotone sur $D_f$
\begin{theorem}[Proposition]
Soit $f : I\rightarrow R$, si $f$ est strictement monotone sur $I$, alors $f$ est injective de I sur R.
\end{theorem}
\begin{myproof}
Pour une fonction strictement décroissante.

$\forall x,x', x'>x f(x)>f(x')$ et donc $f(x)\neq f(x')$
$\forall x,x',x'<x, f(x)<f(x')$ et donc $f(x)\neq f(x')$

Donc, on a bien $\forall x\neq x' \Rightarrow f(x)\neq f(x')$
\end{myproof}
\subsection{Fonctions majorées et minorées}
Soit $f : R\rightarrow R$ Soit $I\in D_R$.

$f$ est majorée sur $I$ s'il existe $M\in R,\forall x\in I, f(x) \leq M$.

$f$ est minorée sur $I$ s'il existe $m\in R,\forall x\in I, f(x) \geq m$.

Elle est bornée sur $I$, si $f$ est majorée et minorée sur $I \iff \exists M\in\mathbb{R}^+, \forall x\in I, |f(x)\leq M$.

Exemple de fonctions majorées

\begin{itemize}
 \item $\cos(x) : \forall x\in\mathbb{R}, |\cos(x)|\leq 1$
 \item $x\longmapsto \exp(x)$ est minorée par 0.
 \item $x\longmapsto -x^2$ est majorée par 30
\end{itemize}

\checkInfo{Montrer que la fonction est non majorée}{
On montre que $\forall M\in\mathbb{R},\exists x\in I, f(x)>M$.}
\subsection{Image directe et image récirproque}
On se donne $f \in F(R,R)$.

Soit $I$ un intervalle de $\mathbb{R}$. $f(I) = \{f(x), x\in I\} = \{y\in\mathbb{R},\exists x\in I, y = f(x)\}$.

Soit $J$ un intervalle de $\mathbb{R}$. $f^{-1}(J) = \{ x\in D_f, f(x)\in J\} = \{y\in\mathbb{R},\exists x\in I, y = f(x)\}$.
\section{Limites d'une fonction en un point ou en l'infini}
Soit $f : \mathbb{R} \rightarrow \mathbb{R}$ une fonction.
\subsection{Limite en un point}
\subsubsection{Voisinage épointé}
\warningInfo{Voisinage épointé de $x_0$}{Un intervalle ouvert contenant $x_0$, privé de $x_0$. On le note $V_{x_0}$. $V_{x_0} = ]x_0-\epsilon,x_0+\epsilon[\setminus \{x_0\}$}

f a une limite finie si :

est définie sur un voisinage épointé de $x_0$ et pour toute suite $(u_n)$ est convergente vers $x_0$ et à valeurs de $x_0$, \[\Lim{\infty} f(u_n) = l\] avec $(u_n)$ tend vers $x_0$.

Cela équivaut à : $\forall \varepsilon >0, \exists \alpha >0, \forall x, 0< |x-x_0|<\alpha \Rightarrow |f(x) - l|<\epsilon$.

 Exemple : $\Lim{0}\frac{0}{x} = 0$, car $x$ est défini au voisinage épointé de 0.
\subsubsection{Limites infinies}

 f a une limite valant $+\infty$ si :
 \begin{itemize}
  \item f st définie sur un voisinage épointé de $x_0$

 \item pour toute suite $(u_n)$ à valeurs dans $V_{x_0}$ et de limite $x_0$, on a : \[\Lim{\infty} f(u_n) = +\infty\]

 \item $\forall A >0, \exists \alpha >0, 0<|x-x_0|<\alpha \Rightarrow f(x)>A$ (équivalente à la proposition précédente)
 \end{itemize}

 De m\^eme en $-\infty$ :

 \begin{itemize}
 \item f st définie sur un voisinage épointé de $x_0$
  \item Pour toute suite $(u_n)$ à valeurs dans $V_{x_0}$, \[\Lim{\infty} f(u_n) = -\infty\]

 \item $\forall A >0, \exists \alpha >0, 0<|x-x_0|<\alpha \Rightarrow f(x)<-A$ (équivalente à la proposition précédente)
 \end{itemize}
\subsection{Limites en + l'infini}
On se donne $f$ définie au voisinage de $+\infty$ : $\exists a\in\mathbb{R}$, f est définie sur $]a,+\infty[$.
\subsubsection{Limite finie $l$}
$\forall \epsilon >0, \exists A>0, x>A \Rightarrow |f(x)-l|<\epsilon$ Schéma 2
\subsubsection{Limite + infinie}
La limite vaut $+\infty$
si $\forall A>0, \exists R>0, x>R \Rightarrow f(x)>A$.
\subsubsection{Limite - infinie}
Schéma 3
La limite vaut $-\infty$
si $\forall A>0, \exists R>0, x>R \Rightarrow f(x)<-A$.
\subsection{Limites en - l'infini}
On se donne $f$ définie au voisinage de $-\infty$ : $\exists a\in\mathbb{R}$, f est définie sur $]-\infty,A[$.
\subsubsection{Limite finie $l$}
$\forall \epsilon >0, \exists R>0, x<-R \Rightarrow |f(x)-l|<\epsilon$
\subsubsection{Limite + infinie}
La limite vaut $+\infty$
si $\forall A>0, \exists R>0, x<-R \Rightarrow f(x)>A$.
\subsubsection{Limite - infinie}
La limite vaut $-\infty$
si $\forall A>0, \exists R>0, x<-R \Rightarrow f(x)<-A$.
\section{Limites et opérations}
Soit $a\in \mathbb{R}\cup \{\pm \infty\}$, avec $f,g$ définies sur un voisinage épointé de $a$. On suppose que $f,g$ a une limite $b \in\mathbb{R}\cup \{\pm \infty\}$, alors on a pour $f+g$:
Tableaux à récupérer :

\subsection{Somme}
$$\begin{array}{|l|c|c|c|c|c|c|}
\hline
\Lim{x\to \alpha}f & \ell & \ell & +\infty & -\infty & + \infty \\
\hline
\Lim{x\to \alpha}g & \ell' & \pm\infty  & +\infty & -\infty & -\infty \\
\hline
\Lim{x\to \alpha}f+g & \ell + \ell' & \pm\infty  & +\infty & -\infty &  \text{F. I.} \\
\hline
\end{array}$$
\subsection{Produit}
$$\begin{array}{|l|c|c|c|}
\hline
\Lim{x\to \alpha}f & \ell  & \ell \neq 0 &  0 \\
															&       &  $ ou $\pm \infty &   \\
\hline
\Lim{x\to \alpha}g & \ell' & \pm \infty & \pm \infty \\
\hline
\Lim{x\to \alpha}f\times g & \ell \times \ell' & \pm \infty & \text{F. I.} \\
\hline
\end{array}$$
\subsection{Quotient}
$$\begin{array}{|l|c|c|c|c|c|c|}
\hline
\Lim{x\to \alpha}f & \ell  & \ell  & 0 & \ell \neq 0 & \pm \infty &  \pm \infty  \\
\hline
\Lim{x\to \alpha}g & \ell'\neq 0 & \pm \infty & 0 & 0 & \ell' &  \pm \infty \\
\hline
\Lim{x\to \alpha}\frac{f}{g} & \cfrac{\ell}{\ell'} & 0^+\text{ ou }0^- & \text{F. I.} & \pm \infty & \pm \infty &  \text{F. I.} \\
\hline
\end{array}$$
\subsection{Limite d'une fonction composée}
Soit $f = v\circ u$.
\begin{flalign*}
\Lim{x\to a}u(x) &= b\\
\Lim{x\to b}v(x) &= c\\
\Lim{x\to a}f(x) &= c\\
\end{flalign*}
\section{Fonctions continues}
Soit $I$ un intervalle ouvert de $\mathbb{R}$. Soit $f:I\to \mathbb{R}$, f est définie sur $I$.
\subsection{Définition}
$f$ est continue en $x_0$ si $\Lim{x_0} f = f(x_0)$. Ce qui est équivalent à $\forall \varepsilon >0,\exists  \alpha >0, |x-x_0|<\alpha \Rightarrow |f(x)-f(x_0)|<\varepsilon$. Ce qui est équivalent à $\forall (U_n), \Lim{\infty}f(U_n) = f(x_0)$. On a donc : $\Lim{\infty} f(U_n) = f(\Lim{\infty} U_n)$.
\subsection{Exemples}
\begin{itemize}
 \item Fonction polynomiales, sinus, cosinus, tangente, exponentielle, logarithme, sont
 \item fonction valeur abosolue continue
\end{itemize}

On dit que $f$ est continue sur l'intervalle $I$ si elle est continue en tout $x$ de $I$.
\section{Continuité et opérations}
On prend 2 fonctions $f$ et $g$ continues sur $I$. Alors $f+g, fg$ sont continues sur $I$ et $\frac{f}{g}$ est continue en tout point de $I$ tel que $g(x)\neq 0$.

\begin{theorem}[Continuité des composées]
Soient $f : I\to J$ une fonction continue sur$I$, à valeurs dans $I\in\mathbb{R}$ et $g:I\to J\in\mathbb{R}$. Alors $g\circ f$ est continue sur $I$.
\end{theorem}
\begin{myproof}
Soit $x_0\in I$

Soit $\epsilon >0$ $\exists ? \alpha >0,$ tel que $|x-x_0| < \alpha \Rightarrow |g(f(x)) - g(f(x_0))|<\varepsilon$.

Soit $\varepsilon >0$ donné. On cherche $\alpha$ tel que $y_0 = f(x_0)$. Notons alors $y=f(x)$. $g$ est continue en $x_0$, donc $\exists \eta >0$, $|y-y_0|<\eta \Rightarrow |g(y)-g(y_0)|<\varepsilon$.

Or, $f$ est continue en $x_0$, donc $\exists \alpha >0, |x-x_0| \Rightarrow |f(x)-f(x_0) < \eta \Rightarrow |g(f(x))-g(f(x_0))| < \varepsilon$ car $|y-y_0|<\eta \Rightarrow |g(y)-g(y_0)|<\varepsilon$

Donc $g\circ f$ est bien continue en $x_0$.
\end{myproof}
\subsection{Théorème des valeurs intermédiaires}
\begin{theorem}[TVI]
Soit $f:I\to\mathbb{R}$ une fonction et $(a \leq b)\in I$. On suppose $f$ continue sur $[a,b]$.

Alors $\forall y_0 \in [f(a),f(b)], \exists x_0\in[a,b], y_0=f(x_0)$.
\end{theorem}
\begin{theorem}[Variante du TVI]
Il est équivalent à :

si $f$ est continue sur $[a,b]$ et $f(a)\times f(b) \leq 0$, alors $\exists c\in[a,b], f(c)=0$.
\end{theorem}
\begin{myproof}[Par dichotomie]
Supposons par exemple que $f (a) \leq 0 \leq f (b)$, de sorte que $0 \in [f (a), f (b)]$ (l’autre cas s’y ramène en considérant $−f$ ). On construit par récurrence deux suites $(a_n )$ et $(b_n )$
de la façon suivante.

On part de $a_0 := a$, $b_0 := b$, et supposant construits $a_n$ et $b_n$ tels que $f(a_n)\leq 0\leq f(b_n)$,  on considère la valeur de f en $(a_n + b_n )/2$ , milieu du segment $[a_n , b_n ]$.

On construit alors$ a_{n+1}$ et $b_{n+1}$ ainsi :
\begin{itemize}
 \item si $f(\frac{a_n+b_n}{2})<0$, on pose $a_{n+1} = \frac{a_n+b_n}{2}$ et $b_{n+1} = b_n$.
 \item Sinon, c'est à dire si $f(\frac{a_n+b_n}{2}) \geq 0$, on pose $a_{n+1} = a_n$ et $b_{n+1} = \frac{a_n+b_n}{2}$.
\end{itemize}
On voit ainsi que :
\begin{enumerate}
 \item $a_n\leq a_{n+1}\leq b_{n+1}\leq b_n$
 \item $0\leq b_n-a_n = \frac{b-a}{2^n}$ (On divise par 2 à chaque fois la longueur initiale $b-a$
 \item $f(a_n)\leq0\leq f(b_n)$
\end{enumerate}
En particulier, $\Lim{\infty} |b_n-a_n| = 0$ et les 2 suites sont adjacentes.

Elles convergent donc vers une unique limite $c$. On a $c\in[a,b]$ car $a_n\in[a,b]$ et comme $f$ est continue, on a : $\Lim{\infty} f(a_n) = f(c) = f(b_n)$.

Les doubles inégalités 2 et 3 impliquent que $c=0$.
\end{myproof}
% \begin{myproof}[Par dichotomie, voir page 634]
% $f$ continue sur $[a,b]$ et l'on suppose que $f(a)\times f(b) < 0$.
%
% On définit 2 suites adjacentes $a_n,b_n$ de la façon suivante : $a_0 = a$, $b_0 = b$ et $f(a_n)f(b_n)\leq 0$. On a aussi $<0b_n-a_n = \frac{b-a}{2^n}$.
%
% On construit $a_n$ et $b_n$ par récurrence :
%
% Soit $c_1$ = $\frac{a_0+b_0}{2}$. Si $f(a)f(c) \leq0$, alors $a_1=a_0, b_1=c$.
%
% Sinon, on a nécessairement $f(c)f(b) \leq 0$, car alors $f(a)f(c)>0$ comme $f(a)f(b)\leq 0$.
%
% On a $f(c)f(b)\leq 0$. Dans ce cas, on pose $a_1=c, b_1=b_0$. on a $b_1-a_1 = \frac{b-a}{2}$. Donc $a_1\leq b_1$ et $f(a_1)f(b_1) \leq 0$.
%
% Hérédité : Supposons que nous ayons construit $a_n$ et $b_n$ vérifiant $H_n$.
%
% On pose $c_{n+1} = \frac{a_n+b_n}{2}$. On a : $a_n\leq c_{n+1}\leq b_n$
%
% Si $f(a_n)f(c_{n+1} \leq 0$, alors $ a_{n+1} = a_n$ et $b_{n+1} = \frac{a_n+b_n}{2}$.
%
% Sinon, i.e $f(a_n)f(c_{n+1}) >0$, $f(a_n)$ et $f(c_{n+1})$ sont du m\^eme signe, alors comme $f(a_n)$ et $f(b_n)$ sont de signe contraire, on en déduit $f(b_n)f(c_{n+1}) \leq 0$, d'où $a_{n+1} = c_{n+1} = \frac{a_n+b_n}{2}$ et $b_{n+1} = b_n$.
%
% Dans tous les cas, $a_n\leq a_{n+1} \leq b_{n+1}\leq b_n$ et $(a_n)$ est croissante et $(b_n)$ est décroissante. On a $f(a_{n+1})f(b_{n+1})\leq 0$ et $b_{n+1} - a_{n+1} = b_n - \frac{a_n+b_n}{2} = \frac{b-a}{2^n\cdot 2}$ ou $\frac{a_n+b_n}{2}-a_n = \frac{b-a}{2^{n+1}}$.
%
% Donc $H_n\Rightarrow H_{n+1}$.
%
% $(b_n)$ est décroissante et minorée par $b_n-a_n\geq 0$ et $(a_n)$ est croissante et $b_n-a_n \to 0$
%
% Donc les suites $(a_n), (b_n)$ convergent vers un m\^eme réel $c$ tel que $a\leq c\leq b$.
%
% De plus, $\forall n\in\mathbb{N}, f(a_n)f(b_n)\leq 0$ et $f$ continue sur $[a,b]$, donc $\Lim{\infty} f(a_n) = f(\Lim{\infty} a_n) = f(c)$.
%
% De m\^eme, $\Lim{\infty} f(b_n) = f(c)$, donc $\Lim{\infty} f(a_n)f(b_n) \leq 0$, c'est à dire $f(c)^2 \leq 0$, donc $f(c) = 0$.
% \end{myproof}
\subsection{Théorème de Heine}
\begin{theorem}[Théorème de Heine]
L'image continue d'un intervalle fermé et borné est un intervalle fermé et borné.

Soient $ a<b \in\mathbb{R}$ et $f:[a,b] \to \mathbb{R}, f$ continue sur $[a,b]$, alors $\exists m\in\mathbb{R},M\in\mathbb{R}, m\leq M$ tels que $f([a,b] = [m,M]$ en particulier $\exists x_0\in[a,b], f(x_0) = m$ et $\exists x_1\in[a,b], f(x_1) = M$
\end{theorem}
Corrlolaire : $si f$ est continue sur $[a,b]$, alors $f$ est bornée sur $[a,b]$ et atteint ses bornes.
\begin{myproof}{}
On montre d'abord que la fonction est bornée, puis qu'elle atteint ses bornes.

Montrons que la fonction est bornée. Soit $f:[a;b]\to \mathbb{R}$, avec $a<b$. On suppose la fonction $f$ non majorée (hypothèse de la démonstration par l'absurde). Dans ce cas, $\forall M,\exists t \in[a,b], t\geq M$.

En posant $M=n\in\mathbb{N}$, on a $t_n\in[a,b],f(t_n) \geq n = M$. La suite obtenue est bornée, on peut en extraire une suite convergente $(t_{n_k})$ de limite $\alpha$.

Nous avons donc $f(t_{n_k}) \geq n_k, \forall k\in\mathbb{N}$.

$f$ est continue, donc la suite $(f(t_{n_k}))\to f(\alpha)$.

Or, d'après $f(t_{n_k}) \geq n_k$, la suite devrait tendre vers $+\infty$. Il y a contradiction, donc $f$ est majorée. On applique ce qui précède pour montrer que $f$ est minorée.

Montrons qu'elle atteint ses bornes.

Notons maintenant $\alpha$ sa borne inférieure et supposons qu'elle n'est pas atteinte pas $f$. Posons alors la fonction $\displaystyle g:[a,b] \to \mathbb{R}, g(t) = \frac{1}{f(t)-\alpha}$. $g$ est bien continue sur $[a,b]$ par composition.

Soit $M>0$ donné. Par définition de la borne inférieure, nous savons qu'il existe $t \in[a,b]$ tel que $\alpha\leq f(t)<\alpha+\frac{1}{M}$, et donc que $g(t) > M$. $M$ étant arbitraire, $g$ n'est pas majorée. Or, cela contredit la première partie de la démonstration.
\end{myproof}

% \begin{myproof}{Incomplète}

% Lemne : Soit $f$ continue sur $[a;b]$, alors $f$ est bornée sur $[a,b]$.
%
% On montre que $f$ est majorée, (alors -$f$ le sera aussi et $f$ sera minorée.)
%
% On fait un raisonnement par l'absurde.
%
% Supposons que $f$ n'est pas majorée sur $[a,b]$. Alors $\forall n\in\mathbb{N}, \exists t_n \in[a,b]$ tel que $f(t_n) \geq n$. On a aussi contruit une suite $(t_n)$ de $[a,b]$. On prend la suite extraite de $(t_{\phi(n)})$ tel que $\phi$ est strcitement croissante et $v_n = t_{\phi(n)}$. $v_n$ est alors une suite extraite de $(t_n)$.
%
% Théorème de Bolsano-Weir : Si $(t_n)$ est une suite à valeurs dans un intervalle fermé borné alors on peut extraire de $(t_n)$ une sous-suite convergente.
%
% Soit $t_{\phi(n)}$ une suite extraite de $t_n$ qui converge. Notons $c = \Lim{\infty} t_{\phi(n)}$. On sait que $\Lim{\infty} f(t_{\phi(n)}) = +\infty$. Or, $f$ est continue en $c\in[a,b]$, donc $\Lim{\infty} f(t_{\phi(n)}) = f(\Lim{\infty} t_{\phi(n)}) = f(c)$. Contradiction avec $t_{\phi(n)}$ non bornée. Finalement, si $f$ continue sur $[a,b]$, alors elle est majorée sur $[a,b]$ et aussi minorée.
%
% Soit $m$ le plus grand des minorants de $f([a,b])$, c'est à dire que $\forall t\in[a,b],m\leq f(t)$ t $\forall \varepsilon >0, m+\varepsilon$ ne minore pas $f([a,b])$.
%
% Donc $\exists t\in[a,b], m\leq f(t)<m+\varepsilon$. On démontrer par l'absurde. Supposons par l'absurde que $\forall t \in [a,b], f(t)>m$. Soit alors $g(t) = \frac{1}{f(t)-m}$. $g$ est continue sur $[a,b]$ comme composée de fonctions continues sur $[a,b]$.
%
% or, $\forall n\in\mathbb{N}, m+\frac{1}{n}$ n'est pas minorant de $f([a,b])$, donc $\exists t_n \in[a,b], m<f(t_n)\leq m+\frac{1}{n} \Rightarrow g(t_n) = \frac{1}{f(t_n)-m} \geq \frac{1}{m+\frac{1}{n}-m} = n$, donc $g$ est non majorée.
%
% Conclusion : $H$ est fausse, donc non H est vraie, donc $\exists t_0\in[a,b],f(t_1) = m$ et de m\^eme, $\exists t, \in[a,b] f(t_n) = M$. Alors, d'après le TVI, $\exists x\in[t0,t_1] \in [a,b]$, donc $f([a,b]) = [m,M]$
% \end{myproof}

\subsection{Réciproque d'une application continue strictement monotone}

\begin{theorem}[]
Si $f$ est continue sur $[a,b]$ et strictement monotone sur $[a,b]$, alors $f$  réalise une bijection de $[a,b]$ dans $J=[f(a),f(b)]$ et $f^{-1} : J\to I$ sa récirproque, de m\^eme monotonie sur $J$
\end{theorem}
\subsection{Remarque}
\warningInfo{Non équivalence entre la dérivabilité et la continuité}{Continuité $\neq$ dérivabilité : dérivabilité$ \Rightarrow$ continuité, mais pas l'inverse. La fonction valeur absolue est continue en 0 mais non dérivable en 0.}
\section{Fonctions dérivables}
Soient $a,b \in\mathbb{R}, a<b$.
\subsection{Généralité}
\begin{theorem}[Définition 1]
Soit $x_0 \in[a,b]$.

$f$ est dérivable en $x_0$ si $\Lim{0} \frac{f(x_0+h)-f(x_0)}{h}$ existe et est finie. On note alors cette limite $f(x_0)$.

On dit que $f$ est dérivable sur $[a,b]$ si $f$ est dérivable en tout $x_0 \in[a,b]$, et dérivable à droite en $a$ et à gauche en $b$.

On définit $f' : [a,b]\to\mathbb{R}, x\to f'(x)$.
\end{theorem}
$\frac{f(x_0+h)-f(x_0)}{h}$ est la pente de la corde reliant $f(x_0+h)$ et $f(x_0)$.
\begin{theorem}[Définition alternative et meilleure]
Soit $x_0 \in]a,b[$ $f$ est dérivable en $x_0$ s'il existe $l\in\mathbb{R}$ et une fonction $\varepsilon$ définie au voisinage de 0 tels que $f(x_0+h) = f(x_0) + lh+h\varepsilon(h)$ où $\Lim{0} \varepsilon(h)=0$. On note $l=f'(x_0)$

On a :$f(x) = f(x_0) + f'(x_0)h+h\varepsilon(h)$ et $f(x_0) + f'(x_0)h$ qui est l'équation de la tangente en $x_0$ à la courbe.
\end{theorem}
Preuve rapide :  $f(x_0+h) = f(x_0) + lh+h\varepsilon(h) \iff \varepsilon(h) = \frac{f(x_0+h)-f(x_0)}{h} -l$
\subsection{Dérivées}
$(f\circ u)' = u'\times(f'\circ u)$

\begin{theorem}[Dérivée de la réciproque]
On donne un intervalle $I,J\in\mathbb{R}$ et $f:I\to J$. On suppose $f$ dérivable sur $I$ et que $f$ est bijective de $I\to J$. On note $f^{-1} :J\to I$ la réciproque. Elle est dérivable en $y_0\in J \iff f'(f^{-1}(y_0))\neq 0$

On a alors : $(f^{-1})'_{y_0} = \frac{1}{f'(f^{-1}(y_0))} = \frac{1}{f'(x_0)}$ avec $x_0 = f^{-1}(y_0)$.
\end{theorem}

Rappel : La courbe de $f^{-1}$ est symétrique de la courbe de $f$ par rapport à $y=x$

Moyen mnomotechnique : $f^{-1}\circ f = Id \iff f'(x)(f^{-1})'(f(x)) = 1$
\begin{myproof}
On pose $f^{-1}(y_0) = x_0$ Si $f^{-1}$ est dérivable en $x$.

On a $\Lim{y\to y_0}\frac{f^{-1}(y)-f^{-1}(y_0)}{y-y_0}$.

On pose alors $ y = f(x)$ et $x_0 = f^{-1}(y_0)$. Donc $y\to y_0 \iff x\to x_0$ car les fonctions sont continues.

Donc $\Lim{x\to x_0} \frac{x-x_0}{f(x)-f(x_0)} = \frac{1}{f'(x_0)}$ si $f'(x_0)\neq 0$.
\end{myproof}
\section{Théorème des accroissements finis et de Rolle}
$a<b$
\subsection{Théorème des accroissements finis}
\begin{theorem}[]
Soit $f:[a,b]\to \mathbb{R}$. Si $f$ est continue sur $[a,b]$ et dérivable sur $]a,b[$. Alors $\exists c\in]a,b[, \frac{f(b)-f(a)}{b-a} = f'(c)$
\end{theorem}

\begin{myproof}
On suppose $f$ continue sur $[a,b]$ et dérivable sur $]a,b[$.

On considère une fonction auxilliaire $\varphi(t) = (t-a)(f(b)-f(a)) - (b-a)(f(t)-f(a))$.

$\varphi$ est continue sur $[a,b]$ et dérivable sur $]a,b[$ car $f$ l'est.

$\varphi(b) = 0 = \varphi(a)$

D'après le Théorème de Rolle, il existe $c\in]a,b[, \varphi'(c) = 0$.

Or, $\varphi'(t) = f(b)-f(a) - (b-a)f'(t)$, donc $\varphi'(c) = 0 \iff f'(c) = \frac{f(b)-f(a)}{b-a}$.
\end{myproof}
\subsection{Théorème de Rolle}
\begin{theorem}[]
Soit $f:[a,b]\to \mathbb{R}$. Si $f$ est continue sur $[a,b]$ et dérivable sur $]a,b[$ et $f(a) = f(b)$. Alors $\exists c\in]a,b[, f'(c) = 0$
\end{theorem}

\begin{myproof}
$f$ étant continue sur $[a, b]$, elle est bornée et atteint
sa borne inférieure $\alpha$ et sa borne supérieure $\beta$. Prenons
donc $c$ et $d$ dans $[a, b]$ tels que $f (c) = \alpha$ et $f (d) = \alpha$ .

Si $\alpha = \beta$ , alors la fonction est en fait constante, et donc en tous les points $c\in ]a, b[$, la dérivée s’annule.

Sinon, on a $\alpha \neq \beta$ , et donc l’un des deux est différent de $f (a) = f (b)$. Disons par exemple
que $f (c) = \alpha < f (a) = f (b)$. Donc $c \neq a$ et $c \neq b$, soit $c \in ]a, b[$ et $f'(c) = 0$.

En effet,
$\Lim{x\to x_0^-} \frac{f(x)-f(x_0)}{x-x_0} = \Lim{x\to x_0^-} \frac{f(x)-m}{x-x_0} \leq 0$.

Mais $f'(x_0) = \Lim{x\to x_0^+} \frac{f(x)-f(x_0)}{x-x_0} = \Lim{x\to x_0^+} \frac{f(x)-m}{x-x_0} \geq 0$.

Donc $f'(x_0) \geq 0 $ et $\leq 0$, donc $f'(x_0) = 0$.


% Hypothèse : $f$ continue sur $[a,b]$, $f$ dérivable sur $]a,b[$, $f(a)=f(b)$.
%
% $f$ est contnue sur $[a,b]$ donc $\exists m?M$, tel que $[m,M] = f([a,b])$. Donc en particuluer $\exists x_0x_1\in[a,b]$, $f(x_0) = m$ et $f(x_1) = M$.
%
% Si $m=M$, alors $f = m =M$ et $\forall x\in ]a,b[, f'(x) = 0$
%
% Si $m<M$, alors comme $f(a) = f(b)$. Supposons $f(x_0) = m \neq f(a)=f(b)$, doù $x_0\in]a,b[$.
%
% Montrons que $f'(x_0) = 0$. $\Lim{x\to x_0^-} \frac{f(x)-f(x_0)}{x-x_0} = \Lim{x\to x_0^-} \frac{f(x)-m}{x-x_0} \leq 0$.
%
% Mais $f'(x_0) = \Lim{x\to x_0^+} \frac{f(x)-f(x_0)}{x-x_0} = \Lim{x\to x_0^+} \frac{f(x)-m}{x-x_0} \geq 0$.
%
% Donc $f'(x_0) \geq 0 $ et $\leq 0$, donc $f'(x_0) = 0$.
\end{myproof}
Théorème de Rolle généralisé :

1er cas : f =cst = l, do,c $\forall x_0\in\mathbb{R}, f'(x_0) = 0$

2e cas : $f$ n'est pas constante. Alors $\exists a \in\mathbb{R}, f(a)\neq l$.

Quitte à prendre $if$ au lieu de $f$, on peut choisir $f(a)>l$. (moment ou l'on choisit de faire le max ou le min).

On veut montrer que $f$ admet un max local sur $\mathbb{R}$

$f$ est continue sur un intervalle fermé borné $[-M,M]$ donc atteint ses bornes. $f(-M,M])$.

On veut éliminier les intervalles à l'infini. :$f(a)>l$ et on choisit $ \varepsilon = \frac{f(a)-l}{2}$.

Comme $\Lim{\infty} f = l,\exists M\in\mathbb{R}, |x|>M \Rightarrow |f(x)-l| < \varepsilon$, donc $l-\frac{f(a)-l}{2} < f(x) < l+\frac{f(a)-l}{2} \leq f(a)$.

$\forall x\in ]-\infty,-M]$ ou $[M,+\infty[, f(x)<f(a)$.

$f$ est continue sur $[-M,M]$, donc $\exists x_0 \in ]-M,M[$, $\exists x_1 \in[-M,M]$

$f([-M,M]) = [f(x_1),f(x_0)]$. Comme $f(x_0) \geq f(a)$, on a $\forall x\in\mathbb{R},f(x_0\geq f(x)$. $x_0$ est un max de $f$ sur $\mathbb{R}$. Comme $f$ est dériavble, $f'(x_0) = 0$.
\section{Méthode}
\subsection{Montrer la limite finie d'une fonction en un point $x_0$}
On doit montrer que $\forall \varepsilon >0, \exists \alpha >0, \forall x, 0< |x-x_0|<\alpha \Rightarrow |f(x) - l|<\epsilon$. Pour un $\varepsilon$ donné, il faut donc trouver un $\alpha$.

Pour ce faire, on conserve dans la fonction ce qui tend vers 0 quand $x\rightarrow x_0$ et on majore en valeur absolue la limite du reste par une quantité dépendante de $x_0$ et non $x$.

Le but est d'obtenir une expression de $\alpha$ en fonction de $\varepsilon$.

Exemple : Montrons que $\Lim{x\rightarrow x_0} x^2 \rightarrow x_0^2$.

$\forall \varepsilon >0, \exists ? \alpha >0, \forall x, 0< |x-x_0|<\alpha \Rightarrow |f(x) - l|<\epsilon$

$f(x)-l = x^2-x_0^2 = (x-x_0)(x+x_0)$

$x-x_0$ tend vers 0 et on veut montrer que $(x+x_0)$ est majorée par une quantité indépendante de $x$.

On pose $k_0 = \max(|x_0-1|;|x_0+1|) > 0$.

Donc, si $x\in ]x_0-1;x_0+1[ \iff |x-x_0|<1$, on a $ |x+x_0| \leq 2k_0$.

Donc $|x-x_0||x+x_0| \leq 2k_0|x-x_0| <\varepsilon$ et $|x^2-x_0^2| \leq 2k_0|x-x_0| <\varepsilon$

On tire de l'égalité $2k_0|x-x_0| <\varepsilon$ que $|x-x_0| <\frac{\varepsilon}{2k_0}$.

Donc en choisissant $\alpha = \min(1; \frac{\varepsilon}{2k_0})$, on a bien l'implication souhaitée.
\subsection{Lever une forme indéterminée}
Le but est d'enlever le terme qui ``perturbe notre analyse'' en le factorisant puis en effectuant une simplification.

Ainsi, pour une FI $\frac{\infty}{\infty}$, on met le terme de plus haut degré en facteur.

En revanche, pour une FI $0/0$, on met en facteur $x-$ la valeur pour laquelle l'expression s'annule. Si elle s'annule $0$, on met simplement $x$ en facteur.

On oeut aussi faire par croissance comparée : l'exp l'emporte sur n'importe quelle fraction artionnelle.
\subsection{Étudier une fonction}
\begin{enumerate}
\item Calculer la dérivée
\item Donner le signe de la dérivée
\item  Donner la tableau de variation
\item Calculer les limites et indiquer les asymptotes
\item Tracer la courbe de la fonction
\end{enumerate}

\subsection{Montrer qu'une fonction n'a pas de limite en $x_0$}
On utilise la définition de la limite avec les suites : $\Lim{\infty}f = l$ si $\forall (u_n),\Lim{\infty} U_n = l \Rightarrow \Lim{\infty}f(U_n) = l$. On montre donc qu'il y a 2 suites tendant vers la limite $x_0$ en $\infty$, mais donc la limite de $f$ quand on les injecte est différente.

\checkInfo{Exemple}{
Montrons que $\sin(\frac{1}{x})$ n'a pas de limite en 0.

Posons $u_k = \frac{1}{2k\pi}$ Ici, $\Lim{\infty} u_k = 0$

Posons $v_k = \frac{1}{\frac{\pi}{2} + 2k\pi}$ Ici, $\Lim{\infty} v_k = 0$

Les 2 suites tendent bien vers $0$, pourtant $\Lim{\infty} \sin(\pi/2 + 2k\pi) = 1$ et $\Lim{\infty} \sin(2k\pi) = 0$.

Les limites sont différentes, donc $\sin(\frac{1}{x})$ n'a pas de limite en 0.

}
\subsection{Étudier la continuité d'une fonction en un point}
On donne la limite à droite et à gauche, puis la valeur atteinte par la fonction au point. Si les 3 valeurs sont égales, la fonction est continue en ce point.
\subsection{Étudier la dérivabilité d'une fonction en un point}
Il faut d'abord montrer qu'elle est continue en ce point !

On étudie la limite du taux d'accroissement de la fonction en ce point : $\Lim{x\to x_0} = \frac{f(x)-f(x_0)}{x-x_0}$, que l'on compare avec $f'(x_0)$.

%On dérive la fonction puis on donne la limite de la dérivée en ce point à droite et à gauche, puis la valeur atteinte par la dérivée au point. Si les 3 valeurs sont égales, la fonction est dérivable en ce point.
\subsection{Créer un intervalle fermé borné à partir d'intervalles infinis pour appliquer le TVI}
Si $\Lim{+\infty} = + \infty$, par définition de la limite, $\forall A>0, \exists R>0, x>R \Rightarrow f(x)>A$. En particulier, pour $A = 1$, $\exists R>0, x>R \Rightarrow f(x)>1$. Donc $\exists a = 2R, f(a)>1$

Si $\Lim{-\infty} = - \infty$, par définition de la limite, $\forall A>0, \exists R'>0, x>R' \Rightarrow f(x)<-A$. En particulier, pour $A = 1$, $\exists R'>0, x>R' \Rightarrow f(x)<-1$. Donc $\exists b = 2R, f(b)<-1$

On peut désormais appliquer le TVI sur l'intervalle $[a,b]$.
\subsection{Résoudre un problème impliquant de montrer l'existence d'un intervalle}
On doit montrer qu'il existe un intervalle $\tau$ durant lequel la fonction augmente de $m$ points.
On modélise la problème par une fonction continue sur un certain intervalle $[a,b]$.

Il faut alors montrer que $\exists x\in [a,b], f(x)-f(x-\tau) = m \iff f(x)-f(x-\tau) -m =0$

On pose alors une fonction $g$ sur l'intervalle $[a,b-\tau]$ telle que $g(x) = f(x)-f(x-\tau)-3$

On calcule $g$ aux bornes de son intervalle, $a$ et $b-\tau$ pour montrer que $g(a)$ et $g(x-\tau)$ sont de signe opposé. Comme $g$ est continue, d'après le TVI, il existe $c$ tel que $g(c) = 0$.
\subsection{Utiliser Rolle}
\begin{itemize}
 \item On nous parle de la dérivée

\item Cette dernière doit s'annuler

\item L'intervalle d'annulation de la dérivée est ouvert, donc on utilise Rolle.
\end{itemize}

Utilisation :

\begin{enumerate}
 \item on vérifie qu'elle est continue sur $[a,b]$
 \item On vérifie qu'elle est dérivable sur $]a,b[$.
 \item On calcule $f(a)$ et $f(b)$
\item On conclue : Donc d'après le Théorème de Rolle, $\exists c\in]a,b[, f'(c)=0$.
\end{enumerate}
\subsection{Utiliser la bijectivité pour montrer qu'il existe une unique solution telle que $f(x_0) = c$ sur un intervalle}
$f$ est continue sur l'intervalle, par exemple $\mathbb{R}$ et strictement croissante ou décroissante.

On calcule les images des limites ou bornes de l'intervalle, $[a,b]$.

Donc $f$ réalise une bijection de l'intervalle $\mathbb{R}$ dans  $[a,b]$.

Or, $c\in[a,b]$

Donc $\exists! x_0\in \mathbb{R}, f(x_0) = c$
\end{document}


