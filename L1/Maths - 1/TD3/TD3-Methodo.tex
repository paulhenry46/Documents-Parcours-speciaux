% !TeX spellcheck = en_US
\documentclass[french]{yLectureNote}

\title{Méthodologie}
\subtitle{Langage mathématique}
\author{Paulhenry Saux}
\date{\today}
\yLanguage{Français}

\professor{C.Dartyge}

\usepackage{graphicx}%----pour mettre des images
\usepackage[utf8]{inputenc}%---encodage
\usepackage{geometry}%---pour modifier les tailles et mettre a4paper
%\usepackage{awesomebox}%---pour les boites d'exercices, de pbq et de croquis ---d\'esactiv\'e pour les TP de PC
\usepackage{tikz}%---pour deiffner + d\'ependance de chemfig
\usepackage{tkz-tab}
\usepackage{chemfig}%---pour deiffner formules chimiques
\usepackage{chemformula}%---pour les formules chimiques en \'equation : \ch{...}
\usepackage{tabularx}%---pour dimensionner automatiquement les tableaux avec variable X
\usepackage{awesomebox}%---Pour les boites info, danger et autres
\usepackage{menukeys}%---Pour deiffner les touches de Calculatrice
\usepackage{fancyhdr}%---pour les en-t\^ete personnalis\'ees
\usepackage{blindtext}%---pour les liens
\usepackage{hyperref}%---pour les liens (\`a mettre en dernier)
\usepackage{caption}%---pour la francisation de la l\'egende table vers Tableau
\usepackage{pifont}
\usepackage{array}%---pour les tableaux
\usepackage{lipsum}
\usepackage{yFlatTable}
\usepackage{multicol}
\newcommand{\Lim}[1]{\lim\limits_{\substack{#1}}\:}
\renewcommand{\vec}{\overrightarrow}
\begin{document}

\setcounter{chapter}{2}

	\chapter{Fonctions}
\section{Méthode}
\subsection{Montrer la limite finie d'une fonction en un point $x_0$}
On doit montrer que $\forall \varepsilon >0, \exists \alpha >0, \forall x, 0< |x-x_0|<\alpha \Rightarrow |f(x) - l|<\epsilon$. Pour un $\varepsilon$ donné, il faut donc trouver un $\alpha$.

Pour ce faire, on conserve dans la fonction ce qui tend vers 0 quand $x\rightarrow x_0$ et on majore en valeur absolue la limite du reste par une quantité dépendante de $x_0$ et non $x$.

Le but est d'obtenir une expression de $\alpha$ en fonction de $\varepsilon$.

Exemple : Montrons que $\Lim{x\rightarrow x_0} x^2 \rightarrow x_0^2$.

$\forall \varepsilon >0, \exists ? \alpha >0, \forall x, 0< |x-x_0|<\alpha \Rightarrow |f(x) - l|<\epsilon$

$f(x)-l = x^2-x_0^2 = (x-x_0)(x+x_0)$

$x-x_0$ tend vers 0 et on veut montrer que $(x+x_0)$ est majorée par une quantité indépendante de $x$.

On pose $k_0 = \max(|x_0-1|;|x_0+1|) > 0$.

Donc, si $x\in ]x_0-1;x_0+1[ \iff |x-x_0|<1$, on a $ |x+x_0| \leq 2k_0$.

Donc $|x-x_0||x+x_0| \leq 2k_0|x-x_0| <\varepsilon$ et $|x^2-x_0^2| \leq 2k_0|x-x_0| <\varepsilon$

On tire de l'égalité $2k_0|x-x_0| <\varepsilon$ que $|x-x_0| <\frac{\varepsilon}{2k_0}$.

Donc en choisissant $\alpha = \min(1; \frac{\varepsilon}{2k_0})$, on a bien l'implication souhaitée.
\subsection{Lever une forme indéterminée}
Le but est d'enlever le terme qui ``perturbe notre analyse'' en le factorisant puis en effectuant une simplification.

Ainsi, pour une FI $\frac{\infty}{\infty}$, on met le terme de plus haut degré en facteur.

En revanche, pour une FI $0/0$, on met en facteur $x-$ la valeur pour laquelle l'expression s'annule. Si elle s'annule $0$, on met simplement $x$ en facteur.

On oeut aussi faire par croissance comparée : l'exp l'emporte sur n'importe quelle fraction artionnelle.
\subsection{Étudier une fonction}
\begin{enumerate}
\item Calculer la dérivée
\item Donner le signe de la dérivée
\item  Donner la tableau de variation
\item Calculer les limites et indiquer les asymptotes
\item Tracer la courbe de la fonction
\end{enumerate}

\subsection{Montrer qu'une fonction n'a pas de limite en $x_0$}
On utilise la définition de la limite avec les suites : $\Lim{\infty}f = l$ si $\forall (u_n),\Lim{\infty} U_n = l \Rightarrow \Lim{\infty}f(U_n) = l$. On montre donc qu'il y a 2 suites tendant vers la limite $x_0$ en $\infty$, mais donc la limite de $f$ quand on les injecte est différente.

\checkInfo{Exemple}{
Montrons que $\sin(\frac{1}{x})$ n'a pas de limite en 0.

Posons $u_k = \frac{1}{2k\pi}$ Ici, $\Lim{\infty} u_k = 0$

Posons $v_k = \frac{1}{\frac{\pi}{2} + 2k\pi}$ Ici, $\Lim{\infty} v_k = 0$

Les 2 suites tendent bien vers $0$, pourtant $\Lim{\infty} \sin(\pi/2 + 2k\pi) = 1$ et $\Lim{\infty} \sin(2k\pi) = 0$.

Les limites sont différentes, donc $\sin(\frac{1}{x})$ n'a pas de limite en 0.

}
\subsection{Étudier la continuité d'une fonction en un point}
On donne la limite à droite et à gauche, puis la valeur atteinte par la fonction au point. Si les 3 valeurs sont égales, la fonction est continue en ce point.
\subsection{Étudier la dérivabilité d'une fonction en un point}
Il faut d'abord montrer qu'elle est continue en ce point !

On étudie la limite du taux d'accroissement de la fonction en ce point : $\Lim{x\to x_0} = \frac{f(x)-f(x_0)}{x-x_0}$, que l'on compare avec $f'(x_0)$.

%On dérive la fonction puis on donne la limite de la dérivée en ce point à droite et à gauche, puis la valeur atteinte par la dérivée au point. Si les 3 valeurs sont égales, la fonction est dérivable en ce point.
\subsection{Créer un intervalle fermé borné à partir d'intervalles infinis pour appliquer le TVI}
Si $\Lim{+\infty} = + \infty$, par définition de la limite, $\forall A>0, \exists R>0, x>R \Rightarrow f(x)>A$. En particulier, pour $A = 1$, $\exists R>0, x>R \Rightarrow f(x)>1$. Donc $\exists a = 2R, f(a)>1$

Si $\Lim{-\infty} = - \infty$, par définition de la limite, $\forall A>0, \exists R'>0, x>R' \Rightarrow f(x)<-A$. En particulier, pour $A = 1$, $\exists R'>0, x>R' \Rightarrow f(x)<-1$. Donc $\exists b = 2R, f(b)<-1$

On peut désormais appliquer le TVI sur l'intervalle $[a,b]$.
\subsection{Résoudre un problème impliquant de montrer l'existence d'un intervalle}
On doit montrer qu'il existe un intervalle $\tau$ durant lequel la fonction augmente de $m$ points.
On modélise la problème par une fonction continue sur un certain intervalle $[a,b]$.

Il faut alors montrer que $\exists x\in [a,b], f(x)-f(x-\tau) = m \iff f(x)-f(x-\tau) -m =0$

On pose alors une fonction $g$ sur l'intervalle $[a,b-\tau]$ telle que $g(x) = f(x)-f(x-\tau)-3$

On calcule $g$ aux bornes de son intervalle, $a$ et $b-\tau$ pour montrer que $g(a)$ et $g(x-\tau)$ sont de signe opposé. Comme $g$ est continue, d'après le TVI, il existe $c$ tel que $g(c) = 0$.
\subsection{Utiliser Rolle}
\begin{itemize}
 \item On nous parle de la dérivée

\item Cette dernière doit s'annuler

\item L'intervalle d'annulation de la dérivée est ouvert, donc on utilise Rolle.
\end{itemize}

Utilisation :

\begin{enumerate}
 \item on vérifie qu'elle est continue sur $[a,b]$
 \item On vérifie qu'elle est dérivable sur $]a,b[$.
 \item On calcule $f(a)$ et $f(b)$
\item On conclue : Donc d'après le Théorème de Rolle, $\exists c\in]a,b[, f'(c)=0$.
\end{enumerate}
\subsection{Utiliser la bijectivité pour montrer qu'il existe une unique solution telle que $f(x_0) = c$ sur un intervalle}
$f$ est continue sur l'intervalle, par exemple $\mathbb{R}$ et strictement croissante ou décroissante.

On calcule les images des limites ou bornes de l'intervalle, $[a,b]$.

Donc $f$ réalise une bijection de l'intervalle $\mathbb{R}$ dans  $[a,b]$.

Or, $c\in[a,b]$

Donc $\exists! x_0\in \mathbb{R}, f(x_0) = c$
\end{document}


