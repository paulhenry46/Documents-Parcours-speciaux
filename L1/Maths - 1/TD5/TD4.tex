% !TeX spellcheck = en_US
\documentclass[french]{yLectureNote}

\title{Mathématiques}
\subtitle{Langage mathématique}
\author{Paulhenry Saux}
\date{\today}
\yLanguage{Français}

\professor{C.Dartyge}

\usepackage{graphicx}%----pour mettre des images
\usepackage[utf8]{inputenc}%---encodage
\usepackage{geometry}%---pour modifier les tailles et mettre a4paper
%\usepackage{awesomebox}%---pour les boites d'exercices, de pbq et de croquis ---d\'esactiv\'e pour les TP de PC
\usepackage{tikz}%---pour deiffner + d\'ependance de chemfig
\usepackage{tkz-tab}
\usepackage{chemfig}%---pour deiffner formules chimiques
\usepackage{chemformula}%---pour les formules chimiques en \'equation : \ch{...}
\usepackage{tabularx}%---pour dimensionner automatiquement les tableaux avec variable X
\usepackage{awesomebox}%---Pour les boites info, danger et autres
\usepackage{menukeys}%---Pour deiffner les touches de Calculatrice
\usepackage{fancyhdr}%---pour les en-t\^ete personnalis\'ees
\usepackage{blindtext}%---pour les liens
\usepackage{hyperref}%---pour les liens (\`a mettre en dernier)
\usepackage{caption}%---pour la francisation de la l\'egende table vers Tableau
\usepackage{pifont}
\usepackage{array}%---pour les tableaux
\usepackage{lipsum}
\usepackage{yFlatTable}
\usepackage{multicol}
\newcommand{\Lim}[1]{\lim\limits_{\substack{#1}}\:}
\renewcommand{\vec}{\overrightarrow}
\DeclareMathOperator{\card}{card}
\begin{document}

\setcounter{chapter}{4}

	\chapter{Combinatoire et dénombrement}
\section{Cardinalité}
\subsection{Introduction}
\begin{theorem}[Équipotence]
2 ensembles $E$ et $F$ sont dits équipotents (écrit $E\sim F$) s'il existe une bijection $\varphi$ de $E\to F$
\end{theorem}
\begin{theorem}[Propriétés de la relation équipotence]
Elle est réflexive : $\forall E, E\sim E$. $\varphi = Id_E$

Elle est symétrique : $E\sim F \Rightarrow F\sim E$ car $\varphi^{-1} F\to E$ est bijective

Emme est transitive : $\forall E,F,G$, si $E\sim F$ et $F\sim G$, alors $E\sim G$ car si $\varphi : E\to F$ et $\psi :F\to G$ bijectives alors $\psi \circ \varphi : E\to G$ est bijective car composée de bijections.
\end{theorem}
\checkInfo{Conséquences}{
$E$ et $F$ ont le m\^eme cardinal $\iff E\sim F$}
\section{Ensembles finis}
\subsection{Notations}
Pour $n\in\mathbb{N}$, on définit $n!$ par $0!=1$, $1!=1$, $(n+1)! = n!\times (n+1)$
\subsection{Définition}
\begin{theorem}[Définition]
$E$ est fini si $E=\varnothing$ ou s'il existe $n\in\mathbb{N}$, tel que $E \sim [\![1;n]\!]$
\end{theorem}
\subsection{Popriétés et propositions}
\subsubsection{Lemmes fondamentaux}
\begin{lemma}
Il existe une injection de $ [\![1;n]\!]  \to [\![n,m]\!] \iff n\leq m$\label{injection_fini}
\end{lemma}
\begin{lemma}
Il existe une bijection de $[\![1;m]\!]  \to [\![1,n]\!] \iff n=m$\label{bijection_fini}
\end{lemma}

Si $E\neq \varnothing$, alors les Lemme \ref{injection_fini} et \ref{bijection_fini} montrent l'unicité de l'entier $n$ tel que $E\sim [\![1;n]\!]$. On écrit alors $\card(E) = |E| = n$ et si $E=\varnothing, n=0$

Rmq : Si $E$ et $F$ sont de cardinal finis, alors $E\sim F \Rightarrow \card(E) = \card(F)$
\subsubsection{cardinalité et surjectivité}
Soient $E$ et $F$ 2 ensembles finis (c'est faux avec des ensembles infinis). Soit $f:E\to F$ une application.
\begin{itemize}
 \item Si $f$ est injective, $Card(E)\leq \card(F)$
  \item Si $f$ est surjective, $\card(E)\geq \card(F)$
    \item Si $\card(E) = \card(F)$, $f$ est bijective $\iff$ injective $\iff$ surjective
\end{itemize}

\subsubsection{Autres propositions}

Soit $E$ un ensemble fini et $A\subset E$, alors
\begin{itemize}
 \item $A$ est fini
 \item $\card(A) \leq \card(E)$
 \item $A=E \iff \card(A) = \card(E)$
\end{itemize}

Soit $E$ un ensemble fini et $A_1\to A_k$ le sous-ensemble de $E$ tels que $A_1\cap A_j = \varnothing$ si $i\neq j$. Alors $\cup^k_{i=1} = \sum^k_{i=1} \card(A_i)$

Soit $A,B \subset E$ : $\card(A\cup B) = \card(A) + \card(B) - \card(A\cap B)$

$\card(A\setminus B) = \card(A) - \card(A\cap B)$
\subsubsection{Lemme des bergers}
\begin{lemma}[Lemme des bergers]

Principe : Si un ensemble E possède une partition en p sous-ensembles contenant chacun r éléments, alors E contient$ p \times r$ éléments.
Soit $E$ un ensemble fini et $F$ un ensemble et soit $f:E\to F$ une application.\label{bergers}

Compter les éléments de $E$ revient à compter les éléments de l'image réciproque de f

$\card(E) = \sum_{y\in F} = \card(f^{-1}(\{y\})$

En effet, $f$ est une application donc $\forall y \neq y' \in F$, $f^{-1}(\{y\}) \cap f^{-1}(\{y'\}) = \varnothing$ %car $x\in \cap \Rightarrow f(x) = y = f(x) = y'$ Contradiction

$\forall x\in E, f(x)$ existe donc $x\in f^{-1}(\{f(x)\})$. D'où $\cup_{y\in F} f^{-1}(\{y\}) = E$

D'où $\card(E) = \sum \card(f^{-1}(\{y\}))$ car $f^{-1}(\{y\}) \subset E$ donc de cardinal fini et comme E est fini, $\{y\in F, \card(f^{-1}(\{y\})) \neq 0\}$ est également fini
\end{lemma}
\subsubsection{Principe des tiroirs}
\begin{lemma}[Principe des tiroirs]

Soit $E,F$ 2 ensembles finis tel que $card(E) \geq \card(F)$ alors il n'existe pas d'affectation injective de $E\to F$, i.e soit $f:E\to F$, $\exists y\in F,$ tel que  $f^{-1}(\{y\})$ contient 2 éléments.\label{tiroirs}
\end{lemma}
\section{Analyse combinatoire}
\begin{theorem}[]\label{comb_1}
Le nombre d'applications de X vers Y, de cardinaux respectifs $n,p\geq 1$ est $p^n$
\end{theorem}
\subsection{Fonction caractéristique}
\begin{theorem}[Fonction caractéristique]
Soit $X$ un ensemble et $A$ une partie de $X$. Une fonction caractéristique de $A$ est l'application $\chi_A$ de $X$ vers $\{0,1\}$, prenant la valeur $1$ sur $A$ et $0$ sur $X\backslash A$.
\end{theorem}
La fonction prend la valeur $1$ si $x \in A$ et $0$ si $x\notin A$
\begin{theorem}[]
L'application  $A\to \chi_A$ est une bijection de l'ensemble des parties de $X$ vers ( l'ensemble des applications de $X$ vers $\{0,1\}$.)
\end{theorem}
\begin{theorem}[]
L'ensemble des parties de $X$ est fini de cardinal $2^n$
\end{theorem}
\begin{myproof}
D'après le théorème précédant, comme l'application $A\to \chi_A$ est bijective, $card(P(x) = $ card(l'ensemble des applications de $X$ vers $\{0,1\}$.).
En appliquant le théorème \ref{comb_1},
en prenant $Y = \{0.1\}$ et $p = 2$, on obtient $2^n$.
\end{myproof}
\subsection{Arrangement}
Un arrangement parmis $n$ objets est une suite de $p$ objets distincts pris parmi les $n$ objets donnés. On note le nombre d'arangements $A^p_n$. Trouver $A$ revient à trouver le nombre d'injections de $[1,p]$ dans l'ensemble des $n$ objets.

\begin{theorem}[]
\[A^p_n = n(n-1)\dots(n-p+1) = \frac{n!}{(n-p)!}\]
\end{theorem}
\begin{myproof}
Prenons ici la première égalité comme définition de $A^p_n$.  Pour tout entier p, notons $P(p)$ la propriété suivante :
pour tout ensemble X de cardinal $p$ et tout ensemble fini Y de cardinal $n\geq p$  le cardinal de l'ensemble des injections de $X\to Y$, noté $I(X,Y)$ est $A^p_n$.

$P(0)$ est vraie : $\forall$ ensembles $Z$, il y a une seule application de l'ensemble vide dans $Z$, et elle est injective.

Soit maintenant $p\geq 1$ un entier tel que $P(p-1)$ est vraie. Nous devons montrer qu'alors le cardinal de $I(X,Y)$ est $A^p_n$.

Fixons un élément $a$ de $X$, et soit $X' = X \backslash \{a\}$, qui est de cardinal $p-1$.

À toute injection $f$ de $I(X,Y)$, on peut associer bijectivement le couple $(g,b)$ où $g$ est la restriction de $f$ à $X'$ et $b$ est $f(a)$.

On a donc : $card(I(X,Y)) = card(I(X',Y)) \times (n-p+1) = A^{p-1}_n (n-p+1) = A^p_n$

Note : $card(f(a)$ correspond à la dernière possibilité, une fois que celles de $X'$ sont prises. C'est pourquoi son cardinal est $(n-p+1)$
\end{myproof}
\begin{theorem}[]
Le nombre de bijection d'un ensemble X vers Y de m\^eme cardinal est $A^n_n = n!$.
\end{theorem}
\subsection{Combinaisons}
On appelle combinaison de $n$ éléments de $X$ pris $p$ à $p$ toute partie de $X$ à $p$ éléments. On le note $C^p_n$. L'ordre n'a pas d'importance.
\begin{theorem}[]
Le nombre de parties à $p$ éléments de $X$ est $C^p_n = \frac{n!}{p!(n-p)!} = \frac{A^p_n}{p!}$.
\end{theorem}
\begin{myproof}
Si $f : [1,p] \to X$ est une injection, $f([1,p])$ est une partie à $p$ éléments de $X$, d'où une application $\psi : f\to f([1,p])$ de $I([1,p],X)$ dans $P_p(X)$.

Soit $B$ une partie à $p$ éléments de $X$.

$\psi^{-1}(\{B\})$ est formée des injections $f : [1,p] \to X$ ayant pour image $B$, i.e. des bijections de $[1,p]$ sur $B$. D'après le théorème précédant, la réciproque est donc de cardinal $p!$.

On a alors : $A^p_n = card(I([1,p],X)) = \sum_{B\in P(X)} p! card(P(X))$ et il vient $card(P(X)) = \frac{A^p_n}{p!}$.
\end{myproof}
\subsubsection{Propriétés}
\begin{itemize}
 \item $C^0_n = 1$
 \item $C^n_n = 0$
 \item $C^1_n = n$
 \item Propriété de symétrie : $C^{n-p}_n = C^p_n$
 \item Triangle de Pascal : $C^p_n + C^{p+1}_n = C^{p+1}_{n+1}$
 \item Nombre de parties d'un ensemble : $\sum^n_{k=0} C^p_n = 2^n$
\end{itemize}
\subsection{Autres propriétés}
\begin{itemize}
 \item Cardinal des parties d'un ensemble à $n$ éléments : $2^n$
 \item Nombre de fonction d'un ensemble à $k$ éléments vers un ensemble à $n$ éléments : $n^k$
 \item Nombre de bijections d'un ensemble à $n$ éléments vers un ensemble à $n$ éléments : $n!$. Il s'agit du nombre $n$-uplet de l'un des ensemble, où l'ordre des éléments comptent.
\end{itemize}
\section{En résumé}
\subsection{Tirage}
%:-+-+-+- Engendré par : http://math.et.info.free.fr/TikZ/Arbre/
\begin{center}
% Racine à Gauche, développement vers la droite
\begin{tikzpicture}[xscale=1,yscale=1]
% Styles (MODIFIABLES)
\tikzstyle{fleche}=[->,>=latex,thick]
\tikzstyle{noeud}=[fill=warningColor,circle,draw]
\tikzstyle{feuille}=[fill=informationColor,circle,draw]
\tikzstyle{etiquette}=[midway,fill=white,draw]
% Dimensions (MODIFIABLES)
\def\DistanceInterNiveaux{3}
\def\DistanceInterFeuilles{1}
% Dimensions calculées (NON MODIFIABLES)
\def\NiveauA{(0)*\DistanceInterNiveaux}
\def\NiveauB{(1.6666666666666665)*\DistanceInterNiveaux}
\def\NiveauC{(3)*\DistanceInterNiveaux}
\def\NiveauD{(4)*\DistanceInterNiveaux}
\def\InterFeuilles{(-1)*\DistanceInterFeuilles}
% Noeuds (MODIFIABLES : Styles et Coefficients d'InterFeuilles)
\node[noeud] (R) at ({\NiveauA},{(2)*\InterFeuilles}) {Avec remise};
\node[noeud] (Ra) at ({\NiveauB},{(0.5)*\InterFeuilles}) {Ordonné};
\node[feuille] (Raa) at ({\NiveauC},{(0)*\InterFeuilles}) {$n^k$};
\node[feuille] (Rab) at ({\NiveauC},{(1)*\InterFeuilles}) {$K^p_n$};
\node[noeud] (Rb) at ({\NiveauB},{(3)*\InterFeuilles}) {Ordonné};
\node[noeud] (Rba) at ({\NiveauC},{(2.5)*\InterFeuilles}) {$n=p$ ?};
\node[feuille] (Rbaa) at ({\NiveauD},{(2)*\InterFeuilles}) {$A^p_n$};
\node[feuille] (Rbab) at ({\NiveauD},{(3)*\InterFeuilles}) {$n!$};
\node[feuille] (Rbb) at ({\NiveauC},{(4)*\InterFeuilles}) {$C^p_n$};
% Arcs (MODIFIABLES : Styles)
\draw[fleche] (R)--(Ra) node[etiquette] {Oui};
\draw[fleche] (Ra)--(Raa) node[etiquette] {Oui};
\draw[fleche] (Ra)--(Rab) node[etiquette] {Non};
\draw[fleche] (R)--(Rb) node[etiquette] {Non};
\draw[fleche] (Rb)--(Rba) node[etiquette] {Oui};
\draw[fleche] (Rba)--(Rbaa) node[etiquette] {Non};
\draw[fleche] (Rba)--(Rbab) node[etiquette] {Oui};
\draw[fleche] (Rb)--(Rbb) node[etiquette] {Non};
\end{tikzpicture}
\end{center}
%:-+-+-+-+- Fin
\subsection{Rangement}
%:-+-+-+- Engendré par : http://math.et.info.free.fr/TikZ/Arbre/
\begin{center}
% Racine à Gauche, développement vers la droite
\begin{tikzpicture}[xscale=1,yscale=1]
% Styles (MODIFIABLES)
\tikzstyle{fleche}=[->,>=latex,thick]
\tikzstyle{noeud}=[fill=warningColor,circle,draw]
\tikzstyle{feuille}=[fill=informationColor,circle,draw]
\tikzstyle{etiquette}=[midway,fill=white,draw]
% Dimensions (MODIFIABLES)
\def\DistanceInterNiveaux{3}
\def\DistanceInterFeuilles{1}
% Dimensions calculées (NON MODIFIABLES)
\def\NiveauA{(0)*\DistanceInterNiveaux}
\def\NiveauB{(1.6666666666666665)*\DistanceInterNiveaux}
\def\NiveauC{(3)*\DistanceInterNiveaux}
\def\NiveauD{(4)*\DistanceInterNiveaux}
\def\InterFeuilles{(-1)*\DistanceInterFeuilles}
% Noeuds (MODIFIABLES : Styles et Coefficients d'InterFeuilles)
\node[noeud] (R) at ({\NiveauA},{(2)*\InterFeuilles}) {$\geq 1$ dans une case};
\node[noeud] (Ra) at ({\NiveauB},{(0.5)*\InterFeuilles}) {Discernable};
\node[feuille] (Raa) at ({\NiveauC},{(0)*\InterFeuilles}) {$n^k$};
\node[feuille] (Rab) at ({\NiveauC},{(1)*\InterFeuilles}) {$K^p_n$};
\node[noeud] (Rb) at ({\NiveauB},{(3)*\InterFeuilles}) {Discernable};
\node[noeud] (Rba) at ({\NiveauC},{(2.5)*\InterFeuilles}) {$n=p$ ?};
\node[feuille] (Rbaa) at ({\NiveauD},{(2)*\InterFeuilles}) {$A^p_n$};
\node[feuille] (Rbab) at ({\NiveauD},{(3)*\InterFeuilles}) {$n!$};
\node[feuille] (Rbb) at ({\NiveauC},{(4)*\InterFeuilles}) {$C^p_n$};
% Arcs (MODIFIABLES : Styles)
\draw[fleche] (R)--(Ra) node[etiquette] {Oui};
\draw[fleche] (Ra)--(Raa) node[etiquette] {Oui};
\draw[fleche] (Ra)--(Rab) node[etiquette] {Non};
\draw[fleche] (R)--(Rb) node[etiquette] {Non};
\draw[fleche] (Rb)--(Rba) node[etiquette] {Oui};
\draw[fleche] (Rba)--(Rbaa) node[etiquette] {Non};
\draw[fleche] (Rba)--(Rbab) node[etiquette] {Oui};
\draw[fleche] (Rb)--(Rbb) node[etiquette] {Non};
\end{tikzpicture}
\end{center}
%:-+-+-+-+- Fin
\begin{theorem}[Formules]
\begin{itemize}
 \item $A^p_n = \frac{n!}{(n-p)!}$
 \item $C^p_n = \frac{n!}{p!(n-p)!}$
 \item $K^p_n = \frac{(n+p-1)!}{p!(n-1)!}$
\end{itemize}

\end{theorem}

\end{document}
