% !TeX spellcheck = en_US
\documentclass[french]{yLectureNote}

\title{Mathématiques}
\subtitle{Langage mathématique}
\author{Paulhenry Saux}
\date{\today}
\yLanguage{Français}

\professor{C.Dartyge}

\usepackage{graphicx}%----pour mettre des images
\usepackage[utf8]{inputenc}%---encodage
\usepackage{geometry}%---pour modifier les tailles et mettre a4paper
%\usepackage{awesomebox}%---pour les boites d'exercices, de pbq et de croquis ---d\'esactiv\'e pour les TP de PC
\usepackage{tikz}%---pour deiffner + d\'ependance de chemfig
\usepackage{tkz-tab}
\usepackage{chemfig}%---pour deiffner formules chimiques
\usepackage{chemformula}%---pour les formules chimiques en \'equation : \ch{...}
\usepackage{tabularx}%---pour dimensionner automatiquement les tableaux avec variable X
\usepackage{awesomebox}%---Pour les boites info, danger et autres
\usepackage{menukeys}%---Pour deiffner les touches de Calculatrice
\usepackage{fancyhdr}%---pour les en-t\^ete personnalis\'ees
\usepackage{blindtext}%---pour les liens
\usepackage{hyperref}%---pour les liens (\`a mettre en dernier)
\usepackage{caption}%---pour la francisation de la l\'egende table vers Tableau
\usepackage{pifont}
\usepackage{array}%---pour les tableaux
\usepackage{lipsum}
\usepackage{yFlatTable}
\usepackage{multicol}
\newcommand{\Lim}[1]{\lim\limits_{\substack{#1}}\:}
\renewcommand{\vec}{\overrightarrow}
\begin{document}

\setcounter{chapter}{3}

	\chapter{Intégrales}
\section{Primitives}

\subsection{Définition}
Soit $f: [a,b] \to \mathbb{R}$ une fontion définie sur $[a,b]$ et soit $F : [a,b] \to \mathbb{R}$ une fonction. $F$ est une primitive de $f$ si $F$ est dérivable sur $[a,b]$ et $F' = f$, i.e $\forall x\in[a,b], F'(x) = f(x)$.

\subsection{Propriétés}
\warningInfo{Existence des primitives}{Il n'existe pas forcément une primitive aux fonctions.}
Si $F$ et $G$ sont 2 primitives de $f$ sur $[a,b]$, alors $\exists k\in\mathbb{R},\forall x\in[a,b],F(x) = G(x)+k$, i.e $f$ et $g$ diffèrent d'une constante
\section{Techniques}
\subsection{Intégration par parties}
Soient $u,v$ 2 fonctions $C^1$\marginElement{\marginTitle{Classe $C^n$} $f$ est $C^n$ sur $[a,b]$ si $f$ est $n$ fois dérivables sur $[a,b]$ et $f^n$ est continue sur $[a,b]$. $C^{\infty}$ si $\forall n\in\mathbb{N}, f^n$ existe.} (continues de dérivée continues) sur $[a,b]$. Alors
\begin{theorem}[Formule]
\[\int_a^b u(x) v'(x) \, dx  = \Big[u(x) v(x)\Big]_a^b - \int_a^b u'(x) v(x) \, dx\]
\end{theorem}
\subsection{Changement de variable dans une intégrale}
\begin{theorem}[]
Soit $f : [a,b]\to \mathbb{R}$ une fonction intégrable

Soit $\varphi [\alpha,\beta] \to [a,b]$ une application bijective et $C^1$. On note $\varphi^{-1}$ l'application réciproque

Alors $\int^b_a f(t)\mathrm{d}t = \int^{\varphi^{-1}(b)}_{\varphi^{-1}(a)} f(\varphi(u))\varphi'(u)\mathrm{d}u$.
\end{theorem}
\subsection{Méthode}
\warningInfo{Vérifications à faire}{Avant d'intégrer, il faut toujours vérifier que la fonction est intégrable, c'est à dire qu'elle est monotone ou continue sur $[a,b]$.}
\subsubsection{Première méthode}
Il faut que la fonction $u$ choisie soit bijective pour appliquer cette méthode !
\begin{enumerate}
 \item On pose $t = \varphi(u)$
 \item $\mathrm{d}t = \varphi'(u)\mathrm{d}u$
 \item Valeurs aux bornes $t=a \Rightarrow u = \varphi^{-1}(a)$ et $t=b \Rightarrow u = \varphi^{-1}(b)$
\end{enumerate}
\subsubsection{Variante}
Variante dans le calcul de primitive. On n'exige pas le fait que cela soit bijectif

Soit $f$ une fonction continue sur $[a,b]$

\[\int f(t)\mathrm{d}t = \int f\circ \varphi (u) \varphi'(u)\mathrm{d}u\]

On pose $t = \varphi(u)$ et $\mathrm{d}t = \varphi'(u)\mathrm{d}u$

Si $F = \int f$ et $f$ continue sur $[a,b]$, $(F\circ \varphi)' = F'\circ \varphi \times \varphi' = f\circ \varphi \times \varphi'$
\subsection{En pratique}
\subsubsection{Exemple 1}
Calculons : \[\int^1_0 \sqrt{e^x-1}\]

On va effectuer un changement de variable pour tenter d'enlever la racine.

On pose donc $u=\sqrt{e^x-1}$. Calculons maintenant $\mathrm{d}u$ pour en déduire $\mathrm{d}x$ : $\mathrm{d}u = \frac{e^x}{2\sqrt{e^x-1}}\mathrm{d}x = \frac{e^x-1 +1}{2u}\mathrm{d}x = \frac{u^2 +1}{2u}\mathrm{d}x$.\marginTips{Le but est de simplifier l'expression au maximum et l'exprimer le plus possible en fonction de $u$.}
Finalement, on obtient $\mathrm{d}x = \frac{2u}{u^2+1} \mathrm{d}u$
\infoInfo{Remarque}{Comme la fonction $u$ est bijective, on aurait aussi pu calculer sa réciproque pour obtenir une nouvelle expression de $x$ : $u = \sqrt{e^x-1} \iff u^2 = e^x-1 \iff u^2+1 = e^x \iff x= \ln(u^2+1)$. On peut ensuite exprimer directement $\mathrm{d}x = \frac{2u}{u^2+1}\mathrm{d}u$}

On applique maintenant la fonction $u$\marginCritical{et non sa réciproque !} aux bornes de l'intégrale : On obtient $x=0 \Rightarrow u(0) =0, x=\ln(2)\Rightarrow u(\ln(2)) = 1$.

On peut maintenant reécrire l'intégrale :
\begin{flalign*}
I &= \int^1_0 u \times \frac{2u}{u^2+1}\mathrm{d}u\\
&= \int^1_0  \frac{2u^2}{u^2+1}\mathrm{d}u\\
&= 2\int^1_0  \frac{u^2 -1+1}{u^2+1}\mathrm{d}u\\
&= 2\int^1_0  1-\frac{1}{u^2+1}\mathrm{d}u\\
&= 2[u -\tan^{-1}(u)]^1_0\\
&= 2-\frac{2\pi}{4}\\
\end{flalign*}
\subsubsection{Exemple 2}
Calculons : \[\int \sin^5(x)\cos^3(x)\mathrm{d}x\]

On pose $u=\sin(x)$. Calculons maintenant $\mathrm{d}u$ pour en déduire $\mathrm{d}x$ : $\mathrm{d}u = \cos(x)\mathrm{d}x \iff \mathrm{d}x = \frac{\mathrm{d}u}{\cos(x)}$.
\warningInfo{Remarque}{Comme la fonction $u$ n'est pas bijective, on ne peut pas écrire de manière équivalente que $x=\arcsin(u)$}
On peut maintenant reécrire l'intégrale :
\begin{flalign*}
F(x) &= \int \sin^5(x)\cos^3(x)\mathrm{d}x\\
&= \int u^5\cos(x)^3\frac{\mathrm{d}u}{\cos(x)}\\
&=\int u^5\cos(x)^2\mathrm{d}u\\
&=\int u^5(1-\sin(x)^2)\mathrm{d}u\\
&=\int u^5(1-u^2)\mathrm{d}u\\
&=\int u^5-u^7\mathrm{d}u\\
&= [\frac{u^6}{6}-\frac{u^8}{8}] = [\frac{\sin(x)^6}{6}-\frac{\sin(x)^8}{8}]
\end{flalign*}
\subsection{Formulaire}
\begin{multicols}{2}
\begin{tabular}{_l^l}
\tableHeaderStyle%
Fonction & Primitive\\
$x^n$ & $\frac{1}{n+1}x^{n+1}+k$ si $n\neq -1$\\
$\frac{1}{x^n}$ & $-\frac{1}{(n-1)x^{n-1}}+k$\\
$\frac{a}{x}$ & $a \ln x+k$\\
$\frac{1}{\sqrt{x}}$ & $2\sqrt{x}+k$\\
$\cos x$ & $\sin x +k$\\
$\sin x$ & $-\cos x +k$\\
$e^x$ & $e^x +k$\\
$u'u^n$ & $\frac{1}{n+1}u^{n+1}+k$\\
$\frac{u'}{u^n}$ & $-\frac{1}{n-1}\times\frac{1}{u^{n-1}}+k$\\
$\frac{u'}{\sqrt{u}}$ & $2\sqrt{u}+k$
\end{tabular}

\begin{tabular}{_l^l}
\tableHeaderStyle%
Fonction & Primitive\\
$u'\cos u$ & $\sin u+k$\\
$u'\sin u$ & $-\cos u +k$\\
$\frac{u'}{u}$ & $\ln u +k$\\
$u'\sqrt{u}$ & $\frac{2}{3}(u)^{3/2} + k$\\
$u'e^u$ & $e^u$\\
$u'\cosh u$ & $\sinh u$\\
$u'\sinh u$ & $\cosh u$\\
$\frac{-1}{\sqrt{1-x^2}}$& $\cos^{-1}$\\
$\frac{1}{\sqrt{1-x^2}}$& $\sin^{-1}$\\
$\frac{1}{1+x^2}$& $\tan^{-1}$
\end{tabular}
\end{multicols}
% \section{Intégrales de Riemann}
% Les fonction $f$ désignent une fonction définie et bornée sur $[a,b]$. C'est à dire les fonctions continues ou monotones.
% \subsection{Somme de Riemann}
% \subsubsection{Idée générale}
% Idée : On sait calculer l'aire d'un rectangle, donc de plusieurs rectangles
%
% Si c'est intégrable, la plus petite aire des fonctions escalier vaut la plus grande.
% \subsubsection{Intégrales des fontions en escalier subidvision d'un intervalle}
% \begin{theorem}[Définition]
% Une subdivision de l'intervalle [a,b] est une suite finie de réels strictement croissant, $\sigma = (x_k)$ et $x_0=a<x_1<x_2<\dots<x_{n-1}<x_n=b$
%
% Le pas de la subdivision $\delta$ est le nombre $\max_{i=0\dots,n-1} (x_{i+1}-x_i)$
% \end{theorem}
%
%
% Exemple : Subidivision régulière : $\delta = \frac{b-a}{n}$. Alors $x_0 = a, x_1 = a+\frac{b-a}{n}, x_2 = a+2\frac{b-a}{n}, x_k = a+k\frac{b-a}{n}$.
% \begin{theorem}[Définition d'une fonction en escalier]
% Une fonction $\varphi$ est dite enescalier s'il existe sur [a,b] une subdivision $\sigma=(x_k)$ de $[a,b]$ telle que $\varphi$ est constante sur $]x_i,x_{i-1}[, \forall i \in {0,\dots,-n-1},$ i.e. $\forall i\in {0,\dots,-n-1},\exists \varphi_i \in\mathbb{R}$ telle que $\varphi(t) = \varphi_i \forall t\in ]x_i,x_{i+1}[$.
%
% On défini alors pour une telle fonction $\varphi$ le nombre $S(\varphi,\sigma) = \sum_{k=0}^{n-1} \varphi_k(x_{kh}-x_k)$ = aire sous la courbe en escalier
% \end{theorem}
% \begin{theorem}[Proposition]
% $S(\varphi,\sigma)$ ne dépend pas de $\sigma$. On note alors $S(\varphi,\sigma) = S(\varphi)$ qui est l'aire sous la courbe de $y=\varphi(t)$
%
% On note alors $\int^b_a \varphi(t)\mathrm{d}t = S(\varphi)$
% \end{theorem}
\section{Propriétés}
\subsection{Propriétés}
Soient $\varphi,\psi$ 2 fonction en escalier

\begin{itemize}
 \item $\int$ est linéaire : $\int_b^a \varphi + \lambda \psi(t) \mathrm{d}t = \int_b^a \mathrm{d}t + \lambda \int_b^a \psi(t) \mathrm{d}t$
 \item $\in^a_b$ est croissante sur les fonctions. Donc si $\varphi \geq 0$, alors $\int^b_a\varphi(t)\mathrm{d}t \geq 0$ et si $\varphi \leq \psi, \int \varphi \leq \int \psi$
 \item Inégalité triangulaire : $|\int_b^a \varphi(t)| \leq \int^a_b |\varphi(t)|$
 \item Relation de Chasle : $\int^y_x\varphi + \int^z_y = \int^z_x\varphi$ : $\int^x_x \varphi = 0$ et $\int^x_y\varphi = -\int^y_x \varphi$.
\end{itemize}
\subsection{Intégrale d'une fonction définie sur [a,b] et bornée sur [a,b]}
\begin{theorem}[Définition]
Une fonction $f$ est intégrable sur $[a,b]$ s'il existe $\forall \varepsilon >0$ 2 fonctions en escalier $\varphi,\psi$, telles que $\varphi \leq f\leq \psi$ et $\int^b_a (\psi(t)-\varphi(t) \mathrm{d}t < \varepsilon$
\end{theorem}
Dans ce cas, on peut définir le plus grand des minorants et le plus petit des majorants, qui sont égaux : $\int^b_a\varphi = \int^b_a\psi$.
\subsection{Autres}
\begin{theorem}[Proposition]
L'intégrale des fonctions intégrables sur [a,b] présente les m\^emes propriétés que l'intégrale des fonctions en escalier, c'est à dire linéarité, croissance, inégalité triangulaire, relation de Chasle.
\end{theorem}
\subsection{Théorème fondamental de l'analyse}
\begin{theorem}[Théorème]
Soit $f$ une fonction de $[a;b]\to\mathbb{R}$ continue sur $[a,b]$. Alors, la fonction $F : [a,b]\to\mathbb{R}, x\to \int^x_a f(t_\mathrm{d}t$ est une primitive de $f$. Donc $F$ est dérivable sur $[a,b]$ et $F'=f$. De plus, si $G$ est une primitive de $f$ alors $\int^b_af(t)\mathrm{d}t = G(b)-G(a)$.
\end{theorem}




\end{document}


