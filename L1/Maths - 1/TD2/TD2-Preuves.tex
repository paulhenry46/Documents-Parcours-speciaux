% !TeX spellcheck = en_US
\documentclass[french]{yLectureNote}

\title{Mathématiques}
\subtitle{Langage mathématique}
\author{Paulhenry Saux}
\date{\today}
\yLanguage{Français}

\professor{C.Dartyge}

\usepackage{graphicx}%----pour mettre des images
\usepackage[utf8]{inputenc}%---encodage
\usepackage{geometry}%---pour modifier les tailles et mettre a4paper
%\usepackage{awesomebox}%---pour les boites d'exercices, de pbq et de croquis ---d\'esactiv\'e pour les TP de PC
\usepackage{tikz}%---pour deiffner + d\'ependance de chemfig
\usepackage{tkz-tab}
\usepackage{chemfig}%---pour deiffner formules chimiques
\usepackage{chemformula}%---pour les formules chimiques en \'equation : \ch{...}
\usepackage{tabularx}%---pour dimensionner automatiquement les tableaux avec variable X
\usepackage{awesomebox}%---Pour les boites info, danger et autres
\usepackage{menukeys}%---Pour deiffner les touches de Calculatrice
\usepackage{fancyhdr}%---pour les en-t\^ete personnalis\'ees
\usepackage{blindtext}%---pour les liens
\usepackage{hyperref}%---pour les liens (\`a mettre en dernier)
\usepackage{caption}%---pour la francisation de la l\'egende table vers Tableau
\usepackage{pifont}
\usepackage{array}%---pour les tableaux
\usepackage{lipsum}
\usepackage{yFlatTable}
\usepackage{multicol}
\newcommand{\Lim}[1]{\lim\limits_{\substack{#1}}\:}
\renewcommand{\vec}{\overrightarrow}
\begin{document}

\setcounter{chapter}{1}

	\chapter{Suites réelles }

\section{Limites d'une suite}
Soit u une suite réelle.

\begin{theorem}[Unicité d'une limite d'une suite convergente]
Si $(U_n)$ est convergente, sa limite l est unique on note $l = \Lim{n+\infty} u_n$.
\end{theorem}

\begin{myproof}[Démonstration par l'absurde]
Supposons que $(U_n)$ admette $l_1$ et $l_2$ comme limite, avec $l_1\neq l_2$. Nous avons donc, pour tout $\epsilon > 0$ :
\begin{flalign}
\exists &N_1\in\mathbb{N}, \forall n\geq N_1, |u_n-l_1| \leq \epsilon\\
\exists &N_2\in\mathbb{N}, \forall n\geq N_2, |u_n-l_2| \leq \epsilon
\end{flalign}
On pose alors $\epsilon = \frac{|l_1-l_2|}{3}$. Il existe donc $N_1$ et $N_2$ tel que les 2 assertions sont vraies.

Choisissons un nombre entier supérieur à  $N_1$ et $N_2$, comme $\max(N_1,N_2)$.

Pour cette valeur de $n$, nous avons à la fois $|u_n-l_1|<\epsilon$ et $|u_n-l_2|<\epsilon$.

Par l'inégalité triangulaire, on obtient :
\[3\epsilon = |l_1-l_2| = |(u_n-l_2)-(u_n-l_1)| \leq |u_n-l_1|+|u_n-l_2| \leq 2\epsilon\]

Le nombre réel vérifie à la fois $\epsilon >0$ et $3\epsilon \leq 2\epsilon$, ce qui est absurde. Donc $l_1=l_2$.
% Supposons qu'il existe 2 limites $l_1$ et $l_2$ et que $(U_n)$ converge.
%
% $|l_1-l_2| = |l_1-u_n+u_n-l_2| \leq |l_1-u_n|+|u_n-l_2|$.
%
% $(U_n)$ converge vers $l_1$, donc $\forall \epsilon > 0, \exists N_1 \in \mathbb{N}, n \geq N_1 \Rightarrow |u_n-l_1| < \epsilon$.
%
% $(U_n)$ converge vers $l_2$, donc $\forall \epsilon > 0, \exists N_2 \in \mathbb{N}, n \geq N_1 \Rightarrow |u_n-l_2| < \epsilon$.
%
% Donc pour $n\geq N_1+N_2$ et $|l_1-l_2| \leq \epsilon + \epsilon = 2 \epsilon$.
%
% Donc $|l_1-l_2| < \epsilon$ et $|l_1-l_2| = 0$ donc $l_1 = l_2$ et la limite est unique.
\end{myproof}

\begin{theorem}[Borne d'une suite convergente]
Toute suite convergente est bornée
\end{theorem}
\begin{myproof}
Soit $(U_n)_{n\in\mathbb{N}}$ une suite réelle convergente. Notons l sa limite $\in \mathbb{R}$.
Elle est convergente $\iff \forall \epsilon > 0, \exists N_{\epsilon} \in \mathbb{N}, \forall n \in \mathbb{N}, n\geq N_{\epsilon} \Rightarrow |u_n-l|\leq \epsilon$.

Pour $\epsilon = 1$ :
\begin{flalign*} \exists N_1 \in \mathbb{N}, n \geq N_1 &\Rightarrow |u_n-l| \leq 1\\
&\Rightarrow -1 \leq u_n-l\leq 1\\
&\Rightarrow -1+l \leq u_n\leq 1+l\\
\end{flalign*}

Donc pour $n\geq N_1, |u_n| \leq \max(|-1+l|, |1+l|)$

Donc pour $n<N_1$, le nombre de terme de $(U_n)$ est fini.

Donc $M = \max(|u_k|)$ existe et est fini, avec $0\leq k\leq N_1-1$ et $|U_n| \leq M, \forall 0 \leq n\leq N_1-1$.
\end{myproof}
\section{Limites}
\subsection{Suites adjacentes}

\begin{theorem}[Convergence de suites adjacentes]
Si les suites $(U_n)$ et $(V_n)$ sont adjacentes, alors elles sont convergentes de m\^eme limite.

\end{theorem}
\begin{myproof}
$\forall n\in\mathbb{N}, u_n\geq v_n\geq v_0$ car $(v_n)$ est croissante. Donc $(U_n)$ est décroissante et minorée par $v_0$ donc convergente. Notons $l_1 = \Lim{+\infty} U_n$.

De m\^eme, $\forall n\in\mathbb{N}v_n\leq u_n\leq, u_0$ car $(u_n)$ est décroissante. Donc $(v_n)$ est croissante et majorée par $v_0$ donc convergente. Notons $l_2 = \Lim{+\infty} v_n$.

De plus, $\Lim{+\infty} U_n-V_n = 0 = l_1-l_2 \iff l_1 = l_2$

Donc, si elles sont adjacentes, $(U_n)$ et $(V_n)$ convergent vers un unique $l$.
\end{myproof}

\subsubsection{Somme des limites}
\begin{myproof}[Somme des limites l et l']
On veut savoir si $\forall\epsilon > 0, \exists ? N \in \mathbb{N}, n\geq N \Rightarrow |u_n + v_n - (l+l')|< \epsilon$.

On sait que d'après l'inégalité triangulaire, $|u_n + v_n - (l+l')| = |u_n -l + v_n -l'| \leq |u_n-l|+|v_n-l'|$.

Or $\Lim{\infty} u_n = l$, donc $\exists N_1\in\mathbb{N},  n \geq N_1$, alors $|u_n-l| \leq \frac{\epsilon}{2}$.

De m\^eme, $\Lim{\infty} v_n = l'$, donc $\exists N_2\in\mathbb{N},  n \geq N_2$, alors $|u_n-l'| \leq \frac{\epsilon}{2}$

Donc si $n\geq N_1+N_2$, alors $|u_n-l| \leq \frac{\epsilon}{2}$ et $|v_n-l'| \leq \frac{\epsilon}{2}$

Donc $|u_n + v_n - (l+l')| \leq \frac{\epsilon}{2} + \frac{\epsilon}{2} = \epsilon$. Donc $u_n+v_n$ est convergente et $\Lim{\infty} u_n+v_n = l$.
\end{myproof}
\subsubsection{Produit des limites}
\begin{myproof}[Produit des limites l et l']
Soit $(u_n), (v_n)$ 2 suites convergentes. On veut démontrer que $\Lim{\infty}u_nv_n = \Lim u_n \Lim{\infty}v_n$. Notons que $\Lim{\infty}u_n = l \in \mathbb{R}$ et $\Lim{\infty}v_n = l' \in \mathbb{R}$. On doit démontrer que la définition de la limite existe pour la suite $u_nv_n$, i.e, $\forall n\in\mathbb{N}, n\geq N_0 \Rightarrow |u_nv_n-ll'|\leq \epsilon >0$.

\begin{flalign*}
|u_nv_n-ll'| &= |(u_n-l)v_n+lv_n-ll'|\\
&= |v_n(u_n-l)+l(v_n-l')|\\
&\leq |v_n||u_n-l|+|l||v_n-l'|
\end{flalign*}
$(v_n)$ convergen, donc est bornée,  donc, $\exists M>0, \forall n\in\mathbb{N} |v_n|\leq M$.
\begin{flalign*}
&\leq M|u_n-l|+|l||v_n-l'|\\
&\leq M\frac{\epsilon}{2M}+|l|\frac{\epsilon}{2M+1+|l|}
\end{flalign*}
On a donc : \begin{itemize}
             \item $\Lim{\infty} u_n = l$, donc pour $\epsilon' = \frac{\epsilon}{2M} >0, \exists N_1 \in \mathbb{N}$, donc $n\geq N_1 \Rightarrow |u_n-l| < \frac{\epsilon}{2M}$.
             \item $\Lim{\infty} v_n = l'$, donc pour $\epsilon'' = \frac{\epsilon}{2(1+|l|)} >0, \exists N_2 \in \mathbb{N}$, donc $n\geq N_2 \Rightarrow |u_n-l| < \frac{\epsilon}{2(1+|l|)}$.
            \end{itemize}

            Donc $\forall n\geq N_1+N_2$, on a $|u_nv_n-ll'| \leq M\frac{\epsilon}{2M}+|l|\frac{\epsilon}{2M+1+|l|}$, puis $|u_nv_n-ll'| \leq \frac{\epsilon}{2}+1$ et  $|u_nv_n-ll'| \leq \epsilon$

            On a bien l'inégalité, CQFD
\end{myproof}
\subsection{Limites et inégalités}
\begin{theorem}[Théorème des gendarmes]
Si on a 3 suites réelles avec $u_n\leq w_n\leq v_n$. Si $u_n$ et $v_n$ sont convergentes de m\^eme limite $l$, alors $w_n$ est convergente vers $l$.
\end{theorem}
\begin{myproof}
Soit $\epsilon >0$, on cherche $N_0 \in\mathbb{N}, n\geq N_0 \Rightarrow |w_n-l|\leq \epsilon$, avec $\Lim{\infty}u_n = l =\Lim{\infty}v_n$ et  $u_n\leq w_n\leq v_n, \forall n\in\mathbb{N}$.

\begin{flalign*}
 |w_n-l|&= |w_n-u_n+u_n-l|\\
 &\leq |w_n-u_n|+|u_n-l|\\
 &\leq |v_n-u_n|+|u_n-l|\\
 &\leq |v_n-l+l-u_n|+|u_n-l|\\
 &\leq |v_n-l|+|l-u_n|+|u_n-l|\\
 &\leq |v_n-l|+2|u_n-l|\\
 &\leq \frac{\epsilon}{2}+2\frac{\epsilon}{4}\\
\end{flalign*}
On a : $\Lim{\infty} u_n = l$, donc $\exists N\in\mathbb{N},n\geq N_1 \Rightarrow |u_n-l|\leq \frac{\epsilon}{4} $ et $\Lim{\infty} v_n = l$, donc $\exists N\in\mathbb{N},n\geq N_1 \Rightarrow |v_n-l|\leq \frac{\epsilon}{2}$.

Donc pour $N_3 = N_1+N_2$, on a : $n\geq N_0 \Rightarrow |w_n-l| \leq \frac{\epsilon}{2} + 2\times \frac{\epsilon}{4} = \epsilon$. Ce qui démontre bien l'égalité souhaitée.
\end{myproof}
\end{document}


