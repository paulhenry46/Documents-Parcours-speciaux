% !TeX spellcheck = en_US
\documentclass[french]{yLectureNote}

\title{Mathématiques}
\subtitle{Langage mathématique}
\author{Paulhenry Saux}
\date{\today}
\yLanguage{Français}

\professor{C.Dartyge}

\usepackage{graphicx}%----pour mettre des images
\usepackage[utf8]{inputenc}%---encodage
\usepackage{geometry}%---pour modifier les tailles et mettre a4paper
%\usepackage{awesomebox}%---pour les boites d'exercices, de pbq et de croquis ---d\'esactiv\'e pour les TP de PC
\usepackage{tikz}%---pour deiffner + d\'ependance de chemfig
\usepackage{tkz-tab}
\usepackage{chemfig}%---pour deiffner formules chimiques
\usepackage{chemformula}%---pour les formules chimiques en \'equation : \ch{...}
\usepackage{tabularx}%---pour dimensionner automatiquement les tableaux avec variable X
\usepackage{awesomebox}%---Pour les boites info, danger et autres
\usepackage{menukeys}%---Pour deiffner les touches de Calculatrice
\usepackage{fancyhdr}%---pour les en-t\^ete personnalis\'ees
\usepackage{blindtext}%---pour les liens
\usepackage{hyperref}%---pour les liens (\`a mettre en dernier)
\usepackage{caption}%---pour la francisation de la l\'egende table vers Tableau
\usepackage{pifont}
\usepackage{array}%---pour les tableaux
\usepackage{lipsum}
\usepackage{yFlatTable}
\usepackage{multicol}
\newcommand{\Lim}[1]{\lim\limits_{\substack{#1}}\:}
\renewcommand{\vec}{\overrightarrow}
\begin{document}

\setcounter{chapter}{1}

	\chapter{Suites réelles }
\section{Généralités}
\subsection{Définition}
Une suite réelle u est une application de $\mathbb{N} \rightarrow \mathbb{R}$ et $n \longmapsto u_n$. On écrit $u_n$ au lieu de $u(n)$

$u_n$ est le nieme terme de la suite.

On écrit aussi $u = (u_n)_{n\in\mathbb{N}}$.\marginCritical{$u_n$ ne désigne pas la suite $u_n$. Il faut écrire $(u_n)_{n\in\mathbb{N}}$.}

Les suites $(u_n)$ et $v_n$ sont égales si $\forall n\in \mathbb{N}, u_n = v_n$.

On note :
$S(\mathbb{R}) = \mathbb{R}^{\mathbb{N}}$ l'ensemble des suites réelles.

Exemple

$u = (n+1)_{n\in\mathbb{N}}$

$0 = (0)_{n\in\mathbb{N}}$

$v = ((-1)^n)_{n\in\mathbb{N}}$

\subsection{Opérations sur les suites}
On définit sur $S(\mathbb{R})$ les opérations $+,\times$ et $\cdot$

$u+v = (u_n+v_n)_{n\in\mathbb{N}}$

$uv = (u_nv_n)_{n\in\mathbb{N}}$

$\lambda u = (\lambda u)_{n\in\mathbb{N}}$

Pour diviser par $(v_n)$, elle doit \^etre tout le temps non nulle.

\subsection{Variations}
Soit $u\in S(\mathbb{R})$ et $n_0 \in \mathbb{N}$.

\begin{theorem}[Suite croissante]
La suite est croissante à partir du rang $n_0$ si $\forall n\geq n_0, u_{n+1} \geq u_n$ et u est stictement croissante à partir de $n_0$ si $n\geq n_0$ on a $u_{n+1} > u_n$.
\end{theorem}

\begin{theorem}[Suite décroissante]
La suite est décroissante à partir du rang $n_0$ si $\forall n\geq n_0, u_{n+1} \leq u_n$ et u est stictement croissante à partir de $n_0$ si $n\geq n_0$ on a $u_{n+1} < u_n$.
\end{theorem}

Exemple : $(n^2)_{n\in\mathbb{N}}$ croissante

$(\frac{1}{n})_{n\in\mathbb{N}}$ décroissante

$\neg(u_n$ est croissante à partir du premier terme)$ = (\exists n\in\mathbb{N}, u_{n+1}<u_n) $

\begin{theorem}[Majoré et minorant]
La suite est majorée si $\exists M\in\mathbb{R}, \forall n\in\mathbb{N}, u_n\leq M$.

La suite est minorée si $\exists m\in\mathbb{R}, \forall n\in\mathbb{N}, u_n\geq m$.

La suite est bornée si elle est minorée et majorée.
\end{theorem}

u bornée $\iff \exists M\in \mathbb{R}^+, \forall n\in \mathbb{N}, |u_n|\leq M$.

\subsubsection{Valeur absolue}
$|x| = x$ si $x>0$ et $-x$ si $x<0$.

$|x-y|$ mesure la distance e x à y.

$|xy| = |x||y|$

$|x+y|\leq |x|+|y|$ (inégalité triangulaire)
\begin{myproof}
Preuve : On étudie la différence des carrés : $(|x|+|y|)^2 - (|x+y|)^2 = x^2+2|x||y| + y^2 - (x^2+xy+y^2) = 2(|xy|-xy)\geq 0$. Donc $(|x|+|y|)^2 \geq (|x+y|)^2$. Or la fonction $x^2$ est croissante, donc l'inégalité est vérifiée.
\end{myproof}
$|x| = 0 \rightarrow x =0$


\section{Limites d'une suite}
Soit u une suite réelle.

\begin{theorem}[Définition]
On dit que la suite u est convergente (CV) si existe $l\in\mathbb{R}$ tel que $\forall \epsilon > 0,  \exists N\in\mathbb{N}, \forall n\in\mathbb{N}, n\geq N \Rightarrow  |u_n-l| < \epsilon$. On dit que l est une limite de $(U_n)$.
\end{theorem} Voir schéma
\begin{theorem}[Unicité d'une limite d'une suite convergente]
Si $(U_n)$ est convergente, sa limite l est unique on note $l = \Lim{n+\infty} u_n$.
\end{theorem}

\begin{myproof}[Démonstration par l'absurde]
Supposons que $(U_n)$ admette $l_1$ et $l_2$ comme limite, avec $l_1\neq l_2$. Nous avons donc, pour tout $\epsilon > 0$ :
\begin{flalign}
\exists &N_1\in\mathbb{N}, \forall n\geq N_1, |u_n-l_1| \leq \epsilon\\
\exists &N_2\in\mathbb{N}, \forall n\geq N_2, |u_n-l_2| \leq \epsilon
\end{flalign}
On pose alors $\epsilon = \frac{|l_1-l_2|}{3}$. Il existe donc $N_1$ et $N_2$ tel que les 2 assertions sont vraies.

Choisissons un nombre entier supérieur à  $N_1$ et $N_2$, comme $\max(N_1,N_2)$.

Pour cette valeur de $n$, nous avons à la fois $|u_n-l_1|<\epsilon$ et $|u_n-l_2|<\epsilon$.

Par l'inégalité triangulaire, on obtient :
\[3\epsilon = |l_1-l_2| = |(u_n-l_2)-(u_n-l_1)| \leq |u_n-l_1|+|u_n-l_2| \leq 2\epsilon\]

Le nombre réel vérifie à la fois $\epsilon >0$ et $3\epsilon \leq 2\epsilon$, ce qui est absurde. Donc $l_1=l_2$.
% Supposons qu'il existe 2 limites $l_1$ et $l_2$ et que $(U_n)$ converge.
%
% $|l_1-l_2| = |l_1-u_n+u_n-l_2| \leq |l_1-u_n|+|u_n-l_2|$.
%
% $(U_n)$ converge vers $l_1$, donc $\forall \epsilon > 0, \exists N_1 \in \mathbb{N}, n \geq N_1 \Rightarrow |u_n-l_1| < \epsilon$.
%
% $(U_n)$ converge vers $l_2$, donc $\forall \epsilon > 0, \exists N_2 \in \mathbb{N}, n \geq N_1 \Rightarrow |u_n-l_2| < \epsilon$.
%
% Donc pour $n\geq N_1+N_2$ et $|l_1-l_2| \leq \epsilon + \epsilon = 2 \epsilon$.
%
% Donc $|l_1-l_2| < \epsilon$ et $|l_1-l_2| = 0$ donc $l_1 = l_2$ et la limite est unique.
\end{myproof}

On utilise Valeur absolue d'une expression pour la majorer par une valeur. Il vaut mieux dire que $|(-1)^n|\leq 1$ que $-1\leq|(-1)^n|\leq 1$. Utiliser ensuite l'inégalité triangulaire.

\begin{theorem}[Borne d'une suite convergente]
Toute suite convergente est bornée
\end{theorem}
\begin{myproof}
Soit $(U_n)_{n\in\mathbb{N}}$ une suite réelle convergente. Notons l sa limite $\in \mathbb{R}$.
Elle est convergente $\iff \forall \epsilon > 0, \exists N_{\epsilon} \in \mathbb{N}, \forall n \in \mathbb{N}, n\geq N_{\epsilon} \Rightarrow |u_n-l|\leq \epsilon$.

Pour $\epsilon = 1$ :
\begin{flalign*} \exists N_1 \in \mathbb{N}, n \geq N_1 &\Rightarrow |u_n-l| \leq 1\\
&\Rightarrow -1 \leq u_n-l\leq 1\\
&\Rightarrow -1+l \leq u_n\leq 1+l\\
\end{flalign*}

Donc pour $n\geq N_1, |u_n| \leq \max(|-1+l|, |1+l|)$

Donc pour $n<N_1$, le nombre de terme de $(U_n)$ est fini.

Donc $M = \max(|u_k|)$ existe et est fini, avec $0\leq k\leq N_1-1$ et $|U_n| \leq M, \forall 0 \leq n\leq N_1-1$.
\end{myproof}
\section{Limites}
\subsection{Limites et monotonie}
\begin{theorem}[]
Toute suite croissante et majorée converge vers son plus petit majorant

Toute suite décroissante et minorée converge vers son plus grand minorant.
\end{theorem}
\infoInfo{Remarque}{Si $(U_n)$ est croissante et si $\Lim{+\infty} U_n = l \in\mathbb{R}$, alors $\forall n\in \mathbb{N}, u_n \leq l$ car $(U_n)$ est croissante..


En effet, si $x\leq u_n$ et si $u_{n+1}>u_n$, alors $\epsilon = U_{n+1}-U_n >0$ et
$\forall n\in\mathbb{N}, n>n_0, u_n, \geq x+\frac{\epsilon}{2}$ et $|u_n-x|>\frac{\epsilon}{2}$. Donc $x$ ne peut \^etre la limite de $(U_n)$.

De plus, $\forall \epsilon >0, l-\epsilon$ n'est pas la limite de $(U_n)$ dnc $\exists n\in\mathbb{N}, l-\epsilon < u_n \leq l$ et $l$ est bien le plus petit des minorants de $(U_n)$.
}
\subsection{Suites adjacentes}
\begin{theorem}[Définition]
2 suites $(U_n)$ et $(V_n)$ sont adjacentes si
\begin{itemize}
 \item $(U_n)$ est décroissante
 \item $(V_n)$ est croissante
 \item $\Lim{+\infty} U_n-V_n = 0$
 \item $u_n\geq v_n, \forall n\in\mathbb{N}$.
\end{itemize}
\end{theorem}

\begin{theorem}[Définition]
Si les suites $(U_n)$ et $(V_n)$ sont adjacentes, alors elles sont convergentes de m\^eme limite.

\end{theorem}
\begin{myproof}
$\forall n\in\mathbb{N}, u_n\geq v_n\geq v_0$ car $(v_n)$ est croissante. Donc $(U_n)$ est décroissante et minorée par $v_0$ donc convergente. Notons $l_1 = \Lim{+\infty} U_n$.

De m\^eme, $\forall n\in\mathbb{N}v_n\leq u_n\leq, u_0$ car $(u_n)$ est décroissante. Donc $(v_n)$ est croissante et majorée par $v_0$ donc convergente. Notons $l_2 = \Lim{+\infty} v_n$.

De plus, $\Lim{+\infty} U_n-V_n = 0 = l_1-l_2 \iff l_1 = l_2$

Donc, si elles sont adjacentes, $(U_n)$ et $(V_n)$ convergent vers un unique $l$.
\end{myproof}
\subsection{Limites infinies}
Soit $(u_n)$ une suite réelle.
\begin{theorem}[définition]
Elle a pour limite $+\infty$ si $\forall A>0, \exists N_A \in \mathbb{N}, \forall n, n\geq N_A \Rightarrow u_n \geq A$.

Elle a pour limite $-\infty$ si $\forall A>0, \exists N_A \in \mathbb{N}, \forall n, n\geq N_A \Rightarrow u_n \leq -A$.
\end{theorem}
\subsection{Suites et opérations}
On considère 2 suites réelles $(u_n)$ et $(v_n)$. On suppose que $\Lim{+\infty} = l/\pm\infty$.
\subsubsection{Somme des limites}
\begin{center}
\begin{tabular}{_l^l^l^l}
\tableHeaderStyle
Limite $(v_n)$ & $-\infty$ & l & $+\infty$\\
$-\infty$ & $-\infty$ & $-\infty$ & FI\\
l' & $-\infty$ & l+l' & $+\infty$\\
$+\infty$ & FI & $+\infty$ & $+\infty$
\end{tabular}
\end{center}
\subsubsection{Produit des limites}
\begin{center}
\begin{tabular}{_l^l^l^l^l}
\tableHeaderStyle
Limite $(v_n)$ et $(u_n)$ & $-\infty$ & l & 0 & $+\infty$\\
$-\infty$ & $+\infty$ & signe de $l$$ \times -\infty$ & FI &$-\infty$\\
l' & signe de $l'$ $ \times -\infty$  & $ll'$ & 0 &  signe de $l'$ $ \times +\infty$\\
0 & FI & 0 & 0&FI\\
$+\infty$ & $-\infty$ & signe de $l'$ $ \times +\infty$ & FI&$+\infty$
\end{tabular}
\end{center}
Voir tableau des limites
\begin{myproof}[Somme des limites l et l']
On veut savoir si $\forall\epsilon > 0, \exists ? N \in \mathbb{N}, n\geq N \Rightarrow |u_n + v_n - (l+l')|< \epsilon$.

On sait que d'après l'inégalité triangulaire, $|u_n + v_n - (l+l')| = |u_n -l + v_n -l'| \leq |u_n-l|+|v_n-l'|$.

Or $\Lim{\infty} u_n = l$, donc $\exists N_1\in\mathbb{N},  n \geq N_1$, alors $|u_n-l| \leq \frac{\epsilon}{2}$.

De m\^eme, $\Lim{\infty} v_n = l'$, donc $\exists N_2\in\mathbb{N},  n \geq N_2$, alors $|u_n-l'| \leq \frac{\epsilon}{2}$

Donc si $n\geq N_1+N_2$, alors $|u_n-l| \leq \frac{\epsilon}{2}$ et $|v_n-l'| \leq \frac{\epsilon}{2}$

Donc $|u_n + v_n - (l+l')| \leq \frac{\epsilon}{2} + \frac{\epsilon}{2} = \epsilon$. Donc $u_n+v_n$ est convergente et $\Lim{\infty} u_n+v_n = l$.


\end{myproof}
\begin{myproof}[Produit des limites l et l']
Soit $(u_n), (v_n)$ 2 suites convergentes. On veut démontrer que $\Lim{\infty}u_nv_n = \Lim u_n \Lim{\infty}v_n$. Notons que $\Lim{\infty}u_n = l \in \mathbb{R}$ et $\Lim{\infty}v_n = l' \in \mathbb{R}$. On doit démontrer que la définition de la limite existe pour la suite $u_nv_n$, i.e, $\forall n\in\mathbb{N}, n\geq N_0 \Rightarrow |u_nv_n-ll'|\leq \epsilon >0$.

\begin{flalign*}
|u_nv_n-ll'| &= |(u_n-l)v_n+lv_n-ll'|\\
&= |v_n(u_n-l)+l(v_n-l')|\\
&\leq |v_n||u_n-l|+|l||v_n-l'|
\end{flalign*}
$(v_n)$ convergen, donc est bornée,  donc, $\exists M>0, \forall n\in\mathbb{N} |v_n|\leq M$.
\begin{flalign*}
&\leq M|u_n-l|+|l||v_n-l'|\\
&\leq M\frac{\epsilon}{2M}+|l|\frac{\epsilon}{2M+1+|l|}
\end{flalign*}
On a donc : \begin{itemize}
             \item $\Lim{\infty} u_n = l$, donc pour $\epsilon' = \frac{\epsilon}{2M} >0, \exists N_1 \in \mathbb{N}$, donc $n\geq N_1 \Rightarrow |u_n-l| < \frac{\epsilon}{2M}$.
             \item $\Lim{\infty} v_n = l'$, donc pour $\epsilon'' = \frac{\epsilon}{2(1+|l|)} >0, \exists N_2 \in \mathbb{N}$, donc $n\geq N_2 \Rightarrow |u_n-l| < \frac{\epsilon}{2(1+|l|)}$.
            \end{itemize}

            Donc $\forall n\geq N_1+N_2$, on a $|u_nv_n-ll'| \leq M\frac{\epsilon}{2M}+|l|\frac{\epsilon}{2M+1+|l|}$, puis $|u_nv_n-ll'| \leq \frac{\epsilon}{2}+1$ et  $|u_nv_n-ll'| \leq \epsilon$

            On a bien l'inégalité, CQFD
\end{myproof}
\subsection{Limites et inégalités}
\begin{theorem}[Inégalités]
Supposons que $u_n\leq v_n$. On a : $\Lim{\infty}u_n = +\infty \Rightarrow \Lim{\infty}v_n = +\infty$ et $\Lim{\infty}v_n = -\infty \Rightarrow \Lim{\infty}u_n = -\infty$
\end{theorem}
Si les 2 suites sont convergentes, $\Lim{\infty}u_n \leq \Lim{\infty}v_n$.\marginCritical{Les inégalités strictes deviennent larges quand on passe à la limite. Par exemple, $\forall n\in\mathbb{N}, u_n>0$ et $u_n$ converge, alors $\Lim{\infty}u_n \geq 0$.}
\begin{theorem}[Théorème des gendarmes]
Si on a 3 suites réelles avec $u_n\leq w_n\leq v_n$. Si $u_n$ et $v_n$ sont convergentes de m\^eme limite $l$, alors $w_n$ est convergente vers $l$.
\end{theorem}
\begin{myproof}
Soit $\epsilon >0$, on cherche $N_0 \in\mathbb{N}, n\geq N_0 \Rightarrow |w_n-l|\leq \epsilon$, avec $\Lim{\infty}u_n = l =\Lim{\infty}v_n$ et  $u_n\leq w_n\leq v_n, \forall n\in\mathbb{N}$.

\begin{flalign*}
 |w_n-l|&= |w_n-u_n+u_n-l|\\
 &\leq |w_n-u_n|+|u_n-l|\\
 &\leq |v_n-u_n|+|u_n-l|\\
 &\leq |v_n-l+l-u_n|+|u_n-l|\\
 &\leq |v_n-l|+|l-u_n|+|u_n-l|\\
 &\leq |v_n-l|+2|u_n-l|\\
 &\leq \frac{\epsilon}{2}+2\frac{\epsilon}{4}\\
\end{flalign*}
On a : $\Lim{\infty} u_n = l$, donc $\exists N\in\mathbb{N},n\geq N_1 \Rightarrow |u_n-l|\leq \frac{\epsilon}{4} $ et $\Lim{\infty} v_n = l$, donc $\exists N\in\mathbb{N},n\geq N_1 \Rightarrow |v_n-l|\leq \frac{\epsilon}{2}$.

Donc pour $N_3 = N_1+N_2$, on a : $n\geq N_0 \Rightarrow |w_n-l| \leq \frac{\epsilon}{2} + 2\times \frac{\epsilon}{4} = \epsilon$. Ce qui démontre bien l'égalité souhaitée.
\end{myproof}
Si une suite est encadrée, la limite de la suite l'est aussi.
La limite est un point fixe de $(u_n)$.
\checkInfo{Partie entière}{$\forall x\in\mathbb{R}, \exists! n\in\mathbb{N}$, tel que $n\leq x<n+1$. C'est noté $E(x)=n$.}
\section{Méthodes}
\subsection{Lever une Forme Indéterminée}
\subsubsection{Du type Infini/Infini}
On donne d'abord le type de FI puis on met en facteur le terme variant de plus haut degré en facteur.
\checkInfo{Exemple}{Voir le 2 et 3 du TD 2.4

\begin{flalign*}
U_n &= \frac{2n+(-1)^n}{5n + (-1)^{n+1}}\\
&= \frac{2n(1+\frac{(-1)^n}{2n})}{5n(1+\frac{(-1)^{n+1}}{5n})}\\
&= \frac{2(1+\frac{(-1)^n}{2n})}{5(1+\frac{(-1)^{n+1}}{5n})}\\
\end{flalign*}

$\Lim{+\infty} \frac{(-1)^n}{2n} = \Lim{+\infty} \frac{(-1)^n}{2n} = 0$

Donc $\Lim{+\infty} (U_n) = \frac{2}{5} \times \frac{1}{1}$.
}
\subsubsection{Du type + Infini - Infini avec des racines}
On multiplie par la quantité conjuguée au numérateur et au dénominateur

\[ (\sqrt{a}-\sqrt{b}) = (\sqrt{a}-\sqrt{b})\frac{\sqrt{b}-\sqrt{a}}{\sqrt{b}-\sqrt{a}} = \frac{a-b}{\sqrt{a}+\sqrt{b}}  \]
\checkInfo{Exemple}{Voir le 4 du TD 2.4
\begin{flalign*} (\sqrt{2n+1}-\sqrt{2n-1}) &= (\sqrt{2n+1}-\sqrt{2n-1})\frac{\sqrt{2n-1}-\sqrt{2n+1}}{\sqrt{2n-1}-\sqrt{2n+1}}\\
&= \frac{(2n+1)-(2n-1)}{\sqrt{2n+1}+\sqrt{2n-1}}\\
&= \frac{2}{\sqrt{2n+1}+\sqrt{2n-1}}
\end{flalign*}
}
\subsubsection{Avec des exponentielles}
Les exponentielles dominent les polyn\^omes. Donc On le met en facteur et on utilise la croissance comparée :
\begin{theorem}[Croissance comparée]
$\Lim{+\infty} \frac{n^x}{y^n} = 0$.
\end{theorem}
\subsection{Montrer que 2 suites sont adjacentes (2.11)}
\subsubsection{Montrer que les 2 suites sont définies}
Il peut \^etre utile de montrer que les 2 suites sont définies et toujours positives.

Pour cela, on peut faire un raisonnement par récurrence.
\subsubsection{Montrer qu'une suite est supérieure à l'autre}
On fait la différence $u_n-v_n$ et selon le signe, on conclue.
\subsubsection{On étudie la monotonie des suites}
Il faut montrer que l'une est croissante et l'autre décroissante. Pour cela, on fait la différence $u_{n-1}-u_n$ et on conclue, pareil pour $(v_n)$.
\subsubsection{Montrer que les suites sont convergentes}
Une suite est croissante et majorée par le premier terme de l'autre suite ; l'autre suite est décroissante et minorée par le premier terme de la première suite.

Les deux suites sont convergentes et on note $l$ et $l'$ leur limite respective.

\subsubsection{Montrer que l=l'}

Les 2 suites sont convergentes, donc en exprimant la limite du terme $n+1$ d'une suite en fonction de l'autre suite, on peut montrer que $l=l'$.
\subsubsection{Conclusion}
Toutes les conditions sont réunies pour dire que les 2 suites sont adjacentes.
\subsubsection{Exemple}
\checkInfo{Énoncé}{On pose $a_0 >0$ et $b_0 >0$. On a : $\frac{2}{u_{n+1}} = \frac{1}{u_n}+\frac{1}{v_n}$ et $v_{n+1} = \frac{u_n+v_n}{2}$. On veut montrer que les suites sont adjacentes.}
\begin{enumerate}
 \item On montre d'abord que les suites sont positives, par récurrence :

L'initialisation est immédiate d'après l'énonce ($a_0 >0$ et $b_0 >0$).

Hérédité : On suppose $(H_n)$. Donc $u_n>0$ et $v_n>0$. On a alors : $v_{n+1} = \frac{u_n+v_n}{2} > 0$. De plus, $\frac{1}{u_n} > 0$ et $\frac{1}{v_n}>0$

Donc la suite est bien définie et supérieure à 0.
\item On montre qu'une suite est supérieure à l'autre.

On a $u_{n+1} = \frac{2u_nv_n}{v_n+u_n}$, donc la différence $u_{n+1}-v_{n+1}$ vaut :

\begin{flalign*}
 u_{n+1}-v_{n+1} &= \frac{2u_nv_n}{v_n+u_n} - \frac{u_n+v_n}{2}\\
 &= \frac{4u_nv_n}{2(v_n+u_n)} - \frac{(u_n+v_n)^2}{2(u_n+v_n)}\\
 &= \frac{4u_nv_n-(u_n+v_n)^2}{2(v_n+u_n)}\\
 &= \frac{4u_nv_n-(u_n^2+2u_nv_n+v_n^2)}{2(v_n+u_n)}\\
  &= \frac{4u_nv_n-u_n^2-2u_nv_n-v_n^2}{2(v_n+u_n)}\\
    &= \frac{2u_nv_n-u_n^2-v_n^2}{2(v_n+u_n)}\\
    &= \frac{-(-2u_nv_n+u_n^2+v_n^2)}{2(v_n+u_n)}\\
    &= \frac{-(u_n+v_n)^2}{2(v_n+u_n)} \leq 0\\
\end{flalign*}
Donc $v_n \geq u_n$
\item On étudie la monotonie de $(v_n)$ et $(u_n)$ : $v_{n+1}-v_n = \frac{u_n+v_n}{2}-v_n$ Mais $v_n \geq u_n$, donc $\frac{u_n+v_n}{2}-v_n \leq 0$

De plus, $u_{n+1}-u_n = \frac{u_n(-u_n+v_n)}{u_n+v_n} \geq 0$, car comme $v_n \geq u_n$, on a $-u_n+v_n \geq 0$
\item On montre que les suites sont convergentes
$(u_n)$ est majorée par $v_1$ et est croissante, donc elle convergente et on note $l$ sa limite.

$(v_n)$ est minorée par $u_1$ et est décroissante, donc elle convergente et on note $l'$ sa limite.
\item On montre que $l=l'$

On sait que $v_{n+1} = \frac{u_n+v_n}{2}$, donc :

\begin{flalign*}
\Lim{\infty} u_{n+1} &= \Lim{\infty} \frac{u_n+v_n}{2}\\
l' &= \frac{l+l'}{2}\\
&= l
\end{flalign*}
Les 2 suites sont bien adjacentes.
\end{enumerate}

\subsection{Utiliser la définition de la convergence (2.10)}
\subsubsection{Montrer qu'une suite est inférieure à un certain nombre supérieur à sa limite}
Soit $\Lim{+ \infty} = l$. On souhaite montrer que $u_n \leq x$.

D'après la définition de la limite, $\forall \epsilon > 0, \exists N_0 \in \mathbb{N}, n\geq N_0 \Rightarrow |U_n-l| \leq \epsilon$. Donc :

\begin{flalign*}
&u_n-l \leq \varepsilon\\
&\iff -\varepsilon \leq u_n-l\leq \varepsilon\\
&\iff l-\varepsilon \leq u_n\leq \varepsilon + l
\end{flalign*}
Ainsi, en prenant $\varepsilon$ tel que $l+\varepsilon = x$, $\exists n_0\in\mathbb{N}, n\geq n_0 \Rightarrow u_n \leq x$.
\subsection{Exercice type : Montrer qu'une suite converge (2.6/2.12)}
\subsubsection{Encadrer la suite}
On trouve un majorant et un minorant de la suite, souvent par récurrence. Si la suite est positive, un minorant est 0.
\subsubsection{On montre qu'elle est croissante/décroissante}
\subsubsection{On montre sa convergence}
On déduit qu'elle est convergente et, comme on sait que la limite est un point fixe, on peut écrire que $\Lim{\infty} u_{n+1} = \Lim{\infty} u_n = l$ et déterminer $l$ à l'aide de cette relation.
\checkInfo{Exemple}{On a : $u_{n+1}^2 = 1 + u_n$. On peut donc écrire : $l^2 = 1+l$
Seul le nombre d'or $\tau$ vérifie cette relation, donc $l= \tau$

}
\end{document}


