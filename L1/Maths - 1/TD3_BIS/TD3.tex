% !TeX spellcheck = en_US
\documentclass[french]{yLectureNote}

\title{Mathématiques}
\subtitle{Langage mathématique}
\author{Paulhenry Saux}
\date{\today}
\yLanguage{Français}

\professor{C.Dartyge}

\usepackage{graphicx}%----pour mettre des images
\usepackage[utf8]{inputenc}%---encodage
\usepackage{geometry}%---pour modifier les tailles et mettre a4paper
%\usepackage{awesomebox}%---pour les boites d'exercices, de pbq et de croquis ---d\'esactiv\'e pour les TP de PC
\usepackage{tikz}%---pour deiffner + d\'ependance de chemfig
\usepackage{tkz-tab}
\usepackage{chemfig}%---pour deiffner formules chimiques
\usepackage{chemformula}%---pour les formules chimiques en \'equation : \ch{...}
\usepackage{tabularx}%---pour dimensionner automatiquement les tableaux avec variable X
\usepackage{awesomebox}%---Pour les boites info, danger et autres
\usepackage{menukeys}%---Pour deiffner les touches de Calculatrice
\usepackage{fancyhdr}%---pour les en-t\^ete personnalis\'ees
\usepackage{blindtext}%---pour les liens
\usepackage{hyperref}%---pour les liens (\`a mettre en dernier)
\usepackage{caption}%---pour la francisation de la l\'egende table vers Tableau
\usepackage{pifont}
\usepackage{array}%---pour les tableaux
\usepackage{lipsum}
\usepackage{yFlatTable}
\usepackage{multicol}
\newcommand{\Lim}[1]{\lim\limits_{\substack{#1}}\:}
\renewcommand{\vec}{\overrightarrow}
\begin{document}

\setcounter{chapter}{2}

	\chapter{Fonctions Réciproques et croissance comparée}
\section{Fonctions trigonométriques}
\subsection{Réciproques de fonctions usuelles}
	\begin{tabular}{_l^l^l}
		\tableHeaderStyle%
		Fonction & Réciproque & Domaine de Def\\
		$\cos$ & $\arccos$ & $[-1,1]\to[0,\pi]$\\
		$\sin$ & $\arcsin$ & $[-1,1]\to[-\frac{\pi}{2},\frac{\pi}{2}]$\\
		$\tan$ & $\arctan$ & $\mathbb{R}\to[-\frac{\pi}{2},\frac{\pi}{2}]$\\
	\end{tabular}
\subsection{Fonctions hyperboliques}
\subsubsection{Définition}
On définit 3 nouvelles fonctions :
\begin{theorem}[Fonctions hyperboliques]
\begin{itemize}
 \item $\cosh(x) = \frac{e^x+e^{-x}}{2}$
  \item $\sinh(x) = \frac{e^x-e^{-x}}{2}$
  \item $\tanh(x) = \frac{\sinh(x)}{\cosh(x)}$
\end{itemize}
\end{theorem}
\subsubsection{Propriétés}
	\begin{tabular}{_l^l}
		\tableHeaderStyle%
		Fonction & Dérivée\\
		$\cosh(x)$ & $\sinh(x)$\\
		$\sinh(x)$ & $\cosh(x)$\\
		$\tanh(x)$ & $\frac{1}{\cosh^2(x)}$
	\end{tabular}

De plus, on a : $\cosh^2-\sinh^2 = 1$.
\subsection{Fonctions hyperboliques inverses}
\begin{tabular}{_l^l^l}
		\tableHeaderStyle%
		Fonction & Inverse & Domaine de Def\\
		$\cosh(x)$ & arcosh & $[-1,+\infty]\to\mathbb{R}$\\
		$\sinh(x)$ & arsinh & $\mathbb{R}\to\mathbb{R}$\\
		$\tanh(x)$ & artanh & $]-1,1[\to\mathbb{R}$\\
	\end{tabular}
\section{Croissance comparée}
\begin{theorem}[Limites à connaître]
On pose $a,n >0$
\begin{itemize}
 \item $\Lim{+\infty} \frac{e^{ax}}{x^n} = +\infty$
 \item $\Lim{+\infty} \frac{x^a}{(\ln(x)^n} = +\infty$
  \item $\Lim{-\infty} |x|^n e^{ax} = 0$
  \item $\Lim{0^+} |\ln(x)|^nx^a = 0$
\end{itemize}

\end{theorem}
\end{document}


