% !TeX spellcheck = en_US
\documentclass[french]{yLectureNote}

\title{Électrocinétique}
\subtitle{Physique}
\author{Paulhenry Saux}
\date{\today}
\yLanguage{Français}

\professor{Allard}%allard@irsamc.ups-tlse.fr
\usepackage{graphicx}%----pour mettre des images
\usepackage[utf8]{inputenc}%---encodage
\usepackage{geometry}%---pour modifier les tailles et mettre a4paper
%\usepackage{awesomebox}%---pour les boites d'exercices, de pbq et de croquis ---d\'esactiv\'e pour les TP de PC
\usepackage{tikz}%---pour deiffner + d\'ependance de chemfig
\usepackage{tkz-tab}
\usepackage{chemfig}%---pour deiffner formules chimiques
\usepackage{chemformula}%---pour les formules chimiques en \'equation : \ch{...}
\usepackage{tabularx}%---pour dimensionner automatiquement les tableaux avec variable X
\usepackage{awesomebox}%---Pour les boites info, danger et autres
\usepackage{menukeys}%---Pour deiffner les touches de Calculatrice
\usepackage{fancyhdr}%---pour les en-t\^ete personnalis\'ees
\usepackage{blindtext}%---pour les liens
\usepackage{hyperref}%---pour les liens (\`a mettre en dernier)
\usepackage{caption}%---pour la francisation de la l\'egende table vers Tableau
\usepackage{pifont}
\usepackage{array}%---pour les tableaux
\usepackage{lipsum}
\usepackage{yFlatTable}
\usepackage{multicol}
\newcommand{\Lim}[1]{\lim\limits_{\substack{#1}}\:}
\renewcommand{\vec}{\overrightarrow}
\begin{document}

	\chapter{Introduction}
	%\section{Introduction}
\begin{definition}
Une étude de l'écoulement de charges ($e^-$) dans des milieux matériels (métaux, conducteurs)
\end{definition}
\subsection{Analogie avec l'écoulement d'un fluide}
Débit = Surface $\times$ vitesse : Analogue du courant électrique.

La mise en mouvement d'un fluide est faite par une différence d'énergie potentielle de pesanteur. L'écoulement se fait vers l'état de plus basse énergie : Analogue de la tension (différence de potentielle)
\section{Charge et courant}
\subsection{Charge}
\subsubsection{Définition et valeurs}
\begin{definition}[Charge électrique]
Propriété fondamentale de la matière qui caractérise l'intéraction entre les champs électromagnétiques

Elle peut \^etre > 0 et < 0 et son unité est le Coulomb.

Elle est quantifiée : c'est un multiple entier de la charge élémentaire \(e  = 1.602\cdot 10^{-19}\) C
\end{definition}
\begin{itemize}
 \item électron : $q_{e-} = -e$
 \item proton : $q_{e} = +e$
 \item neutron : $q_n = 0$
\end{itemize}
\subsubsection{Couplage entre particule}
Le couplage entre une particule de masse $m$ et un champ EM s'écrit sous la forme d'une force appelée force de Lorenz, avec \(\vec{E}\) le champ électrique et \(\vec{B}\) le champ magnétique : \[\vec{F} = q(\vec{E} + \vec{v} \wedge \vec{B})\]
\subsection{Courant}
\subsubsection{Définition}
\begin{definition}[Courant]
Le débit de charge dans un conducteur, noté $i(t) = \frac{\mathrm{d} q(t)}{\mathrm{d}t}$
\end{definition}
Il est mesuré par l'intensité du courant et s'exprime en Ampère : $C/s$

Il peut \^etre continue (les charges se déplacent dans le m\^eme sens) ou alternatif (sens du déplacement oscille).
\subsubsection{ODG}
\(1A \Rightarrow D = \frac{I}{e} = 6\cdot 10^{18} \)
\section{Potentiel électrique, tension et dip\^oles}
\subsection{Définitions}
Pour faire se déplacer des charges, il faut qu'elles soient attirées par la force de Lorenz vers les potentiels plus faible.

Le travail de la force électrique pour amener les charges de A à B sécrit comme \(W_{AB} = \int \vec{F}\cdot \vec{u}\), avec \(\vec{F} = - \vec{\nabla} E_p\).
\begin{definition}[Tension électrique]

On appelle tension électrique la différence de potentielle \(u_{AB} = V_A - V_B = \frac{Ep(A)-Ep(B)}{q}\). Elle traduit le changement de potentiel de la charge $q$ entre $A$ et $B$ et s'exprime en Volt.
\end{definition}
\begin{definition}[Dip\^ole]
Un composant qui possède deux p\^oles, appelées bornes. Ils consomment ou fournissent de l'énergie/puissance.
\end{definition}
\begin{definition}[Puissance]
Elle correspond au débit d'énergie à travers le dip\^ole et s'exprime en Watt (J/s). Dans un circuit la puissance fournie est intégralement consommée.
\end{definition}
\begin{definition}[Circuit]
Un ensemble de dip\^oles reliés par des fils électriques idéaux (conducteur parfait). Deux points reliés par un conduteur parfait sont au m\^eme potentiel
\end{definition}
\subsection{Conventions}
\subsubsection{Dip\^ole fournissant de l'énergie}
On l'appelle générateur. La puissance le traversant est négative : $P<0$ et le sens positif du courant est orienté comme la tension.
\subsubsection{Dip\^ole consommant de l'énergie}
On l'appelle récepteur. La puissance le traversant est positive : $P>0$ et le sens positif du courant est orienté dans le sens opposé à la tension.

Schéma 1
\section{Dip\^oles élémentaires}
Ils permettent en les assemblant  de modéliser le comportement électrique de tout circuit électrique réel. Ils sont parcourus par un courant et ont une tension à leur borne. On caractérise leur propriété par une courbe courant/tension de la forme \(i = f(u)\). Cette courbe se nomme carractéristiqye courant/tension.
\subsection{Générateur de tension idéal}
Il impose une tension $U_0$ assez grande entre ses bornes quelque soit le dip\^ole auquel il est connecté. \ref{schema 2}
\subsection{Générateur de courant idéal}
Il maintient un courant $I_0$ constant quelque soit les dip\^oles connectés \ref{schema 3}
\subsection{Résistance}
C'est un dip\^ole récepteur dont la tension entre ses bornes est proportionnelle au courant qui le traverse avec un coefficient de proportionalité $R$ en $\Omega$. On a $U=R\times I$. C'est le seul dip\^ole à dissiper l'énergie sous forme de chaleur par effet Joule \ref{schema 4}
\subsection{Condensateur idéal}
C'est un sip\^ole dont la tension à ses bornes et le courant sont reliés par $i = C \frac{\mathrm{d}U}{\mathrm{d}t}$ ou $U = \frac{q}{C}$ avec $q$ la charge accumulée dans le condensateur. et $C$ la capacité du condensateur qui s'exprime en Farad $F$.

Pour une tension constante, on a un courant nul, donc il se comporte comme un interrupteur ouvert.

%On encerlce 2 plaques avec un solant au milieu

La capcité dépend de la surface des plaques, de la distance entre plaque et la permitivé de l'isolant.

\ref{schema 5}
\subsection{Bobine idéale}
C'est un dip\^ole dont la tension à ses bornes et le courant sont reliés par : \(U = L \frac{\mathrm{d}i}{\mathrm{d}t}\) où $L$ est l'inductance propre de la bobine (en Henry H) .

Pour un courant constant, on a une tension nulle, donc cela se comporte comme un court-circuit.

Pour la fabriquer, on réalise un enroulement de fil conducteur (souvant autour d'un barreau ferro-magnétique). \(L\) dépend ud nombre de tour, du diamètre du tube et des prorpriétés du barrea, la permeabilité lagnétique relative

\ref{schema 6}
\end{document}

