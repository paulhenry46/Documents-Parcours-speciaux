% !TeX spellcheck = en_US
\documentclass[french]{yLectureNote}

\title{Mécanique}
\subtitle{Analyse dimensionnelle}
\author{Paulhenry Saux}
\date{\today}
\yLanguage{Français}

\professor{S.Deheuvels}%sebastien.deveuhels.irap.omp.eu

\usepackage{graphicx}%----pour mettre des images
\usepackage[utf8]{inputenc}%---encodage
\usepackage{geometry}%---pour modifier les tailles et mettre a4paper
%\usepackage{awesomebox}%---pour les boites d'exercices, de pbq et de croquis ---d\'esactiv\'e pour les TP de PC
\usepackage{tikz}%---pour deiffner + d\'ependance de chemfig
\usepackage{tkz-tab}
\usepackage{chemfig}%---pour deiffner formules chimiques
\usepackage{chemformula}%---pour les formules chimiques en \'equation : \ch{...}
\usepackage{tabularx}%---pour dimensionner automatiquement les tableaux avec variable X
\usepackage{awesomebox}%---Pour les boites info, danger et autres
\usepackage{menukeys}%---Pour deiffner les touches de Calculatrice
\usepackage{fancyhdr}%---pour les en-t\^ete personnalis\'ees
\usepackage{blindtext}%---pour les liens
\usepackage{hyperref}%---pour les liens (\`a mettre en dernier)
\usepackage{caption}%---pour la francisation de la l\'egende table vers Tableau
\usepackage{pifont}
\usepackage{array}%---pour les tableaux
\usepackage{lipsum}
\usepackage{yFlatTable}
\usepackage{multicol}
\newcommand{\Lim}[1]{\lim\limits_{\substack{#1}}\:}
\renewcommand{\vec}{\overrightarrow}
\begin{document}

	\chapter{Analyse Dimensionnelle }

	\section{Grandeurs physiques et unités associées}
	\subsection{Grandeurs physiques et Dimensions}
	\checkInfo{Définitions}{
	Grandeur physique : Propriété d'un système que l'on peut mesurer ou calculer

	Dimension : la nature d'une grandeur. Elle se note $[G]$.}

	Deux grandeurs $G$ et $G'$ qui ont la m\^eme dimension sont dites homogènes. On note $[G] = [G']$ ou $G \sim G'$


	\subsection{Système international}

	\begin{center}
\begin{tabular}{ll}
\tableHeaderStyle%
Dimension & Unité\\
Longueur (L) & mètre (m)\\
Masse (M) & kilogramme (kg)\\
Temps (T) & seconde (s)\\
Intensité du courant (I) & ampère (A)\\
Température ($\theta$) & kelvin (K)\\
Qt de matière (N)& mole (mol)\\
intensité lumineuse (J)& candela (cd)
\end{tabular}
\end{center}
\section{Dimension d'une grandeur physique}
\subsection{Dans le SI}
Pour toute grandeur physique G, il existe une unique décomposition dans le SI du type : $[G] = [L]^a[T]^b[M]^c[I]^d[\theta]^e[N]^f[J]^g$ Trouver la dimension de G dan le système revient à déterminer les valeurs des exposants.

\subsection{Grandeur sans dimension}
On dit qu'une grandeur G est sans dimension\marginCritical{Elle peut cependant avoir une unité} si [G] = 1


\criticalInfo{Liste des grandeurs sans dimension}{
\begin{itemize}
 \item angles
 \item fonctions usuelles, $\cos, \tan, \ln, \exp, \log$
 \item arguments des fonction usuelles
\end{itemize}
}
\subsection{Dimension d'un vecteur}
La dimension d'un vecteur correspond à la dimension de sa norme.

\subsection{Dimension et unité de grandeurs dérivées}
\tipsInfo{Méthode pour déterminer la dimension d'une grandeur dans le SI}{Il faut trouver une formule qui fait intervenir cette grandeur + des grandeurs du SI (ou de dimensions connues).}

\subsubsection{Dimensions à conna\^itre}
	\begin{tabular}{_l^l^l}
		\tableHeaderStyle%
		Type & Exemple d'unités & Dimension\\
		Énergie & J & $[M][L]^2[T]^{-2}$\\
		Force & N & $[M][L][T]^{-2}$\\
		Accélération & $m\cdot s^{-2}$ & $[L][T]^{-2}$\\
		Vitesse & Km/h & $[L][T]^{-1}$\\
		Charge & C & $[I][T]$\\
	\end{tabular}

\subsection{Problème aux dimensions}

On suppose qu'une grandeur physique G dépend d'un ensemble d'autres grandeurs $g_i$ (intuition physique). On voudrait écrire que la grandeur $G = g_i^{\alpha_1} \times g_i^{\alpha_2} \times g_i^{\alpha_i}$.

On détermine les valeurs des exposants $\alpha_i$ à partir des dimensions des grandeurs.

\checkInfo{Exemple : Période d'oscillation d'un pendule}{On veut déterminer la période P des oscillations de la masse $m$.

Elle peut dépendre de $g, l, \theta, m$. Le tout peut être multiplié par une constante sans dimension.

Dimension des grandeurs : $[P] = [T]$

$[g] = [L][T]^{-2}$

$[l] = [L]$.

$[m] = [M]$.

\begin{flalign*}
[M]^0[L]^0[T]^1 &= [g]^{\alpha}[l]^{\beta}[m]^{\lambda}\\
&= ([L][T]^{-2})^\alpha[L]^{\beta}[M]^{\lambda}\\
&= [M]^\lambda[L]^{\alpha+\beta}[T]^{-2\alpha}\\
\end{flalign*}



On procède par identification pour créer un système d'équations et trouver les coefficients.

\[
\begin{cases}0 = \lambda\\
0 = \alpha+\beta\\
1 = -2\alpha
\end{cases} \Rightarrow
\begin{cases}\lambda = 0\\
0 = \alpha+\beta\\
\alpha = -0.5
\end{cases} \Rightarrow
\begin{cases}\lambda = 0\\
\beta = 0.5\\
\alpha = -0.5
\end{cases}
\]

On remplace les valeurs dans l'expression de départ.

\begin{flalign*}
[M]^0[L]^0[T]^1 &= [g]^{\alpha}[l]^{\beta}[m]^{\lambda}\\
&= [g]^{-0.5}[l]^{0.5}[m]^{0}\\
&= [g]^{-0.5}[l]^{0.5}\\
\end{flalign*}

}

\criticalInfo{Valeur à conna\^itre : Masse volumique de l'air}{
$\rho_{air} = 1.2\:kg\cdot m^{-3}$.}

\tipsInfo{Lois d'échelle}{Quand on compare 2 systèmes, il n'est pas nécessaire de conna\^itre la valeur de la constante multiplicative. En effet, ces dernières s'annulent si on fait le rapport de 2 systèmes.}
\end{document}

