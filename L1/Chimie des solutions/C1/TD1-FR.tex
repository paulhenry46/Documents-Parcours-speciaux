% !TeX spellcheck = en_US
\documentclass[french]{yLectureNote}

\title{Chimie des solutions}
\subtitle{Chimie}
\author{Paulhenry Saux}
\date{\today}
\yLanguage{Français}

\professor{IHallery}%isabelle.hallery@univ-tlse3.fr
\usepackage{graphicx}%----pour mettre des images
\usepackage[utf8]{inputenc}%---encodage
\usepackage{geometry}%---pour modifier les tailles et mettre a4paper
%\usepackage{awesomebox}%---pour les boites d'exercices, de pbq et de croquis ---d\'esactiv\'e pour les TP de PC
\usepackage{tikz}%---pour deiffner + d\'ependance de chemfig
\usepackage{tkz-tab}
\usepackage{chemfig}%---pour deiffner formules chimiques
\usepackage{chemformula}%---pour les formules chimiques en \'equation : \ch{...}
\usepackage{tabularx}%---pour dimensionner automatiquement les tableaux avec variable X
\usepackage{awesomebox}%---Pour les boites info, danger et autres
\usepackage{menukeys}%---Pour deiffner les touches de Calculatrice
\usepackage{fancyhdr}%---pour les en-t\^ete personnalis\'ees
\usepackage{blindtext}%---pour les liens
\usepackage{hyperref}%---pour les liens (\`a mettre en dernier)
\usepackage{caption}%---pour la francisation de la l\'egende table vers Tableau
\usepackage{pifont}
\usepackage{array}%---pour les tableaux
\usepackage{lipsum}
\usepackage{yFlatTable}
\usepackage{multicol}
\newcommand{\Lim}[1]{\lim\limits_{\substack{#1}}\:}
\renewcommand{\vec}{\overrightarrow}
\begin{document}

	\chapter{Transformation chimique et équilibre chimique }
	\section{Description d'un système isgège d'une transformation chimique}
	\subsection{Description d'un système}
	\begin{itemize}
	 \item Au repos : On considère que le système est immobile
	 \item Fermé : n'échange pas de matière avec l'extérieur mais échange d'énergie : exemple : montage au reflux/becher
	 \item En contact avec une seule source de chaleur à température constante
	\end{itemize}
\subsubsection{Variables pour décrire l'état physique}
Température, Pression, Volume. Il suffit d'obtenir 2 des 3 pour connaitre l'évolution du système
\subsubsection{Variables pour décrire la composition}
On écrit les constituants physicochimiques, notés $A_i$ avec leur état en indice.

% On se sert de \begin{itemize}
%                \item la concentration $C_i = \frac{n_i}{V_{tot}}$
%                \item la fraction molaire
%               \end{itemize}
\begin{definition}[Fraction molaire]
\[x_i^{(\alpha)} = \frac{n_i^{(\alpha)}}{\sum n_j^{(\alpha)} = n_{tot}} \leq 1\] avec $\alpha$ la phase
\end{definition}
\criticalInfo{Phase}{Pour le nombre total de mole, il faut bien cacluler les quantités de matière présente dans la phase et non dans tout le milieu : pour un gaz, on ne prend que les gaz et pour un liquide qu'un liquide !}
\begin{definition}[Grandeur intensive]
Grandeur indépendante de la quantité de matière : Température, Pression, Concentration, Fraction molaire
\end{definition}

\begin{definition}[Grandeur extensive]
Grandeur dépendante de la quantité de matière :  Quantité de matière, masse
\end{definition}
\subsection{Pression partielle}
On utilise le modèle des gaz parfaits.\marginWarning{Dans le modèle des gaz parfaits, les molécules n'ont pas d'influence sur les autres sauf quand elles se rentrent dedans.} La pression peut indiquer des éléments stoechiométriques.\marginTips{En effet, si la quantité de matière reste contsante durant la réaction, la pression reste constante. Il faut alors que la somme des coefficient stoechiométriques des gaz soit nulle.}
% \begin{theorem}[Équation des gaz parfaits pour un seul gaz]
%  \[pV = nRT\]
% \end{theorem}

% \begin{definition}
% \[P = \frac{F}{S}\]
% \end{definition}
\begin{theorem}[Mélange des gaz parfaits]
On écrit pour chaque gaz $i$ la m\^eme relation. Tout se passe comme si chaque gaz occupait la totalité du volume.
 \[p_iV = n_iRT\]
\end{theorem}
La pression totale $p$ est la somme des pressions partielles $p_i$.



\subsection{Modélisation de la transformtion}
% Notation : $(n_{H2})_0$ dans l'état initial
%
% $(n_{H2})_c$ consommé
%
% $(n_{H2})_f$ formé

On prend en compte les proportions dans lesquelles sont consommés les réactifs et sont formés les produits, les charges et les éléments
\begin{definition}[Écriture formelle et coefficient]
$0 = \sum \nu_i A_i$ avec $\nu_i$ le coefficient stoechiométrique algébrisé de $A_i$. Si $A_i$ est un produit ($\nu_i >0$) ou un réactif ($\nu_i <0$).
\end{definition}
Écriture formelle :

\subsection{Avancement d'une réaction}

\begin{definition}[Avancement molaire]
\[\frac{n_i-(n_i)_0}{\nu_i} = \xi \iff n_i = (n_i)_0 + \nu_i \xi\]
\end{definition}

\begin{definition}[Avancement volumétrique]
\[x = \frac{\xi}{V_{tot}}\]
\end{definition}
On peut donc écrire $ [A_i] = [A_i]_0 + \nu_i x$.

L'avancement maximal, noté $\xi_{max} \geq 0$ est l'avancement calculé si la transformation est totale dans le sens direct.\marginTips{Pour le trouver, on compare le rapport $|\frac{n_0}{\nu}|$ de chaque réactif et on prend le plus petit, qui est l'avancement maximal.}

L'avancement minimal, noté $\xi_{min} \leq 0$ est l'avancement calculé si la transformation est totale dans le sens indirect.
On peut également se servir du taux d'avancement :\marginWarning{En effet, à un instant $t$, $\alpha_t=\frac{\xi_t}{\xi_{max}}$, donc $\xi_t = \alpha_t \xi_{max}$. C'est utile quand on n'a pas la valeur des quantités initiales de matière.}
\begin{definition}[Taux d'avancement]
\[\alpha = \frac{\xi-\xi_{min}}{\xi_{max} - \xi{min}}\]
\end{definition}
\subsection{Taux de conversion}
\criticalInfo{Taux de conversion}{Le taux de conversion $\alpha_i$ est défini par rapport à un constituant alors que le taux d'avancemet est le m\^eme pour tous les réactifs.} Si les proportions initiales de A et de B sont différentes des proportions stoechiométriques , seul le taux de conversion du réactif minoritaire sera égal au taux d'avancement de réaction.

On peut cependant l'utiliser pour obtenir une nouvelle expression de l'avancement. On a \(\alpha_{i,t} = \frac{n_{i,0} - n_{i,t}}{n_{i,0}}\)

\criticalInfo{Taux de conversion}{Le taux de conversion est défini par rapport à un constituant alors que le taux d'avancemet est le m\^eme pour tous les réactifs. On peut cependant l'utiliser pour obtenir une nouvelle expression de l'avancement. On a \(\alpha_{i,t} = \frac{n_{i,0} - n_{i,t}}{n_{i,0}}\)}
\section{Composition du système dans l'état final}
\subsection{Transformtion totale/limitée}
Dans une transformation totale, un des réactif est totalement consommé.
\begin{itemize}
 \item dans le sens direct : $\xi_{EF} = \xi{max}$ et $\alpha = 1$
 \item dans le sens indirect : $\xi_{EF} = \xi{min}$ et $\alpha = 0$
\end{itemize}

En revanche, dans une transformation limitée, toues les réactifs sont encore présents à l'état final : $\xi_{min} < \xi_{EF} < \xi_{max}$ et $0<\alpha_{EF} <1$
\subsubsection{Constante d'équilibre d'une réaction chimique}
Il y a équilibre quand le milieu n'évolue plus alors que tous les réactifs et produits sont présents. Il y a un équilibre dynamique.\marginCheck{La constante d'équilibre se calcule toujours à la fin de la réaction et n'évolue pas ! (C'est le quotient de réaction qui évolue)}

% Notations : $\to$ : réaction totale dans le sens direct ou $\leftrightarrows$ pour une réaction conduisant à état d'équilibre chimique.

% Il existe une constante qui ne dépend que de la température et indépendante de la composition initiale.

\begin{definition}[Constante d'équilibre]
\[K° = \Pi a(A_i)_{eq}^{\nu_i}\] avec $a(A_i)_{eq}$ est l'activité du constituant physicochimique $A_i$ à l'équilibre et $\nu_i$ le coefficient stoechiométrique algébrisé.
\end{definition}

\begin{center}
\begin{tabular}{ll}
$A_i$ & $a(A_i)$\\
Gaz parfait & $\frac{p_i}{p^0}$\\
Soluté & $\frac{C_i}{C^0}$\\
Solvant & 1\\
Solide pur & 1\\
Mélange & $x_i$
\end{tabular}
\end{center}
\subsection{Calcul de K}
On la calcule  à partir des éléments de la réaction pour une tenpérature donnée et elle ne dépend pas des quantités de matière initiales.

En mumtipliant les réaction par des coefficients, on peut trouver le K d'une réaction avec des combinaisons linéaires. Si on a \(K_1\) pour une réaction et qu'on multiplie l'équation par 2, on a $K_1^2$. On écrit le $K$ de la réaction.

Si on peut écrire l'équation l'équation de réaction (1) comme $(2)\times n + (3)\times m$, alors $K(1) = K(2)^n\times K(3)^m$.

Pour de petites valeurs, on a $pK° = -\log(K°)$.
\tipsInfo{En pratique pour les gaz}{On a :\(a = \frac{p_i}{P_0} = \frac{x_iP_t}{P_0} = \frac{n_iP_t}{n_tP_0}\). Il faut que la pression soit en bar pour \^etre dans le m\^eme unité que $P_0$.}
\subsection{Composition dans l'état final}
Si $K° > 10^{4}$, la réaction est considérée comme totale et $\xi_{EF} = \xi_{max}$, si $K° > 10^{-4}$ la réaction est considérée comme totale dans le sens indirect et $\xi_{EF} = \xi_{min}$. Si $K°$ est entre les deux valeurs, on doit résoudre une équation.


\end{document}

