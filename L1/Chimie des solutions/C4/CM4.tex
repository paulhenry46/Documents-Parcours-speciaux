% !TeX spellcheck = en_US
\documentclass[french]{yLectureNote}

\title{Chimie des solutions}
\subtitle{Chimie}
\author{Paulhenry Saux}
\date{\today}
\yLanguage{Français}

\professor{IHallery}%isabelle.hallery@univ-tlse3.fr
\usepackage{graphicx}%----pour mettre des images
\usepackage[utf8]{inputenc}%---encodage
\usepackage{geometry}%---pour modifier les tailles et mettre a4paper
%\usepackage{awesomebox}%---pour les boites d'exercices, de pbq et de croquis ---d\'esactiv\'e pour les TP de PC
\usepackage{tikz}%---pour deiffner + d\'ependance de chemfig
\usepackage{tkz-tab}
\usepackage{chemfig}%---pour deiffner formules chimiques
\usepackage{chemformula}%---pour les formules chimiques en \'equation : \ch{...}
\usepackage{tabularx}%---pour dimensionner automatiquement les tableaux avec variable X
\usepackage{awesomebox}%---Pour les boites info, danger et autres
\usepackage{menukeys}%---Pour deiffner les touches de Calculatrice
\usepackage{fancyhdr}%---pour les en-t\^ete personnalis\'ees
\usepackage{blindtext}%---pour les liens
\usepackage{hyperref}%---pour les liens (\`a mettre en dernier)
\usepackage{caption}%---pour la francisation de la l\'egende table vers Tableau
\usepackage{pifont}
\usepackage{array}%---pour les tableaux
\usepackage{lipsum}
\usepackage{yFlatTable}
\usepackage{multicol}
\newcommand{\Lim}[1]{\lim\limits_{\substack{#1}}\:}
\renewcommand{\vec}{\overrightarrow}
\begin{document}
\setcounter{chapter}{3}
\chapter{Équilibres acido-basique}
\section{Équilibres acido-basiques}
\subsection{Acido-basicité au sens de Bronsted}
\begin{definition}[Acidobasicité de Bronsted]
L'acide peut céder un proton H+ et une base peut capter un proton H+. Une espèce pouvant faire ls 2 est ampholyte.
\end{definition}
À tout acide est associée une base conjuguée et réciproquement : C'est le couple acido-basique.

\marginTips{Dans tous les cas, la base possède un DNL pouvant fixer un proton et un acide}
\checkInfo{Exemple d'espèce ampholyte}{L'eau participe à 2 couples acido-basiques. On a \(H_3O⁺/H_2O\) et \(H_2O/HO^-\). Il y a aussi le couple de l'acide carbonique : \(H_2CO_3/HCO_3^-\) et  \(HCO_3^-/CO_3^{2-}\) }

\subsubsection{Formules de Lewis}
Elles permettent de prévoir le comportement d'une molécule
\warningInfo{Composés à conna\^itre}{Il y a l'acide acétique \(CH_3COOH\), l'ion carbonate \(CO_3^{2-}\), l'ion hydrogénocarbonate \(HCO_3^-\), l'ion hydronium \(H_3O^+\) et l'ion hydroxyde \(HO^-\).}

Une liaison C-H n'est pas assez polarisée pour avoir une liaison. En revanche une liaison OH l'est assez pour pouvoir céder l'hydrogène.

Un composé pouvant accepter/céder 2 protons est un diacide/dibase. En général, on parle de polyacide/polybase.
\subsection{Réactions acido-basique}
C'est une réaction d'échange de protons où il n'apparait pas dans l'équation de réaction.

\begin{proposition}[Réaction d'autoprotolyse de l'eau]
\(2H_2O = H_3O^++OH^-\)
\end{proposition}
\begin{proposition}[Produit ionique de l'eau]
C'est la constante d'équilibre de la réaction d'autoprotolyse de l'eau. On a \(K_e = [H_3O^+][HO^-]\) dont la valeur à 25° est de \(10^{-14}\).
\end{proposition}
Dans l'eau pure, les 2 concentrations valent \(10^{-7}\)
\subsection{Constante d'acidité Ka}
\subsubsection{Acide fort/acide faible - Constante d'acidiité Ka}
Dès qu'on met un acide dans l'eau, il va agir avec l'eau. On a \(AH+H_2O = A^-+H_3O^+\).

\begin{definition}[Acide fort /faible]
La réaction avec l'eau d'un acide fort est totale alors que celle d'un acide faible est limitée
\end{definition}
\begin{definition}[Solution acide]
On a \([H_3O^+]>[HO^-]\)
\end{definition}
\begin{definition}[Constante d'acidité]
\(K_a = \frac{[A^-][H_3O^+]}{[AH]}\)
\end{definition}
On a \(pK_a (H_3O^+/H_2O) = 0\) et \(pk_a(H_2O/OH^-) = 14\)
\subsubsection{Base forte/base faible - Constante de basicité Ka}
Dès qu'on met une base dans l'eau, il va agir avec l'eau. On a \(A^-+H_2O = AH+HO^-\).

\begin{definition}[Base fort /faible]
La réaction avec l'eau d'une base forte est totale alors que celle d'une base faible est limitée
\end{definition}
\begin{definition}[Solution basique]
On a \([H_3P^+]<[HO^-]\)
\end{definition}
\begin{definition}[Constante de basicité]
\(K_b = \frac{[AH][HO^-]}{[A^-]} = \frac{K_e}{K_a}\)
\end{definition}
\section{Échelle de pka}
\subsection{Force comparée dans acides et bases}
\begin{proposition}[Acides/Bases forts à connaître]
On a \(HCl,HNO_3, H_2SO_4\) en acides forts et \(NaOH, Ba(OH)\) à conna\^itre.
\end{proposition}
\criticalInfo{Comparaison}{\(H_3O^+\) est l'acide le plus fort dans l'eau et \(HO^-\) la base la plus forte.}
\subsection{Classement sur le critère des Ka}
Un acide faible est d'autant plus faible que le Ka du couple est petit, que le pKa est grand

Une base faible est d'autant plus faible que le Ka du couple est grand, que le pKb est petit

\subsection{Influence de la dilution}
\criticalInfo{Acide faible}{CH3COOH se comporte comme un acide faible}
\begin{proposition}
La force d'un acide dépend du Ka mais aussi de sa concentration. En première approximation, on ne prend en compte que Ka.
\end{proposition}
Quand on dilue un acide faible à l'infini, on peut obtenir un acide fort.
% \chapter{Équilibre de plusieurs réactions acido-basiques }
% \section{Les outils}
% \subsection{pH d'une solution aqueuse}
% \begin{definition}[Définition du pH]
% \(pH = -\log(a(H_3O^+)) \simeq -\log([H_3O^+])\) si la concentration totale en ion est suffisament faible.
% \end{definition}
% \begin{definition}[Définition du pOH]
% \(pH = -\log(a(HO^-)) \simeq -\log([HO^-])\) si la concentration totale en ion est suffisament faible.
% \end{definition}
% La somme des 2 vaut 14.
% \subsection{Diagramme de prédominance}
% Pour trouver la valeur à laquelle les 2 espèces sont dans les m\^emes proportions,  on éxprime l'une des 2 en fonction du Ka pour obtenir le pKa. Si le pH est supérieur à ce dernier, $A^-$ prédomine.
% \subsection{Diagramme de prépondérance}
% Une espèce est prépondérante devant une autre si il y a 10 fois plus de cette espèce que l'autre espèce. On recule et avance de 0,5 unités de pH pour l'eau. Pour les autres espèces, on pkA+1 et pkA-1 pour un prépondérance de $A^-$ et $AH$.
% \subsection{Constante d'équilibre d'une réaction}
% La constante d'équilibre peut se déduire des Ka des 2 couples mis en jeu. On a
% \begin{theorem}[Constante d'équilibre]
%  Pour une réaction de la forme Acide1+Base2=Base1+Acide2, on a \(K° = \frac{Ka(1}){Ka(2)}\).
% \end{theorem}
% Si le \(pK_a(2)>pK_a(1)\) et \(|\Delta pK_a| >4\), alors la réaction est quasi-totale dans le sens direct et la constante d'équilibre est \(K°=10^{+|\Delta pKa|}\).
%
% Si le \(pK_a(2)<pK_a(1)\) et \(|\Delta pK_a| >4\), alors la réaction est quasi-négligeable dans le sens direct et la constante d'équilibre est \(K°=10^{-|\Delta pKa|}\).
% \section{Méthode de la réaction prépondérante}
% \subsection{Réaction prépondérante}
% Par définition, c'est la réaction qui modifie le plus la composition du milieu.
%
% Par hypothèse\marginCritical{Cela dépend aussi de la composition initiale mais aussi des conditions initiales (dilution, réactif en excès)}, c'est celle qui a la constante d'équilibre la plus élevée. Quand il n'y a que es acides ou des bases, c'est la réaction de l'acide le plus fort sur la base la plus forte.
% \subsubsection{Méthode}
%
% \begin{enumerate}
%  \item Identifier les espèces prépondérantes
%  \item Classer les couples présents sur une échelle de \(pK_a\)
%  \item Entourer les espèces prépondérantes présentes initialement
%  \item On identifie l'acide le plus fort et la base la plus forte
%  \item Les 2 réagissent ensembles
% \end{enumerate}
% \subsection{Principe}
% On image la succession des réactions prépondérantes qui conduisent à des solutions équivalentes. On s'arrette quand $K \leq 10^4$.
% \subsection{Validation}
% Quand on connait le Ka, la concentration de la forme acide et de la forme basique, on peut trouver la concentration en [H3O+] car \(K_a = \frac{[A-][H_3O^+]}{[AH]}\).
% \subsection{pH}
% Pour une solution d'acide fort à la concentration $C_0$, $pH=-\log(C_0)$ si $pH<6.5$
%
% Pour une solution de base forte à la concentration $C_0$, $pOH=-\log(C_0)$ si $pH>7.5$
%
% Pour une solution d'acide faible à la concentration $C_0$, $pH=-\frac{1}{2}(pKa-\log(C_0))$ si $pH<pKa-1$ et $pH<6.5$
%
% Pour une solution de base faible à la concentration $C_0$, $pOH=-\frac{1}{2}(pKb-\log(C_0))$ si $pH>pKa+1$ et $pH>7.5$
\end{document}

