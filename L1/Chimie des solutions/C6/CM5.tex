% !TeX spellcheck = en_US
\documentclass[french]{yLectureNote}

\title{Chimie des solutions}
\subtitle{Chimie}
\author{Paulhenry Saux}
\date{\today}
\yLanguage{Français}

\professor{IHallery}%isabelle.hallery@univ-tlse3.fr
\usepackage{graphicx}%----pour mettre des images
\usepackage[utf8]{inputenc}%---encodage
\usepackage{geometry}%---pour modifier les tailles et mettre a4paper
%\usepackage{awesomebox}%---pour les boites d'exercices, de pbq et de croquis ---d\'esactiv\'e pour les TP de PC
\usepackage{tikz}%---pour deiffner + d\'ependance de chemfig
\usepackage{tkz-tab}
\usepackage{chemfig}%---pour deiffner formules chimiques
\usepackage{chemformula}%---pour les formules chimiques en \'equation : \ch{...}
\usepackage{tabularx}%---pour dimensionner automatiquement les tableaux avec variable X
\usepackage{awesomebox}%---Pour les boites info, danger et autres
\usepackage{menukeys}%---Pour deiffner les touches de Calculatrice
\usepackage{fancyhdr}%---pour les en-t\^ete personnalis\'ees
\usepackage{blindtext}%---pour les liens
\usepackage{hyperref}%---pour les liens (\`a mettre en dernier)
\usepackage{caption}%---pour la francisation de la l\'egende table vers Tableau
\usepackage{pifont}
\usepackage{array}%---pour les tableaux
\usepackage{lipsum}
\usepackage{yFlatTable}
\usepackage{multicol}
\newcommand{\Lim}[1]{\lim\limits_{\substack{#1}}\:}
\renewcommand{\vec}{\overrightarrow}
\begin{document}
\setcounter{chapter}{5}
\chapter{Titrages }
\begin{definition}[Dosage]
Détermination de la concentration d'une espèce
\end{definition}
\section{Réalisation d'un titrage}
\subsection{Principe}
On étudie la réaction de titrage \(aA+bB+cC = dD+eE\) avec C en excès. On cherche à déterminer la concentration de A (introduit dans le becher le plus souvent et appelé prise d'essai), on le titre.
\begin{proposition}[Caractérisation d'une réaction de titrage]
Elle est quasi-totale, rapide et unique
\end{proposition}
\begin{proposition}[Relation à l'équivalence]
\(\frac{n(A)_0}{a} = \frac{n(B)_{aj}}{b}\)
\end{proposition}
\warningInfo{Influence de l'addition d'eau dans le becher}{Aucune influence sur la valeur du volume équivalent Ve mais applaitit la courbe et la rend moins précise}
\subsection{Repérage de l'équivalence par pH-métrie}
On utilise le plus souvent un acide fort ou une base forte.
\subsubsection{Méthode graphique}
Méthode des tangentes : repérage du milieu du saut. (OK pour AF+BF)

Méthode des physiciens : On trace les tangentes au plats, avant et après l'équivalence puis on trace les perpendiculaires aux 2 tangentes. On trouve le milieu des 2 droites, puis on trace la droite liant les 2 points

Méthode de la dérivée : on prend le milieu du pic.
\subsection{Repérage par indicateur coloré}
\begin{definition}[Indicateur coloré acidobasique]
Espèce acido-basique dont la forme acide et basique ont des couleurs différentes.
\end{definition}
On choisit l'indicateur coloré pour que sa zone de virage contiennent le pH d'équivalence
\section{Titrage d'un monoacide}
\subsection{Titrage d'un monoacide fort par base forte}
Plus l'acide est dilué, plus l'amplitude du saut est faible
\warningInfo{Concentration}{On ne peut pas doser par titrage un AF avec \(c_0 \leq 10^{-4}\)}
\subsection{Titrage d'un monoacide faible}
\tipsInfo{Forme de la courbe}{Si la courbe n'est pas plate près de v=0, c'est un acide faible}
La dolution aplatit la courbe de titrage mais seulement après l'équivalence seulement.

Plus le pkA augmente, plus la courbe est aplatit avant le Ve. La constante de récation diminue et l'emplitude du saut diminue.

Plus l'amplitude du saut est faible, moins la réaction de titrage est favorisée.
\section{Solution tampon}
\begin{definition}
C'est une solution dont le pH varie peu.
\end{definition}
\checkInfo{Composition}{Il s'agit d'un acide faible et sa base faible conjuguée dans des proportions senmblables. La solution fonctionne à \(V_e/2\). Le pkA du couple doit \^etre le plus proche possible du pH que l'on souhaite obtenir. Les propriétés du tampon sont conservées autour de \(pk_a-1\) et \(pk_A+1\).
}
\section{Titrage d'un polyacide}
\subsection{Condition de séparation des acidités}
Diagralle de distribution : H2A : le premier, HA-, le second et A2- le dernier.

Si l'écart de \(pk_A\) est trop faible, on ne peut apercevoir le saut du premier changement de dosage.

\criticalInfo{Écart}{2 acidités sont dosés séparément si la différence de pkA est d'au moins 4.}

On peut trouver l'acidité dosée avec le pH.

\checkInfo{Exempple}{Pour un \(pkA_1 = 4\) et un \(pk_A2 = 6\), si pH(E) = 9, Espèce prépondérante est \(A^{2-}\) et l'équation est donc \(H_2A+2Ho^- = A^{2-}+H_2O\)}

\subsection{Choix du saut à exploiter}
Dans le cas, où les 2 acides sont dosés séparément, il faut choisit quelle équivalence choisir.

\checkInfo{Exemple}{
Pour \(E_1 : H_2A+HO- = HA^-+H_2O\).

Pour \(E_2 : H_2A+2HO^- = A^{2-} + 2H_2O\)

Les 2 donnent le m\^eme résultat et on remarque que \(V_{e2} = 2V_{e1}\).
}
Le résultat le plus précis correspond à la pente la plus grande.
\section{Exemples}
\subsection{Titrages successifs}
\subsubsection{mélange AcOH et NH4Cl par de la soude}
\begin{itemize}
 \item On titre un mélange de AcOH (C1, pKA1 = 4,8) et NH4Cl (C2, pKA2 = 9,2) (bécher, V0) avec NaOH de concentration $C_B$.
 \item
La différence de pkA est bien supérieure à 4, donc les 2 espèces seront titrés séparément.
\item On remarque que  $NH_4Cl$ se dissout pour donner $NH_4^+$ et $Cl^-$.
\item En plaçant les couples $AcOH/AcO^-$ et $NH_4^+/NH_3$ sur un diagramme de prédominance, on identifie la première RP, puis la deuxième, quand tout le AcOH est consommé.
\end{itemize}
Les 2 RP sont donc
\begin{enumerate}
 \item $AcOH_{(aq)} + OH^-_{(aq)} \to ACO^-_{(aq)} + H2O_{(l)}$
 \item $NH_{4(aq)}+HO^-_{(aq)}\to NH_{3(aq)} + H2O_{(l)}$
\end{enumerate}
L'eau étant le solvant, elle est toujours considérée en excès, et on ne cherche pas à calculer sa concentration. Pour des questions de mise en page, elle n'apparait pas dans le tableau
\begin{center}
\begin{tabular}{lllll}
RP1 & $AcOH_{(aq)}$ & $OH^-_{(aq)}$ & $ACO^-_{(aq)}$ & $K_1 = \frac{K{a1}}{K_e}=10^{9.2}$\\
$V=0$ & $C_B\times V_{eq1}$ & 0 & 0 & $pH=\frac{1}{2}(pk_{a1}-\log(\frac{C_b\times V_{eq1}}{V_T}))$ \\
$0<V<V_{eq1}$ & $C_B\times (V_{eq1}-V)$ & 0 & $C_B\times V$ & $pH=pk_{a1}+\log(\frac{C_B\times V}{C_B\times (V_{eq1}-V)})$\\
$V=V_{eq1}$ & 0 & 0 & $C_B\times V_{eq1}$ & $pH=\frac{1}{2}(pk_{a1}+pK_{a2}+\log(\frac{C_B \times V_{eq1}}{C_B \times (V_{eq2}-V_{eq1})}))$\\
RP2 & $NH_4^+$ & $HO^-$ & $NH_3$ & $K_2=\frac{K_{a2}}{Ke}=10^{4.8}$\\
$V=V_{eq1}$ & $C_B(V_{eq2}-V_{eq1})$ & 0 & 0 & $pH=\frac{1}{2}(pk_{a1}+pK_{a2}+\log(\frac{C_B \times V_{eq1}}{C_B \times (V_{eq2}-V_{eq1})}))$\\
$V_{eq1}<V<V_{eq2}$ & $C_B(V_{eq2}-V)$ & 0 & $C_B(V-V_{eq1})$ & $pH=pk_{a2}+\log(\frac{C_B(V-V_{eq1})}{C_B(V_{eq2}-V)})$\\
$V=V_{eq2}$ & 0 & 0 & $C_B(V_{eq2}-V_{eq1})$ & $pH=\frac{1}{2}(14+pk{a2}+\log(\frac{C_B(V_{eq2}-V_{eq1})}{V_T}))$\\
$V_{eq}<V$ & 0 & $C_B(V-V_{eq2})$ & $C_B(V_{eq2}-V_{eq1})$ & $pH=14+\log(\frac{C_B(V-V_{eq2})}{V_T})$
\end{tabular}
\end{center}
\begin{itemize}
 \item Si on avait dosé un diacide au lieu de 2 espèces acides, on aurait obtenu les m\^emes relations de quantités de matière et de pH dans le tableau.
 \item Dans le cas d'un diacide, \(V_{eq2}=2\times V_{eq1}\)
 \item Justification des formules de pH
 \begin{enumerate}
  \item Il n'y a que des acides faibles dans le milieu, donc le pH est imposé par l'acide faible le plus fort, ici ACOH.
  \item le pH est imposé par la solution tampon d'acide et base conjuguée.
  \item Le pH est imposé par le mélange de la base faible ACO- et de l'acide faible NH4+.
  \item Le pH est imposé par le mélange de la base faible ACO- et de l'acide faible NH4+.
  \item le pH est imposé par la solution tampon d'acide et base conjuguée.
  \item Il n'y a que des bases faibles dans le milieu, donc le pH est imposé par la base faible le plus forte, ici NH3.
  \item Le pH est imposé par la base forte du milieu, ici $HO^-$
 \end{enumerate}

\end{itemize}


\end{document}

