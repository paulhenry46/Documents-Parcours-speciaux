% !TeX spellcheck = en_US
\documentclass[french]{yLectureNote}

\title{Linux et Python : Une introduction}
\subtitle{Mathématiques et Physique}
\author{Paulhenry Saux}
\date{\today}
\yLanguage{Français}

\professor{Juliette Jinenez Jaimes}% jimenez@irsamc.ups-tlse.fr

\usepackage{graphicx}%----pour mettre des images
\usepackage[utf8]{inputenc}%---encodage
\usepackage{geometry}%---pour modifier les tailles et mettre a4paper
%\usepackage{awesomebox}%---pour les boites d'exercices, de pbq et de croquis ---d\'esactiv\'e pour les TP de PC
\usepackage{tikz}%---pour deiffner + d\'ependance de chemfig
\usepackage{tkz-tab}
\usepackage{chemfig}%---pour deiffner formules chimiques
\usepackage{chemformula}%---pour les formules chimiques en \'equation : \ch{...}
\usepackage{tabularx}%---pour dimensionner automatiquement les tableaux avec variable X
\usepackage{awesomebox}%---Pour les boites info, danger et autres
\usepackage{menukeys}%---Pour deiffner les touches de Calculatrice
\usepackage{fancyhdr}%---pour les en-t\^ete personnalis\'ees
\usepackage{blindtext}%---pour les liens
\usepackage{hyperref}%---pour les liens (\`a mettre en dernier)
\usepackage{caption}%---pour la francisation de la l\'egende table vers Tableau
\usepackage{pifont}
\usepackage{array}%---pour les tableaux
\usepackage{lipsum}
\usepackage{yFlatTable}
\usepackage{multicol}
\newcommand{\Lim}[1]{\lim\limits_{\substack{#1}}\:}
\renewcommand{\vec}{\overrightarrow}
\begin{document}

%\setcounter{chapter}{2} Login Générique : fsi.local et mot de passe : pdfl1

	\chapter{Systèmes UNIX}
\section{Commandes de base}
\subsubsection{Bases}
	\begin{tabular}{_l^l^l}
		\tableHeaderStyle%
		Commande & Explication\\
		pwd & Affiche le répertoire dans lequel on se trouve\\
		ls & liste le contenu du répertoire\\
		ls -all & liste le contenu du répertoire avec les permissions\\
		man & Affiche la Doc de la commande qui le suit\\
		echo ``texte'' & Affiche le texte\\
	\end{tabular}
\subsection{Modification des fichiers}
\begin{tabular}{_l^l^l}
		\tableHeaderStyle%
		Commande & Explication\\
		touch file.txt & Crée un fichier vide nommé file.txt\\
		nano file.txt & Modifie le fichier file.txt ou le crée s'il n'existe pas\\
		mv file.txt new\_name.txt & Renomme le fichier file.txt en new\_name.txt\\
	\end{tabular}
	\subsubsection{Nano}
	Nano est un éditeur de texte utilisable dans le Terminal adapté pour les débutants. Pour enregistrer le fichier, faire Ctrl+S et pour quitter Nano, Ctrl+x.
	\subsection{Modification des répertoires}
	\begin{tabular}{_l^l^l}
		\tableHeaderStyle%
		Commande & Explication\\
		mkdir dossier & Crée un dossier dans le répertoire courant\\
		rmdir dossier& Supprime un dossier vide\\
		rmdir -r dossier & Supprime un dossier et son contenu\\
		mv dossier /emplacement2/ & déplace le dossier dans le dossier emplacement2\\
	\end{tabular}
\subsection{Navigation}
\begin{tabular}{_l^l^l}
		\tableHeaderStyle%
		Commande & Explication\\
		cd dossier & Aller dans le dossier\\
		cd ..& Aller dans le dossier parent\\
		cd & Aller dans l'emplacement par défaut\\
	\end{tabular}

	La commande ``cd'' accepte un répertoire, donc les commandes ``cd dossier1'' suivie de ``cd dossier2'' ont le m\^eme effet que ``cd dossier1/dossier2''.
	\section{Filtres}
	Le filtre > permet d'écrire le résultat de ce qui précède vers, par exemple un fichier.

\checkInfo{Exemple}{
	Je veux enregistrer le manuel de la commande ``ls'' dans un fichier nommé documentation.txt. Pour cela, je sais que le manuel est donné par ``man ls''; je fais donc : man ls > documenttation.txt.}

Le filtre | permet de passer le résultat de ce qui précède vers une autre commande.

	\section{Permissions}
	\subsection{Utilisateurs, groupes, fichier}
	Dans les systèmes UNIX, à chaque fichiers sont associés les permissions de :
	\begin{itemize}
	 \item Un propriétaire : par défaut celui qui crée le fichier
	 \item Un groupe : Par défaut le groupe du m\^eme nom que le propriétaire
	 \item Tous les autres
	\end{itemize}
\subsection{Lecture des permissions}
Quand on fait la commande ``ls -all'', la première indique une suite de lettre x,r,w,-. Ce sont les permissions.

Il y a 10 caractères en tout, que l'on peut partager en 4.

\begin{itemize}
 \item Le premier indique s'il s'agit d'un dossier (d) ou d'un fichier (-)
 \item Les 3 autres caractères correspondent aux permissions du propriétaire
 \item les 3 autres aux permissions du groupe
 \item les 3 derniers aux permissions des autres utilisateurs.
\end{itemize}

Chaque ensemble de 3 caractère peut contenir un ``r'', un ``w'', un ``x'' et des ``-'', symbolisant respectivement les droits de lecture, écriture et d'exécution. L'absence de l'un de ces droit est marqué par un ``-''.
\checkInfo{Exemple}{Analysons les permissions {\color{blue}-}{\color{red}rwx}{\color{orange}r-x}{\color{purple}r- -}.

L'élément est un {\color{blue}fichier}

Les permissions du propriétaire sont {\color{red}rwx}, celles des membres du groupe sont {\color{orange}r-x} et celles des autres sont {\color{purple}r- -}.

On en déduit que le propriétaire peut lire, écrire et exécuter le fichier, les membres du groupe peuvent le lire et l'exécuter et les autres peuvent seulement le lire.}
\subsection{Modification des permissions}
On utilise la commande chmod, qui adopte la syntaxe suivante :

\texttt{chmod [option] [mode] [nom\_fichier\_dossier]}

L'option la plus utilisée est -R, pour appliquer la commande de manière récursive.

Le mode correspond au coeur de la commande. On y indique à quels utilisateurs on applique les modification et l'opération sur les permissions.
\subsubsection{Syntaxe du mode}
On indique d'abord à qui on applique les modifications

	\begin{tabular}{_l^l^l}
		\tableHeaderStyle%
		Lettre & Utilisateur(s)\\
		u & Propriétaire\\
		g & Groupe\\
		o & 8 Les Autres\\
	\end{tabular}

On indique ensuite l'opération à effectuer :

	\begin{tabular}{_l^l^l}
		\tableHeaderStyle%
		Signe & Opération\\
		+ & Ajout des permissions\\
		- & retrait des permissions\\
		= & Édition des permissions
	\end{tabular}

On indique ensuite les lettres de permissions (r,w,x) que l'on souhaite utiliser.

\checkInfo{Exemples}{
Pour supprimer la permission d'éxécution au propriétaire, on utilise : \texttt{chmod u-x fichier.txt}

Pour ajouter la permission de modification au groupe et indiquer que les autres ne peuvent que lire le fichier, on tape : \texttt{chmod g+w,o=r fichier.txt}

Pour supprimer la permission de lecture aux autres et ajouter la permissions d'éxécution au propriétaire, on tape \texttt{chmod o-r, u+x fichier.txt}}
\end{document}


