% !TeX spellcheck = en_US
\documentclass[french]{yLectureNote}

\title{Atomistique}
\subtitle{La matière à l'échelle atomique}
\author{Paulhenry Saux}
\date{\today}
\yLanguage{Français}

\professor{J.Cuny}%sebastien.deveuhels.irap.omp.eu

\usepackage{graphicx}%----pour mettre des images
\usepackage[utf8]{inputenc}%---encodage
\usepackage{geometry}%---pour modifier les tailles et mettre a4paper
%\usepackage{awesomebox}%---pour les boites d'exercices, de pbq et de croquis ---d\'esactiv\'e pour les TP de PC
\usepackage{tikz}%---pour deiffner + d\'ependance de chemfig
\usepackage{tkz-tab}
\usepackage{chemfig}%---pour deiffner formules chimiques
\usepackage{chemformula}%---pour les formules chimiques en \'equation : \ch{...}
\usepackage{tabularx}%---pour dimensionner automatiquement les tableaux avec variable X
\usepackage{awesomebox}%---Pour les boites info, danger et autres
\usepackage{menukeys}%---Pour deiffner les touches de Calculatrice
\usepackage{fancyhdr}%---pour les en-t\^ete personnalis\'ees
\usepackage{blindtext}%---pour les liens
\usepackage{hyperref}%---pour les liens (\`a mettre en dernier)
\usepackage{caption}%---pour la francisation de la l\'egende table vers Tableau
\usepackage{pifont}
\usepackage{array}%---pour les tableaux
\usepackage{lipsum}
\usepackage{yFlatTable}
\usepackage{multicol}
\newcommand{\Lim}[1]{\lim\limits_{\substack{#1}}\:}
\renewcommand{\vec}{\overrightarrow}
\begin{document}
\setcounter{chapter}{4}
\chapter{Nomenclature}
\section{Représentation des molécules en chimie organique}
\subsection{Représentation non structurale}
Il y a la formule brute.
\begin{theorem}[Nombre d'insaturation]
\[N_i = N_c+ 1-\frac{1}{2}n_h+\frac12 n_N - \frac12 n_x\]
C'est le nombre de cycles et de liaisons multiples
\end{theorem}
Pour $C_2H_6O : 0$, $C_6H_6 :4$, $C_4H_6O_3 :2$

\begin{definition}[Isomère]
Molécule avec la m\^eme formule brute mais avec des agencements d'atome différentes.
\end{definition}
\subsection{Représentation strcuturale}
Il y a la structure de Lewis
\subsubsection{Formule développée}
lewis mais sans les DNL. Elle représente l'ensemble des liaisons.
\subsubsection{Formule semi-developpée}
On ne représente plus les liaisons $C-H$ et on condense la notation des fonctions.

Exemple : $CH_3-CH_2-OH$
\subsubsection{Formule topologique}
On n'écrit plus $C$ et $H$.
\subsection{Nomenclature}
Alcane = Suffixe en ane

Chaine carbonée principale = la plus longue

Quand on est en ramfication, on a un truc en yle. On prend la numérotation avec les chiffres les plus petits. S'il y a en a plusieurs, on les met par ordre alphabétique.

Alcène : S'il y a une liaison double : 3-methylhex-X-ene, avec X le plus petit.
\subsection{Isomérie}
2 isomères de constitution sont 2 molécules qui ont la m\^eme formule brute mais pas la m\^eme formule développée.
\begin{itemize}
 \item Isomérie de fonction : 2 isomères de constitution avec des fonctions différentes
 \item Isomérie de cha\^ine : M\^eme fonction mais chaine carbonnée différentes
 \item Isomère de position : m\^eme focntion et chaine mais position de la fonction différente.
\end{itemize}
Plus la température d'ébulation est grande plus les intéractions inter moléculaires sont grandes. Géométrie de la fonction : Isomères avec des propriétés physiques différentes
\subsection{Représentation spatiale}
\subsubsection{Représentation de cram}
Le truc habituel de VSEPR (trait gras, trait pointillé, trait normal, etc)
\subsubsection{Projection de Newman}
On l'utilise pour les molécules avec au moins une liaison carbone-carbone. Je met mon oeuil le long de la liaison carbone-carbone

Conformères : Conformation : A basse temprétaure la forme décalée est la forme la plus stable : Meme moléculaire avec des conformations différentes
\section{Stereoismoérie}
Stéréoisomères : Molécules identiques mais non superposables

Plusieurs types :
\begin{itemize}
 \item de géométrie
 \begin{itemize}
  \item sur les doubles liaisons
 \end{itemize}

\end{itemize}


\end{document}

